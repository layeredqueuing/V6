%% -*- mode: latex; mode: outline-minor; fill-column: 108 -*-
%% Title:  userman
%%
%% $HeadURL: http://rads-svn.sce.carleton.ca:8080/svn/lqn/trunk/doc/userman/userman.tex $
%% Original Author:     Greg Franks <greg@sce.carleton.ca>
%% Created:             Tue Oct  4 2005
%%
%% ----------------------------------------------------------------------
%% $Id: userman.tex 12086 2014-07-11 14:34:15Z greg $
%% ----------------------------------------------------------------------

\documentclass{report}
\usepackage[T1]{fontenc}
\usepackage{times}
\usepackage{bnf,epsf,dcolumn,verbatim,subfig,makeidx,rotating,color,listings,multicol}
\usepackage{hyperref}
\newcolumntype{d}[1]{D{.}{.}{#1}}
% Some useful definitions...

\marginparwidth 72pt
\oddsidemargin 0pt
\evensidemargin 0pt
\topmargin -32pt
\textwidth 6.5in
\textheight 8.7in

\makeindex
\newcommand{\flag}[2]{\texttt{-#1}\emph{#2}\index{-#1@\texttt{-#1}}}
\newcommand{\longopt}[1]{\texttt{--#1}\index{--#1@\texttt{--#1}}}
\newcommand{\ctrlparam}[1]{\texttt{#1}\index{#1@\texttt{#1}}}
\newcommand{\optarg}[2]{\emph{#1}{#2}\index{#1@\textit{#1}}}
\newcommand{\pragma}[1]{\emph{#1}\index{#1@\textit{#1}}}
\newcommand{\manpage}[2]{\emph{#1(#2)}\index{#1}}
\newcommand{\indexerror}[1]{\index{error!#1}\index{#1!error}}
\newcommand{\dindex}[2]{\index{#1!#2}\index{#2!#1}}
\newcommand{\schemaelement}[1]{\texttt{#1}\index{element!#1}\index{#1@\textit{#1}}}
\newcommand{\attribute}[1]{\texttt{#1}\index{attribute!#1}\index{#1@\texttt{#1}}}
\newcommand{\schematype}[1]{\textbf{#1}\index{type!#1}\index{#1@\textbf{#1}}}

\lstset{
  basicstyle=\small,
  keywordstyle=\bfseries,
  identifierstyle=,
  commentstyle=\slshape,
  stringstyle=\ttfamily,
  showstringspaces=false
  frame=leftline,
  numbers=left,
  numberstyle=\tiny,
  stepnumber=1,
  numbersep=-4pt
}


\begin{document}

\title{Layered Queueing Network Solver and Simulator User Manual}
\author{Greg Franks \and Peter Maly \and Murray Woodside \and Dorina
  C. Petriu \and Alex Hubbard \and Martin Mroz}
\date{Department of Systems and Computer Engineering\\
  Carleton University\\
  Ottawa ON K1S~5B6\\
  \texttt{\{cmw,greg\}@sce.carleton.ca}\\[1cm]
  \today\\[1cm]
  $\ $Revision: 12086 $\ $ } \maketitle \clearpage
\tableofcontents
\listoffigures
\listoftables
\clearpage
\begin{abstract}
  
The Layered Queuing Network\index{Layered Queueing Network} (LQN) model is a canonical form for extended
queueing networks\index{queueing network!extended}\index{queueing network!layered} with a layered structure.
The layered structure arises from servers at one level making requests to servers at lower levels as a
consequence of a request from a higher level. LQN was developed for modeling software systems, but it
applies to any extended queueing network with multiple resource possession\index{resource!possession}, in
which multiple resources are held in a nested fashion.

This document describes the elements found in Layered Queueing Network Model, the results produced when a
LQN model is solved, and the input and output file formats.  It also describes the method used to invoke the
analytic and simulation solvers, and the possible errors that can arise when solving a model.  The reader is
referred to ``Tutorial Introduction to Layered Modeling of Software Performance''~\cite{sw:woodside-xx} for
constructing models.
\end{abstract}

%%  -*- mode: latex; mode: outline-minor; fill-column: 108 -*-
%% Title:  model
%% $HeadURL: http://rads-svn.sce.carleton.ca:8080/svn/lqn/trunk-V6/doc/userman/model.tex $
%%
%% Doc: /home/greg/usr/src/doc/userman/model.tex
%% Original Author:     Greg Franks <greg@sce.carleton.ca>
%% Created:             Tue Jul 18 2006
%%
%% ----------------------------------------------------------------------
%% $Id: model.tex 17289 2024-09-13 18:55:26Z greg $
%% ----------------------------------------------------------------------

\chapter{The Layered Queueing Network Model}
\label{sec:lqn}

Figure~\ref{fig:bookstore} illustrates the LQN notation with an
example of an on-line e-commerce system. In an LQN, software
resources\index{resource!software} are all called
``tasks''\index{task}, have queues\index{queue} and provide classes of
service which are called ``entries''\index{entry}.  The
demand\index{demand} for each class of service\index{service!class}
can be specified through ``phases''\index{phase}, or for more complex
interactions, using ``activities''\index{activity}.  In
Figure~\ref{fig:bookstore}, a task is shown as a parallelogram,
containing parallelograms for its entries and rectangles for
activities. Processor\index{processor} resources are shown as circles,
attached to the tasks that use them. Stacked icons\index{icon!stacked}
represent tasks or processors with multiplicity\index{multiplicity},
making it a multiserver\index{multiserver}. A multiserver may
represent a multi-threaded task, a collection of identical users, or a
symmetric multiprocessor with a common scheduler.
Multiplicity\index{multiplicity} is shown on the diagram with a label
in braces. For example there are five copies of the task `Server' in
Figure~\ref{fig:bookstore}.

\begin{figure}[htbp]
  \centering
  \epsfxsize=\textwidth
  \epsffile{bookstore/bookstore.eps}
  \caption{Notation}
  \label{fig:bookstore}
\end{figure}

Entries and activities have directed arcs\index{arcs} to other entries
at lower layers to represent service
requests\index{service!request}\index{request} (or
messages)\footnote{requests may jump over
  layers\index{layer!spanning}, such as the request from the
  Administrator task to the InventoryMgr task.}. A request from an
entry or an activity to an entry may return a
reply\index{request!reply}\index{reply} to the requester (a
synchronous request\index{request!synchronous}, or
\emph{rendezvous}\index{rendezvous}) indicated in
Figure~\ref{fig:bookstore} by solid arrows with closed arrowheads.
For example, task Administrator makes a request to task BackorderMgr
who then makes a request to task InventoryMgr. While task InventoryMgr
is servicing the request, tasks BackorderMgr and Administrator are
blocked\index{request!blocked}.  A request may be
forwarded\index{forwarding}\index{request!forward} to another entry
for later reply, such as from InventoryMgr to CustAccMgr.  Finally a
request may not return any reply at all (an asynchronous
request\index{request!asynchronous} or
\emph{send-no-reply}\index{send-no-reply}, shown as an arrow with an
open arrow head, for example, the request from task ShoppingCart to
CustAccMgr.

The first way that the demand at entries can be specified is through
phases\index{phase}.  The parameters of an
entry\index{entry!parameters} are the mean number of requests for
lower entries (shown as labels in parenthesis on the request arcs),
and the mean total host demand for the entry (in units of time, shown
as a label on the entry in brackets). An entry may continue to be busy
after it sends a reply\index{reply}, in an asynchronous ``second
phase''\index{phase!second}\index{phase!asynchronous} of
service~\cite{perf:franks-99b} so each parameter is an array of values
for the first and second phase.  Second phases are a common
performance optimization, for example for transaction cleanup and
logging, or delayed write operations.

The second way that demand can be specified is through
activities\index{activity}.  Activities are the lowest level of
granularity in a performance model and are linked together in a
directed graph to indicate precedence\index{precedence!activity}.
When a request arrives at an entry, it triggers the first activity of
the activity graph\index{activity graph}.  Subsequent activities may
follow sequentially, or may fork\index{fork} into multiple paths which
later join\index{join}.  The fork may take the form of an `AND' which
means that all the activities on the branch\index{branch!AND} after
the fork can run in parallel, or in the form of an `OR', which chooses
one of the branches\index{branch!OR} with a specified
probability\index{branch!probability}.  In Figure~\ref{fig:bookstore},
a request that is received by entry ``SCE3'' of task ``ShoppingCart''
is processed using an activity called ``SCE3A95'' that represents the
main thread\index{thread} of control, then the main thread is
OR-Forked into two branches, one of which is later
AND-forked\index{AND-fork} into three threads.  The three threads,
starting with activities `AFBA109', `AFBA130' and `AFBA133'
respectively, run in parallel.  The first thread
replies\index{reply!activity}\index{activity!reply} to the entry
through activity `OJA110' then ends.  The remaining two threads join
into one thread at activity `AJA131'.  When both `OJA110' and `AJA131'
terminate, the task can accept a new request.

The holding time\index{holding time} for one class of service is the
entry service time, which is not a constant parameter but is
determined by its lower servers.  Thus the essence of layered queuing
is a form of simultaneous resource
possession\index{resource!possession!simultaneous}. In software
systems delays and congestion are heavily influenced by
synchronous\index{request!synchronous} interactions such as remote
procedure calls\index{remote procedure call} (RPCs) or
rendezvous\index{rendezvous}, and the LQN model captures these delays
by incorporating the lower layer queueing and service into the service
time of the upper layer server.  This ``active
server''\index{server!active}\index{active server}
feature~\cite{srvn:woodside-94-ieeetc-srvn} is the key difference between layered
and ordinary queueing networks.

\section{Model Elements}
\label{sec:elements}

Figure~\ref{fig:meta-model} shows the
\emph{meta-model}\index{meta model} used to describe Layered Queueing
Networks\index{Layered Queueing Network}.  This model is unique in
that it is more closely aligned with the architecture of a software
system that it is with a conventional queueing network model such as
Performance Model Interchange
Format\index{Performance Model Interchange Format}
(PMIF)~\cite{perf:smith-99-jss-pmif,perf:smith-2004-pmif2}.  The latter consists of
stations with queues and visits, whereas a LQN has processors, tasks
and requests.

A Layered Queueing Network is a directed graph.  Nodes in the
graph\index{node} consist of tasks\index{task},
processors\index{processor}, entries\index{entry},
activities\index{activity}, and precedence\index{precedence}.
Arcs\index{arcs} in the graph consist of requests\index{request} from
one node to another.  The model objects are described below.

\begin{figure}[htbp]
  \centering
%  \epsfxsize=\textwidth
  \epsffile{model/meta-model.eps}
  \caption{LQN Meta Model}
  \label{fig:meta-model}
\end{figure}

\subsection{Processors}
\label{sec:processors}

Processors\index{processor|(textbf} are used by the
activities\index{activity} within a performance model to consume
\emph{time}.  They are \emph{pure servers}\index{server!pure} in that
they only accept requests from other servers and clients.  They can be
actual processors in the system, or may simply be place holders for
tasks representing customers and other logical resources.

Each processor has a single queue for requests.  Requests may be
scheduled using the following queueing disciplines:
\begin{description}
\item[FIFO] First-in, first out\index{scheduling!fifo} (first-come,
  first-served).  Tasks are served in the order in which they arrive.
\item[PRI] Priority, preemptive resume\index{priority!preemptive resume}\index{scheduling!pri}.  Tasks with
  priorities\index{processor!priority}\index{priority!processor} higher than the task currently running on
  the processor will preempt the running task.  Priorities\index{priority!highest} range from zero to
  positive infinity, with a priority of zero being the highest.  The default priority for all tasks is zero.
\item[HOL] Head-of-line priority\index{priority!head of line}\index{scheduling!hol}.  Tasks
  with higher priorities will be served by the processor first.  Tasks
  in the queue will not preempt a task running on the processor even
  though the running task may have a lower priority.
\item[PS] Processor sharing\index{processor!sharing|textbf}\index{scheduling!ps}.  The processor
  runs all tasks ``simultaneously''.  The rate of service by the
  processor is inversely proportional to the number of executing
  tasks.  For \emph{lqsim}\index{lqsim!scheduling}, processor
  sharing\index{scheduling!processor sharing} is implemented as
  \emph{round-robin}\index{round robin}\index{scheduling!round robin}
  -- a \emph{quantum}\index{quantum} must be specified.
\item[INF] Infinite (delay)\index{processor!delay}\index{scheduling!delay}\index{scheduling!infinite}. 
\item[RAND] Random scheduling\index{scheduling!random}.  The processor selects a task at random. 
\item[CFS] Completely fair scheduling\index{scheduling!cfs}\index{scheduling!completely fair}~\cite{perf:li-2009-mascots-fairshare}.  Tasks are scheduled within groups\index{group} using
  round-robin scheduling and groups are scheduled according to their share\index{share}.  A
  \emph{quantum}\index{quantum} must be specified.  This scheduling discipline is implemented on the
  simulator only at present.
\end{description}

Each processor can have multiple cores, all of which are served by the common queue (see
\S\protect\ref{sec:multiplicity}).  The processor
multiplicity\index{processor!multiplicity}\index{multiplicity!processor} must be a integer greater than zero
or the special constant
\texttt{@infinity}\index{multiplicity!"@infinity@\texttt{"@infinity}}\index{"@infinity@\texttt{"@infinity}}.  If
the multiplicity is set to \texttt{@infinity}, the processor is coerced to a delay
server\index{processor!delay}.  \index{processor|)}

Priorities\index{priority!highest} range from zero to positive
infinity, with a priority of zero being the lowest.  The default
priority for all tasks is zero.\index{processor|)}

\subsection{Groups}
\label{sec:groups}

Groups\index{group|(\textbf}\cite{perf:li-2009-mascots-fairshare} are used to divide up a processor's
execution time up into \emph{shares}\index{share}.  The tasks within a group divide the share up among
themselves evenly.  Groups can only be created on processors running the scheduling discipline
\emph{completely fair scheduling},\index{scheduling!completely fair}.  \index{group|)}.

Shares may either be \emph{guaranteed}\index{share!guarantee} or \emph{capped}\index{share!cap}.  Guarantee
shares act as a floor for the share that a group receives.  If surplus CPU time is available (i.e., the
processor is not fully utilized), tasks in a guaranteed group can exceed their share\index{share!exceed}.
Cap shares act as a hard ceiling.  Tasks within these groups will never receive more than their share of CPU
time.  

Note: Completely fair scheduling is a form of priority scheduling\index{scheduling!priority}.  With layered
models, calls made by tasks within groups to lower level servers can cause \emph{priority
  inversion}\index{priority!inversion}.  Cap scheduling tends to behave better than guaranteed scheduling
for these cases.

\subsection{Tasks}
\label{sec:tasks}

Tasks\index{task|(textbf} are used in layered queueing networks to
represent resources.  Resources include, but are not limited to:
actual tasks (or processes) in a computer system,
customers\index{customer}, buffers\index{buffers}, and hardware
devices.  In essence, whenever some entity requires some sort of
service, requests between tasks involved.

A task has a queue\index{task!queue} for requests and runs on a
processor.  Items are served from the queue in a first-come,
first-served manner.  Different classes of
service\index{service!class} are specified using
\emph{entries}\index{entry} (c.f.~\S\ref{sec:entries}).  Tasks may
also have internal concurrency\index{concurrency}, specified using
\emph{activities}\index{activity} (c.f.~\S\ref{sec:activities}).

Requests can be served using the following scheduling
methods\index{scheduling!task}:
\begin{description}
\item[FIFO] First-in, first out\index{scheduling!fifo} (first-come,
  first-served).  Requests are served in the order in which they
  arrive.  This scheduling discipline is the default for tasks.
\item[PRI] Priority, preemptive resume\index{priority!preemptive resume}\index{scheduling!pri}.  Requests
  arriving at entries with priorities\index{entry!priority}\index{priority!entry} higher than entry that
  task is currently processing will preempt the execution of the current entry.
  Priorities\index{priority!highest} range from zero to positive infinity, with a priority of zero being the
  highest.  The default priority for all entries is zero.
\item[INF] Infinite (delay)\index{task!delay}\index{scheduling!delay}\index{scheduling!infinite}. 
\item[HOL] Head-of-line priority\index{priority!head of line}\index{scheduling!hol}.
  Requests arriving at entries with higher priorities will be served
  by the task first.  Requests in the queue will not preempt the
  processing of the current entry by the task.
\end{description}

<<<<<<< .working
Each task can have multiple homogenous
threads\index{task!multiplicity}\index{task!threads}\index{threads!homogenous}, all of which are served by
the common queue (see \S\protect\ref{sec:multiplicity}).  The task
multiplicity\index{task!multiplicity}\index{multiplicity!task} must be a integer greater than zero or the
special constant
\texttt{@infinity}\index{multiplicity!"@infinity@\texttt{"@infinity}}\index{"@infinity@\texttt{"@infinity}}.  If
the multiplicity is set to \texttt{@infinity}, the task is coerced to a delay server\index{task!delay}.
||||||| .merge-left.r17282
Priorities\index{priority!highest} range from zero to positive
infinity, with a priority of zero being the highest.  The default
priority for all entries is zero.
=======
Priorities\index{priority!highest} range from zero to positive
infinity, with a priority of zero being the lowest.  The default
priority for all entries is zero.
>>>>>>> .merge-right.r17288

The subclasses of \emph{task} are:
\begin{description}
\item[\emph{Reference Task:}] Reference tasks\index{task!reference}\index{reference task|textbf} are used to
  represent customers\index{customer} in the layered queueing network.  They are like normal tasks in that
  they have entries and can make requests.  However, they can never receive requests and are always found at
  the top of a call graph.  They typically generate traffic in the underlying closed queueing
  model\index{queueing model!closed} by making rendezvous\index{rendezvous!reference task} requests to
  lower-level servers.  Reference tasks can also generate traffic in the underlying open queueing
  model\index{queueing model!open} by making send-no-reply requests instead of rendezvous requests.
  However, open class customers are more typically represented using open arrivals which is simply encoded
  as a parameter to an entry.
  
  \emph{Bursty} reference tasks\index{task!reference!bursty|textbf}\index{reference task!bursty|textbf} are
  a special case of reference tasks where the service time for the slices\index{slice} are random variables
  with a \emph{Pareto} distribution\index{Pareto distribution} (c.f.~\S\ref{sec:slices}).
\item[\emph{Semaphore Task:}] Semaphore tasks\index{task!semaphore}\index{semaphore task|textbf} are used to
  model passive resources\index{resource!passive} such as buffers.  They always have two entries which are
  used to \emph{signal}\index{entry!signal}\index{signal} and \emph{wait}\index{entry!wait}\index{wait} the
  semaphore.  The wait entry must be called using a synchronous request whereas the signal entry can be
  called using any type of request.  Once a request is accepted by the wait entry, no further requests will
  be accepted until a request is processed by the signal entry.  The signal and wait entries do not have to
  called from a common task.  However, the two entries must share a common call graph, and the call graph
  must be deterministic.  The entries themselves can be defined using phases or activies and can make
  requests to other tasks.  Counting semaphores\index{semaphore!counting} can be modeled using a
  multiserver.
\item[\emph{Synch Task:}] Synchronization tasks\index{task!synchronization}
  \index{synchronization task|textbf} are used...  Cannot be a multiserver.\typeout{Write me!}
\end{description}\index{task|)}


\subsection{Entries}
\label{sec:entries}

Entries\index{entry|(textbf} service requests and are used to
differentiate the service provided by a task.  An entry can accept
either synchronous\index{message!synchronous}\index{message|see{request}}, or
asynchronous\index{message!asynchronous} requests, but not both.
Synchronous requests are part of the \emph{closed} queueing
model\index{queueing model!closed}\index{closed model} whereas
asynchronous requests are part of the \emph{open} 
model\index{queueing model!open}\index{open model}.  Message types are
described in Section~\ref{sec:requests} below.

Entries also generate the replies for synchronous
requests\index{rendezvous}.  Typically, a reply to a message is
returned to the client who originally sent the message.  However,
entries may also \emph{forward}\index{forwarding} the reply.  The next
entry which accepts the forwarded reply may forward the message in
turn, or may reply back to the originating client.  For example, in
Figure~\ref{fig:bookstore}, entry `IME8' on task `IventoryMgr'
forwards the request from entry `BME2' on task `BackorderMgr' to entry
`CAME5' on task `CustAccMgr'.  The reply from `CAME2' will be sent
directly back to `BME2'.

The parameters for an entry can be specified using either
phases\index{phase} or activities\index{activity}\footnote{The
  meta-model in Figure~\protect\ref{fig:meta-model} only shows
  activities, phases are a notational short-hand.}.  The activity
method is typically used when a task has complex internal behaviour
such as forks\index{fork} and joins\index{join}, or if its behaviour
is specified as an activity graph\index{activity graph} such as those
used by Smith and Williams~\cite{perf:smith-2002}.  The phase method
is simply a short hand notation for specifying a sequence of one to
three activities, with the
reply\index{reply!phase}\index{reply!activity} being generated by the
first activity in the sequence.  Figure~\ref{fig:entry-specification}
shows both methods for specifying a two-phase client calling a
two-phase server.

\begin{figure}[htbp]
  \centering
  \subfloat[Phases]{\epsffile{model/entry-phases.eps}}
  \subfloat[Activities]{\epsffile{model/entry-activities.eps}}
  \caption{Entry Specification}
  \label{fig:entry-specification}
\end{figure}

Regardless of the specification method used for an entry, its
behaviour as a server to its clients is by \emph{phase}\index{phase},
shown in Figure~\ref{fig:phase}.  Phases consume time on processors
and make requests to entries.  Phase one is a \emph{service phase} and
is similar to the service given by a station in a queueing network.
Phase one ends after the server sends a reply\index{reply!phase}.
Subsequent phases are \emph{autonomous}\index{autonomous phase}\index{phase!autonomous} phases
which are launched by phase one.  These phases operate in parallel
with the clients which initiated them.  The simulator and analytic
solver limit the number of phases to three.

\begin{figure}[htbp]
  \centering
  \epsffile{model/phase.eps}
  \caption{Phases for an Entry.\protect\index{phase}}
  \label{fig:phase}
\end{figure}

\subsection{Activities}
\label{sec:activities}

Activities\index{activity|textbf} are the lowest-level of
specification in the performance model.  They are connected together
using ``Precedence'' (c.f.~\S\ref{sec:precedence})\index{precedence}
to form a directed graph\index{directed graph} to represent more than
just sequential execution scenarios.

Activities consume time on processors.  The \emph{service time}\index{service time} is defined by a mean and
variance, the latter through \index{coefficient of variation}\emph{coefficient of variation squared}
\footnote{The squared coefficient of variation is variance divided by the square of the mean.}. The service
time between requests to lower level servers is assumed to be exponentially distributed (with the exception
of \emph{bursty reference tasks}\index{task!reference!bursty}) so the total service time is the sum of a
random number of exponentially distributed random variables.

Activities also make requests to entries on other tasks.  The distribution of requests to lower level
servers is set by the \emph{call order}\index{call order} for the activity which is either
\emph{stochastic}\index{stochastic} or \emph{deterministic}\index{deterministic}.  If the call order is
deterministic, the activity makes the exact number of requests specified to the lower level servers.  The
number of requests is integral; the order of requests to different entries is not defined.  If the call
order is stochastic, the activity makes a random number of requests to the lower level servers.  The mean
number of requests is specified by the value specified.  Requests are assumed to be geometrically
distributed.

For entries which accept rendezvous\index{rendezvous} requests,
replies must be generated.  If the entry is specified using phases,
the reply\index{reply!implicit} is implicit after phase
one\index{phase!reply}.  However, if the entry is specified using
activities\index{activity!reply}, one or more of the activities must
explicitly\index{reply!explicitly} generate the reply.  Exactly one
reply must be generated for each request.\index{entry|)}

\subsubsection{Slices}\index{slice|(textbf}
\label{sec:slices}

Activities consume time by making requests to the processor associated with the task.  The service time
demand specified for an activity is divided into {\em slices} between requests to other entries, shown in
the UML Sequence Diagram in Figure~\ref{fig:slices}. The mean number of slices is always $1 + Y$ where $Y$
is total total number of requests made by the activity.  

\begin{figure}[htbp]
  \centering
  \epsffile{model/slice.eps}
  \caption{Slices\protect\index{slice}.  The \emph{slice time} is shown using the label $\zeta$.}
  \label{fig:slices}
\end{figure}

By default, the demand of a \emph{slice} is assumed to be exponentially distributed~\cite{srvn:woodside-94-ieeetc-srvn}
but a variance may be specified through the \index{coefficient of variation|textbf}\emph{coefficient of
  variation squared} ($\textit{cv}^2 = \sigma^2 / \overline{s}^2$) parameter for the entry or activity.  The
method used to solve the model depends on the solver being used:
\begin{description}
\item[Analytic Solver:] All servers with $\textit{cv}^2 \ne 1$ use the HVFCFS MVA approximation
  from~\cite{queue:reiser-79}.  
\item[Simulator:] The simulator uses the following distributions for generating random variates for slice
  times provided that the task is \emph{not} a bursty reference task.
  \begin{description}
  \item[$\textit{cv}^2 = 0$:] deterministic.
  \item[$0 < \textit{cv}^2 < 1$:] gamma\index{distribution!gamma}.
  \item[$\textit{cv}^2 = 1$:] exponential\index{distribution!exponential}.
  \item[$\textit{cv}^2 > 1$:] bizarro...
  \end{description}
  If the task is a bursty reference task\index{reference task!bursty}, then the simulator generates random
  variates for slice times according to the Pareto distribution\index{distribution!Pareto}. The scale $x_m >
  0$ and shape $k > 0$ parameters for the distribution are derived from the service time $s$ and coefficient
  of variation squared $\textit{cv}^2$ parameters for the corresponding activity (or phase).
  \begin{eqnarray*}
    k & = & \sqrt{\frac{1}{\textit{cv}^2} + 1} + 1 \\
    x_m & = & s \times \frac{(k - 1)}{k}
  \end{eqnarray*}
  On-off\index{on-off behaviour} behaviour can simulated by using two or more phases at the client, where on
  phase corresponds to the on period and makes requests to other servers, while the other phase corresponds
  to the off period.
\end{description}
\index{slice|)}

\subsection{Precedence}
\label{sec:precedence}

\emph{Precedence}\index{precedence|(textbf} is used to connect
activities within a task to from an \emph{activity graph}.  Referring
to Figure~\ref{fig:meta-model}, precedence is subclassed into
`\textbf{Pre}'\index{pre-precedence} (or \emph{`join'}\index{join!precedence})
and `\textbf{Post}'\index{post-precedence} (or \emph{`fork'}\index{fork!precedence}).  To
connect one activity to another, the source activity connects to a
\emph{pre-}precedence (or a \emph{join-}list\index{join-list}).  The
\emph{pre-}precedence then connects to a \emph{post-}precedence (or a
\emph{fork-}list\index{fork-list}) which, in turn, connects to the
destination activity.  Table~\ref{tab:activity-notation} summarizes
the precedence types.

\begin{table}[htbp]
  \begin{center}
    \begingroup\makeatletter\ifx\SetFigFont\undefined%
    \gdef\SetFigFont#1#2#3#4#5{%
      \reset@font\fontsize{#1}{#2pt}%
      \fontfamily{#3}\fontseries{#4}\fontshape{#5}%
      \selectfont}%
    \fi\endgroup%
    \leavevmode
    \begin{tabular}{|m{1in}|m{3cm}|m{3.5in}|}
      \hline
      Name & Icon & Description\\
      \hline
      \hline
%
% Join lists (pre-)
%
      Sequence\index{precedence!sequence}
      & \centering\setlength{\unitlength}{4144sp}%
      \begin{picture}(1104,339)(529,107)
        \thinlines
        \put(541,119){\makebox(1080,315){}}
        \put(1081,389){\line( 0,-1){225}}
      \end{picture}%
      & Transfer of control from an activity to a join-list.\\
      \hline
      And-Join\index{precedence!and-join}\index{and~join} 
      & \centering\setlength{\unitlength}{4144sp}%
      \begin{picture}(1104,429)(529,-28)
        \thinlines
        \put(541,-16){\makebox(1080,585){}}
        \put(1081,119){\circle{180}}
        \multiput(856,344)(5.78571,-5.78571){29}{\makebox(1.5875,11.1125){\SetFigFont{5}{6}{\rmdefault}{\mddefault}{\updefault}.}}
        \put(1018,182){\vector( 1,-1){0}}
        \put(1081, 67){\makebox(0,0)[b]{\smash{{\SetFigFont{10}{12.0}{\rmdefault}{\mddefault}{\updefault}\&}}}}
        \multiput(1306,344)(-5.64286,-5.64286){29}{\makebox(1.5875,11.1125){\SetFigFont{5}{6}{\rmdefault}{\mddefault}{\updefault}.}}
        \put(1148,186){\vector(-1,-1){0}}
      \end{picture}%
      & A Synchronization point for concurrent activities.\\
      \hline
      Quorum-Join\index{precedence!quourm-join}\index{quorum~join} 
      & \centering\setlength{\unitlength}{4144sp}%
      \begin{picture}(1104,429)(529,-28)
        \thinlines
        \put(541,-16){\makebox(1080,585){}}
        \put(1081,119){\circle{180}}
        \multiput(856,344)(5.78571,-5.78571){29}{\makebox(1.5875,11.1125){\SetFigFont{5}{6}{\rmdefault}{\mddefault}{\updefault}.}}
        \put(1018,182){\vector( 1,-1){0}}
        \put(1081, 67){\makebox(0,0)[b]{\smash{{\SetFigFont{10}{12.0}{\rmdefault}{\mddefault}{\itdefault}n}}}}
        \multiput(1306,344)(-5.64286,-5.64286){29}{\makebox(1.5875,11.1125){\SetFigFont{5}{6}{\rmdefault}{\mddefault}{\updefault}.}}
        \put(1148,186){\vector(-1,-1){0}}
      \end{picture}%
      & A Synchronization point for concurrent activities where only $n$ branches must finish.\\
      \hline
      Or-Join\index{precedence!or-join}\index{or~join} 
      & \centering\setlength{\unitlength}{4144sp}%
      \begin{picture}(1104,429)(529,17)
        \thinlines
        \put(541, 29){\makebox(1080,405){}}
        \put(1081,164){\circle{180}}
        \multiput(856,389)(5.64286,-5.64286){29}{\makebox(1.5875,11.1125){\SetFigFont{5}{6}{\rmdefault}{\mddefault}{\updefault}.}}
        \put(1014,231){\vector( 1,-1){0}}
        \multiput(1306,389)(-5.67857,-5.67857){29}{\makebox(1.5875,11.1125){\SetFigFont{5}{6}{\rmdefault}{\mddefault}{\updefault}.}}
        \put(1147,230){\vector(-1,-1){0}}
        \put(1081,112){\makebox(0,0)[b]{\smash{{\SetFigFont{10}{12.0}{\rmdefault}{\mddefault}{\updefault}+}}}}
      \end{picture}%
      & \mbox{} \\
      \hline
      \hline
%
% Fork lists (post-precedence)
%
      Sequence\index{precedence!sequence}
      & \centering\setlength{\unitlength}{4144sp}%
      \begin{picture}(1104,339)(529,107)
        \thinlines
        \put(541,119){\makebox(1080,315){}}
        \put(1081,389){\vector( 0,-1){225}}
      \end{picture}%
      & Transfer of control from fork-list to activity\\
      \hline
      And-Fork\index{precedence!and-fork}\index{and~fork} 
      & \centering\setlength{\unitlength}{4144sp}%
      \begin{picture}(1104,474)(529,-28)
        \thinlines
        \put(541,-16){\makebox(1080,450){}}
        \multiput(1019,192)(-5.62069,-5.62069){30}{\makebox(1.5875,11.1125){\SetFigFont{5}{6}{\rmdefault}{\mddefault}{\updefault}.}}
        \put(856, 29){\vector(-1,-1){0}}
        \multiput(1146,189)(5.71429,-5.71429){29}{\makebox(1.5875,11.1125){\SetFigFont{5}{6}{\rmdefault}{\mddefault}{\updefault}.}}
        \put(1306, 29){\vector( 1,-1){0}}
        \put(1081,254){\circle{180}}
        \put(1081,202){\makebox(0,0)[b]{\smash{{\SetFigFont{10}{12.0}{\rmdefault}{\mddefault}{\updefault}\&}}}}
      \end{picture}%
      & Start of concurrent execution.  There can be
      any number of forked paths. \\
      \hline
      Or-Fork\index{precedence!or-fork}\index{or~fork} 
      & \centering\setlength{\unitlength}{4144sp}%
      \begin{picture}(1104,474)(529,-28)
        \thinlines
        \put(541,-16){\makebox(1080,450){}}
        \put(1081,254){\circle{180}}
        \put(1081,202){\makebox(0,0)[b]{\smash{{\SetFigFont{10}{12.0}{\rmdefault}{\mddefault}{\updefault}+}}}}
        \put(901,119){\makebox(0,0)[rb]{\smash{{\SetFigFont{8}{9.6}{\rmdefault}{\mddefault}{\updefault}$p$}}}}
        \put(1261,119){\makebox(0,0)[lb]{\smash{{\SetFigFont{8}{9.6}{\rmdefault}{\mddefault}{\updefault}$1-p$}}}}
        \multiput(1144,191)(5.78571,-5.78571){29}{\makebox(1.5875,11.1125){\SetFigFont{5}{6}{\rmdefault}{\mddefault}{\updefault}.}}
        \put(1306, 29){\vector( 1,-1){0}}
        \multiput(1017,190)(-5.75000,-5.75000){29}{\makebox(1.5875,11.1125){\SetFigFont{5}{6}{\rmdefault}{\mddefault}{\updefault}.}}
        \put(856, 29){\vector(-1,-1){0}}
      \end{picture}%
      & A branching point where one of the paths
      is selected with probability $p$.  There can be any number of branches.\\
      \hline
      Loop\index{precedence!loop}\index{loop} 
      & \centering\setlength{\unitlength}{4144sp}%
      \begin{picture}(1104,474)(529,-28)
        \thinlines
        \put(541,-16){\makebox(1080,450){}}
        \put(1081,299){\circle{180}}
        \put(1081,209){\vector( 0,-1){180}}
        \put(1017,235){\vector(-1,-1){206}}
        \put(1145,235){\vector( 1,-1){206}}
        \put(856,119){\makebox(0,0)[rb]{\smash{{\SetFigFont{8}{9.6}{\rmdefault}{\mddefault}{\updefault}$n_1$}}}}
        \put(1036, 29){\makebox(0,0)[rb]{\smash{{\SetFigFont{8}{9.6}{\rmdefault}{\mddefault}{\updefault}$n_2$}}}}
        \put(1081,226){\makebox(0,0)[b]{\smash{{\SetFigFont{10}{12.0}{\rmdefault}{\mddefault}{\updefault}*}}}}
      \end{picture}%
      & Repeat the activity an average of $n$ times.\\
      \hline
    \end{tabular}
    \caption{\label{tab:activity-notation}Activity graph notation.}
  \end{center}
\end{table}

The semantics of an activity graph\index{activity graph!semantics} are
as follows.  For AND-forks\index{AND-fork}, AND-joins\index{AND-join} and QUORUM-joins\index{QUORUM-join},
each branch of a join must originate from a common fork, and each
branch of the join must have a matching branch from the fork.
Branches\index{branch!AND} from AND-forks need not necessarily join, either explictily by a ``dangling''
thread not participating in a join, or implicitly through a quorum 
join\index{join!quorum}\index{quorum join}, where only  a subset of the branches must join while ignoring
the rest. However, all threads started by a fork must terminate before the task
will accept a new message (i.e., there is an implied join collecting
all threads at the end of a task's cycle).  Branches to an AND-join do
not necessarily have to originate from a fork -- for this case each
branch must originate from a unique entry.  This case is used to
synchronize\index{server!synchronize}\index{synchronization server}
two or more clients at the server.

For OR-forks\index{OR-fork}, the sum of the
probabilities\index{branch!probability}\index{probability!branch} of
the branches must sum to one -- there is no ``default'' operation.
AND-forks may join at OR-joins\index{OR-join}.  The threads from the
AND-fork implicitly join when the task cycle completes.  OR-joins may
be called directly from entries.  This case is analogous to running
common code for different requests to a task.

LOOPs\index{LOOP} consist of one or more
branches\index{branch!loop count}\index{loop count}, each of which is
run a random number of times with the specified mean, followed by an
optional deterministic branch\index{branch!deterministic}
exit\index{branch!exit} which is followed after all the looping has
completed.

Replies\index{activity!reply} can only occur from activities in
\emph{pre-}precedence (\emph{and-}join) lists.  Activities cannot
reply to entries from a loop branch because the number of times that a
branch is executed is a random number.\index{precedence|)}

\subsection{Requests}
\label{sec:requests}

Service requests\index{request|(textbf} from one task to another can be
one of three types: rendezvous\index{rendezvous},
forwarded\index{forwarding}, and send-no-reply\index{send-no-reply},
shown in Figure~\ref{fig:request-types}.  A rendezvous request is a
blocking synchronous request -- the client is suspended while the
server processes the request.  A send-no-reply request is an
asynchronous request -- the client continues execution after the send
takes place.  A forwarded request results when the reply to a client
is redirected to a subsequent server which, may forward the request
itself, or may reply to the originating client.\index{request|)}

\begin{figure}[htbp]
  \centering
  \subfloat[Rendezvous]{\hspace{1cm}\epsffile{model/request-rnv.eps}\hspace{1cm}}
  \subfloat[Forwarding]{\hspace{1cm}\epsffile{model/request-fwd.eps}\hspace{1cm}}
  \subfloat[Send-no-reply]{\hspace{1cm}\epsffile{model/request-snr.eps}\hspace{1cm}}
  \caption{Request Types.\index{request!types}}
  \label{fig:request-types}
\end{figure}

\section{Multiplicity and Replication}
\label{sec:replication}\label{sec:multiplicity}\index{replication|(textbf}\index{multiplicity|(textbf}

One common technique to improve the performance of a system is to add
copies of servers.  The performance model supports two techniques:
multiplicity\index{multiplicity|textbf} and
replication\index{replication|textbf}.  Multiplicity is the simpler
technique of the two as a single queue is served by multiple servers.
Replication requires a more elaborate specification because the queues
of the servers are also copied, so requests must be routed to the
various queues.  Multi-servers can be replicated.
Figure~\ref{fig:multiplicity-replication} shows the underlying
queueing models for each technique.

\begin{figure}[htbp]
  \centering
  \subfloat[Multi-server]{\hspace{1cm}\epsffile{replication/multiserver.eps}\hspace{1cm}}
  \subfloat[Replicated]{\hspace{1cm}\epsffile{replication/replicaserver.eps}\hspace{1cm}}
  \caption{Multiple copies of servers.}
  \label{fig:multiplicity-replication}
\end{figure}

Replication reduces the number of nodes in the layered queueing model
by combining tasks and processors with identical behaviour into a
single object, shown in Figure~\ref{fig:replicated-flat}.  The left
figure shows three identical clients making requests to two identical
servers.  The right figure is the same model, but specified using
replication.  Labels within angle brackets in
Figure~\ref{fig:replicated-flat}(b) denote the number of replicas.

\begin{figure}[htbp]
  \centering
  \subfloat[Flat]{\epsffile{replication/flat.eps}}
  \subfloat[Replicated]{\epsffile{replication/replicated.eps}}
  \caption{Replicated Model}
  \label{fig:replicated-flat}
\end{figure}

Replication also introduces the notion of
\emph{fan-in}\index{fan-out|textbf} and
\emph{fan-in}\index{fan-in|textbf}, denoted using the
\texttt{O=}\emph{n} and \texttt{I=}\emph{n} labels on the request from
t1 to t2 in Figure~\ref{fig:replicated-flat}(b).  Fan-out represents
the number of replicated servers that a client task calls.  Similarly,
fan-in represents the number of replicated clients that call a server.
The product of the number of clients and the fan-out to a server must
be the same as the product of the number of servers and the fan-in to
the server.  Further, both fan-in and fan-out must be integral and
non-zero.  

The total number of requests that a client makes to a server is the product of the mean number of requests
and the fan-out.  If the performance of a system is being evaluated by varying the replication parameter of
a server, the number of requests to the server must be varied inversely with the number of server replicas
in order to retain a constant number of requests from the client.
\index{replication|)}\index{multiplicity|)}

\section{A Brief History}
\label{sec:history}

LQN~\cite{perf:franks-2009-ieeese-lqn} is a combination of Stochastic Rendezvous
Networks~\cite{srvn:woodside-94-ieeetc-srvn} and the Method of Layers~\cite{perf:rolia-95-ieeese-mol}.


%%% Local Variables: 
%%% mode: latex
%%% mode: outline-minor 
%%% fill-column: 108
%%% TeX-master: "userman"
%%% End: 

%%  -*- mode: latex; mode: outline-minor; fill-column: 108 -*- 
%% Title:  results
%%
%% $HeadURL: http://rads-svn.sce.carleton.ca:8080/svn/lqn/trunk-V6/doc/userman/results.tex $
%% Original Author:     Greg Franks <greg@sce.carleton.ca>
%% Created:             Tue Jul 18 2006
%%
%% ----------------------------------------------------------------------
%% $Id: results.tex 10901 2012-05-22 02:12:44Z greg $
%% ----------------------------------------------------------------------

\chapter{Results}
\label{sec:results}

Both the analytic solver and the simulator calculate:
\begin{itemize}
\item throughput bounds (lqns only)\index{throughput!bounds},
\item mean delay for rendezvous\index{queueing delay!task} and
  send-no-reply requests,
\item variances for the rendezvous and send-no-reply request delays
  (lqsim only),
\item mean delay for joins\index{join!delay},
\item entry service times and variances\index{service time},
\item distributions for the service time\index{service time!distribution}\marginpar{lqsim}
\item task throughputs\index{throughput} and
  utilizations\index{utilization!task},
\item processor utilizations\index{utilization!processor}  and
  queueing delays\index{queueing delay!processor}.
\end{itemize}
Figure~\ref{fig:results} shows some of these results for the model
shown in Figure~\ref{fig:bookstore}, after solving the model
analytically using \manpage{lqns}{1}.  The interpretation of these
results are describe below in Section~\ref{sec:model-results}.

\begin{figure}[htbp]
  \centering
  \epsffile{bookstore/bookstore-result.eps}
  \caption{Results.}
  \label{fig:results}
\end{figure}

Results can be saved in three different formats:
\begin{enumerate}
\item in a human-readable form.
\item in a ``parseable''\index{output!parseable} form suitable for
  processing by other programs.  The grammar for the parseable output
  is described in Section \ref{sec:old-grammar} on
  page~\pageref{sec:old-grammar}.
\item in XML\index{output!XML} (again suitable for by processing by
  other programs).  The schema for the XML output is shown in
  Section~\ref{sec:xml-grammar} on page~\pageref{sec:xml-grammar}.
\end{enumerate}
If input to the solver is in XML\index{input!XML}, then output will be
in XML.  Human-readable output\index{output!human readable} will be
produced by default except if output is redirected using the
\flag{o}{output} flag and either XML or parseable output is being
generated.  Conversion\index{output!conversion} from parseable output
to XML, and from either parseable or XML output to the human-readable
form, can be accomplished using \manpage{lqn2ps}{1}.

\section{Header}

The human-readable output from the the analytic solver and simulator
consists of three parts.  Part 1 of the output consists of solution
statistics and other header information and is described in detail in
Sections~\ref{sec:analytic-header-out} and
\ref{sec:simulator-header-out} below.  Part 2 of the output lists the
input and is not described further.  Part 3 contains the actual
results.  These results are described in
Section~\ref{sec:model-results}, starting on
page~\pageref{sec:model-results}.  The section headings here
correspond to the section headings in the output file.

\subsection{Analytic Solver (lqns)}
\label{sec:analytic-header-out}

Figure~\ref{fig:output-lqns} shows the header information output by
the analytic solver.  The first line of the output shows the version
of the solver and where it was run.  This information is often useful
when reporting problems with the solver.  The lines labeled
\texttt{Input} and \texttt{Output} are the input and output file names
respectively.  The line labelled \texttt{Command line} shows all the
arguments used to invoke the solver.  The \texttt{Comment} field
contains the information found in the comment field of the general
information field of the input file (c.f.~\S\ref{sec:general-in},
\S\ref{sec:LqnModelType}).  Next, optionally, the output lists any
pragma used.  Much of this information is also present if the
simulator is used to solve the model.  The remainder of the header
lists statistics accumulated during the solution of the model and is
solver-specific.

\begin{sidewaysfigure}
  \centering
  \small
  \begin{verbatim}
Generated by lqns, version 3.9 (Darwin 6.8.Darwin Kernel Version 6.8: Wed Sep 10 15:20:55 PDT 2003;  Power Macintosh)

Copyright the Real-Time and Distributed Systems Group,
Department of Systems and Computer Engineering
Carleton University, Ottawa, Ontario, Canada. K1S 5B6

Input:  bookstore.lqn
Output: bookstore.out
Command line: lqns -p
Tue Nov  1 21:37:54 2005

Comment: lqn2fig -Lg bookstore.lqn

    #pragma multiserver          = conway

Convergence test value: 7.51226e-07
Number of iterations:   5

MVA solver information: 
Submdl  n   k srv   step()     mean    stddev     wait()     mean    stddev        User      System     Elapsed   
1       5   2   4       44      8.8    1.4697       4776    955.2    299.82  0:00:00.01  0:00:00.00  0:00:00.00 
2       9   1   1       51   5.6667   0.94281        594       66    22.627  0:00:00.00  0:00:00.00  0:00:00.00 
3       9   8   3      240   26.667    9.4751 4.0365e+05    44850     32163  0:00:00.19  0:00:00.00  0:00:00.21 
4       9  10   3      271   30.111    7.0623 7.7481e+05    86090     40554  0:00:01.15  0:00:00.00  0:00:01.19 
5       9   2   1       70   7.7778    1.6178       3408   378.67    181.73  0:00:00.00  0:00:00.00  0:00:00.00 
6       5   0   0        0        0         0          0        0         0  0:00:00.00  0:00:00.00  0:00:00.00 
Total  46   0   0      676   14.696    12.464 1.1872e+06    25809     41253  0:00:01.35  0:00:00.00  0:00:01.40 

    greg-frankss-Computer.local. Darwin 6.8
    User:     0:00:01.35
    System:   0:00:00.00
    Elapsed:  0:00:01.40
\end{verbatim}
  \caption{Analytic Solver Status Output.}
  \label{fig:output-lqns}
\end{sidewaysfigure}

\begin{description}
\item[\texttt{convergence test value:}] The \texttt{convergence test
    value}\index{convergence!test value} is the root of the mean of
  the squares of the difference in the utilization of all of the
  servers from the last two iterations of the solver.  If this value
  is less than the \texttt{convergence value}
  (c.f.~\S\ref{sec:LqnModelType}, \ref{sec:general-in}) specified in
  the input file, then the results are considered
  valid\index{results!valid}.
\item[\texttt{number of iterations:}] The \texttt{number of
    iterations}\index{number of iterations} shows the number of times
  the solver has performed its ``outer iteration''.  If the number of
  iterations exceeds the iteration limit\index{iteration limit} set by
  the model file, the results are considered invalid.
\item[\texttt{MVA solver information:}] This table shows the amount of
  effort the solver expended solving each submodel.  The first column
  lists the submodel number.  Next, the column labelled `n' indicates
  the number of times the MVA solver was run on the submodel.  The
  columns labelled `k' and `srv' show the number of
  chains\index{chain} and servers in the submodel respectively.  The
  next three columns show the number of times the core MVA
  \texttt{step()}\index{step()@\texttt{step()}} function was called.
  The following three columns show the number of time the
  \texttt{wait()}\index{wait()@\texttt{wait()}} function, responsible
  for computing the queueing delay at a server, is called.  Finally,
  the last three columns list the time the solver spends solving each
  submodel.
\end{description}
Finally, the solver lists the name of the machine the it was run on,
the time spent executing the solver code, the time spent by the system
on behalf of lqns, and the total elapsed time.

\subsection{Simulator (lqsim)}
\label{sec:simulator-header-out}

Figure~\ref{fig:output-lqsim} shows the header information output by
the simulator after execution is completed.  The first line of the
output shows the version of the simulator and where it was run.  The
lines labeled \texttt{Input} and \texttt{Output} are the input and
output file names respectively.  The \texttt{Comment} field contains
the information found in the comment field of the general information
field of the input file (c.f.~\S\ref{sec:general-in},
\S\ref{sec:LqnModelType}).  Next, optionally, the output lists
any pragma used.  The remainder of the header lists statistics
accumulated during the solution of the model and is specific to the
simulator.

\begin{figure}[htbp]
  \centering
  \begin{verbatim}
Generated by lqsim, version 3.9 (Linux 2.4.20-31.9  i686),

Copyright the Real-Time and Distributed Systems Group,
Department of Systems and Computer Engineering,
Carleton University, Ottawa, Ontario, Canada. K1S 5B6

Wed Nov  2 11:42:25 2005

Input: bookstore.lqn
Output: bookstore.out
Comment: lqn2fig -Lg bookstore.lqn


Run time: 4.34765E+09
Number of Statistical Blocks: 15
Run time per block: 2.89651E+08
Max confidence interval: 7.32
Seed Value: 1130948006

    epsilon-13.sce.carleton.ca Linux 2.4.20-31.9
    User:     0:04:47.78
    System:   0:00:00.07
    Elapsed:  0:14:27.66
\end{verbatim}
  \caption{Simulator Status Output.}
  \label{fig:output-lqsim}
\end{figure}

\begin{description}
\item[\texttt{Run time:}] The total run time in simulation time units.
\item[\texttt{Number of Statistical Blocks:}] The number of
  statistical blocks\index{block!simulation} collected (when producing
  confidence intervals).  
\item[\texttt{Run time per block:}] The run time in simulation units
  per block.  This value, multiplied by the number of statistical
  blocks and the initial skip period will total to the run time.
\item[\texttt{Seed Value:}] The seed used by simulator.  
\end{description}
Finally, the simulator lists the name of the machine that it was run
on, the time spent executing the simulator code, the time spent by the
system on behalf of lqsim, and the total elapsed time.

\label{sec:model-results}

\section{Type 1 Throughput Bounds}\marginpar{lqns}
\label{sec:bounds-out}\index{throughput!bounds|textbf}

The \emph{Type 1 Throughput Bounds} are the ``guaranteed not to
exceed'' throughputs for the entries listed.  The value is calculated
assuming that there is no contention
delay\index{contention delay}\index{delay!contention} to underlying
servers.

\section{Mean Delay for a Rendezvous}
\label{sec:rendezvous-delay-out}\index{rendezvous!delay|textbf}

The \emph{Mean Delay for a Rendezvous} is the queueing
time\index{queueing time} for a request from a client to a server.  It
does not include the time the customer spends at the server (see
Figure~\ref{fig:service-time}).  To find the \emph{residence
  time}\emph{residence time}, add the queueing time to the \emph{phase
  one service time}\index{service time!phase one} of the request's
server.

\section{Variance of Delay for a Rendezvous}\marginpar{lqsim}
\label{sec:rendezvous-variance-out}\index{rendezvous!variance|textbf}

The \emph{Variance of Delay for a Rendezvous} is the variance of the
queueing time\index{queueing time!variance} for a request from a
client to the server.  It does not include the variance of the time
the customer spends at the server (see Figure~\ref{fig:service-time}).
This result is only available from the simulator.

\section{Mean Delay for a Send-No-Reply Request}
\label{sec:snr-delay-out}\index{send-no-reply!delay|textbf}

The \emph{Mean delay for a send-no-reply request} is the time the
request spends in queue and in service in phase one at the
destination.  Phase two is treated as a `vacation' at the server.

\section{Variance of Delay for a Send-No-Reply Request}\marginpar{lqsim}
\label{sec:snr-variance-out}\index{send-no-reply!variance|textbf}

\section{Arrival Loss Probabilities}
\label{sec:arrival-loss}\index{send-no-reply!loss probability|textbf}

The \emph{Arrival Loss Probabilities}\index{arrival loss probabilities}...

\section{Mean Delay for a Join}
\label{sec:join-delay-out}\index{join!delay|textbf}

The \emph{Mean Delay for a Join}\index{join} is the maximum of the sum
of the service times for each branch of a fork.  The source activity
listed in the output file is the first activity prior to the fork
(e.g., AFBA112 in Figure~\ref{fig:results}).  Similarly, the
destination activity listed in the output file is the first activity
after the join (AJA131).  The variance of the join
time\index{join!variance} is also computed.


\begin{figure}[htbp]
  \centering
  \epsffile{timing-diagrams/join-time.eps}
  \caption{Service Time Components for Join.}
  \label{fig:service-time}
\end{figure}

\section{Service Times}
\label{sec:service-time-out}\index{service time|(textbf}

The \emph{service time} is the total time a phase\index{phase!service
  time} or activity\index{activity!service time} uses processing a
request.  The time consists of four components, shown in
Figure~\ref{fig:service-time}:
\begin{enumerate}
\item Queueing for the processor\index{processor!queueing} (shown as
  items 1, 4, 6 and 8 in Figure~\ref{fig:results-service-time}.(b)).
\item Service at the processor (items 2, 5 and 9)
\item Queueing for serving tasks (item 6), and
\item Phase one service time\index{service time!phase one} at serving
  tasks (items 3 and 7).
\end{enumerate}
Queuing at processors and tasks and can occur because of contention
from other tasks (items 1, 6, and 8), or from second phases from
previous requests.  For example, entry SE3 is queued at the processor
because the processor is servicing the second phase of entry SCE3.

\begin{figure}[htbp]
  \centering
  \epsffile{timing-diagrams/service-time.eps}
  \caption{Service Time Components for Entry `SCE3'.}
  \label{fig:results-service-time}
\end{figure}

Using the results shown in Figure~\ref{fig:results}, the
service time for entry SE3 ($21.7$) is the sum of:
\begin{itemize}
\item the processor wait ($0.767$),
\item it's own service time ($6\times 10^{-6}$),
\item the queueing time to entry SCE3 ($0$),
\item the phase one service time at entry SCE3 ($11.6$),
\item the queueing time to entry CE1 ($3.83\times 10^{-10}$), and
\item the phase one service time at entry CE1 ($10$)
\end{itemize}

Queueing time for serving tasks is shown in the \emph{Mean Delay for a
  Rendezvous} section of the output.  (c.f.
\S\ref{sec:rendezvous-delay-out}).  Queueing time for the processor is
shown in the \emph{Utilization and Waiting per Phase for Processor} of
the output (c.f.
\S\ref{sec:processor-wait-utilization-out}).\index{service time|)}

\section{Service Time Variance}
\label{sec:service-time-variance-out}\index{service time!variance|textbf}

The \emph{Service Time
  Variance}\index{service time!variance}\index{variance!service time}
section lists the variance of the service time
(c.f.~\S\ref{sec:service-time-out}) for the phases and activities in
the model.

\section{Probability Maximum Service Time Exceeded}\marginpar{lqsim}
\label{sec:service-time-exceeded-out}\index{service time!probability exceeded|textbf}

The \emph{probability maximum service time exceeded}\index{service time!maximum exceeded} is output by the
simulator for all phases and activities with a \attribute{max-service-time}.  This result is the probability
that the service time is greater than the value specified.  In effect, it is a histogram with two bins.

\section{Service Time Distributions for Entries and Activities}\marginpar{lqsim}
\label{sec:service-time-distribution-out}\index{service time!distribution|textbf}

\emph{Service Time Distributions}\index{service time!distributions} are generated by the simulator by
setting the \attribute{service-time-distribution} parameter (c.f.~\S\ref{sec:ActivityDefBase},
\S\ref{sec:entry}, \S\ref{sec:activity}) for an entry or activity.  A histogram of \attribute{number-bins}
bins between \attribute{min} and \attribute{max} is generated.  Samples that fall either under or over this
range are stored in their own under-flow\index{histogram!underflow} or over-flow\index{histogram!overflow}
bins respectively.  The optional \attribute{x-samples} parameter can be used to set the sampling behaviour
to one of:
\begin{description}
\item[linear] Each bin is of equal width, found by dividing the histogram range by the number of bins.  If
  the \attribute{x-samples} is not set, this behaviour is the default.
\item[log] The logarithm of the range specified is divided by \attribute{number-bins}.  This has the effect
  of making the width of the bins small near \attribute{min}, and large near \attribute{max}.  A minimum
  value of zero is \textbf{not} allowed.
\item[sqrt] The square root of the range specified is divided by \attribute{number-bins}.  Bins are smallest
  near \attribute{bin} are smaller than those near \attribute{max}.
\end{description}

The results of the histogram collection, shown in Figure~\ref{fig:histogram}, consist of the
mean\index{service time!mean}, standard deviation,\index{service time!standard deviation},
skew\index{service time!skew} and kurtosis\index{service time!kurtosis} of the sampled range, followed by
the histogram itself.  Each entry of the histogram contains the probability of the sample falling within the
bucket, and, if available, the confidence intervals of the sample.

\begin{sidewaysfigure}
\begin{verbatim}
Service time distributions for entries and activities:

SCE3             PHASE 1: 
    Mean =   11.58, Stddev =   8.457, Skew =  0.8501, Kurtosis = -0.2496
       Begin      End       Prob.      +/-95%     +/-99%
           0        1  0.03355     0.001048   0.001412   |          *
           1        2  0.03786     0.001605   0.002163   |            *
           2        3  0.05406     0.002026   0.002731   |                 *
           3        4  0.06333     0.002031   0.002737   |                   *
           4        5  0.06545     0.001631   0.002199   |                    *
           5        6  0.06369     0.001578   0.002127   |                   *
           6        7  0.06049     0.001692   0.00228    |                  *
           7        8  0.05591     0.001822   0.002456   |                 *
           8        9  0.05133     0.001272   0.001714   |                *
           9       10  0.0472      0.001767   0.002382   |              *
          10       11  0.04318     0.001618   0.002181   |             *
          11       12  0.03931     0.001185   0.001597   |            *
          12       13  0.03579     0.001073   0.001446   |           *
          13       14  0.03231     0.001654   0.002229   |          *
          14       15  0.02952     0.001033   0.001392   |         *
          15       16  0.02677     0.001189   0.001603   |        *
          16       17  0.0243      0.001058   0.001425   |       *
          17       18  0.02214     0.001087   0.001466   |       *
          18       19  0.02001     0.001122   0.001512   |      *
          19       20  0.01806     0.001016   0.001369   |      *
          20       21  0.01653     0.0009079  0.001224   |     *
          21       22  0.01499     0.001018   0.001372   |     *
          22       23  0.01365     0.0007152  0.0009639  |    *
          23       24  0.01229     0.000955   0.001287   |    *
          24       25  0.0112      0.0008691  0.001171   |   *
          25       26  0.009997    0.0006182  0.0008331  |   *
          26       27  0.009227    0.0007344  0.0009898  |   *
          27       28  0.008282    0.0006896  0.0009293  |   *
          28       29  0.007444    0.0005936  0.0007999  |  *
          29       30  0.006802    0.0005752  0.0007751  |  *
             overflow  0.06532     0.001561   0.002104   | *
\end{verbatim}
  \caption{Histogram output}
  \label{fig:histogram}
\end{sidewaysfigure}

The statistics for the histogram\index{histogram!statistics} are found by multiplying the mid-point of the
range defined by \texttt{begin} and \texttt{end}, not counting either the overflow or underflow bins.  If
the mean value reported by the histogram is substantially different than the actual service time of the
phase or activity, then the range of the histogram is not sufficiently large.

\section{Semaphore Holding Times}
\label{sec:semaphore-holding}\index{utilization!semaphore|textbf}

The \emph{Semaphore Holding Times} section lists the average time a semaphore\index{semaphore!service time}
token is held (it's service time), the variance of the holding time, and the utilization of
semaphore\index{semaphore!utilization}\index{utilization!semaphore}.  Figure~\ref{fig:semaphore-stats} shows
how these values are found.

\begin{figure}
  \centering
  \epsffile{timing-diagrams/holding-time.eps}
  \caption{Time components of a semaphore task.}
  \label{fig:semaphore-stats}
\end{figure}

\section{Throughputs and Utilizations per Phase}
\label{sec:througput-utilization-out}\index{throughput|textbf}\index{utilization!task|textbf}

The \emph{Throughputs and Utilizations per Phase} section lists the
throughput by entry and activity, and the utilization by phase and
activity. The utilization is the \emph{task utilization}, i.e., the
reciprocal of the service time\index{service time} for the task
(c.f.~\ref{sec:service-time-out}).  The processor
utilization\index{utilization!processor}\index{processor!utilization} for the task is listed under
\emph{Utilization and Waiting per Phase for Processor}
(see~\S\ref{sec:processor-wait-utilization-out}).

\section{Arrival Rates and Waiting Times}
\label{sec:open-wait-out}\index{open arrival!waiting time|textbf}\index{waiting time!open arrival|textbf}

The \emph{Arrival Rates and Waiting Times} section is only present in
the output when \emph{open arrivals} are present in the input.  This
section shows the arrival rate (\emph{Lambda}\index{lambda}) and the
waiting time\index{waiting time}.  The waiting time includes the
service time at the task.

\section{Utilization and Waiting per Phase for Processor}
\label{sec:processor-wait-utilization-out}\index{utilization!processor|textbf}\index{queueing time!processor|textbf}

The \emph{Utilization and Waiting per Phase for Processor} lists the
processor utilization and the queueing time for every entry and
activity running on the processor.


%%% Local Variables: 
%%% mode: latex
%%% mode: outline-minor 
%%% fill-column: 108
%%% TeX-master: "userman"
%%% TeX-master: "userman"
%%% End: 

%% -*- mode: latex; mode: outline-minor; fill-column: 108 -*-
%% Title:  schema
%%
%% $HeadURL: http://rads-svn.sce.carleton.ca:8080/svn/lqn/branches/merge-V5-V6/doc/userman/schema.tex $
%% Original Author:     Greg Franks <greg@sce.carleton.ca>
%% Created:             Tue Jul 18 2006
%%
%% ----------------------------------------------------------------------
%% $Id: schema.tex 11982 2014-04-15 21:32:15Z greg $
%% ----------------------------------------------------------------------

\chapter{XML Grammar}
\label{sec:xml-grammar}

The definition of LQN models using XML\index{XML Grammar|(textbf}\index{Grammar!XML|(textbf} is an evolution of
the original SRVN file format (c.f.~\S\ref{ch:srvn} and~Appendix~\ref{sec:input-file-bnf}).  The
new XML format is based on the work done in~\cite{perf:wu-2003}, with
further refinement for general usage.  There are new features in the
XML format to support new concepts for building and assembling models
using components\index{components}.  The normal LQN tool suite (like
\manpage{lqns}{1} and \manpage{lqsim}{1}) do not support these new
features, however other tools outside the suite are being written to
utilize the new parts of the XML format.

\section{Basic XML File Structure}
\label{sec:xml-file-structure}

In XML, layered models are specified in a bottom-up order, which is
the reverse of how layered models are typically presented.  First, a
processor is defined, then within the processor block, all the tasks
than run on it are defined.  Similarly, within each task block all the
entries that are associated with it are defined, etc.  A simplified
layout of an incomplete LQN model written in XML is shown in
Figure~\ref{fig:xml-file-layout}.

\lstset{language=XML,basicstyle=\ttfamily,numbersep=10pt}
\begin{lstlisting}[float,caption={XML file layout.},label=fig:xml-file-layout]
<lqn-model>
   <solver-params>
      <pragma/>
   </solver-params>
   <processor>
      <task>
         <entry>
            <entry-phase-activities>
               <activity>
                  <synch-call/>
                  <asynch-call/>
               </activity>
               <activity> ... </activity>
            </entry-phase-activities>
         </entry>
         <entry> ... </entry>
         <task-activities>
            <activity/>
            <precedence/>
         </task-activities>
      </task>
      <task> ... </task>
   </processor>
   <processor> ... </processor>
</lqn-model>
\end{lstlisting}

Activity graphs (specified by task-activities) belong to a task, and
hence are siblings to entry elements.  The element
entry-activity-graph specifies an activity graph contained within one
entry, but is not supported by any of the LQN tools.  The concept of
phases still exists, but now each phase is an activity, and is defined
in the entry-phase-activities element.

\section{Schema Elements}
\label{sec:schema-elements}

The XML definition for layered models consists of three files:
\begin{description}
\item[\texttt{lqn.xsd}\index{lqn.xsd}:] lqn.xsd is the root of the schema.
\item[\texttt{lqn-sub.xsd}\index{lqn-sub.xsd}] ...
\item[\texttt{lqn-core.xsd}\index{lqn-core.xsd}] lqn-core is the
  actual model specfication and is included by lqn.xsd.
\end{description}
All three files should exist in the same location.  If the solver cannot located the \texttt{lqn.xsd} file,
it will emit an error\footnote{See the error message ``The primary document entity could not be opened'' on
  \pageref{error:primary-document}.} and stop\indexerror{primary document}.

Figure~\ref{fig:Schema} shows the schema for Layered Queueing Networks
using Unified Modeling Language notation.  The model is defined starting from
\texttt{lqn-model}.  Unless otherwise specified in the figure, the order of
elements in the model is from left to right, i.e., \texttt{<solver-params>}
always preceeds \texttt{<processor>} in the input file.  Optional elements are
shown using a multiplicity of zero for an association.  Note that results
(optional, shown in blue) are part of the schema.

\begin{figure}[htbp]
%  \epsfxsize=\textwidth 
  \epsffile{xml-schema/schema.eps}
  \caption[LQN Schema]{LQN Schema.  Elements shown in \color{blue}blue\color{black}\mbox{} are results found in the output.
    Elements shown in \color{red}red\color{black}\mbox{} are not implemented.  Unless otherwise indicated, all elements are
    ordered from left to right.}
  \label{fig:Schema}
\end{figure}

\subsection{LqnModelType}
\label{sec:LqnModelType}

The first element in a layered queueing network XML input\index{input!XML} file is
\schemaelement{lqn-model}, which is of type \schematype{LqnModelType} and is shown in
Figure~\ref{fig:LqnModelType}.  \textbf{LqnModelType} has five elements, namely:
\schemaelement{run-control}, \schemaelement{plot-control}, \schemaelement{solver-params},
\schemaelement{processor} and \schemaelement{slot}.  \texttt{Run-control} and \texttt{plot-control} are not
not implemented.  \texttt{Processor} is described under Section~\ref{sec:ProcessorType}.  \texttt{Slot} is
described in~\cite{perf:wu-2003}.  The attributes for \textbf{LqnModelType} are shown in
Table~\ref{tab:LqnModelType}.

\begin{figure}[htbp]
%  \epsfxsize=\textwidth 
  \epsffile{xml-schema/lqn-model-schema.eps}
  \caption{Top-level LQN Schema.}
  \label{fig:LqnModelType}
\end{figure}

\begin{table}[htbp]
  \centering
  \begin{tabular}[l]{|l|l|l|l|p{2.5in}|}
    \hline
    \textbf{Name} & \textbf{Type} & \textbf{Use} & \textbf{Default} &
    \textbf{Comments} \\
    \hline
    \attribute{name} & string & optional & & The name of the model. \\
    \hline
    \attribute{description} & string & optional & & A description
    of the model. \\
    \hline
    \attribute{lqn-schema-version} & integer & fixed & 1.0 & The version of the schema
    (used by the solver in case of substantial schema changes for model
    conversion.) \\
    \hline
    \attribute{lqncore-schema-version} & integer & fixed & 1.0 & \\
    \hline
    \attribute{xml-debug} & boolean & optional & false & \\
    \hline
  \end{tabular}
  \caption{\label{tab:LqnModelType}Attributes for elements of type \schematype{LqnModelType}
    from Figure~\protect\ref{fig:LqnModelType}.}
\end{table}

The element \schemaelement{solver-params} is used to set various
operating parameters for the analytic solver, and to record various
output statistics after a run completes.  It contains the elements
\schemaelement{result-general} and \schemaelement{pragma}.  The
attributes for \schemaelement{solver-params} are shown in
Table~\ref{tab:solver-params}.  These attributes are mainly used to
control the analytic solver.  Refer to
Section~\ref{sec:lqns-stopping-criteria} for more information.  The
attributes for \schemaelement{result-general} are shown in
Table~\ref{tab:result-general}.  Refer to
Sections~\ref{sec:analytic-header-out} and
\ref{sec:simulator-header-out} for the interpretation of header
information.  The attributes for \schemaelement{pragma} are show in
Table~\ref{tab:pragma}.  Refer to Section~\ref{sec:lqns-pragmas} for
the pragmas supported by lqns and to Section~\ref{sec:lqsim-pragmas}
for the pragmas supported by lqsim.

\begin{table}[htbp]
  \centering
  \begin{tabular}[l]{|l|l|l|l|p{3in}|}
    \hline
    \textbf{Name} & \textbf{Type} & \textbf{Use} & \textbf{Default} &
    \textbf{Comments} \\
    \hline
    \attribute{conv\_val}   & float   & optional & 1 & Convergence
    value\index{lqns!convergence value} for lqns
    (c.f~\S\protect\ref{sec:lqns-stopping-criteria}).  Ignored by lqsim.\\
    \hline
    \attribute{it\_limit}   & integer & optional & 50 & Iteration limit\index{iteration limit} for lqns
    (c.f~\S\protect\ref{sec:lqns-stopping-criteria}).  Ignored by lqsim.\\
    \hline
    \attribute{print\_int}  & integer & optional & 0 & Print interval for intermediate
    results.  The \flag{t}{print} must be specified to lqns\index{print interval!lqns} to generate
    output after \emph{it\_limit} iterations.  Blocked
    statistics\index{statistics!blocked} must be specified to
    lqsim\index{print interval} using
    the \flag{A}{n}, \flag{B}{n}, or \flag{C}{n} flags. \\
    \hline
    \attribute{underrelax\_coeff} & float & optional & 0.5 & Under-relaxation
    coefficient for lqns (c.f~\S\protect\ref{sec:lqns-stopping-criteria}).
    Ignored by lqsim. \\
    \hline
  \end{tabular}
  \caption{\label{tab:solver-params}Attributes of element \schemaelement{solver-params} from
    Figure~\protect\ref{fig:LqnModelType}.} 
\end{table}

\begin{table}[htbp]
  \centering
  \begin{tabular}[l]{|l|l|l|l|p{2.8in}|}
    \hline
    \textbf{Name} & \textbf{Type} & \textbf{Use} & \textbf{Default} &
    \textbf{Comments} \\
    \hline
    \hline
    \attribute{conv-val} & float & required & & Convergence value
    (c.f.~\protect\ref{sec:analytic-header-out}) \\
    \hline
    \attribute{valid} & enumeration & required & & Either \texttt{YES} or
    \texttt{NO}. \\
    \hline
    \attribute{iterations} & float & optional & & The number of iterations of the
    analytic solver or the number of blocks for the simulator.\\
    \hline
    \attribute{elapsed-time} & string & optional & & The wall-clock time used by
    the solver. \\
    \hline
    \attribute{system-cpu-time} & string & optional & & The CPU time spent in
    kernel-mode. \\
    \hline
    \attribute{user-cpu-time} & string & optional & & The CPU time spent in user
    mode. \\
    \hline
    \attribute{platform-info} & string & optional & & The operating system and CPU
    type. \\
    \hline
    \attribute{solver-info} & string & optional & & The version of the solver. \\
    \hline
  \end{tabular}
  \caption{\label{tab:result-general}Attributes of element \schemaelement{result-general} from
    Figure~\protect\ref{fig:LqnModelType}.} 
\end{table}

\begin{table}[htbp]
  \centering
  \begin{tabular}[l]{|l|l|l|l|p{3in}|}
    \hline
    \textbf{Name} & \textbf{Type} & \textbf{Use} & \textbf{Default} &
    \textbf{Comments} \\
    \hline
    \hline
    \attribute{param} & string & required & & The name of the parameter.
    (c.f.~\protect\ref{sec:lqns-pragmas}, \S\ref{sec:lqsim-pragmas}) \\
    \hline
    \attribute{value} & string & required & & the value assigned to the pragma. \\
    \hline
  \end{tabular}
  \caption{\label{tab:pragma}Attributes of element \schemaelement{pragma} from
    Figure~\protect\ref{fig:LqnModelType}.} 
\end{table}

%%
%% Processor
%%

\subsection{ProcessorType}
\label{sec:ProcessorType}

Elements of type \schematype{ProcessorType}, shown in Figure~\ref{fig:ProcessorType} are used to define the
processors in the model.  They contain an optional \texttt{result-processor} element and elements of either
\schematype{GroupType} or \schematype{TaskType}.  The \attribute{scheduling} attribute must by set to
\texttt{cfs}, for completely fair scheduling\index{scheduling!completely fair}, if \schematype{GroupType}
elements are present and to any other type if \schematype{GroupType} are not found.  \schematype{GroupType}
and \schematype{TaskType} elements may not be both be defined in a processor.

Element \schemaelement{result-processor} is of type \schematype{OutputResultType} and is described in
Section~\ref{sec:OutputResultType}.  Element \schemaelement{task} is described in
Section~\ref{sec:TaskType}.  The attributes of \schematype{ProcessorType}, described in \ref{sec:processor},
are shown in Table~\ref{tab:ProcessorType}.

\begin{figure}[htbp]
%  \epsfxsize=\textwidth 
  \centering
  \epsffile{xml-schema/processor-schema.eps}
  \caption{Processor Schema.}
  \label{fig:ProcessorType}
\end{figure}
\begin{table}[htbp]
  \centering
  \begin{tabular}[l]{|l|l|l|l|p{3in}|}
    \hline
    \textbf{Name} & \textbf{Type} & \textbf{Use} & \textbf{Default} &
    \textbf{Comments} \\
    \hline
    \attribute{name}         & string         & required & & \\
    \hline
    \attribute{multiplicity} & integer        & optional & 1 & See \S\protect\ref{sec:multiplicity} \\
    \hline
    \attribute{speed-factor} & float          & optional & 1.0 & Scaling factor for
    the processor. \\
    \hline
    \attribute{scheduling}   & enumeration & optional & fcfs & The allowed
    scheduling types are \texttt{fcfs}, \texttt{hol}, \texttt{pp},
    \texttt{rand}, \texttt{inf}, \texttt{ps-hol}, \texttt{ps-pp} and \texttt{cfs}. See
    \S\protect\ref{sec:processors}\index{scheduling!processor}\index{scheduling!head of line}. \\
    \hline
    \attribute{replication}  & integer        & optional & 1 & See \S\protect\ref{sec:replication} \\
    \hline
    \attribute{quantum}      & float          & optional & 0.0 & Mandatory for
    processor sharing\index{processor!sharing} scheduling when using lqsim. \\
    \hline
  \end{tabular}
  \caption{\label{tab:ProcessorType}Attributes for elements of type \schematype{ProcessorType}.}
\end{table}
%%
%%
%% Group
%%
\subsection{GroupType}
\label{sec:GroupType}

Optional elements of type \schematype{GroupType}, shown in Figure~\ref{fig:ProcessorType}, are used to
define groups of tasks for processors running completely fair scheduling\index{scheduling!completely fair}.
Each group must contain a minimum of one task.  The attributes of \schematype{GroupType} are shown in
Table~\ref{tab:GroupType}. 

\begin{table}[htbp]
  \centering
  \begin{tabular}[l]{|l|l|l|l|p{2.8in}|}
    \hline
    \textbf{Name} & \textbf{Type} & \textbf{Use} & \textbf{Default} &
    \textbf{Comments} \\
    \hline
    \attribute{name}         & string         & required & & \\
    \hline
    \attribute{share}        & float          & required & & The fraction of the processor allocated to this
    group. \\
    \hline
    \attribute{cap}          & boolean        & optional & false & If true, shares are
    \emph{caps}\index{share!cap} (ceilings).  Otherwise, shares are guarantees\index{share!guarantee}
    (floors) \\
    \hline
  \end{tabular}
  \caption{\label{tab:GroupType}Attributes for elements of type \schematype{GroupType}}
\end{table}
%%
%%
%% Task
%%
\subsection{TaskType}
\label{sec:TaskType}

Elements of type \schematype{TaskType}, shown in Figure~\ref{fig:TaskType}, are used to define the tasks in
the model.  These elements contain an optional \schemaelement{result-task} element, one or more elements of
\textbf{EntryType}, and optionally, elements of \schemaelement{service} and \schemaelement{task-activities}.
Element \schemaelement{result-task} is of type \schematype{OutputResultType}, and is described in
Section~\ref{sec:OutputResultType}.  Element \schemaelement{entry} is described in
Section~\ref{sec:EntryType}.  The attributes of \schematype{TaskType}, described in Section~\ref{sec:task},
are shown in Table~\ref{tab:TaskType}.

\begin{figure}[htbp]
  \centering
  \epsffile{xml-schema/task-schema.eps}
  \caption{TaskType}
  \label{fig:TaskType}
\end{figure}

\begin{table}[htbp]
  \centering
  \begin{tabular}[l]{|l|l|l|l|p{2.8in}|}
    \hline
    \textbf{Name} & \textbf{Type} & \textbf{Use} & \textbf{Default} &
    \textbf{Comments} \\
    \hline
    \attribute{name}         & string         & required & & \\
    \hline
    \attribute{multiplicity} & integer & optional & 1 & See \S\protect\ref{sec:multiplicity}.\\
    \hline
    \attribute{priority}     & integer & optional & 0 & The priority used by the
    processor for scheduling.  See \S\protect\ref{sec:processors}. \\
    \hline
    \attribute{queue-length} & integer & optional & 0 & Maximum queue size (for
    open-class requests only).  See \S\protect\ref{sec:tasks}. \\
    \hline
    \attribute{replication}  & integer & optional & 1 & See \S\protect\ref{sec:replication}\\
    \hline
    \attribute{scheduling}   & enumeration & optional & FCFS & The scheduling
    of requests at the task.  The allowed
    scheduling types are \texttt{ref}, \texttt{fcfs}, \texttt{hol}, \texttt{pri},
    \texttt{inf}, \texttt{burst}, and \texttt{poll} and \texttt{semaphore}. See \S\protect\ref{sec:tasks}.\index{scheduling!task} \\
    \hline
    \attribute{activity-graph} & enumeration & required & &
    \texttt{yes} or \texttt{no}\\
    \hline
    \hline
    \attribute{think-time}   & float         & optional & 0 & Reference tasks only.  Customer think time. \\
    \hline
    \hline
    \attribute{initially}    & integer       & optional & \emph{multiplicity} & Semaphore tasks only.  Set the initial
    number of semaphore tokens to zero.  By default, the number of tokens is set to the multiplicity of the task.  \\
    \hline
  \end{tabular}
  \caption{\label{tab:TaskType}Attributes for elements of type \schematype{TaskType}}
\end{table}

\subsection{FanInType and FanOutType}
\label{sec:FanInType}
\label{sec:FanOutType}

\begin{table}[htbp]
  \centering
  \begin{tabular}[l]{|l|l|l|l|p{2.5in}|}
    \hline
    \textbf{Name} & \textbf{Type} & \textbf{Use} & \textbf{Default} &
    \textbf{Comments} \\
    \hline
    \attribute{source}     & integer & required &  & (See \S\ref{sec:replication}) \\
    \hline
    \attribute{value}      & integer & required &  & (See \S\ref{sec:replication}) \\
    \hline
  \end{tabular}
  \caption{\label{tab:FanInType}Attributes for elements of type \schematype{FanInType}.}
\end{table}

\begin{table}[htbp]
  \centering
  \begin{tabular}[l]{|l|l|l|l|p{2.5in}|}
    \hline
    \textbf{Name} & \textbf{Type} & \textbf{Use} & \textbf{Default} &
    \textbf{Comments} \\
    \hline
    \attribute{dest}       & integer & required &  & (See \S\ref{sec:replication}) \\
    \hline
    \attribute{value}      & integer & required &  & (See \S\ref{sec:replication}) \\
    \hline
  \end{tabular}
  \caption{\label{tab:FanOutType}Attributes for elements of type \schematype{FanOutType}.}
\end{table}

\subsection{EntryType}
\label{sec:EntryType}

Elements of type \schematype{EntryType}, shown in Figure~\ref{fig:EntryType}, are used to define the entries
of tasks.  Entries can be specified one of three ways, based on the attribute \attribute{type} of an
\schemaelement{entry} element, namely:
\begin{figure}[htbp]
  \centering
%  \epsfxsize=\textwidth 
  \epsffile{xml-schema/entry-schema.eps}
  \caption{Schema for type \schematype{EntryType}.}
  \label{fig:EntryType}
\end{figure}
\begin{description}
\item[\texttt{ph1ph2}] The entry is specified using phases.  The
  phases are specified using an \schemaelement{entry-phase-activities}
  element which is of the \schematype{ActivityPhasesType} type.
  Activities defined within this element must have a unique
  \attribute{phase} attribute.  
\item[\texttt{graph}] The entry is specified as an activity
  graph\index{activity graph} defined within the entry.  The demand is
  specified using elements of type \schematype{ActivityEntryDefType}.
  This method of defining an entry is not supported currently.
\item[\texttt{none}] The entry is specified using an activity graph
  defined within the task.  A \schemaelement{task-activities} element
  of type \schematype{ActivtyDefType} must be present and one of the
  activities defined within this element must have a
  \attribute{bound-to-entry} attribute.  The
  \schematype{TaskActivityGraph} type is defined in
  Section~\ref{sec:TaskActivityGraph}.
\end{description}
\schematype{ActivityPhasesType},   \schematype{ActivityEntryDefType} and
\schematype{ActivtyDefType} are all based on
\schematype{ActivityDefBase}, described in
Section~\ref{sec:ActivityDefBase}.  They only differ in the way the
start of the graph is identified, and in the case of
\schematype{ActivityPhasesType}, the way the activities are connected.  

The attributes for \schematype{EntryType}, described in
Section~\ref{sec:entry}, are shown in Table~\ref{tab:EntryType}.  The optional
element \schemaelement{result-entry} is of type
\textbf{OutputResultType}, and is described in
Section~\ref{sec:OutputResultType}.  The optional element
\schemaelement{forwarding} is used to describe the probability of
forwarding a request to another entry; it is described in
Section~\ref{sec:MakingCallType}.

\begin{table}[htbp]
  \centering
  \begin{tabular}[l]{|l|l|l|l|p{2.5in}|}
    \hline
    \textbf{Name} & \textbf{Type} & \textbf{Use} & \textbf{Default} &
    \textbf{Comments} \\
    \hline
    \attribute{name} & string & required & & The entry name\\
    \hline
    \attribute{type} & enumeration & required & & \texttt{PH1PH2}, \texttt{GRAPH}, or \texttt{NONE} \\
    \hline
    \attribute{open-arrival-rate} & float & optional & \index{open arrival} & \\
    \hline
    \attribute{priority} & integer & optional & & (c.f.~\ref{sec:tasks}) \\
    \hline
    \attribute{sempahore} & enumeration & optional & & \texttt{signal} or \texttt{wait} (c.f.~\ref{sec:tasks}) \\
    \hline
  \end{tabular}
  \caption{\label{tab:EntryType}Attributes for elements of type \schematype{EntryType}.}
\end{table}

\subsection{ActivityGraphBase}
\label{sec:ActivityGraphBase}

Elements of type \schematype{ActivityGraphBase}, shown in
Figure~\ref{fig:ActivityGraphBase}, are used to define activities
(c.f.~\ref{sec:activities})\index{activity} and their relationships to
each other.  They are used by elements of both \schematype{EntryType} and
\schematype{TaskActivityGraph} types.

\begin{figure}[htbp]
  \centering
%  \epsfxsize=\textwidth 
  \epsffile{xml-schema/activity-schema.eps}
  \caption{Schema diagram for the type \schematype{ActivityGraphBase}}
  \label{fig:ActivityGraphBase}
\end{figure}

Elements of the \schematype{ActivityGraphBase} consist of a sequence
of one or more \schemaelement{activity} elements followed by a
sequence of \schemaelement{precedence} elements.  \texttt{Activity}
elements are used to store the demand for an
activity\index{activity!demand}\index{demand} and requests to other
servers (through the \schematype{ActivityDefType}) and, optionally,
results through elements of \schematype{ActivityDefType}.
\texttt{Precedence} elements are defined by the
\schematype{PrecedenceType} in Section~\ref{sec:PrecedenceType}.

\subsection{TaskActivityGraph}
\label{sec:TaskActivityGraph}

Task Activity Graphs\index{activity graph!task}, defined using
elements of type \schematype{TaskActivityGraph} and shown in
Figure~\ref{fig:ActivityGraphBase}, are used to specify the behaviour
of a task using activities.  This type is almost the same as
\schematype{EntryActivityGraph}, except that the activity that replies
to an entry\index{reply!activity} must explicitly specify the entry
for which the reply is being generated.  The actual activity graph is
defined using elements of type \schematype{ActivityGraphBase},
described in Section~\ref{sec:ActivityGraphBase}.  The attributes for
elements \schemaelement{reply-entry} and
\schemaelement{reply-activity} are shown in
Tables~\ref{tab:reply-entry} and \ref{tab:reply-activity}
respectively.

\begin{table}[htbp]
  \centering
  \begin{tabular}[l]{|l|l|l|l|p{3in}|}
    \hline
    \textbf{Name} & \textbf{Type} & \textbf{Use} & \textbf{Default} &
    \textbf{Comments} \\
    \hline
    \attribute{name} & string & required & & The name of the
    entry for which the list of \schemaelement{reply-activity} elements
    generate replies.\\
    \hline
  \end{tabular}
  \caption{\label{tab:reply-entry}Attributes of element \schemaelement{reply-entry} from
    Figure~\protect\ref{fig:ActivityGraphBase}.} 
\end{table}

\begin{table}[htbp]
  \centering
  \begin{tabular}[l]{|l|l|l|l|p{3in}|}
    \hline
    \textbf{Name} & \textbf{Type} & \textbf{Use} & \textbf{Default} &
    \textbf{Comments} \\
    \hline
    \attribute{name} & string & required & & The name of the
    activity which generates a reply.  The entry is either implicitly
    defined if this element is defined within an
    \schematype{EntryType}, or part of list defined within a
    \schemaelement{reply-element}. \\
    \hline
  \end{tabular}
  \caption{\label{tab:reply-activity}Attributes of element \schemaelement{reply-activity} from
    Figure~\protect\ref{fig:ActivityGraphBase}.} 
\end{table}

\subsection{ActivityDefBase}
\label{sec:ActivityDefBase}

The type \schematype{ActivityDefBase}, shown in Figure~\ref{fig:ActivityGraphBase}, is used to define the
parameters for an activity, such as demand\index{demand} and call-order\index{call order}.  This type is
extended by \schematype{ActivityPhasesType}, \schematype{EntryActivityDefType}, and
\schematype{ActivityDefType} to define the requests\index{request} from an activity to an entry, and to
connect the activity graph\index{activity graph!connection} to the requesting entry.
Table~\ref{tab:ActivityDefBase} lists the parameters used as attributes and the attributes used by the three
sub-types.  Refer to Section~\ref{sec:activity} for more information on these parameters.  Refer to
\schematype{MakingCallType} (\S\ref{sec:MakingCallType}) for the \schemaelement{Activity-CallGroup} used to
make requests to other entries\footnote{\schemaelement{Call-List-Group} is not defined at present.}. Refer
to \schematype{OutputResultForwardingANDJoinDelay} (\S\ref{sec:OutputResultJoinDelayType}) for
\schemaelement{result-join-delay} and \schemaelement{result-forwarding} for join-delay\index{join!delay} and
forwarding\index{forwarding} results respectively.  Refer to \schematype{OutputDistributionType}
(\S\ref{sec:OutputDistributionType}) for \schemaelement{service-time-distribtion}.  Finally, refer to
\schematype{OutputResultType} (\S\ref{sec:OutputResultType}) for \schemaelement{result-activity}.  This
element contains most of the results for an activity\index{activity!results}\index{results!activity} or
phase\index{results!phase}\index{phase!results}.

\begin{table}[htbp]
  \centering
  \begin{tabular}[l]{|l|l|l|l|p{2in}|}
    \hline
    \textbf{Name} & \textbf{Type} & \textbf{Use} & \textbf{Default} &
    \textbf{Comments} \\
    \hline
    \attribute{name}             & string & required & & \\
    \hline
    \attribute{host-demand-mean} & float  & required & & The mean
    service time\index{service time} demand for the activity.\\
    \hline
    \attribute{host-demand-cvsq} & float  & optional & 1.0 & The
    squared coefficient of variation~\index{coefficient of variation} for the activity.\\
    \hline
    \attribute{think-time}       & float  & optional & 0.0 & \\
    \hline
    \attribute{max-service-time} & float  & optional & 0.0 & \\
    \hline
    \attribute{call-order}       & enumeration & optional &
    STOCHASTIC & \texttt{STOCHASTIC} or \texttt{DETERMINISTIC} \\
    \hline
    \hline
    \multicolumn{5}{|c|}{\schematype{ActivtyPhasesType}}\\
    \hline
    \attribute{phase} & integer & required & & 1, 2, or 3 \\
    \hline
    \hline
    \multicolumn{5}{|c|}{\schematype{ActivtyEntryDefType}}\\
    \hline
    \attribute{first-activity} & string & required & &  \\
    \hline
    \hline
    \multicolumn{5}{|c|}{\schematype{ActivtyDefType}} \\
    \hline
    \attribute{bound-to-entry}   & string & optional & & If set, this
    activity is the start of an activity
    graph\index{activity graph!start}. \\
    \hline
  \end{tabular}
  \caption{\label{tab:ActivityDefBase}Attributes for elements of type \schematype{ActivityDefBase}.}
\end{table}


\subsection{MakingCallType}
\label{sec:MakingCallType}

The type \schematype{MakingCallType}, shown in
Figure~\ref{fig:MakingCallType}, is used to define the parameters for
requests\index{request} to entries.  This type is extended by
\schematype{ActivityMakingCallType} and
\schematype{EntryMakingCallType} to defined requests from activities
to entries and for forwarding requests from entry to entry
respectively.  Requests from activities to entries can be either
synchronous, (i.e., a \emph{rendezvous}\index{rendezvous}), through a
\schemaelement{sync-call} element, or asynchronous (i.e., a
\emph{send-no-reply}\index{send-no-reply}), through a
\schemaelement{async-call} element.  Section~\ref{sec:requests}
defines the parameters for a request.
Table~\ref{tab:MakingCallType} lists the attributes for the
types.

\begin{figure}[htbp]
  \centering
%  \epsfxsize=\textwidth 
  \epsffile{xml-schema/call-schema.eps}
  \caption{Schema diagram for the group \schematype{MakingCallType}.}
  \label{fig:MakingCallType}
\end{figure}

\begin{table}[htbp]
  \centering
  \begin{tabular}[l]{|l|l|l|l|p{3in}|}
    \hline
    \textbf{Name} & \textbf{Type} & \textbf{Use} & \textbf{Default} &
    \textbf{Comments} \\
    \hline
    \attribute{dest}       & string  & required &   & The name of the entry to which the requests are made. \\
    \hline
    \hline
    \multicolumn{5}{|c|}{\schematype{ActivityMakingCallType}}\\
    \hline
    \hline
    \attribute{calls-mean} & float   & required &   & The mean number of requests. \\
    \hline
    \hline
    \multicolumn{5}{|c|}{\schematype{EntryMakingCallType}}\\
    \hline
    \hline
    \attribute{prob} & float   & required &   & The probability of forwarding requests. \\
    \hline
  \end{tabular}
  \caption{\label{tab:MakingCallType}Attributes for elements of type \schematype{MakingCallType}.}
\end{table}

%
% Precedence
%

\subsection{PrecedenceType}
\label{sec:PrecedenceType}

The type \schematype{PrecedenceType}, shown in
Figure~\ref{fig:PrecedenceType}, is used to connect one activity to
another within an activity graph\index{activity graph}.  Each element
of this type contains exactly one \schemaelement{pre} element and,
optionally, one \schemaelement{post} element.  The pre elements are
referred to as \emph{join}-lists\index{join-list} as all of the
branches associated with the activities in the join-list must finish
(i.e.~``join'') before the activities in the subsequent post element
can start.  The post element itself is referred to as a
\emph{fork}-list\index{fork-list}.

\begin{figure}[htbp]
  \centering
  \epsffile{xml-schema/precedence-schema.eps}
  \caption{Schema diagram for the type \schematype{PrecedenceType}.}
  \label{fig:PrecedenceType}
\end{figure}

Elements of \schematype{PrecedenceType} can be of one of five types:
\begin{description}
\item[\schematype{SingleActivityListType}:] Elements of this type have
  no attributes and a sequence of exactly one \schemaelement{activity}
  element of \schematype{ActivityType}.
\item[\schematype{ActivityListType}:] Elements of this type have no
  attributes and a sequence one or more \schemaelement{activity}
  elements of \schematype{ActivityType}.
\item[\schematype{AndJoinListType}:] Elements of this type have an optional \schemaelement{quorum} element
  and a sequence of one or more or more \schemaelement{activity} elements of \schematype{ActivityType}.
  Table~\ref{tab:AndJoinListType} show the attributes of \schematype{AndJoinListType}.
\item[\schematype{OrListType}:] Elements of this type have no
  attributes and a sequence one or more \schemaelement{activity}
  elements of \schematype{ActivityOrType}.  These elements specify an
  activity name and a branch
  probability\index{branch!probability}\index{probability!branch}.
  Table~\ref{tab:ActivityOrType} show the attributes of
  \schematype{ActivityOrType}.
\item[\schematype{ActivityLoopListType}:] Elements of this type have
  one optional attribute and a sequence one or more
  \schemaelement{activity} elements of \schematype{ActivityLoopType}.
  These elements specify an activity name and a loop
  count\index{branch!loop count}\index{loop count}.  The optional
  attribute is used to specify the activity that is executed after all
  the loop\index{loop} branches complete.
  Tables~\ref{tab:ActivityLoopListType} and \ref{tab:ActivityLoopType}
  show the attributes of \schematype{ActivityLoopListType} and
  \schematype{ActivityLoopType} respectively.
\end{description}

\begin{table}[htbp]
  \centering
  \begin{tabular}[l]{|l|l|l|l|p{3in}|}
    \hline
    \textbf{Name} & \textbf{Type} & \textbf{Use} & \textbf{Default} &
    \textbf{Comments} \\
    \hline
    \attribute{name} & string & required & & \\
    \hline
    \attribute{quorum} & integer & optional & 0 & The number of branches which must complete for the join to
    finish.  If this attribute is not specified, then all of the branches must finish, which makes this
    object an AND-Join\index{join!and}\index{join!quorum}\index{quorum~join}\\
    \hline
  \end{tabular}
  \caption{\label{tab:AndJoinListType}Attributes for elements of type \schematype{AndJoinListType}.}
\end{table}

\begin{table}[htbp]
  \centering
  \begin{tabular}[l]{|l|l|l|l|p{3in}|}
    \hline
    \textbf{Name} & \textbf{Type} & \textbf{Use} & \textbf{Default} &
    \textbf{Comments} \\
    \hline
    \attribute{name} & string & required & & \\
    \hline
    \attribute{prob} & float & optional & 1.0 & The probability that the branch is
    taken, on average (c.f.~\S\ref{sec:precedence}) \\
    \hline
  \end{tabular}
  \caption{\label{tab:ActivityOrType}Attributes for elements of type \schematype{ActivityOrType}.}
\end{table}

\begin{table}[htbp]
  \centering
  \begin{tabular}[l]{|l|l|l|l|p{3in}|}
  \hline
  \textbf{Name} & \textbf{Type} & \textbf{Use} & \textbf{Default} &
  \textbf{Comments} \\
  \hline
  \attribute{end} & string & required & & \\
  \hline
  \end{tabular}
  \caption{\label{tab:ActivityLoopListType}Attributes for elements of type \schematype{ActivityLoopListType}.}
\end{table}

\begin{table}[htbp]
  \centering
  \begin{tabular}[l]{|l|l|l|l|p{3in}|}
    \hline
    \textbf{Name} & \textbf{Type} & \textbf{Use} & \textbf{Default} &
    \textbf{Comments} \\
    \hline
    \attribute{count} & float & optional & 1.0 & The number of times the loop is
    executed, on average (c.f.~\S\ref{sec:precedence})\\
    \hline
  \end{tabular}
  \caption{\label{tab:ActivityLoopType}Attributes for elements of type \schematype{ActivityLoopType}.}
\end{table}

\subsection{OutputResultType}
\label{sec:OutputResultType}

The type \schematype{OutputResultType}, shown in Figure~\ref{fig:OutputResultType}, is used to create
elements that store results described earlier in Section~\ref{sec:results}.  \schematype{OutputResultType}
is a subtype of \schematype{ResultContentType}.  This latter type defines the result element's attributes.
Elements of this \schematype{OutputResultType} can contain two elements of type
\schematype{ResultContentType}, which contain the $\pm 95\%$ and $\pm 99\%$ confidence intervals, provided
that these results are available.  The attributes for elements of \schematype{ResultContentType} are listed
in Table~\ref{tab:ResultContentType} and are used to store the actual results produced by the solver.  Note
that all the attributes are optional: elements of this type will only have those attributes which are
relevant.

\begin{figure}[htbp]
  \centering
  \epsffile{xml-schema/result-schema.eps}
  \caption{Schema diagram for type \schematype{OutputResultType}}
  \label{fig:OutputResultType}
\end{figure}

\begin{table}[htbp]
  \centering
  \begin{tabular}[l]{|l|l|p{2.5in}|c|}
    \hline
    \textbf{Name} & \textbf{Type} & \textbf{Comments} & \textbf{(xref)}\\
    \hline
    \hline
    \attribute{proc-utilization} & float & Processor utilization for a task, entry, or activity. & \S\ref{sec:processor-wait-utilization-out}\index{utilization!processor}\\
    \hline
    \attribute{proc-waiting} & float & Waiting time at a processor for an activity. & \S\ref{sec:processor-wait-utilization-out}\index{queueing time!processor} \\
    \hline
    \attribute{phaseX-proc-waiting} & float & Waiting time at a processor for phase \emph{X} of an entry. & \S\ref{sec:processor-wait-utilization-out})\index{queueing time!processor} \\
    \hline
    \hline
    \attribute{open-wait-time} & float & Waiting time for open arrivals. \index{waiting time!open arrival} & \S\ref{sec:open-wait-out} \\
    \hline
    \attribute{service-time} & float & Activity service time. & \S\ref{sec:service-time-out}\index{service time}\\
    \hline
    \attribute{loss-probability} & float & Probability of dropping an asynchronous message. & \S\ref{sec:rendezvous-delay-out}\index{loss probability}\\
    \hline
    \attribute{phaseX-service-time} & float & Service time for phase X of an entry. & \S\ref{sec:service-time-out}\index{service time}\\
    \hline
    \attribute{service-time-variance} & float & Variance for an activity. & \S\ref{sec:service-time-variance-out}\index{service time!variance} \\
    \hline
    \attribute{phaseX-service-time-variance} & float & Variance for phase \emph{X} of an entry. & \S\ref{sec:service-time-variance-out}\index{service time!variance} \\
    \hline
    \attribute{phaseX-utilization} & float & Utilization for phase \emph{X} of an entry. & \S\ref{sec:througput-utilization-out} \\
    \hline
    \attribute{prob-exceed-max-service-time} & float & & \S\ref{sec:service-time-distribution-out}\index{service time!probability exceeded} \\
    \hline
    \attribute{squared-coeff-variation} & float & Squared coefficient of variation\index{coefficient of variation} over all phases of an entry & \S\ref{sec:service-time-variance-out}\index{service time!variance} \\
    \hline
    \attribute{throughput-bound} & float & Throughput bound for an entry. & \S\ref{sec:bounds-out}\index{throughput!bounds}\\
    \hline
    \hline
    \attribute{throughput} & float & Throughput for a task, entry or activity. & \S\ref{sec:througput-utilization-out}\index{throughput}\\
    \hline
    \attribute{utilization} & float & Utilization for a task, entry, activity. & \S\ref{sec:througput-utilization-out}\index{utilization!task}\index{utilization!entry}\\
    \hline
    \hline
    \attribute{waiting} & float & Rendezvous delay & \S\ref{sec:rendezvous-delay-out}\index{rendezvous!delay}\\
    \hline
    \attribute{waiting-variance} & float & Variance of delay for a rendezvous & \S\ref{sec:rendezvous-variance-out}\index{rendezvous!variance}\\
    \hline
  \end{tabular}
  \caption{\label{tab:ResultContentType}Attributes for elements of type \schematype{ResultContentType}.}
\end{table}

\subsection{OutputResultJoinDelayType}
\label{sec:OutputResultJoinDelayType}

The type \schematype{OutputResultJoinDelayType} is similar to \schematype{OutputResultType}.  The attributes
of this type are shown in Table~\ref{tab:OutputResultJoinDelay}.

\begin{figure}[htbp]
  \centering
  \epsffile{xml-schema/result-join-delay-schema.eps}
  \caption{Schema diagram for type \schematype{OutputResultJoinDelayType}}
  \label{fig:OutputResultJoinDelayType}
\end{figure}

\begin{table}[htbp]
  \centering
  \begin{tabular}[l]{|l|l|p{2.5in}|c|}
    \hline
    \textbf{Name} & \textbf{Type} & \textbf{Comments} & \textbf{(xref)}\\
    \hline
    \hline
    \attribute{join-waiting} & float & Join delay\index{join!delay} & \S\ref{sec:join-delay-out} \\
    \hline
    \attribute{join-variance} & float & Join delay variance\index{join!variance} & \S\ref{sec:join-delay-out} \\
    \hline
  \end{tabular}
  \caption{\label{tab:OutputResultJoinDelay}Attributes for elements of type
    \schematype{OutputResultJoinDelayType}.}
\end{table}

\subsection{OutputDistributionType}
\label{sec:OutputDistributionType}

Elements of type \schematype{OutputDistributionType}, shown in Figure~\ref{fig:OutputDistributionType}, are
used to define and store histograms\index{histogram} of phase and activity service times\index{service time!histogram}.
The optional \schemaelement{underflow-bin}, \schemaelement{overflow-bin} and \schemaelement{histogram-bin}
elements, all the elements are of type \schematype{HistogramBinType}, are used to store results.  

The attributes of \schematype{OutputDistributionType} elements are used to both store the parameters for the
histogram, and output statistics.  Refer to Table~\ref{tab:OutputDistributionType}

\begin{figure}[htbp]
  \centering
%  \epsfxsize=\textwidth 
  \epsffile{xml-schema/distribution-schema.eps}
  \caption{Schema for type \schematype{OutputDistributionType}.}
  \label{fig:OutputDistributionType}
\end{figure}

\begin{table}[htbp]
  \centering
  \begin{tabular}[l]{|l|l|l|l|p{2.8in}|}
    \hline
    \textbf{Name} & \textbf{Type} & \textbf{Use} & \textbf{Default} & \textbf{Comments} \\
    \hline
    \attribute{min} & float & required & & The lower bound of the collected histogram data. \\
    \hline
    \attribute{max} & float & required & & The upper bound of the collected histogram data. \\
    \hline
    \attribute{number-bins} & integer & optional & 20 & The number of bins in the distribution. \\
    \hline
    \hline
    \attribute{mid-point} & float & optional & & \\
    \hline
    \attribute{bin-size} & float & optional & &  \\
%     \hline
%     \attribute{mean} & float & optional & & The mean of the distribution. \\
%     \hline
%     \attribute{std-dev} & float & optional & & The standard deviation of the distribution. \\
%     \hline
%     \attribute{skew} & float & optional & & The skew of the distribution. \\
%     \hline
%     \attribute{kurtosis} & float & optional & & The kurtosis of the distribution. \\
    \hline
  \end{tabular}
  \caption{\label{tab:OutputDistributionType}Attributes for elements of type \schematype{OutputDistributionType}.}
\end{table}

\subsection{HistogramBinType}
\label{sec:HistogramBinType}

\begin{table}[htbp]
  \centering
  \begin{tabular}[l]{|l|l|p{2.5in}|c|}
    \hline
    \textbf{Name} & \textbf{Type} & \textbf{Comments} & \textbf{(xref)}\\
    \hline
    \hline
    \attribute{begin} & float & Lower limit of the bin. & \\
    \hline
    \attribute{end} & float & Upper limit of the bin. & \\
    \hline
    \attribute{prob} & float & The probability that the measured value lies within
    \attribute{begin} and \attribute{end}. & \\
    \hline
    \attribute{conf-95} & float & & \\
    \hline
    \attribute{conf-99} & float & & \\
    \hline
  \end{tabular}
  \caption{\label{tab:HistogramBinType}Attributes for elements of type
    \schematype{HistogramBinType}.}
\end{table}

\section{Schema Constraints}
\label{sec:schema-constraints}

The schema contains a set of constraints\index{schema!constraints}
that are checked by the Xerces\index{Xerces} XML
parser~\cite{sw:xerces} to ensure that the model file is
valid.  XML editors can also enforce these constraints so that the
model is somewhat correct before being passed to the simulator or
analytic solver.  The constraints are as follow:
\begin{itemize}
\item All processor must have a unique name.
\item All tasks must have a unique name.
\item All entries must have a unique name.
\item All activities must have a unique name within a given task.
\item All synchronous requests must have a valid destination.
\item All asynchronous requests must have a valid destination.
\item All forwarding requests must have a valid destination.
\item All activity connections (in precedence blocks) must refer to
  valid activities.\index{activity!connection}\index{precedence}
\item All activity replies must refer to a valid entry.
\item All activity loops must refer to a valid activities.
\item Each entry has only one activity bound to it.
\item Phases are restricted to values one through three.
\item All phase attributes\index{attribute!unique phase} within an
  entry must be unique.
\end{itemize}
Further validation is performed by the solver itself.  Refer to
Section~\ref{sec:error-messages} for the error messages generated.

One downside of using the Xerces\index{Xerces!error messages} XML
parser library is that the Xerces tends to give rather cryptic error
messages when compared to other tools.  If an XML file fails to pass
the validation phase, and the error looks cryptic, chances are very
good that there is a genuine problem with the XML input file.  Xerces\index{Xerces}
has a bad habit of coming back with cryptic errors when constraint
checking\index{constraint checking} fails, and only gives you the
general area in the file where the actual problem is.

One easy and convenient solution around this problem is to validate
the XML\index{XML!validation} file using another XML tool.  Tools that
have been found to give more user friendly feedback are
XMLSpy\index{XMLSpy} (any edition), and XSDvalid\index{XSDvalid} (Java
based, freely available).  Another solution is to check if a
particular tool can de-activate schema validation and rely on the
actual tool to do its own internal error checking.  Currently this is
not supported in any of the LQN tools which are XML enabled, but it
maybe implemented later on.

If the XML file validates using other tools, but fails validation with
Xerces\index{Xerces!validation}, or if the XML file fails validation on other tools, but passes
with Xerces then please report the problem.  The likelihood of
validation passing with Xerces and not other tools will be much higher
then the reverse scenario, because Xerces does not rigorously apply
the XML Schema standard as other tools.  Other sources of problems
could be errors in the XML schema itself, or some unknown bug in the
Xerces library.
\index{XML Grammar)|}\index{Grammar!XML)|}

%%% Local Variables: 
%%% mode: latex
%%% mode: outline-minor 
%%% fill-column: 108
%%% TeX-master: "userman"
%%% End: 

%% -*- mode: latex; mode: outline-minor; fill-column: 108 -*-
%%
%%  Created by Martin Mroz on 2009-02-03.
%%  Copyright (c) 2009 __MyCompanyName__. All rights reserved.
%%
%% ------------------------------------------------------------------------
%%  $Id: lqx.tex 14882 2021-07-07 11:09:54Z greg $
%% ------------------------------------------------------------------------
\newcommand{\ModLang}{LQX }
\newcommand{\lqns}{Layer Queueing Network Solver}
\newcommand{\oper}[1]{\texttt{#1}\index{#1@\texttt{#1}}}
\newcommand{\opex}[1]{\texttt{#1}\index{"#1@\texttt{"#1}}}

\lstdefinelanguage{LQX}{
  keywords={foreach,in,if,else,for,while,break,return},
  keywords={[2]solve,print,println,println_spaced,task,processor,entry,throughput,utilization,read_data,array_create},
  keywordstyle=[2]{\itshape},
  morecomment=[l]{//},
}

\chapter{\ModLang Users Guide}
\index{LQX|(textbf}
\section{Introduction to \ModLang}

The \ModLang programming language is a general purpose programming
language used for the control of input parameters to the \lqns system
for the purposes of sensitivity analysis. This language allows a user to
perform a wide range of different actions on a variety of different input
sources, and to subsequently solve the model and control the output of
the resulting data.

\subsection{Input File Format}

The \ModLang programming language follows grammar rules which are very
similar to those of ANSI C and PHP. The main difference between these
languages and \ModLang is that \ModLang is a loosely typed language with
strict runtime type-checking and a lack of variable coercion
(``type casting''). Additionally, variables need not be declared before
their first use. They do, however, have to be initialized. If they
are un-initialized prior to their first use, the program will fail.

\subsubsection{Comment Style}

\ModLang supports two of the most common commenting syntaxes, ``C-style''
and ``C++-style.'' Any time the scanner discovers two forward slashes
side-by-side ({\tt //}), it skips any remaining text on that line
(until it reaches a newline). These are ``C++-style'' comments. The other
rule that the scanner uses is that should it encounter a forward slash
followed by an asterisk (``/*''), it will ignore any text it finds up
until a terminating asterisk followed by a slash (``*/''). The preferred
commenting style in \ModLang programs is to use ``C++-style'' comments
for single-line comments and to use ``C-style'' comments where they
span multiple lines. This is a matter of style.

\subsubsection{Intrinsic Types}

There are five intrinsic types\index{LQX!intrinsic types|(textbf} in the \ModLang programming languages:

\begin{itemize}
\item \textbf{Number}: All numbers are stored in IEEE double-precision floating point format.
\item \textbf{String}: Any literal values between (``) and ('') in the input.
\item \textbf{Null}: This is a special type used to refer to an ``empty'' variable.
\item \textbf{Boolean}: A type whose value is limited to either ``true'' or ``false.''
\item \textbf{Object}: An semi-opaque type used for storing complex objects. See ``Objects.''
\item \textbf{File Handle} File handles to open files for writing/appending or reading. See ``File Handles.''
\end{itemize}

\ModLang also supports a pseudo-intrinsic ``Array'' type. Whereas for any other
object types, the only way to interact with them is to explicitly invoke a
method on them, objects of type Array may be accessed with {\tt operator []}
and with {\tt operator []=}, in a familiar C- and C++-style syntax.

The Object type also allows certain attributes to be exposed as ``properties.''
These values are accessed with the traditional C-style {\tt object.property}
syntax. An example property is the {\tt size} property for an object of
type Array, accessed as {\tt array.size} Only instances of type Object
or its derivatives have properties. Number, String, Null and Boolean
instances all have no properties\index{LQX!intrinsic types|)}.

\subsubsection{Built-in Operators}
\index{LQX!operators|(textbf}

There are eight built in arithmetic operators in the \ModLang programming language:
\begin{itemize}
\item \textbf{\oper{<<}}: Shift the left operand \emph{left} by the amount specified by the right operand.
  Both operands must be non-negative integers.
\item \textbf{\oper{>>}}: Shift the left operand \emph{right} by the amount specified by the right operand.
  Both operands must be non-negative integers.
\item \textbf{\oper{+}}: Add the left operand to the right operand.
\item \textbf{\oper{-}}: Subtract the right operand from the left operand.
\item \textbf{\oper{*}}: Multiply the left operand by the right operand.
\item \textbf{\oper{/}}: Divide the left operand by the right operand.
\item \textbf{\oper{\%}}: Take the modulus of the left operand by the right operand.  This operation is
  implemented using the \texttt{fmod()} function, so both operands can be
  real numbers.
\item \textbf{\oper{**}}: Raise the left operand by the right operand.  This operation is implemented
  using the \texttt{power()} function.
\end{itemize}
All operands must be numeric.

There are six built in comparison operators in the \ModLang programming language:
\begin{itemize}
\item \textbf{\oper{==}}: Return \emph{true} if the left operand equals the right operand.
\item \textbf{\opex{!=}}: Return \emph{true} if the left operand is not equal to the right operand.
\item \textbf{\oper{<=}}: Return \emph{true} if the left operand is less than or equal to the right operand.
\item \textbf{\oper{>=}}: Return \emph{true} if the left operand is greater than or equal to the right operand.
\item \textbf{\oper{>}}:  Return \emph{true} if the left operand is greater than the right operand.
\item \textbf{\oper{<}}:  Return \emph{true} if the left operand is less than the right operand.
\end{itemize}
All operands must be numeric.

There are three built in logical operators in the \ModLang programming language:
\begin{itemize}
\item \textbf{\opex{!}}: Return \emph{true} if the operand is false.
\item \textbf{\oper{\&\&}}: Return \emph{true} if the left and right operands are true, otherwise return
  false.  Short-circuit evaluation is used so if the left operand evaluates to false, the right operand is
  not evaluated.
\item \textbf{\oper{||}}: Return \emph{true} if either the left or right operand is true, otherwise return
  false. Short-circuit evaluation is used so if the left operand evaluates to true, the right operand is not
  evaluated.
\end{itemize}
All operands must be boolean.

There are nine built in assignement operators in the \ModLang programming language:
\begin{itemize}
\item \textbf{\oper{=}}: Set the value of the right operand to the value of the left operand.
\item \textbf{\oper{+=}}: Equivalent to: \texttt{a = a + (b)}.
\item \textbf{\oper{-=}}: Equivalent to: \texttt{a = a - (b)}.
\item \textbf{\oper{*=}}: Equivalent to: \texttt{a = a * (b)}.  Note that \texttt{a *= b + c} is not necessarily the same
  as \texttt{a = a * b + c} because \oper{*=} has lower precedence\index{LQX!operator!precedence} than \oper{+}.
\item \textbf{\oper{/=}}: Equivalent to: \texttt{a = a / (b)}.
\item \textbf{\oper{**=}}: Equivalent to: \texttt{a = a ** (b)}.
\item \textbf{\oper{<<=}}: Equivalent to: \texttt{a = a << (b)}.  The left and right operands must both be
  non-negative integers.
\item \textbf{\oper{>>=}}: Equivalent to: \texttt{a = a >> (b)}.  The left and right operands must both be
\end{itemize}
All operands must be numeric.

\index{LQX!operators|)}

\subsubsection{Operator Precedence and Associativity}
\index{LQX!operators!precedence|(textbf}

\begin{tabular}{|c|l|c|}
  \hline
  pre & operator & associativity \\
  \hline
  1  & \oper{()} & left \\
  2  & \oper{[]} & left \\
  3  & \opex{!}  & left \\
  4  & \oper{**} & right \\
  5  & \oper{*}, \oper{/}, \oper{\%} & left \\
  6  & \oper{+}, \oper{-} & left \\
  7  & \oper{<<}, \oper{>>} & left \\
  8  & \oper{>=}, \oper{<=}\oper{<}, \oper{>} & left \\
  9  & \oper{==}  \opex{!=} & left \\
  10 & \oper{\&\&} & left \\
  11 & \oper{||} & left \\
  12 & \oper{=}, \oper{+=}, \oper{-=}, \oper{*=}, \oper{/=}, \oper{\%=}, \oper{**=}, \oper{<<=}, \oper{>>=} & none \\
  \hline
\end{tabular}

\index{LQX!operators!precedence|)}

\subsubsection{Arrays and Iteration}

The built-in Array type is very similar to that used by PHP. It is actually
a hash table, also known as a ``Dictionary'' or a ``Map'' for which you may use
any object as a key, and any object as a value. It is important to realize that
different types of keys will reference different entries. That is to say that
{\tt integer 0} and {\tt string ``0''} will not yield the same value from the
Array when used as a key.

The Array object exposes a couple of convenience APIs, as detailed in Section~\ref{sec:api}.
These methods are simply short-hand notation for the full function calls they
replace, and provide no additional functionality. Arrays may be created in
three different ways:

\begin{itemize}
\item {\tt array\_create(...)} and {\tt array\_create\_map(key,value,...)}:\\
  The explicit, but long and wordy way of creating an array of objects or a map is by using
  the standard functional API. {\tt array\_create(...)} takes an arbitrary number
  of parameters (from 0 up to the maximum specified, for all practical purposes infinity),
  and returns a new Array instance consisting of {\tt [0=>arg1, 1=>arg2, 2=>arg3, ...]}.

  The other function, {\tt array\_create\_map(key,value,...)} takes an even number of
  arguments, from 0 to 2n. The first argument is used as the key, and the second argument
  used as the value for that key, and so on. The resulting Array instance consists
  of {\tt [arg1=>arg2, arg3=>arg4, ...]}. Both of these methods are documented in Section~\ref{sec:api}.
\item {\tt $[$arg1, arg2, ...$]$}: Shorthand notation for {\tt array\_create(...)}
\item \{{\tt k1=$>$v1, k2=$>$v2, ...}\}: Shorthand notation for {\tt array\_create\_map(...)}
\end{itemize}

The \ModLang language supports two different methods of iterating over the
contents of an Array. The first involves knowing what the keys in the array
actually are. This is a ``traditional'' iteration.

\lstset{language=LQX}
\begin{lstlisting}
  /* Traditional Array Iteration */
  for (idx = 0; key < array.size; idx=idx+1) {
    print("Key ", idx, " => ", array[idx]);
  }
\end{lstlisting}

In the above code snippet, we assume there exists an array which contains
{\tt n} values, stored at indexes 0 through {\tt n-1}, continuously. However,
the language provides a more elegant method for iterating over the contents
of an array which does not require prior knowledge of the contents of the array.
This is known as a ``{\tt foreach}'' loop. The statement above can be rewritten
as follows:

\lstset{language=LQX}
\begin{lstlisting}
  /* More modern array itteration */
  foreach (key, value in array) {
    print("Key ", key, " => ", value);
  }
\end{lstlisting}

This method of iteration is much cleaner and is the recommended way of
iterating over the contents of an array. However, there is little guarantee
of the order of the results in a {\tt foreach} loop, especially when keys
of multiple different types are used.

\subsubsection{Type Casting}

The \ModLang programming language provides a number of built-in methods for
converting between variables of different types. Any of these methods support
any input value type except for the Object type. The following is a non-extensive
list of use cases for each of the different type casting methods and the
results. Complete documentation is provided in Section~\ref{sec:api}.

\begin{multicols}{2}
  \begin{center}
    \begin{tabular}{|p{1.7in}|p{1.0in}|}
      \hline
      \multicolumn{2}{|l|}{\textbf{str(...)}}\\
      \hline
      {\tt str()} & ``'' \\
      {\tt str(1.0)} & ``1'' \\
      {\tt str(1.0, "+", true)} & ``1+true'' \\
      {\tt str([1.0, "t"])} & ``[0=$>$1, 1=$>$t]''\\
      {\tt str(null)} & ``(null)''\\
      \hline
    \end{tabular}
  \end{center}
  \begin{center}
    \begin{tabular}{|p{1.7in}|p{1.0in}|}
      \hline
      \multicolumn{2}{|l|}{\textbf{double(?)}}\\
      \hline
      {\tt double(1.0)} & 1.0\\
      {\tt double(null)}& 0.0\\
      {\tt double("9")} & 9.0\\
      {\tt double(true)}& 1.0\\
      {\tt double([0])} & {\tt null}\\
      \hline
    \end{tabular}
  \end{center}
\end{multicols}

\begin{multicols}{2}
  \begin{center}
    \begin{tabular}{|p{1.7in}|p{1.0in}|}
      \hline
      \multicolumn{2}{|l|}{\textbf{boolean(?)}}\\
      \hline
      {\tt boolean(1.0)} & {\tt true}\\
      {\tt boolean(17.0)} & {\tt true}\\
      {\tt boolean(-9.0)} & {\tt true}\\
      {\tt boolean(0.0)} & {\tt false}\\
      {\tt boolean(null)}& {\tt false}\\
      {\tt boolean("yes")}&{\tt true}\\
      {\tt boolean(true)}& {\tt true}\\
      {\tt boolean([0])} & {\tt null}\\
      \hline
    \end{tabular}
  \end{center}
\end{multicols}
\subsubsection{Keywords}
\index{LQX!keywords|(textbf}

The following strings are keywords in the language: 
\begin{description}
\item[Control Flow]: 
\texttt{break}\index{break@\texttt{break}},
\texttt{else}\index{else@\texttt{else}},
\texttt{foreach}\index{foreach@\texttt{foreach}},
\texttt{for}\index{for@\texttt{for}},
\texttt{function}\index{function@\texttt{function}},
\texttt{if}\index{if@\texttt{if}},
\texttt{in}\index{in@\texttt{in}},
\texttt{return}\index{return@\texttt{return}},
\texttt{while}\index{while@\texttt{while}}.
\item[Constants]:
\texttt{NULL}\index{NULL@\texttt{NULL}},
\texttt{false}\index{false@\texttt{false}},
\texttt{null}\index{null@\texttt{null}},
\texttt{true}\index{true@\texttt{true}}.
\item[File Input/Output]:
\texttt{append}\index{append@\texttt{append}},
\texttt{file\_close}\index{file\_close@\texttt{file\_close}},
\texttt{file\_open}\index{file\_open@\texttt{file\_open}},
\texttt{print\_spaced}\index{print\_spaced@\texttt{print\_spaced}},
\texttt{println\_spaced}\index{println\_spaced@\texttt{println\_spaced}},
\texttt{println}\index{println@\texttt{println}},
\texttt{print}\index{print@\texttt{print}},
\texttt{read\_data}\index{read\_data@\texttt{read\_data}},
\texttt{read\_loop}\index{read\_loop@\texttt{read\_loop}},
\texttt{read}\index{read@\texttt{read}},
\texttt{write}\index{write@\texttt{write}}.
\end{description}
\index{LQX!keywords|)}

\subsubsection{User-Defined Functions}

The LQX programming language has support for user-defined functions. When
defined in the language, functions do not check their arguments types so every
effort must be taken to ensure that arguments are the type that you expect them
to be. The number of arguments will be checked. Variable-length
argument lists are also supported with the use of the
ellipsis ({\tt ...}) notation. Any arguments given that fall into the ellipsis
are converted into an array named ({\tt \_va\_list}) in the functions' scope.
This is a regular instance of Array consisting of 0 or more items and can be
operated on using any of the standard operators.

User-defined functions do \textbf{not} have access to any variables except
their arguments and External (\$-prefixed) and Constant (@-prefixed) variables.
Any additional variables must be passed in as arguments, and all values must
be returned. All arguments are in \textbf{only}. There are no out or inout
arguments supported. All arguments are copied, pass-by-value. The basic syntax
for declaring functions is as follows:

\lstset{language=LQX}
\begin{lstlisting}
  function <name>(<arg1>, <arg2>, ...) {
    <body>
    return (value);
  }
\end{lstlisting}

You can return a value from a function anywhere in the body using the {\tt return}
function. A function which reaches the end of its body without a call to return will
automatically return NULL. {\tt return()} is a function, not a language construct,
and as such the brackets are required. The number of arguments is not limited,
so long as each one has a unique name there are no other constraints.

\subsection{Program Input/Output and External Control}

The \ModLang language allows users to write formatted output to external files and standard output and to read
input data from external files/pipes and standard input. These features may be combined to allow LQNX to be
controlled by a parent process as a child process providing model solving functionality. These capabilities will
be described in the following sections.

\subsubsection{File Handles}

The \ModLang language allows users to open files for program input and output. Handles to these
open files are stored in the symbol table for use by the print() functions for file output and the
read\_data() function for data input. Files may be opened for writing/appending or for reading.
The \ModLang interpreter keeps track of which file handles were opened for writing and which
were opened for reading.

The following command opens a file for writing. If it exists it is overwritten. It is also possible
to append to an existing file. The three options for the third parameter are {\tt write}, {\tt append}, and {\tt read}.

\lstset{language=LQX}
\begin{lstlisting}
  file_open( output_file1, "test_output_99-peva.txt", write );
\end{lstlisting}

To close an open file handle the following command is used:

\lstset{language=LQX}
\begin{lstlisting}
  file_close( output\_file1 );
\end{lstlisting}

\subsubsection{File Output}

Program output to both files and standard output is possible with the print functions. If the first parameter to
the functions is an existing file handle opened for writing output is directed to that file. If the first parameter
is not a file handle output is sent to standard output. Standard output is useful when it is desired to control LQNX
execution from a parent process using pipes. If the given file handle has been opened for reading instead of writing
a runtime error results.

There are four variations of print commands with two options. One option is a newline at the end of the line. It is
possible to specify additional newlines with the {\tt endl} parameter. The second option is controlling the spacing
between columns either by specifying column widths in integers or supplying a text string to be placed between columns.

The basic print functions are {\tt print()} and {\tt println()} with the {\tt ln} specifying a newline at the end.

\lstset{language=LQX}
\begin{lstlisting}
  println( output_file1, "Model run #: ", i, " t1.throughput: ", t1.throughput );

  print( output_file1, "Model run #: ", i, " t1.throughput: ", t1.throughput, endl );
\end{lstlisting}

It should be noted that with the extra {\tt endl} parameter both of these calls will produce the same output.
The acceptable inputs to all print functions are valid file handles, quoted strings, \ModLang variables that
evaluate to numerical or boolean values ( or expressions that evaluate to numerical/boolean values ) as well
as the newline specifier {\tt endl}. Parameters should be separated by commas.

To print to standard output no file handle is specified as follows:

\lstset{language=LQX}
\begin{lstlisting}
  println( "subprocess lqns run #: ", i, " t1.throughput: ", t1.throughput );
\end{lstlisting}

To specify the content between columns the print functions {\tt print\_spaced()} and {\tt println\_spaced() }
are used. The first parameter after the file handle (the second parameter when a file handle is specified) is
used to specify either column widths or a text string to be placed between columns. If no file handle is specified
as when printing to standard output then the first parameter is expected to be the spacing specifier. The specifier
must be either an integer or a string.

The following {\tt println\_spaced()} command specifies the string {\tt ", " } to be placed between columns. It could be used
to create comma separated value (csv) files.

\lstset{language=LQX}
\begin{lstlisting}
  println_spaced( output_file2, ", ", $p1, $p2, $y1, $y2, t1.throughput );
\end{lstlisting}

Example output: 0, 2, 0.1, 0.05, 0.0907554

The following {\tt println\_spaced()} command specifies the integer 12 as the column width.

\lstset{language=LQX}
\begin{lstlisting}
  println_spaced( output_file3, 12, $p1, $p2, $y1, $y2, t1.throughput );
\end{lstlisting}

\subsubsection{Reading Input Data from Files/Pipes}

Reading data from input files/pipes is done with the {\tt read\_data()} function. Data can either be read from a valid
file handle that has been opened for reading or from standard input. Reading data from standard input is useful when
is useful when it is desired to control LQNX execution from a parent process using pipes.
If the given file handle has been opened for writing rather than reading a runtime error results. The first
parameter is either a valid file handle for reading or the strings {\tt stdout} or {\tt - } specifying
standard input. The data that can be read can be either numerical values or boolean values.

There are two forms in which the {\tt read\_data()} function can be used. The first is by specifying a list of
\ModLang variables which correspond to the expected inputs from the file/pipe. This requires the data inputs
from the pipe to be in the expected order.

\lstset{language=LQX}
\begin{lstlisting}
  read_data( input_file, y, p, keep_running );
\end{lstlisting}

The second form  in which the {\tt read\_data()} function can be used is much more robust. It can go into a loop
attempting to read string/value pairs from the input pipe until a termination string {\tt STOP\_READ } is encountered.
The string must corespond to an existing \ModLang variable (either numeric or boolean) and the corresponding
value must be of the same type.

\lstset{language=LQX}
\begin{lstlisting}
  read_data( stdin, read_loop );
\end{lstlisting}

Sample input:

\lstset{language=C++}
\begin{lstlisting}
  y 10.0 p 1.0 STOP_READ
  continue_processing false STOP_READ
\end{lstlisting}

\subsubsection{Controlling LQNX from a Parent Process}

The file output and data reading functions can be combined to allow an LQNX process to be created and controlled
by a parent process through pipes. Input data can be read in from pipes, be used to solve a model with those
parameters and the output of the solve can be sent back through the pipes to the parent process for analysis.
A \ModLang program can easily be written to contain a main loop that reads input, solves the model, and returns
output for analysis. The termination of the loop can be controlled by a boolean flag that can be set from the
parent process.

This section describes an example of how to control LQNX execution from a parent process, in this case a {\tt perl }
script which uses the {\tt open2()} function to create a child process with both the standard input and output
mapped to file handles in the {\tt perl} parent process. This allows data sent from the parent to be read with
{\tt read\_data( stdin, ...)} and output from the \ModLang print statements sent to standard output to be received for
analysis in the parent.

This also provides synchronization between the parent and the child
LQNX processes. The {\tt read\_data()} function
blocks the LQNX process until it has received its expected data. Similarly the parent process can be programmed
to wait for feedback from the child LQNX process before it continues.

The following is an example perl script that can be used to control a LQNX child process.

\lstset{language=Perl}
\begin{lstlisting}
  #!/usr/bin/perl -w
  # script to test the creation and control of an lqns solver subprocess
  # using the LQX language with synchronization

  use FileHandle;
  use IPC::Open2;

  @phases = ( 0.0, 0.25, 0.5, 0.75, 1.0 );
  @calls = ( 0.1, 3.0, 10.0 );

  # run lqnx as subprocess receiving data from standard input
  open2( *lqnxOutput, *lqnxInput, "lqnx 99-peva-pipe.lqnx" );

  for $call (@calls) {
    for $phase (@phases) {
      print( lqnxInput "y ", $call, " p ", $phase, " STOP_READ " );
      while( $response = <lqnxOutput>) !~ m/subprocess lqns run/ ){}
      print( "Response from lqnx subprocess: ", $response );
    }
  }

  # send data to terminate lqnx process
  print( lqnxInput "continue_processing false STOP_READ" );
\end{lstlisting}


The above program invokes the lqnx program with its input file as a child process with {\tt open2()}. Two file
handles are passed as parameters. These will be used to send data over the pipe to the LQNX process to be
received as standard input and to receive feedback from the LQX program which it sends as standard output.

The while loop at line 17 waits for the desired feedback from the model solve before continuing. This example
uses stored data but a real application such as optimization would need to analyze the feedback data to decide
which data to send back in the next iteration therefore this synchronization is important.

When the data is exhausted the LQNX process needs to be told to quit. This is done with the final print statement
which sets the continue\_processing flag to false. This causes the main loop in the LQX program which follows to
quit.

\lstset{language=LQX}
\begin{lstlisting}
  <lqx><![CDATA[

  i = 1;
  p = 0.0;
  y = 0.0;
  continue_processing = true;

  while ( continue_processing ) {

    read_data( stdin, read_loop ); /* read data from input pipe */

    if( continue_processing ) {

      $p1 = 2.0 * p;
      $p2 = 2.0 * (1 - p);
      $y1 = y;
      $y2 = 0.5 * y;
      solve();

      /* send output of solve through stdout through pipe */
      println( "subprocess lqns run #: ", i, " t1.throughput: ", t1.throughput );
      i = i + 1;
    }
  }
  ]]></lqx>

\end{lstlisting}

The variables {\tt p}, {\tt y}, and {\tt continue\_processing} all need to be initialized to their correct
types before the loop begins as they need to exist when the {\tt read\_data()} function searches for them
in the symbol table. This is necessary as they are all local variables. External variables that exist in
the LQN model such as {\tt \$p} and {\tt \$y} don't need initialization.

% -------------------------------------------------------- [Examples]

\subsection{Writing Programs in \ModLang}

\subsubsection{Hello, World Program}

A good place to start learning how to write programs in \ModLang is of course
the traditional Hello World program. This would actually be a single line,
and is not particularly interesting. This would be as follows:

\lstset{language=LQX}
\begin{lstlisting}
  println("Hello, World!");
\end{lstlisting}

The ``{\tt println()}'' function takes an arbitrary number of arguments of any
type and will output them (barring a file handle as the first parameter) to
standard output, followed by a newline.

\subsubsection{Fibonacci Sequence}

This particular program is a great example of how to perform flow control
using the \ModLang programming language. The Fibonacci sequence is an extremely
simple infinite sequence which is defined as the following piecewise function:

\begin{equation}
  \mathrm{fib}(X) =  \left\{ \begin{array}{l l} 1 & x = 0, 1 \\ \mathrm{fib}(x-1) +
      \mathrm{fib}(x-2) & \mathrm{otherwise} \end{array} \right.
\end{equation}

Thus we can see that the Fibonacci sequence is defined as a recursive sequence.
The naive approach would be to write this code as a recursive function. However,
this is extremely inefficient as the overhead of even simple recursion in LQX
can be substantial. The best way is to roll the algorithm into into a loop of some
type. In this case, the loop is terminated when we have reached a target
number in the Fibonacci sequence \{ 1, 1, 2, 3, 5, 8, 13, 21, ...\}.

\lstset{language=LQX}
\begin{lstlisting}
  /* Initial Values */
  fib_n_minus_two = 1;
  fib_n_minus_one = 1;
  fib_n = 0;

  /* Loop until we reach 21 */
  while (fib_n < 21) {
    fib_n = fib_n_minus_one + fib_n_minus_two;
    fib_n_minus_two = fib_n_minus_one;
    fib_n_minus_one = fib_n;
    println("Currently: ", fib_n);
  }
\end{lstlisting}

As you can see, this language is extremely similar to C or PHP. One of the
few differences as far as expressions are concerned is that pre-increment/decrement
and post-increment/decrement are not supported. Neither are short form expressions such
as {\tt +=, -=, *=, /=}, etc.

\subsubsection{Re-using Code Sections}

Many times, there will be code in your \ModLang programs that you would like to invoke
in many places, varying only the parameters. The \ModLang programming language does
provide a pretty standard functions system as described earlier. Bearing in mind the
caveats (some degree of overhead in function calls, plus the inability to see
global variables without having them passed in), we can make pretty ingenious use
of user-defined functions within \ModLang code.

When defining functions, you can specify only the number of arguments, not their
types, so you need to make sure things are what you expect them to be, or your
code may not perform as you expect. We will begin by demonstrating a substantially
shorter (but as described earlier) much less efficient implementation of the
Fibonacci Sequence using functions and recursion.

\lstset{language=LQX}
\begin{lstlisting}
  function fib(n) {
    if (n == 0 || n == 1) { return (1); }
    return (fib(n-2) + fib(n-1));
  }
\end{lstlisting}

Once defined, a function may be used anywhere in your code, even in other user
defined functions (and itself | recursively). This particular example functions
very well for the first 10-11 fibonacci numbers but becomes substantially slower
due to the increased number of relatively expensive function invocations.
\emph{Remember}, {\tt return() } is a function, not a language construct.
The brackets are required.

A much more interesting use of functions, specifically those with variable
length argument lists, is an implementation of the formula for standard deviation
of a set of values:

\lstset{language=LQX}
\begin{lstlisting}
  function average(/*Array<double>*/ inputs) {
    double sum = 0.0;
    foreach (v in inputs) { sum = sum + v; }
    return (sum / inputs.size);
  }

  function stdev(/*boolean*/ sample, ...) {
    x_bar = average(_va_list);
    sum_of_diff = 0.0;

    /* Figure out the divisor */
    divisor = _va_list.size;
    if (sample == true) {
      divisor = divisor - 1;
    }

    /* Compute sum of difference */
    foreach (v in _va_list) {
      sum_of_diff = sum_of_diff + pow(v - x_bar, 2);
    }

    return (pow(sum_of_diff / divisor, 0.5));
  }
\end{lstlisting}

You can then proceed to compute the standard deviation of the variable length of
arguments for either sample or non-sample values as follows, from anywhere in your
program after it has been defined:

\lstset{language=LQX}
\begin{lstlisting}
  stdev(true,  1, 2, 5, 7, 9, 11);
  stdev(false, 2, 9, 3, 4, 2);
\end{lstlisting}

\subsubsection{Using and Iterating over Arrays}

As mentioned in the ``Arrays and Iteration'' under section 1.1 of the Manual,
\ModLang supports intrinsic arrays and {\tt foreach} iteration. Additionally, any
type of object may be used as either a key or a value in the array. The following
example illustrates how values may be added to an array, and how you can
iterate over its contents and print it out. The following snippet creates an array,
stores some key-value pairs with different types of keys and values, looks up a
couple of them and then iterates over all of them.

\lstset{language=LQX}
\begin{lstlisting}
  /* Create an Array */
  array = array\_create();

  /* Store some key-value pairs */
  array[0] = "Slappy";
  array[1] = "Skippy";
  array[2] = "Jimmy";

  /* Iterate over the names */
  foreach ( index,name in array ) {
    print("Chipmunk #", index, " = ", name);
  }

  /* Store variables of different types, shorthand */
  array = {true => 1.0, false => 3.0, "one" => true, "three" => false}

  /* Shorthand indexed creation with iteration */
  foreach (value in [1,1,2,3,5,8,13]) {
    print ("Next fibonacci is ", value);
  }
\end{lstlisting}

\subsection{Actual Example of an LQX Model Program}

The following LQX code is the complete LQX program for the model designated {\tt peva-99}.
The model itself contains a few model parameters which the LQX code configures, notably
{\tt \$p1}, {\tt \$p2}, {\tt \$y1} and {\tt \$y2}. The LQX program is responsible for
setting the values of all model parameters at least once, invoking solve and optionally
printing out certain result values. Accessing of result values is done via the LQNS
bindings API documented in Section~\ref{sec:api}.

The program begins by defining an array of values that it will be setting for each of
the external variables. By enumerating as follows, the program will set the variables
for the cross product of {\tt phase} and {\tt calls}.

\lstset{language=LQX}
\lstset{name=lqx-program}
\begin{lstlisting}
  phase = [ 0.0, 0.25, 0.5, 0.75, 1.0 ];
  calls = [ 0.1, 3.0, 10.0 ];
  foreach ( idx,p in phase ) {
    foreach ( idx,y in calls ) {
    \end{lstlisting}

    Next, the program uses the input values {\tt p} and {\tt y} to compute the values of
    {\tt \$p1}, {\tt \$p2}, {\tt \$y1} and {\tt \$y2}. Any assignment to a variable
    beginning with a {\tt \$} requires that variable to have been defined externally,
    within the model definition. When such an assignment is made the value of the right-hand
    side is effectively put everywhere the left-hand side is found within the model.

    \lstset{ firstnumber= 5  }
    \begin{lstlisting}
      $p1 = 2.0 * p;
      $p2 = 2.0 * (1 - p);
      $y1 = y;
      $y2 = 0.5 * y;
    \end{lstlisting}

    Since all variables have now been set, the program invokes the solve function with
    its optional parameter, the suffix to use for the output file of the current run.
    This particular program outputs {\tt in.out-\$p1-\$p2-\$y1-\$y2} files, so that
    results for a given set of input values can easily be found. As shown in the
    documentation in Section 3, {\tt solve(<opt> suffix)} will return a boolean
    indicating whether or not the solution converged, and this program will abort
    when that happens, although that is certainly not a requirement.

    \lstset{ firstnumber=9}
    \begin{lstlisting}
      if (solve(str($p1,"-",$p2,"-",$y1,"-",$y2)) == false) {
        println("peva-99.xml:LQX: Failed to solve the model properly.");
        abort(1, "Failed to solve the model.");
      } else {
      \end{lstlisting}

      The remainder of the program outputs a small table of results for certain key
      values of interest to the person running the solution using the APIs in Section 3.

      \lstset{ firstnumber=13}
      \begin{lstlisting}
        t0 = task("t0");
        p0 = processor("p0");
        e0 = entry("e0");
        ph1 = phase(e0, 1);
        ctoe1 = call(ph1, "e1");
        println("+-------------------------------------+");
        println("t0 Throughput:  ", t0.throughput        );
        println("t0 Utilization: ", t0.utilization       );
        println("+                -----                +");
        println("e0 Throughput:  ", e0.throughput        );
        println("e0 TP Bound:    ", e0.throughput_bound  );
        println("e0 Utilization: ", e0.utilization       );
        println("+                -----                +");
        println("ph Utilization: ", ph1.utilization      );
        println("ph Svt Variance:", ph1.service_time_variance );
        println("ph Service Time:", ph1.service_time     );
        println("ph Proc Waiting:", ph1.proc_waiting     );
        println("+                -----                +");
        println("call Wait Time: ", ctoe1.waiting_time   );
        println("+-------------------------------------+");
      }
    }
  }
\end{lstlisting}

% ------------------------------------------------------------------------------

\section{API Documentation}
\label{sec:api}
\subsection{Built-in Class: Array}

\begin{tabular}{|p{0.8in}|p{2.2in}||p{3in}|}
  \hline
  \multicolumn{3}{|l|}{\textbf{Summary of Attributes}}\\
  \hline
  numeric & {\tt size} & The number of key-value pairs stored in the array.\\
  \hline
\end{tabular}\\
\\\ \\
\begin{tabular}{|p{0.8in}|p{2.2in}||p{3in}|}
  \hline
  \multicolumn{3}{|l|}{\textbf{Summary of Constructors}}\\
  \hline
  object[Array] & {\tt array\_create(...)} & This method returns a new instance of the Array class,
  where each the first argument to the method is mapped to index numeric(0), the second one to
  numeric(1) and so on, yielding {\tt [0=$>$arg0, 1=$>$arg1, ...]}\\
  object[Array] & {\tt array\_create\_map(k,v,...)} & This method returns a new instance of the Array
  class where the first argument to the constructor is used as the key, and the second is used
  as the value, and so on. The result is a n array {\tt [arg0=$>$arg1, arg2=$>$arg3,...] }\\
  \hline
\end{tabular}  \\
\\\ \\
\begin{tabular}{|p{0.8in}|p{2.2in}||p{3in}|}
  \hline
  \multicolumn{3}{|l|}{\textbf{Summary of Methods}}\\
  \hline
  null & {\tt array\_set(object[Array] a, ? key, ? value)} & This method sets the value {\tt value} of any
  type for the key {\tt key} of any type, for array {\tt a}. The shorthand notation for this operation
  is to use the {\tt operator []}.\\
  ref$<$?$>$ & {\tt array\_get(object[Array] a, ? key) } & This method obtains a reference to the slot in the
  array {\tt a} for the key {\tt key}. If there is no value defined in the array yet for the given key,
  a new slot is created for that key, assigned to NULL, and a reference returned.\\
  boolean & {\tt array\_has(object[Array] a, ? key)} & Returns whether or not there is a value defined
  on array {\tt a} for the given key, {\tt key}.\\
  \hline
\end{tabular}

\subsection{Built-in Global Methods and Constants}

\subsubsection{Intrinsic Constants}
\begin{tabular}{|p{0.8in}|p{2.2in}||p{3in}|}
  \hline
  \multicolumn{3}{|l|}{\textbf{Summary of Constants}}\\
  \hline
  double & {\tt @infinity} & IEEE floating-point numeric infinity.\\
  double & {\tt @type\_un} & The type\_id for an Undefined Variable.\\
  double & {\tt @type\_boolean} & The type\_id for a Boolean Variable.\\
  double & {\tt @type\_double} & The type\_id for a Numeric Variable.\\
  double & {\tt @type\_string} & The type\_id for a String Variable.\\
  double & {\tt @type\_null} & The type\_id for a Null Variable.\\
  \hline
\end{tabular}

\subsubsection{General Utility Functions}

\begin{tabular}{|p{0.8in}|p{2.2in}||p{3in}|}
  \hline
  \multicolumn{3}{|l|}{\textbf{Summary of Methods}}\\
  \hline
  null & {\tt abort(numeric n, string r)} & This call will immediately halt the flow of the program,
  with failure code {\tt n} and description string {\tt r}. This cannot be ``caught'' in any way
  by the program and will result in the interpreter not executing any more of the program.\\
  null & {\tt copyright()} & Displays the \ModLang copyright message.\\
  null & {\tt print\_symbol\_table()} & This is a very useful debugging tool which output the
  name and value of all variables in the current interpreter scope.\\
  null & {\tt print\_special\_table()} & This is also a useful debugging tool which outputs the
  name and value of all special (External and Constant) variables in the interpreter scope.\\
  numeric & {\tt type\_id(? any)} & This method returns the Type ID of any variable, including
  intrinsic types (numeric, boolean, null, etc.) and the result can be matched to the
  constants prefixed with @type (@type\_null, @type\_un, @type\_double, etc.)\\
  null & {\tt return(? any)} & This method will return any value from a user-defined function.
  This method cannot be used in global scope.\\
  \hline
\end{tabular}

\subsubsection{Numeric/Floating-Point Utility Functions}

\begin{tabular}{|p{0.8in}|p{2.2in}||p{3in}|}
  \hline
  \multicolumn{3}{|l|}{\textbf{Summary of Methods}}\\
  \hline
  numeric & {\tt abs(numeric n)} & Returns the absolute value of the argument {\tt n} \\
  numeric & {\tt ceil(numeric n)} & Returns the value of {\tt n} rounded up.\\
  numeric & {\tt exp(numeric n)} & Returns $e^{n}$.\\
  numeric & {\tt floor(numeric n)} & Returns the value of {\tt n} rounded down.\\
  numeric & {\tt log(numeric n)} & Returns $\log(n), n > 0$ (natural log).\\
  numeric & {\tt max(array a)} & Returns the largest value found in the array \texttt{a}.\\
  numeric & {\tt max(numeric a, numeric b, ...)} & Returns the largest value among the numeric args.\\
  numeric & {\tt min(array a)} & Returns the smallest value found in the array \texttt{a}.\\
  numeric & {\tt min(numeric a, numeric b, ...)} & Returns the smallest value among the numeric args.\\
  numeric & {\tt pow(numeric bas, numeric x)} & Returns {\tt bas} to the power {\tt x}.\\
  numeric & {\tt rand()} & Returns a random number between 0 and 1.\\
  numeric & {\tt round(numeric n)} & Returns the value of {\tt n} rounded to the nearest integer.\\
  numeric & {\tt sqrt(numeric n)} & Returns $\sqrt{n}, n \ge 0$.\\
  \hline
  numeric & {\tt normal(numeric a, numeric b)} & Returns a normally distributed random number with mean of {\tt a} and a standard deviation of {\tt b}.\\
  numeric & {\tt gamma(numeric a, numeric b)} & Returns a Gamma distributed random number with a mean of {\tt a} and a shape of {\tt b}.\\
  numeric & {\tt uniform(numeric a, numeric b)} & Returns a uniformily distributed random number between {\tt a} and {\tt b}.\\
  numeric & {\tt poisson(numeric a)} & Returns a Poisson distributed random number with a mean of {\tt a}.\\
  \hline
\end{tabular}

\subsubsection{Type-casting Functions}

\begin{tabular}{|p{0.8in}|p{2.2in}||p{3in}|}
  \hline
  \multicolumn{3}{|l|}{\textbf{Summary of Methods}}\\
  \hline
  string & {\tt str(...)} & This method will return the same value as the function {\tt print(...)}
  would have displayed on the screen. Each argument is coerced to a string and then adjacent
  values are concatenated.\\
  numeric & {\tt double(? x)} & This method will return 1.0 or 0.0 if provided a boolean of
  {\tt true} or {\tt false} respectively. It will return the passed value for a double,
  0.0 for a null and fail (NULL) for an object. If it was passed a string, it will attempt
  to convert it to a double. If the whole string was not numeric, it will return NULL, otherwise
  it will return the decoded numeric value.\\
  boolean & {\tt bool(? x)} & This method will return {\tt true} for a numeric value of (not 0.0), a
  boolean {\tt true} or a string ``true'' or ``yes''. It will return {\tt false} for a numeric
  value 0.0, a NULL or a string ``false'' or ``no'', or a boolean {\tt false}. It will
  return NULL otherwise.\\
  \hline
\end{tabular}

% ------------------------------------------------------------------------------

\section{API Documentation for the LQN Bindings}

\subsection{LQN Class: Document}
\begin{tabular}{|p{1.0in}|p{2.3in}||p{2.8in}|}
  \hline
  \multicolumn{3}{|l|}{\textbf{Summary of Attributes}}\\
  \hline
  \hline
  \multicolumn{3}{|l|}{\emph{Read-Write Attributes}}\\
  \hline
  string &  {\tt comment} & The model comment. \\
  double &  {\tt conv\_val} & The model convergence value for lqns.\\
  double &  {\tt it\_limit} & The iteration limit for lqns.\\
  double &  {\tt print\_int} & Iteration numbers where intermediate results are generated.\\
  double &  {\tt underrelax\_coeff} & The underrelaxation coefficient for lqns.\\
  double &  {\tt seed\_value} & The initial seed value for the random number generator for lqsim.\\
  double &  {\tt number\_of\_blocks} & \\
  double &  {\tt block\_time} & \\
  double &  {\tt precision} & \\
  double &  {\tt warm\_up\_loops} & \\
  double &  {\tt warm\_up\_time} & \\
  \hline
  \hline
  \multicolumn{3}{|l|}{\emph{Read-Only Attributes}}\\
  \hline
  double &  {\tt iterations} & The number of solver iterations/simulation blocks.\\
  double &  {\tt invocation} & The solution invocation number.\\
  double &  {\tt system\_cpu\_time} & Total system time for this invocation.\\
  double &  {\tt user\_cpu\_time} & Total user time for this invocation.\\
  double &  {\tt elapsed\_time} & Total elapsed time for this invocation.\\
  boolean & {\tt valid} & True if the results are valid.\\
  double &  {\tt waits} & The number of times \texttt{wait()} was called.\\
  \hline
\end{tabular}
\\\\\ \\
\begin{tabular}{|p{1.0in}|p{2.3in}||p{2.8in}|}
  \hline
  \multicolumn{3}{|l|}{\textbf{Summary of Constructors}}\\
  \hline
  Document & {\tt document()} & Returns the Document object\\
  \hline
\end{tabular}

\subsection{LQN Class: Processor}
\begin{tabular}{|p{1.0in}|p{2.3in}||p{2.8in}|}
  \hline
  \multicolumn{3}{|l|}{\textbf{Summary of Attributes}}\\
  \hline
  double & {\tt utilization} & The utilization of the Processor\\
  \hline
\end{tabular}
\\\\\ \\
\begin{tabular}{|p{1.0in}|p{2.3in}||p{2.8in}|}
  \hline
  \multicolumn{3}{|l|}{\textbf{Summary of Constructors}}\\
  \hline
  Processor & {\tt processor(string name)} & Returns an instance of Processor from the current LQN model with the given name.\\
  \hline
\end{tabular}

\subsection{LQN Class: Group}
\begin{tabular}{|p{1.0in}|p{2.3in}||p{2.8in}|}
  \hline
  \multicolumn{3}{|l|}{\textbf{Summary of Attributes}}\\
  \hline
  double & {\tt utilization} & The utilization of the Group\\
  \hline
\end{tabular}
\\\\\ \\
\begin{tabular}{|p{1.0in}|p{2.3in}||p{2.8in}|}
  \hline
  \multicolumn{3}{|l|}{\textbf{Summary of Constructors}}\\
  \hline
  Group & {\tt processor(string name)} & Returns an instance of Group from the current LQN model with the given name.\\
  \hline
\end{tabular}

\subsection{LQN Class: Task}
\begin{tabular}{|p{1.0in}|p{2.3in}||p{2.8in}|}
  \hline
  \multicolumn{3}{|l|}{\textbf{Summary of Attributes}}\\
  \hline
  double & {\tt throughput} & The throughput of the Task\\
  double & {\tt utilization} & The utilization of the Task\\
  double & {\tt proc\_utilization} & This Task's processor utilization\\
  Array & {\tt phase\_utilizations} & Individual phase utilizations\\
  \hline
\end{tabular}
\\\\\ \\
\begin{tabular}{|p{1.0in}|p{2.3in}||p{2.8in}|}
  \hline
  \multicolumn{3}{|l|}{\textbf{Summary of Constructors}}\\
  \hline
  Task & {\tt task(string name)} & Returns an instance of Task from the current LQN model with the given name.\\
  \hline
\end{tabular}

\subsection{LQN Class: Entry}
\begin{tabular}{|p{1.0in}|p{2.3in}||p{2.8in}|}
  \hline
  \multicolumn{3}{|l|}{\textbf{Summary of Attributes}}\\
  \hline
  boolean & {\tt has\_phase\_1} & Whether the entry has a phase 1 result\\
  boolean & {\tt has\_phase\_2} & Whether the entry has a phase 2 result\\
  boolean & {\tt has\_phase\_3} & Whether the entry has a phase 3 result\\
  boolean & {\tt has\_open\_wait\_time} & Whether the entry has an open wait time\\
  double & {\tt phase1\_proc\_waiting} & Phase 1 Processor Wait Time\\
  double & {\tt phase1\_service\_time\_variance} & Phase 1 Service Time Variance\\
  double & {\tt phase1\_service\_time} & Phase 1 Service Time\\
  double & {\tt phase1\_utilization} & Phase 1 (task) Utilization\\
  double & {\tt phase1\_pr\_time\_exceeded} & Phase 1 Max Service Time Exceeded\\
  double & {\tt phase2\_proc\_waiting} & Phase 2 Processor Wait Time\\
  double & {\tt phase2\_service\_time\_variance} & Phase 2 Service Time Variance\\
  double & {\tt phase2\_service\_time} & Phase 2 Service Time\\
  double & {\tt phase2\_utilization} & Phase 2 (task) Utilization\\
  double & {\tt phase2\_pr\_time\_exceeded} & Phase 2 Max Service Time Exceeded\\
  double & {\tt phase3\_proc\_waiting} & Phase 3 Processor Wait Time\\
  double & {\tt phase3\_service\_time\_variance} & Phase 3 Service Time Variance\\
  double & {\tt phase3\_service\_time} & Phase 3 Service Time\\
  double & {\tt phase3\_utilization} & Phase 3 (task) Utilization\\
  double & {\tt phase3\_pr\_time\_exceeded} & Phase 3 Max Service Time Exceeded\\
  double & {\tt proc\_utilization} & Entry processor utilization\\
  double & {\tt squared\_coeff\_variation} & Squared coefficient of variation\\
  double & {\tt throughput\_bound} & Entry throughput bound\\
  double & {\tt throughput} & Entry throughput\\
  double & {\tt utilization} & Entry utilization\\
  double & {\tt waiting} & Entry open wait time\\
  \hline
\end{tabular}
\\\\\ \\
\begin{tabular}{|p{1.0in}|p{2.3in}||p{2.8in}|}
  \hline
  \multicolumn{3}{|l|}{\textbf{Summary of Constructors}}\\
  \hline
  Entry & {\tt entry(string name)} & Returns the Entry object for the model entry whose name is given as name\\
  \hline
\end{tabular}

\subsection{LQN Class: Phase}
\begin{tabular}{|p{1.0in}|p{2.3in}||p{2.8in}|}
  \hline
  \multicolumn{3}{|l|}{\textbf{Summary of Attributes}}\\
  \hline
  double & {\tt service\_time} & Phase service time\\
  double & {\tt service\_time\_variation} & Phase service time variance\\
  double & {\tt utilization} & Phase utilization\\
  double & {\tt proc\_waiting} & Phase processor waiting time\\
  double & {\tt pr\_time\_exceeded} & Phase Max Service Time Exceeded\\
  \hline
\end{tabular}
\\\\\ \\
\begin{tabular}{|p{1.0in}|p{2.3in}||p{2.8in}|}
  \hline
  \multicolumn{3}{|l|}{\textbf{Summary of Constructors}}\\
  \hline
  Phase & {\tt phase(object entry, numeric\_int nr)} & Returns the Phase object for a given entry's phase number specified as nr\\
  \hline
\end{tabular}

\subsection{LQN Class: Activity}
\begin{tabular}{|p{1.0in}|p{2.3in}||p{2.8in}|}
  \hline
  \multicolumn{3}{|l|}{\textbf{Summary of Attributes}}\\
  \hline
  double & {\tt proc\_utilization} & The activities' share of the processor utilization\\
  double & {\tt proc\_waiting} & Activities' processor waiting time\\
  double & {\tt service\_time\_variance} & Activity service time variance\\
  double & {\tt service\_time} & Activity service time\\
1091  double & {\tt pr\_time\_exceeded} & Activity Max Service Time Exceeded\\
  double & {\tt squared\_coeff\_variation} & The square of the coefficient of variation\\
  double & {\tt throughput} & The activity throughput\\
  double & {\tt utilization} & Activity utilization\\
  \hline
\end{tabular}
\\\\\ \\
\begin{tabular}{|p{1.0in}|p{2.3in}||p{2.8in}|}
  \hline
  \multicolumn{3}{|l|}{\textbf{Summary of Constructors}}\\
  \hline
  Activity & {\tt activity(object task, string name)} & Returns an instance of Activity from the current LQN model, whose name corresponds to an activity in the given task.\\
  \hline
\end{tabular}

\subsection{LQN Class: Call}
\begin{tabular}{|p{1.0in}|p{2.3in}||p{2.8in}|}
  \hline
  \multicolumn{3}{|l|}{\textbf{Summary of Attributes}}\\
  \hline
  double & {\tt waiting} & Call waiting time\\
  double & {\tt waiting\_variance} & Call waiting time\\
  double & {\tt loss\_probability} & Message loss probability for asynchronous messages\\
  \hline
\end{tabular}
\\\\\ \\
\begin{tabular}{|p{1.0in}|p{2.3in}||p{2.8in}|}
  \hline
  \multicolumn{3}{|l|}{\textbf{Summary of Constructors}}\\
  \hline
  Call & {\tt call(object phase, string destinationEntry)} & Returns the call from an entry's phase (phase) to the destination entry whose name is (dest).\\
  Call & {\tt call(object activity, string destinationEntry)} & Returns the call from a task's activity (activity) to the destination entry whose name is (dest)\\
  \hline
\end{tabular}

\subsection{Pragmas}
\begin{tabular}{|p{1.0in}|p{2.3in}||p{2.8in}|}
  \hline
  \multicolumn{3}{|l|}{\textbf{Summary of Attributes}}\\
  \hline
  string & {\tt value} & Value of pramga.\\
  \hline
\end{tabular}
\\\\\ \\
\begin{tabular}{|p{1.0in}|p{2.3in}||p{2.8in}|}
  \hline
  \multicolumn{3}{|l|}{\textbf{Summary of Constructors}}\\
  \hline
  Pragma & {\tt pragma(string pragma)} & Returns the value for the pragma supplied as an argument.\\
  \hline
\end{tabular}

\subsection{Confidence Intervals}

\begin{tabular}{|p{1.0in}|p{2.3in}||p{2.8in}|}
  \hline
  \multicolumn{3}{|l|}{\textbf{Summary of Constructors}}\\
  \hline
  conf\_int & {\tt conf\_int(object, int level)} & Returns the $\pm$ (level) for the attribute for the object\\
  \hline
\end{tabular}
\index{LQX|)}
%%% Local Variables:
%%% mode: latex
%%% mode: outline-minor
%%% fill-column: 108
%%% TeX-master: "userman"
%%% End:

%% -*- mode: latex; mode: outline-minor; fill-column: 108 -*-
%%
%% Title:  spex
%%
%% $HeadURL: http://franks.dnsalias.com/svn/lqn/trunk/doc/userman/lqns.tex $
%% Original Author:     Alex Hubbard
%% Created:             27 April 2011
%%
%% ------------------------------------------------------------------------
%%  $Id: srvn.tex 16945 2024-01-26 13:02:36Z greg $
%% ------------------------------------------------------------------------
%%
\lstdefinelanguage{LQN}{
  basicstyle=\ttfamily,
  keywords={G,P,U,T,E,A,R,C,p,g,t,s,c,f,y,z,u,F,-1,->,\#pragma},
  classoffset=1,
  morekeywords={<param>,<proc-id>,<group-id>,<task-id>,<entry-id>,<activity-id>,<sched>,<opt-mult>,<opt-cap>,<opt-pri>,<opt-grp>,<opt-think-time>,<opt-obs>,<opt-repl>,<value>,<expression>,<entry-list>,<activity-list>,<expression-list>,<real>,<int>,<string>},
  keywordstyle=\itshape,
  classoffset=0,
  alsoletter={-1<>},
  sensitive=true,
  morecomment=[l]{\# },
  morestring=[b]'',
  index={pragma}
}
\newcommand{\nonterminal}[1]{$<$\emph{#1}$>$\mbox{}}
%%
%%
%%
\chapter{LQN Input File Format}
\label{ch:srvn}

This Chapter describes the original `SRVN' input file format, augmented with the Software Performance
EXperiment driver (SPEX)\index{SPEX} grammar.  In this model format models are specified breadth-first, in
contrast to the XML format described in \S\ref{sec:xml-grammar} where models are specified depth-first.
This specification means that all resources such as processors, tasks and entries, are defined before they
are referenced. Furthermore, each resource is grouped into its own section in the input file.
Figure~\ref{fig:srvn-schema} shows the basic schema and Listing~\ref{lst:lqn-file-layout} shows the basic
layout of the model file.

\begin{figure}[ht]
  \centering
  \epsffile{srvn-schema/srvn-schema.eps}
  \caption{SRVN input schema}
  \label{fig:srvn-schema}
\end{figure}

\lstset{language=LQN,numbersep=10pt,firstnumber=1,texcl}
\begin{lstlisting}[float,caption={LQN file layout},label=lst:lqn-file-layout]
  # \color{blue}Pragmas
  #pragma <param>=<value>

  # \color{red}Parameters (SPEX)
  $var = <expression>
  $var = [ <expression-list> ]

  # \color{blue}General Information
  G "<string>" <real> <int> <int> <real> -1

  # \color{black}Processor definitions
  P 0
    p <proc-id> <sched> <opt-mult> <opt-repl> <opt-obs>
  -1

  # \color{blue}Group definitions
  U 0
    g <group-id> <real> <opt-cap> <proc-id>
  -1

  # \color{black}Task definitions
  T 0
    t <task-id> <sched> <entry-list> -1 <proc-id> <opt-pri> <opt-think-time>
                <opt-mult> <opt-repl> <opt-grp> <opt-obs>
  -1

  # Entry definitions
  E 0
    A <activity-id>
    s <entry-id> <real> ... -1 <opt-obs>
    y <entry-id> <entry-id> <real> ... -1 <opt-obs>
  -1

  # \color{blue}Activity definitions
  A <task-id>
    s <activity-id> <real> <opt-obs>
    y <activity-id> <entry-id> <real> <opt-obs>
  :
    <activity-list> -> <activity-list>
  -1

  # \color{red}Result defintions (SPEX)
  R 0
    plot( $var, $var,... )
    $var = <expression>
  -1

  # \color{red}Convergence defintions (SPEX)
  C 0
    $var = <expression>
  -1
\end{lstlisting}

Each of the sections within the input file begins with a key-letter, as follows:
\begin{description}
\item[\$] SPEX parameters\index{SPEX!parameters} (optional).
\item[G] General solver parameters (optional).
\item[P] Processor definitions.
\item[U] Processor group definitions (optional).
\item[T] Task definitions.
\item[E] Entry definitions.
\item[A] Task activity definitions  (optional).
\item[R] SPEX result definitions\index{SPEX!results}  (optional).
\item[C] SPEX convergence\index{SPEX!convergence}  (optional).
\end{description}
Section~\ref{sec:lqn-file-format} describes the input sections necessary to solve a model, i.e.\ \texttt{P},
\texttt{U} \texttt{T}, \texttt{E}, and \texttt{A}.  Section~\ref{sec:spex} describes the additional input
sections for solving multiple models using SPEX, i.e.\ \texttt{\$}, \texttt{R}, and \texttt{C}.  The
complete input grammar is listed in Appendix~\ref{sec:old-grammar}.

\section{Lexical Conventions}

The section describes the lexical conventions of the SRVN input file format.

\subsection{White Space}

White space\dindex{LQN}{white space}, such as spaces, tabs and new-lines, is ignored except within strings.
Object definitions can span multiple lines.

\subsection{Comments}

Any characters following a hash mark (\#)\index{\#} through to the end of the line are considered to be a
comment\dindex{LQN}{comment} and are generally ignored.  However, should a line begin with optional whitespace
followed by `\texttt{\#pragma}'\index{\#pragma}\index{pragma}, then the remainder of the line will be
treated by the solver as a pragma (more on pragmas below).

\subsection{Identifiers}
\label{sec:lqn-identifiers}\label{LQN!identifiers|textbf}

Identifiers\dindex{LQN}{identifier} are used to name the objects in the model.  They consist of zero or more
leading underscores (`\_'), followed by a character, followed by any number of characters, numbers or
underscores. Punctuation characters and other special characters such as the dollar-sign (`\texttt{\$}') are
not permitted.  Non-numeric identifiers must be a minimum of two characters in length\footnote{Single
  characters are used for section and record keys.}  The following, {\tt 1}, {\tt p1}, {\tt p\_1}, and {\tt
  \_\_P\_21\_proc} are valid identifiers, while {\tt \$proc} and {\tt \$1} are not.

\subsection{Variables}
\label{sec:lqn-variables}\label{LQN!variables|textbf}

Variables \dindex{LQN}{parameter} are used to set values of various objects such as the multiplicity of
tasks and the service times of the phases of entries.  Variables are modifed by
SPEX\dindex{SPEX}{Variables} (see~\S\ref{sec:spex}) to run multiple experiments.  Variables start with a
dollar-sign (`\texttt{\$}') followed by any number of characters, numbers or underscores.  {\tt \$var} and
{\tt \$1} are valid variables while {\tt \$\$} is not.

\section{LQN Model Specification}
\label{sec:lqn-file-format}
\index{LQN|(textbf}

This section describes the mandatory and option input for a basic LQN model file.  SPEX information, namely
\emph{Variables}, (\S\ref{sec:lqn-activity-information}), \emph{Report Information}
(\S\ref{sec:lqn-report-information}) and \emph{Convergence Information}
(\S\ref{sec:lqn-convergence-information}) are described in the section that follows.  All input files are
composed of three mandatory sections: \emph{Processor Information} (\S\ref{sec:lqn-processor-information}),
\emph{Task Information} (\S\ref{sec:lqn-task-information}) and \emph{Entry Information}
(\S\ref{sec:lqn-entry-information}), which define the processors, tasks and entries respectively in the
model.  All of the other sections for a basic model file are optional.  They are: \emph{Pragmas},
\emph{General Information} (\S\ref{sec:lqn-general-information}), \emph{Group Information}
(\S\ref{sec:lqn-group-information}), and \emph{Activity Information}.  The syntax of these specifications
are described next in the order in which they appear in the input model.

\subsection{Pragmas}

Any line beginning with optional whitespace followed by the word `\texttt{\#pragma}'\dindex{LQN!pragma}
defines a pragma which is used by either the analytic solver or the simulator to change its behaviour.  The
syntax for a pramga directive is shown in line~2 in Listing~\ref{lst:lqn-file-layout}.  Pragma's which are
not handled by either the simulator or the analytic solver are ignored.  Pragma's can appear anywhere in the
input file\footnote{Pragma's are processed during lexical analysis.}  though they typically appear first.

\subsection{General Information}
\label{sec:lqn-general-information}

The optional general information section is used to set various control parameters for the analytic solver
LQNS\index{LQNS!parameters}.  These parameters, with the exception of the model comment, are ignored by the
simulator, lqsim.  Listing~\ref{lst:lqn-general-info} shows the format of this section.  Note that these
parameters can also be set using SPEX variables, described below
in~\S\ref{sec:spex-control}\index{SPEX!parameter!control}.

\lstset{language=LQN,basicstyle=\ttfamily,numbersep=10pt,firstnumber=1}
\begin{lstlisting}[caption={General Information},label=lst:lqn-general-info,frame=single]
G "<string>"	# Model title.
  <real> 	# convergence value
  <int>         # iteration limit
  <int>         # Optional print interval.
  <real>        # Optional under-relaxation.
-1
\end{lstlisting}

\subsection{Processor Information}
\label{sec:lqn-processor-information}

Processors\dindex{LQN}{processor} are specified in the processor information section of the input file using
the syntax shown in Listing~\ref{lst:lqn-proc-info}.  The start of the section is identified using
``\texttt{P} \nonterminal{int}'' and ends with ``\texttt{-1}''.  The \nonterminal{int} parameter is either
the number of processor definitions in this section, or zero\footnote{The number of processors,
  \nonterminal{int}, is ignored with all current solvers.}.

\lstset{language=LQN,basicstyle=\ttfamily,numbersep=10pt,firstnumber=1}
\begin{lstlisting}[caption={Processor Information},label=lst:lqn-proc-info,frame=single]
P <int>
  p <proc-id> <sched> <opt-mult>
-1
\end{lstlisting}

Each processor in the model is defined using the syntax shown in line~2 in Listing~\ref{lst:lqn-proc-info}.
Each record in this section beginning with a `\texttt{p}' defines a processor. \nonterminal{proc-id} is
either an integer or an identifier (defined earlier in \S\ref{sec:lqn-identifiers}).  \nonterminal{sched} is
used to define the scheduling discipline for the processor and is one of the code letters listed in
Table~\ref{tab:lqn-proc-sched}.  The scheduling disciplines supported by the model are described in
Section~\ref{sec:processors}.  Finally, the optional \nonterminal{opt-mult} specifies the number of copies
of this processor serving a common queue.  Multiplicity is specified using the syntax shown in
Table~\ref{tab:lqn-multiplicity}.  By default, a single copy of a processor is used for the model.

\begin{table}[h]
  \centering
  \begin{tabular}{|c|l|}
    \hline
    \textbf{\nonterminal{sched}} & \multicolumn{1}{c|}{\textbf{Scheduling Discipline}} \\
    \hline
    \texttt{f} & First-come, first served\index{scheduling!FCFS}.\\
    \texttt{p} & Priority-preemptive resume.\\
    \texttt{r} & Random.\\
    \texttt{i} & Delay (infinite server).\\
    \texttt{h} & Head-of-Line\index{scheduling!head-of-line}.\\
    \texttt{c} \nonterminal{real} & Completely fair share with time quantum  \nonterminal{real}.\index{scheduling!completely fair}\\
    \texttt{s} \nonterminal{real} & Round Robin with time quantum \nonterminal{real}.\\
    \hline
  \end{tabular}
  \caption{Processor Scheduling Disciplines (see~\S\protect\ref{sec:processors}).\dindex{processor}{scheduling}}
  \label{tab:lqn-proc-sched}
\end{table}

\begin{table}[h]
  \centering
  \begin{tabular}{|c|l|}
    \hline
    \textbf{\nonterminal{opt-mult}} & \multicolumn{1}{c|}{\textbf{Multiplicity}} \\
    \hline
    \texttt{m} \nonterminal{int} & \nonterminal{int} identical copies with a common queue. \\
    \texttt{i} & Infinite (or delay).\\
    \hline
    \hline
    \textbf{\nonterminal{opt-repl}} & \multicolumn{1}{c|}{\textbf{Repliplication}} \\
    \hline
    \texttt{r} \nonterminal{int} & \nonterminal{int} replicated copies with separate queues. \\
    \hline
  \end{tabular}
  \caption{Multiplicity and Replication (see~\S\protect\ref{sec:replication}).\dindex{LQN}{multiplicity}}
  \label{tab:lqn-multiplicity}
\end{table}

\subsection{Group Information}
\label{sec:lqn-group-information}

Groups\dindex{LQN}{group} are specified in the group information section of

\begin{lstlisting}[caption={Group Information},label=lst:lqn-group-info,frame=single,firstnumber=1]
U <int>
  g <group-id> <real> <opt-cap> <proc-id>
-1
\end{lstlisting}

\subsection{Task Information}
\label{sec:lqn-task-information}

Tasks\dindex{LQN}{task} are specified in the task information section of the input file using the syntax
shown in Listing~\ref{lst:lqn-task-info}.  The start of the task section is identified using
``\texttt{T}~\nonterminal{int}'' and ends with ``\texttt{-1}''.  The \nonterminal{int} parameter is either
the number of task definitions in this section, or zero.

\begin{lstlisting}[caption={Task Information},label=lst:lqn-task-info,frame=single,firstnumber=1]
T <int>
  t <task-id> <sched> <entry-list> -1 <opt-queue-length> <opt-tokens>  
              <proc-id> <opt-pri> <opt-think-time>
              <opt-mult> <opt-repl> <opt-grp>
  I <task-id> <task-id> <int>    # fan-in for replication
  O <task-id> <task-id> <int>    # fan-out for replication
-1
\end{lstlisting}

Each task definition within this section starts with a `\texttt{t}' and is is defined using the syntax shown
in lines~2 and 3 of Listing~\ref{lst:lqn-task-info}\footnote{Line 3 is a continuation of line 2.}.
\nonterminal{task-id} is an identifier which names the task.  \nonterminal{sched} is used to define the
request distribution for reference tasks\dindex{reference}{task}, or the scheduling discipline for
non-reference tasks.  The scheduling and distribution code letters are shown in
Table~\ref{tab:lqn-task-sched}.  Some disciplines are only supported by the simulator; these are identified
using ``\dag''.  \nonterminal{entry-list} is a list of idententifiers naming the entries of the task.  The
optional \nonterminal{opt-pri} is used to set the priority for the task provided that the processor running
the task is scheduled using a priority discipline.  The optional \nonterminal{opt-think-time} specifies a
think time\index{think time} for a reference task.  The optional \nonterminal{opt-mult} specifies the number
of copies of this task serving a common queue.  Multiplicity is specified using the syntax shown in
Table~\ref{tab:lqn-multiplicity}.  By default, a single copy of a task is used for the model. Finally, the
optional \nonterminal{opt-grp} is used to identify the group that this task belongs to provided that the
task's processor is using fair-share scheduling\index{scheduling!fair share}


\begin{table}[h]
  \centering
  \begin{tabular}{|c|l|}
    \hline
    \multicolumn{2}{|c|}{Reference tasks (customers)\index{task!reference}\index{reference task}.}\\
    \hline
    \textbf{\nonterminal{sched}} & \multicolumn{1}{c|}{\textbf{Request Distribution}} \\
    \hline
    r & Poisson.\\
    b & Bursty\dag.\\
    u & Uniform\dag.\\
    \hline
    \hline
    \multicolumn{2}{|c|}{Non-Reference tasks (servers)\index{task!reference}\index{reference task}.}\\
    \hline
    \textbf{\nonterminal{sched}} & \multicolumn{1}{c|}{\textbf{Queueing Discipline}} \\
    \hline
    n & First come, first served\index{scheduling!FCFS}.\\
    P & Polled service at entries\dag.\\
    h & Head-of-line priority\index{scheduling!head-of-line}.\\
    f & First come, first served\index{scheduling!FCFS}.\\
    i & Infinite (delay) server.\\
    w & Read-Write lock task\dag.\\
    S & Semaphore task\dag.\\
    \hline
  \end{tabular}
  \caption{Task Scheduling Disciplines (see~\S\protect\ref{sec:tasks}).}
  \label{tab:lqn-task-sched}
\end{table}

\begin{table}
  \centering
  \begin{tabular}{|c|l|}
    \hline
    \textbf{Option} & \multicolumn{1}{c|}{\textbf{Parameter}} \\
    \hline
    \nonterminal{integer} & Task priority for tasks running on processors supporting priorities.\\
    \texttt{z} \nonterminal{real} & Think Time for reference tasks.\\
    \texttt{q} \nonterminal{real} & Maximum queue length for asynchronous requests. \\
    \texttt{T} \nonterminal{integer} & Initial tokens at semaphore task\dag.\\
    \texttt{m} \nonterminal{integer} & Task multiplicity.\\
    \texttt{r} \nonterminal{integer} & Task replication.\\
    \texttt{g} \nonterminal{identifier} & Group identifier for tasks running of processors with fair share scheduling.\\
    \hline
  \end{tabular}
  \caption{Optional parameters for tasks (see~\S\protect\ref{sec:tasks}).}
  \label{tab:lqn-task-think-time}
\end{table}

\subsection{Entry Information}
\label{sec:lqn-entry-information}

Entries\dindex{LQN}{entry} are specified in the entry information section starting from
``\texttt{E}~\nonterminal{int}'' and ending with ``\texttt{-1}''.  The \nonterminal{int} parameter is either
the number of entry definitions in this section, or zero.  Each record in the entry section defines a single
parameter for an entry, such as its priority, or a single parameter for the phases of the entry, such as
service time.  Listing~\ref{lst:lqn-entry-info} shows the syntax for the most commonly used parameters.

\begin{lstlisting}[caption={Entry Information},label=lst:lqn-entry-info,frame=single,firstnumber=1,float]
E <int>
  A <entry-id> <activity-id>            # Start activity.
  F <entry-id> <entry-id> <real>        # forward.
  s <entry-id> <real> ... -1            # Service time by phase.
  y <entry-id> <entry-id> <real> ... -1 # Synchronous request by phase.
-1
\end{lstlisting}

All entry records start with a key letter, followed by an \nonterminal{entry-id}, followed by from one to up
to five arguments.  Table~\ref{tab:lqn-entry-specifier} lists all the possible entry specifiers.  The table
is split into six classes, based on the arguments to the parameter.  Records used to specifiy service time
and call rate parameters for phases take a list of from one to three arguments and terminated with a
`\texttt{-1}'.  All other entry records, with the exception of histogram information, take a fixed number of
arguments.   Records which only apply to the simulator are marked with a `\dag'.

\begin{table}
  \centering
  \begin{tabular}{|l|p{1.3in}|p{2.4in}|}
    \hline
    \multicolumn{1}{|c|}{\textbf{Key}}
    &\multicolumn{1}{c|}{\textbf{Paramater}}
    &\multicolumn{1}{c|}{\textbf{Arguments}}\\
    \hline
    \hline
    \multicolumn{3}{|c|}{\emph{One argument}}\\
    \hline
    \texttt{a} \nonterminal{entry-id} \nonterminal{real} & Arrival Rate   & \\
    \hline
    \texttt{A} \nonterminal{entry-id} \nonterminal{activity-id}& Start activity & \\
    \hline
    \texttt{p} \nonterminal{entry-id} \nonterminal{int} & Entry priority & \\
    \hline
    \hline
    \multicolumn{3}{|c|}{\emph{One to three phase arguments}}\\
    \hline
    \texttt{s} \nonterminal{entry-id} \nonterminal{real} $...$ -1 & Service Time. & The entry's \nonterminal{entry-id} and mean service
    time\index{entry!service time} value per phase.\\
    \hline
    \texttt{c} \nonterminal{entry-id} \nonterminal{real} $...$ -1 & Coefficient of Variation Squared.~\index{coefficient of variation} & The entry's
    \nonterminal{entry-id} and $\textit{CV}^2$ value for each phase.\\
    \hline
    \texttt{f} \nonterminal{entry-id} \nonterminal{int} $...$ -1 & Call Order & \texttt{STOCHASTIC} or \texttt{DETERMINISTIC} \\
    \hline
    \texttt{M} \nonterminal{entry-id} \nonterminal{real} $...$ -1 & Max Service Time\dag & Output probability that the service time result exceeds the
    \nonterminal{real} parameter, per phase.\\
    \hline
    \hline
    \multicolumn{3}{|c|}{\emph{Arguments for a single phase}}\\
    \hline
    \multicolumn{2}{|l|}{\texttt{H} \nonterminal{int} \nonterminal{entry-id} \nonterminal{real} :
      \nonterminal{real} \nonterminal{opt-int}}  & Histogram\dag:  An \nonterminal{int} phase, followed by a range from
    \nonterminal{real} to \nonterminal{real}, and an optional \nonterminal{int} buckets. \\
    \hline
    \hline
    \multicolumn{3}{|c|}{\emph{Destination and one argument}}\\
    \hline
    \texttt{F} \nonterminal{entry-id} \nonterminal{real} -1 & Forwarding Probability & Source and Destination entries, and probability reply is forwarded.\\
    \hline
    \hline
    \multicolumn{3}{|c|}{\emph{Destination and one to three phase arguments}}\\
    \hline
    \texttt{y} \nonterminal{entry-id} \nonterminal{entry-id} \nonterminal{real} $...$ -1 & Rendevous Rate & Source and Destination entries, and rate per phase.\\
    \hline
    \texttt{z} \nonterminal{entry-id} \nonterminal{entry-id} \nonterminal{real} $...$ -1 & Send-no-Reply Rate & Source and Destination entries, and rate per phase.\\
    \hline
    \hline
    \multicolumn{3}{|c|}{\emph{Semaphores and Locks}\dag} \\
    \hline
    \texttt{P}  \nonterminal{entry-id} & Signal\dag & Entry \nonterminal{entry-id} is used to \emph{signal}\index{signal} a semaphore
    task\index{semaphore task}.\\
    \hline
    \texttt{V}  \nonterminal{entry-id} & Wait\dag   & \\
    \hline
    \texttt{R}  \nonterminal{entry-id} & Read lock\dag & \\
    \hline
    \texttt{U}  \nonterminal{entry-id} & Read unlock\dag  & \\
    \hline
    \texttt{W}  \nonterminal{entry-id} & Write lock\dag   & \\
    \hline
    \texttt{X}  \nonterminal{entry-id} & Write unlock\dag & \\
    \hline
  \end{tabular}
  \caption{Entry Specifiers\index{LQN!entry}}
  \label{tab:lqn-entry-specifier}
\end{table}

\subsection{Activity Information}
\label{sec:lqn-activity-information}

Activity information\index{LQN!activity}\index{activity!LQN} sections are required to specify the parameters
and connectivity of the activities for a task.  Note that unlike all other sections, each task with
activities has its own activity information section. 

An activity information section starts with ``\texttt{A}~\nonterminal{task-id}'' and ends with
``\texttt{-1}''.  The data within an activity information section is partitioned into two parts.  The first
part lists the parameter data for an activity in a fashion similar to the parameter data for an entry; the
second section defines the connectivity of the activities.  Listing~\ref{lst:lqn-activity-info} shows the
basic syntax.

\lstset{language=LQN,basicstyle=\ttfamily,numbersep=10pt,firstnumber=1}
\begin{lstlisting}[caption={Activity Information},label=lst:lqn-activity-info,frame=single,firstnumber=1,float]
A <task-id>
  s <activity-id> <real>
  c <activity-id> <real>
  f <activity-id> <int>
  y <activity-id> <entry-id> <real>
  z <activity-id> <entry-id> <real>
:
  <activity-list> -> <activity-list>
-1
\end{lstlisting}

\begin{table}
  \centering
  \begin{tabular}{|l|l|p{3.5in}|}
    \hline
    \multicolumn{1}{|c|}{\textbf{Key}}
    &\multicolumn{1}{c|}{\textbf{Paramater}}
    &\multicolumn{1}{c|}{\textbf{Arguments}}\\
    \hline
    \multicolumn{3}{|c|}{\emph{one to three phase arguments}}\\
    \hline
    \texttt{s} & Service Time. & The entry's \nonterminal{entry-id} and mean service time value per phase.\\
    \hline
    \texttt{c} & Coefficient of Variation Squared.~\index{coefficient of variation} & The entry's
    \nonterminal{entry-id} and $\textit{CV}^2$ value for each phase.\\
    \hline
    \texttt{f} & Call Order & \texttt{STOCHASTIC} or \texttt{DETERMINISTIC} \\
    \hline
    \multicolumn{3}{|c|}{\emph{Destination and one to three phase arguments}}\\
    \hline
    \texttt{y} & Rendevous Rate & Source and Destination entries, and rate per phase.\\
    \hline
    \texttt{z} & Send-no-Reply Rate & Source and Destination entries, and rate per phase.\\
    \hline
  \end{tabular}
  \caption{Activity Specifiers\index{LQN!activity}}
  \label{tab:lqn-activity-specifier}
\end{table}

\begin{table}
  \centering
  \begin{tabular}{|l|p{3.5in}|}
    \hline
    \multicolumn{2}{|c|}{\emph{Post (or Join) lists}} \\
    \hline
    \nonterminal{activity-id} & \\
    \hline
    \nonterminal{activity-id} $+$ \nonterminal{activity-id} $+ ...$ & \\
    \hline
    \nonterminal{activity-id} $\&$ \nonterminal{activity-id} $\& ...$ & \\
    \hline
    \hline
    \multicolumn{2}{|c|}{\emph{Pre (or Fork) lists}} \\
    \hline
    \nonterminal{activity-id} & \\
    \hline
    \nonterminal{activity-id} $+$ \nonterminal{activity-id} $+ ...$ & \\
    \hline
    \nonterminal{activity-id} $\&$ \nonterminal{activity-id} $\& ...$ & \\
    \hline
  \end{tabular}
  \caption{Activity Lists\dindex{LQN}{activity list}}
  \label{tab:lqn-activity-lists}
\end{table}

\index{LQN|)}

\section{SPEX: Software Performance Experiment Driver}
\label{sec:spex}
\index{SPEX|(textbf}

SPEX, the \textbf{S}oftware \textbf{P}erformance \textbf{E}x\textbf{P}eriment driver, was originally a Perl
program used to generate and solve multiple layered queueing network models.  With version 5 of the solvers
this functionality has been incorporated into the \texttt{lqiolib}\index{lqiolib} and
\texttt{lqx}\index{lqx} libraries used by the simulator and analytic solver.  The primary benefit of this
change is that analytic solutions can run faster for reasons described
in~\cite{perf:mroz-2009-valuetools-lqx}.

SPEX augments the input file described in \S\ref{sec:lqn-file-format} by adding
\emph{variables}\dindex{spex}{variables} for setting input values, a \emph{Report Information}
(\S\ref{sec:lqn-report-information}) used to format output, and an optional \emph{Convergence Information}
(\ref{sec:lqn-convergence-information}) for feeding result values back into input variables.
Listing~\ref{lst:lqn-file-layout} shows these sections starting with comments in
\color{red}red\color{black}.  The syntax of these sections are described next in the order in which they
appear in the input model.

\subsection{Variables}
\label{sec:spex-variables}\index{SPEX!variables|(textbf}

SPEX variables are used to set and possibly vary various input values to the model, and to record results
from the solution of the model.  There are four types of variables:
control, scalar, array and observation.  Control variables are used to define parameters that control the
execution of the solver.  Scalar and array variables are used to set input parameters to the model.
Finally, observation variables are used to record results such as throughputs and utilizations.

\subsubsection{Control Variables}
\label{sec:spex-control}\index{SPEX!variables!control}

Control variables are used to set parameters that are used to control either the analytic solver
\emph{lqns}, or the simulator \emph{lqsim}.  With the exception of \ctrlparam{\$comment}, all of these
variables can be changed as SPEX executes, though this behaviour may not be appropriate in many cases.
Table~\ref{tab:spex-control} lists all of the control variables accepted by SPEX.  See Table~\ref{tab:spex-obsolete-control} for 
control variables that are no longer recognized.

\begin{table}
  \centering
  \begin{tabular}{|l|l|l|c|}
\hline
  \multicolumn{1}{|c|}{\textbf{Control Variable}}
  &\multicolumn{1}{c|}{\textbf{Type of Value}}
  &\multicolumn{1}{c|}{\textbf{Default Value}}
  & \textbf{Program}\\
  \hline
  \ctrlparam{\$model\_comment}  	    & \nonterminal{string}  & ""      &       \\
  \ctrlparam{\$solver}		            & \nonterminal{string}  & \multicolumn{2}{c|}{\emph{deprecated}} \\
  \hline
  \ctrlparam{\$convergence\_value}	    & \nonterminal{real}    & 0.00001 & lqns  \\
  \ctrlparam{\$iteration\_limit}	    & \nonterminal{int}	    & 100     & lqns  \\
  \ctrlparam{\$print\_interval}		    & \nonterminal{int}	    & 1       & lqns  \\
  \ctrlparam{\$underrelaxation}             & \nonterminal{real}    & 0.9     & lqns  \\
  \hline
  \ctrlparam{\$block\_time}                 & \nonterminal{int}     & 50000   & lqsim \\
  \ctrlparam{\$number\_of\_blocks}          & \nonterminal{int}     & 1       & lqsim \\
  \ctrlparam{\$result\_precision}           & \nonterminal{real}    & --      & lqsim \\
  \ctrlparam{\$seed\_value}                 & \nonterminal{int}     & --      & lqsim \\
  \ctrlparam{\$warm\_up\_loops}             & \nonterminal{int}     & --      & lqsim \\
  \hline
  \ctrlparam{\$spex\_convergence}           & \nonterminal{real}    & 0.001   & spex \\
  \ctrlparam{\$spex\_iteration\_limit}      & \nonterminal{int}     & 50      & spex \\
  \ctrlparam{\$spex\_underrelaxation}       & \nonterminal{real}    & 1.0     & spex \\
  \hline
  \end{tabular}
  \caption{Spex Control Variables}
  \label{tab:spex-control}
\end{table}

\subsubsection{Scalar Variables}
\label{sec:spex-scalar}\index{SPEX!variables!scalar}

Scalar variables are used to set input values for the model and are initialized using any
\nonterminal{ternary-expression} (\texttt{?:}\index{?:}\index{SPEX!ternary expressions}) using this syntax:
\begin{quotation}
    \$name = \nonterminal{ternary-expression}
\end{quotation}
The \nonterminal{ternary-expression} may contain any variables defined previously or subsequently in the
input file; order does not matter.  However, undefined variables and observation variables (more on these
below) are not permitted.  If \nonterminal{ternary-expression} is an actual ternary expression, the test
part must evaluate to a boolean.  Refer to Appendix~\ref{sec:old-grammar},~\S\ref{sec:spex-expressions} for
the complete grammar for \nonterminal{ternary-expression}.

\subsubsection{Array Variables}
\label{sec:spex-array}\index{SPEX!variables!array}

Array variables are used to specify a range of values that an input parameter may take on.  There are two
ways to specify this information:
\begin{lstlisting}[numbersep=10pt,firstnumber=1]
  $name = [x, y, z, ...]
  $name = [a : b, c]
\end{lstlisting}
The first form is used to set the variable \texttt{\$name} to the values in the list, \texttt{x},
\texttt{y}, \texttt{z}, \texttt{...}.  The second form is used the set the variable \texttt{\$name} from the
value \texttt{a} to \texttt{b} using a step size of \texttt{c}.  The value of \texttt{b} must be greater
that \texttt{a}, and the step size must be positive.  Regardless of the format, the values for array
variables must be constants.

During the execution of the solver, SPEX iterates over all of the values defined for each array variable. If
multiple arrays are defined, then SPEX generates the cross-product of all possible parameter values.  Note
that if a scalar variable is defined in terms of an array variable, then the scalar variable will be
recomputed for each model generated by SPEX.

\subsubsection{Observation Variables}
\label{sec:spex-observation}\index{SPEX!variables!observation}

There is a set of special symbols that are used to indicate to SPEX
which result values from the solution of the model are of interest.
This result indication has the following form:
\begin{quotation}
  \%\nonterminal{key}\nonterminal{phase} \$var
\end{quotation}
where \nonterminal{key} is a one or two letter key indicating the type of data to be observed and
\nonterminal{phase} is an optional integer indicating the phase of the data to be observed.  The data, once
obtained from the results of the model, is placed into the variable \$var where it may be used in the Result
Information section described below.

To obtain confidence interval information, the format is
\begin{quotation}
  \%\nonterminal{key}\nonterminal{phase}[confidence] \$var1 \$var2
\end{quotation}
where confidence can be 95 or 99, \$var1 is the mean and \$var2 is the half-width of the confidence interval.

The location of a result indication determines the entity to be observed.  Table~\ref{tab:obs-location}
describes each of the keys and where they may be used.

\begin{table}
  \centering
  \begin{tabular}{|l|l|l|l|l|}
    \hline
    \multicolumn{1}{|c|}{\textbf{Key}}
    &\multicolumn{1}{c|}{\textbf{Phase}}
    &\multicolumn{1}{c|}{\textbf{Description}}
    &\multicolumn{2}{l|}{\textbf{Location}}\\
\hline
  \texttt{\%u}  & no  & Utilization                    & processor declaration          & (p info) \\
                & yes &                                & task declaration               & (t info) \\
                & yes &                                & entry service declaration      & (s info) \\
\hline
  \texttt{\%f}  & no  & Throughput                     & task declaration               & (t info) \\
                & no  &                                & entry service declaration      & (s info) \\
\hline
  \texttt{\%pu} & no  & Processor Utilization          & task declaration               & (t info) \\
                & no  &                                & entry service declaration      & (s info) \\
\hline
  \texttt{\%s}  & yes & Service Time                   & entry service declaration      & (s info) \\
\hline
  \texttt{\%v}  & yes & Service Time Variance          & entry service declaration      & (s info) \\
\hline
  \texttt{\%fb} & no  & Throughput Bound               & entry service declaration      & (s info) \\
\hline
  \texttt{\%pw} & yes & Processor waiting time by task & entry service declaration      & (s info) \\
\hline
  \texttt{\%w}  & yes & Call waiting time              & entry call declaration         & (y info) \\
                & no  &                                & entry open arrival declaration & (a info) \\
\hline
  \texttt{\%x}  & yes & Max Service Time Exceeded      & entry service declaration      & (s info) \\
\hline
  \end{tabular}
  \caption{Observation Key location}
  \label{tab:obs-location}
\end{table}

For any key/location combination that takes a phase argument, if none
is supplied then the sum of the values for all phases is reported.
This also happens if a phase of zero is given.

\label{sec:spex-variables|)}

\subsection{Report Information}
\label{sec:lqn-report-information}\index{SPEX!report}

The purpose of the report information section of the input file is to specify which variable values
(including result indications) are to be printed in the SPEX result file.  The format of this section is
shown in Listing~\ref{lst:lqn-report-info} and consists of either a single plot
function\index{SPEX!report!plot}, or a list of variables (with possible computed results).  

\lstset{language=LQN,basicstyle=\ttfamily,numbersep=10pt,firstnumber=1}
\begin{lstlisting}[caption={Report Information},label=lst:lqn-report-info,frame=single,float]
R <int>
  plot( <var-list> )
  splot( <var-list> )
  $var = <ternary-expression>
  <expression>
-1
\end{lstlisting}

There may be any number of report declarations, however, the integer parameter to \texttt{R} must either be
the number of report declarations present or zero\footnote{The number is ignored; it is present in the
syntax so that the report section matches the other sections.}.

The \nonterminal{ternary-expression} may be any valid ternaray expression as discussed above.  The
\nonterminal{expression} can be any expression, including a lone variable or a function call.  Report
indication variables\index{variables!report indication} and the parameter
variables\index{variables!parameters} may both be used together, but may not be mixed with either
\texttt{plot()}\index{Spex!report!plot} or \texttt{splot()}\index{Spex!report!splot}.

The values of the variables listed in this
section are printed from left to right in the order that they appear in the input file separated by commas.
This output can then be used as input to Gnuplot\index{Gnuplot} or a spreadsheet such as
Excel\index{Excel}.  Output is normally sent to the terminal, but can be redirected using \flag{o}{}\texttt{output} \emph{filename}.

If the \texttt{plot()}\index{Spex!report!plot} or \texttt{splot()}\index{Spex!report!splot} function is used, output will formatted in such a way that it
can be used as input into gnuplot\index{gnuplot}.  The first argument to the \texttt{plot}\index{plot}
function call is the x\index{plot!x} variable for the graph.  The first two arguments to the \texttt{splot}\index{splot}
function call are the x\index{plot!x} and y\index{plot!y} variables for the graph.  Multiple additional arguments are permitted
but should be grouped by result type (ie., throughputs, utilizations).  When using
\texttt{plot}\index{plot}, up to two groups can be plotted with
results in first group being plotted against the left 'y'\index{plot!y} axis, and results from the second on
the right 'y' axis.  When using
\texttt{splot}\index{splot}, only one group of results can be used. Only one plot function can be specified
in the report section at this time.  The actual output is self-contained gnuplot\index{gnuplot} input.

There is a special variable called \texttt{\$0}\index{\$0} which represents the independent variable in the
results tables (the x-axis in plots).  The variable \texttt{\$0} may be set to any expression allowing for
flexibility in producing result tables.  This variable cannot be used as a parameter in the model.

\subsection{Convergence Information}
\label{sec:lqn-convergence-information}

Spex allows a parameter value to be modified at the end of a model solution and
then fed back in to the model.  The model is solved repeatedly until the parameter
value converges.  The convergence section is declared in a manner similar to the
result section:

\lstset{language=LQN,basicstyle=\ttfamily,numbersep=10pt,firstnumber=1}
\begin{lstlisting}[caption={Convergence Information},label=lst:lqn-convergence-info,frame=single]
C <int>
   $var = <ternary-expression>
-1
\end{lstlisting}

Convergence variables must be parameters.  They cannot be
result variables.

\subsection{Differeneces to SPEX 1}

This section outlines differences in the syntax between SPEX\index{SPEX!versions} version 1 and version 2.
SPEX version 1 was processed by a Perl\index{Perl!SPEX} program to convert the model into a conventional LQN
model file.  SPEX version 2 is now parsed directly and converted into LQX\index{SPEX!LQX} internally.

\subsubsection{Array Initialization}

Lists used for array initialization\index{SPEX!arrays} must now be enclosed within square brackets
(`[]')\index{[]}.  Further, the items must be separated using
commas. Figure~\ref{fig:spex-array-initialization} shows the old and new syntax.
\newsavebox{\spexone}
\newsavebox{\spextwo}
\begin{lrbox}{\spexone}
\begin{minipage}{.3\textwidth}
  \lstset{language=LQN,basicstyle=\ttfamily\color{red},numbersep=10pt,firstnumber=1}
\begin{lstlisting}
$array = 1 2 3
$array = 1:10,2
\end{lstlisting}
\end{minipage}
\end{lrbox}
\begin{lrbox}{\spextwo}
\begin{minipage}{.3\textwidth}
  \lstset{language=LQN,basicstyle=\ttfamily\color{blue},numbersep=10pt,firstnumber=1}
\begin{lstlisting}
$array = [1, 2, 3]
$array = [1:10,2]
\end{lstlisting}
\end{minipage}
\end{lrbox}
\begin{figure}
  \centering
  \subfloat[Spex 1]{\usebox{\spexone}}
  \subfloat[Spex 2]{\usebox{\spextwo}}
  \caption{x}
  \label{fig:spex-array-initialization}
\end{figure}

\subsubsection{Perl Expressions}

Perl Expressions are no longer supported in SPEX 2.0\index{SPEX!Perl}\index{Perl!SPEX}.  Rather, a subset of
LQX expressions are used instead.  For SPEX convergence
expressions\index{SPEX!convergence}\index{Convergence!SPEX}, Perl \texttt{if then else} statements must be
converted to use the ternary \texttt{?:}\index{?:}\index{SPEX!if-then-else}\index{SPEX!ternary expressions}
operator.  SPEX 2 cannot invoke Perl functions.

\subsubsection{Line Continuation}

Line continuation\index{line continuation}, where a line is terminated by a backslash (`\textbackslash')\index{\textbackslash}, is not needed
with Version 2 SPEX.  All whitespace, including newlines, is ignored.

\subsubsection{Comments}

In Version 1 of SPEX, all text before a dollar sign (`\$'), or either an upper case `G' or `P' at the start
of a line, was treated as a comment.  Since Version 2 SPEX is parsed directly, all comments must start with
the hash symbol (`\#')\index{\#}.

\subsubsection{String Substitution}

Version 2 SPEX does not support variable substitution of string parameters such as pragmas, and scheduling
types.  This restriction may be lifted in future versions.

\subsubsection{Pragmas}

Version 1 SPEX did not require the hash symbol (`\#')\index{\#} for setting pragmas.  Version 2 SPEX does.

\subsubsection{SPEX AGR}

SPEX AGR\index{SPEX!AGR} is no longer supported.

\subsubsection{Control Variables}
\label{sec:spex-obsolete-control}

Version 2 SPEX does not support the control variables shown in Table~\ref{tab:spex-obsolete-control}.
\begin{table}
  \centering
  \begin{tabular}{|l|}
\hline
  \multicolumn{1}{|c|}{\textbf{Control Variable}}\\
\hline
\ctrlparam{\$coefficient\_of\_variation}  \\
\ctrlparam{\$hosts}                       \\
\hline
  \end{tabular}
  \caption{Obsolete SPEX Control Variables.}
  \label{tab:spex-obsolete-control}
\end{table}

\subsubsection{Random Numbers}

Version 2 SPEX introduces the function \texttt{rand()}\index{rand@\texttt{rand()}}\index{SPEX!random
  numbers} to generate random numbers in the range of $[0..1)$.  To generate a set of experiments with
random parameters, an array (or set of arrays) with the number of elements corresponding to the number of
experiments is required to cause SPEX to iterate (see Section~\ref{sec:spex-and-lqx}).
Listing~\ref{lst:spex-rand} shows the syntax to generate random values for ten experiments.

\lstset{language=LQN,numbersep=10pt,firstnumber=1,texcl}
\begin{lstlisting}[float,caption={SPEX random parameter generation},label=lst:spex-rand]
  $experiments = [1:10,1]             # 10 experiments.
  $experiments, $parameter1 = rand()  # \$experiments is ignored.
  $experiments, $parameter2 = rand()  # \$experiments is ignored.
\end{lstlisting}

\index{SPEX|)}

\subsection{SPEX and LQX}
\label{sec:spex-and-lqx}

SPEX uses LQX\index{LQX} to generate individual model files.  All scalar paramaters are treated as globally
scoped variables in LQX and can be used to set parameters in the model.  If the assignement expression for a
scalar variable does not reference any array variables, it is set prior to the iteration of any loop.
Otherwise, the scalar variable is set during the execution of the innermost loop of the program.

Array variables are used to generate \texttt{foreach} loops in the LQX program.  The variable defining the
array is local (i.e.\ without the `\$') with the name of the SPEX parameter.  Each array variable generates
a for loop; the loops are nested in the order of the definition of the array variable.  The value variable
for the \texttt{foreach} loop is global (i.e.\ with the `\$') with the name of the SPEX parameter and can be
used as a parameter in the model.

If SPEX convergence is used, a final innermost loop is created.  This loop tests the variables defined in
the convergence section for change, and if any of the variables changes by a non-trivial amount, the loop
repeats.

Listing~\ref{lst:spex-file-layout} shows a model defined defined using SPEX syntax.
Listing~\ref{lst:spex-lqx} shows the corresponding LQX program generated by the model file.  

\lstset{language=LQN,basicstyle=\ttfamily,numbersep=10pt,firstnumber=1}
\begin{lstlisting}[float,caption={SPEX file layout.},label=lst:spex-file-layout]
$m_client = [1, 2, 3]
$m_server = [1: 3, 1]
$s_server = $m_server / 2

P 2
  p client i
  p server s 0.1
-1

T 2
  t client r client -1 client m $m_client %f $f_client
  t server n server -1 server m $m_server %u $u_server
-1

E 2
  s client 1 -1
  y client server $s_server -1
  s server 1 -1
-1

R 3
  $0
  $f_client
  $u_server
-1
\end{lstlisting}

\lstset{language=LQX,basicstyle=\ttfamily,numbersep=10pt,firstnumber=1,texcl=false}
\begin{lstlisting}[float,caption={LQX Program for SPEX input.},label=lst:spex-lqx]
m_client = array_create(1, 2, 3);
_0 = 0;
_f_client = 0;
_u_server = 0;
println_spaced(", ", "$0", "$f_client", "$u_server");
foreach( $m_client in m_client ) { 
  for ( $m_server = 1; $m_server <= 3; $m_server = ($m_server + 1)) {
    $s_server = ($m_server / 2);
    _0 = (_0 + 1);
    if (solve()) {
      _f_client = task("client").throughput;
      _u_server = task("server").utilization;
      println_spaced(", ", _0, _f_client, _u_server);
    } else {
      println("solver failed: $0=", _0);
    }
  }
}
\end{lstlisting}




%%% Local Variables:
%%% mode: latex
%%% mode: outline-minor
%%% fill-column: 108
%%% TeX-master: "userman"
%%% End:

%%  -*- mode: latex; mode: outline-minor; fill-column: 108 -*- 
%% Title:  lqns
%%
%% $HeadURL: http://rads-svn.sce.carleton.ca:8080/svn/lqn/trunk/doc/userman/lqns.tex $
%% Original Author:     Greg Franks <greg@sce.carleton.ca>
%% Created:             12 March 2013
%%
%% ----------------------------------------------------------------------
%% $Id: lqns.tex 12964 2017-03-27 19:17:20Z greg $
%% ----------------------------------------------------------------------

\chapter{Invoking the Analytic Solver ``lqns''}
\label{sec:invoking-lqns}
The Layered Queueing Network Solver (LQNS)\index{LQNS} is used to
solved Layered Queueing Network models analytically.
\textbf{Lqns} reads its input from \texttt{filename}, specified at the
command line if present, or from the standard input otherwise.  By
default, output for an input file \texttt{filename} specified on the
command line will be placed in the file \texttt{filename.out}.  If the
\flag{p}{} switch is used, parseable output will also be written into
\texttt{filename.p}. If XML input\index{input!XML} or the \flag{x}{} switch is used, XML output\index{output!XML} will be written to 
\texttt{filename.lqxo}.  This behaviour can be changed using the
\flag{o}{}\texttt{output} switch, described below.  If several files are
named, then each is treated as a separate model and output will be
placed in separate output files.  If input is from the standard input,
output will be directed to the standard output.  The file name `\texttt{-}' is
used to specify standard input.


The \flag{o}{}\texttt{output} option can be used to direct output to the file
\texttt{output} regardless of the source of input.  Output will be XML\index{XML}\index{output!XML}
if XML input\index{XML!input} or if the \flag{x}{} switch is used, parseable output if the \flag{p}{} switch is used,
and normal output otherwise.  Multiple input files cannot be specified
when using this option.  Output can be directed to standard output by
using \flag{o}{}\texttt{-} (i.e., the output file name is `\texttt{-}'.)
\section{Command Line Options}
\label{sec:options}
\begin{description}
\item[\flag{a}{}, \longopt{ignore-advisories}]~\\
Ignore advisories.  The default is to print out all advisories.\index{advisory!ignore}
\item[\flag{b}{}, \longopt{bounds-only}]~\\
This option is used to compute the ``Type 1 throughput bounds''\index{throughput!bounds}\index{bounds!throughput} only.
These bounds are computed assuming no contention anywhere in the model
and represent the guaranteed not to exceed values.
\item[\flag{d}{}, \longopt{debug}=\emph{arg}]~\\
This option is used to enable debug output.\index{debug}
\emph{Arg} can be one of:
\begin{description}
\item[\optarg{activities}{}]
Activities -- not functional.
\item[\optarg{all}{}]
Enable all debug output.
\item[\optarg{calls}{}]
Print out the number of rendezvous between all tasks.
\item[\optarg{forks}{}]
Print out the fork-join matching process.
\item[\optarg{interlock}{}]
Print out the interlocking table and the interlocking between all tasks and processors.
\item[\optarg{joins}{}]
Joins -- not functional.
\item[\optarg{layers}{}]
Print out the contents of all of the layers found in the model.
\item[\optarg{lqx}{}]
Debug LQX parser.
\item[\optarg{overtaking}{}]
Overtaking -- not functional.
\item[\optarg{quorum}{}]
Print out results from pseudo activities used by quorum.
\item[\optarg{xml}{}]
Debug XML.
\end{description}
\item[\flag{e}{}, \longopt{error}=\emph{arg}]~\\
This option is to enable floating point exception handling.\index{floating point!exception}
\emph{Arg} must be one of the following:
\begin{enumerate}
\item \textbf{a}
Abort immediately on a floating point error (provided the floating point unit can do so).
\item \textbf{d}
Abort on floating point errors. (default)
\item \textbf{i}
Ignore floating point errors.
\item \textbf{w}
Warn on floating point errors.
\end{enumerate}
The solver checks for floating point overflow,\index{overflow} division by zero and invalid operations.
Underflow and inexact result exceptions are always ignored.


In some instances, infinities \index{infinity}\index{floating point!infinity} will be propogated within the solver.  Please refer to the
\textbf{stop-on-message-loss} pragma below.
\item[\flag{f}{}, \longopt{fast}]~\\
This option is used to set options for quick solution of a model using One-Step (Bard-Schweitzer) MVA.
It is equivalent to setting \textbf{pragma} \emph{mva}=\emph{one-step}, \emph{layering}=\emph{batched}, \emph{multiserver}=\emph{conway}
\item[\flag{H}{}, \longopt{help}=\emph{arg}]~\\
\item[\flag{I}{}, \longopt{input-format}=\emph{arg}]~\\
This option is used to force the input file format to either \emph{xml} or \emph{lqn}.
By default, if the suffix of the input filename is one of: \emph{.in}, \emph{.lqn} or \emph{.xlqn}
, then the LQN parser will be used.  Otherwise, input is assumed to be XML.
\item[\flag{n}{}, \longopt{no-execute}]~\\
Read input, but do not solve.  The input is checked for validity.  
No output is generated.
\item[\flag{o}{}, \longopt{output}=\emph{arg}]~\\
Direct analysis results to \emph{output}\index{output}.  A filename of `\texttt{-}\index{standard input}'
directs output to standard output.  If \texttt{output}is a directory, all output is saved in \texttt{output/input.out}. If the input model contains a SPEX program with loops, the SPEX output is sent to \texttt{output}; the individual model output files are found in the directory \texttt{output.d}. If \textbf{lqns} is invoked with this
option, only one input file can be specified.
\item[\flag{p}{}, \longopt{parseable}]~\\
Generate parseable output suitable as input to other programs such as
\textbf{lqn2ps(1)} and \textbf{srvndiff(1)}.  If input is from
\texttt{filename}, parseable output is directed to \texttt{filename.p}.
If standard input is used for input, then the parseable output is sent
to the standard output device.  If the \flag{o}{}\texttt{output} option is used, the
parseable output is sent to the file name \texttt{output}.
(In this case, only parseable output is emitted.)
\item[\flag{P}{}, \longopt{pragma}=\emph{arg}]~\\
Change the default solution strategy.  Refer to the PRAGMAS section\index{pragma}
below for more information.
\item[\flag{r}{}, \longopt{rtf}]~\\
Output results using Rich Text Format instead of plain text.  Processors, entries and tasks with high utilizations are coloured in red.
\item[\flag{t}{}, \longopt{trace}=\emph{arg}]~\\
This option is used to set tracing \index{tracing} options which are used to print out various
intermediate results \index{results!intermediate} while a model is being solved.
\emph{arg} can be any combination of the following:
\begin{description}
\item[\optarg{activities}{}]
Print out results of activity aggregation.
\item[\optarg{convergence}{=\emph{arg}}]
Print out convergence arg after each submodel is solved.  This option is useful for tracking the rate of convergence for a model.
The optional numeric argument supplied to this option will print out the convergence value for the specified MVA submodel, otherwise,
the convergence value for all submodels will be printed.
\item[\optarg{delta\_wait}{}]
Print out difference in entry service time after each submodel is solved.
\item[\optarg{forks}{}]
Print out overlap table for forks prior to submodel solution.
\item[\optarg{idle\_time}{}]
Print out computed idle time after each submodel is solved.
\item[\optarg{interlock}{}]
Print out interlocking adjustment before each submodel is solved.
\item[\optarg{joins}{}]
Print out computed join delay and join overlap table prior to submodel solution.
\item[\optarg{mva}{=\emph{arg}}]
Print out the MVA submodel and its solution. A numeric argument supplied to this option will print out only the specified MVA submodel, otherwise, all submodels will be printed.
\item[\optarg{overtaking}{}]
Print out overtaking calculations.
\item[\optarg{print}{}]
Print out intermediate solutions at the print interval specified in the model.  The print interval field in the input is ignored otherwise.
\item[\optarg{quorum}{}]
Print quorum traces.
\item[\optarg{throughput}{}]
Print throughput's values.
\item[\optarg{variance}{}]
Print out the variances calculated after each submodel is solved.
\item[\optarg{wait}{}]
Print waiting time for each rendezvous in the model after it has been computed.
\end{description}
\item[\flag{v}{}, \longopt{verbose}]~\\
Generate output after each iteration of the MVA solver and the convergence value at the end of each outer iteration of the solver.
\item[\flag{V}{}, \longopt{version}]~\\
Print out version and copyright information.\index{version}\index{copyright}
\item[\flag{w}{}, \longopt{no-warnings}]~\\
Ignore warnings.  The default is to print out all warnings.\index{warning!ignore}
\item[\flag{x}{}, \longopt{xml}]~\\
Generate XML output regardless of input format.
\item[\flag{z}{}, \longopt{special}=\emph{arg}]~\\
This option is used to select special options.  Arguments of the form
\emph{nn} are integers while arguments of the form \emph{nn.n} are real
numbers.  \emph{Arg} can be any of the following:
\begin{description}
\item[\optarg{convergence-value}{=\emph{arg}}]
Set the convergence value to \emph{arg}.  
\emph{Arg} must be a number between 0.0 and 1.0.
\item[\optarg{full-reinitialize}{}]
For multiple runs, reinitialize all processors.
\item[\optarg{generate}{=\emph{arg}}]
This option is used for debugging the solver.  A directory named \emph{arg} will be created containing source code for invoking the MVA solver directly.
\item[\optarg{ignore-overhanging-threads}{}]
Ignore the effect of the overhanging threads.
\item[\optarg{iteration-limit}{=\emph{arg}}]
Set the maximum number of iterations to \emph{arg}.
\emph{Arg} must be an integer greater than 0.  The default value is 50.
\item[\optarg{man}{=\emph{arg}}]
Output this manual page.  If an optional \emph{arg}
is supplied, output will be written to the file named \emph{arg}.
Otherwise, output is sent to stdout.
\item[\optarg{min-steps}{=\emph{arg}}]
Force the solver to iterate min-steps times.
\item[\optarg{mol-ms-underrelaxation}{=\emph{arg}}]
Set the under-relaxation factor to \emph{arg} for the MOL multiserver approximation.
\emph{Arg} must be a number between 0.0 and 1.0.
The default value is 0.5.\item[\optarg{overtaking}{}]
Print out overtaking probabilities.\item[\optarg{print-interval}{=\emph{arg}}]
Set the printing interval to \emph{arg}.  The
\flag{d}{} or \flag{v}{} options must also be selected to display intermediate results.
The default value is 10.
\item[\optarg{single-step}{}]
Stop after each MVA submodel is solved.  Any character typed at the terminal except end-of-file will resume the calculation.  End-of-file will cancel single-stepping altogether.
\item[\optarg{skip-layer}{=\emph{arg}}]
Ignore submodel \emph{arg} during solution.
\item[\optarg{tex}{=\emph{arg}}]
Output this manual page in LaTeX format.   If an optional \emph{arg}
is supplied, output will be written to the file named \emph{arg}.
Otherwise, output is sent to stdout.
\item[\optarg{underrelaxation}{=\emph{arg}}]
Set the underrelaxation to \emph{arg}.
\emph{Arg} must be a number between 0.0 and 1.0.
The default value is 0.9.
\end{description}
If any one of \emph{convergence}, \emph{iteration-limit}, or\emph{print-interval} are used as arguments, the corresponding 
value specified in the input file for general information, `G', is
ignored.  
\item[\flag{c}{}, \longopt{convergence}=\emph{arg}]~\\
Set the convergence value to \emph{arg}.  
\emph{Arg} must be a number between 0.0 and 1.0.
\index{convergence!value}\item[\flag{e}{}, \longopt{exact-mva}]~\\
Use Exact MVA to solve all submodels.\index{MVA!exact}
\item[\longopt{hwsw-layering}]~\\
\item[\flag{i}{}, \longopt{iteration-limit}=\emph{arg}]~\\
Set the maximum number of iterations to \emph{arg}.
\emph{Arg} must be an integer greater than 0.  The default value is 50.
\item[\longopt{srvn-layering}]~\\
Solve the model using submodels containing exactly one server.
\item[\longopt{squashed-layering}]~\\
Use only one submodel to solve the model.
\item[\flag{m}{}, \longopt{method-of-layers}]~\\
This option is to use the Method Of Layers solution approach to solving the layer submodels.
\item[\longopt{processor-sharing}]~\\
Use Processor Sharing scheduling at all fixed-rate processors.
\item[\flag{s}{}, \longopt{schweitzer-amva}]~\\
Use Bard-Schweitzer approximate MVA to solve all submodels.\index{MVA!Bard-Schweitzer}
\item[\flag{o}{}, \longopt{no-stop-on-message-loss}]~\\
Do not stop the solver on overflow (infinities) for open arrivals or send-no-reply messages to entries.  The default is to stop with an
error message indicating that the arrival rate is too high for the service time of the entry
\item[\longopt{trace-mva}]~\\
Output the inputs and results of each MVA submodel for every iteration of the solver.
\item[\flag{u}{}, \longopt{underrelaxation}=\emph{arg}]~\\
Set the underrelaxation to \emph{arg}.
\emph{Arg} must be a number between 0.0 and 1.0.
The default value is 0.9.
\item[\longopt{no-variance}]~\\
Do not use variances in the waiting time calculations.
\item[\longopt{reload-lqx}]~\\
Re-run the LQX\index{LQX} program without re-solving the models.  Results must exist from a previous solution run.
This option is useful if LQX print statements are changed.
\item[\longopt{no-header}]~\\
\item[\longopt{debug-lqx}]~\\
Output debugging information as an LQX\index{LQX!debug} program is being parsed.
\item[\longopt{debug-xml}]~\\
Output XML\index{XML!debug} elements and attributes as they are being parsed.   Since the XML parser usually stops when it encounters an error,
this option can be used to localize the error.
\item[\longopt{debug-srvn}]~\\
\end{description}


\textbf{Lqns} exits\index{exit!success} with 0 on success, 1 if the model failed to converge,\index{convergence!failure}
2 if the input was invalid\index{input!invalid}, 4 if a command line argument was\index{command line!incorrect}
incorrect, 8 for file read/write problems and -1 for fatal errors\index{error!fatal}.  If
multiple input files are being processed, the exit code is the
bit-wise OR of the above conditions.
\section{Pragmas}
\label{sec:lqns-pragmas}
\emph{Pragmas}\index{pragma} are used to alter the behaviour of the solver in a
variety of ways.  They can be specified in the input file with
``\#pragma'', on the command line with the \flag{P}{} option, or through
the environment variable \index{environment variable}\emph{LQNS\_PRAGMAS}\index{LQNS\_PRAGMAS@\texttt{LQNS\_PRAGMAS}}.  Command line
specification of pragmas overrides those defined in the environment
variable which in turn override those defined in the input file.  The
following pragmas are supported.  Invalid pragma\index{pragma!invalid} specification at the
command line will stop the solver.  Invalid pragmas defined in the
environment variable or in the input file are ignored as they might be
used by other solvers.
\begin{description}
\item[\optarg{cycles}{=\emph{arg}}]~\\
This pragma is used to enable or disable cycle detection\index{cycle!detection} in the call
graph.\index{call graph}  Cycles may indicate the presence of deadlocks.\index{deadlock}
\emph{Arg} must be one of: 
\begin{description}
\item[\optarg{allow}{}]
Allow cycles in the call graph.  The interlock\index{interlock} adjustment is disabled.
\item[\optarg{disallow}{}]
Disallow cycles in the call graph.
\end{description}
The default is disallow.
\item[\optarg{interlocking}{=\emph{arg}}]~\\
The interlocking\index{interlock} is used to correct the throughputs\index{throughput!interlock} at stations as a
result of solving the model using layers~\cite{perf:franks-95-ipds-interlock}.  This pragma is used to
choose the algorithm used.
\emph{Arg} must be one of: 
\begin{description}
\item[\optarg{none}{}]
Do not perform interlock adjustment.
\item[\optarg{throughput}{}]
Perform interlocking by adjusting throughputs.
\end{description}
The default is throughput.
\item[\optarg{layering}{=\emph{arg}}]~\\
This pragma is used to select the layering strategy\index{layering!strategy} used by the solver.
\emph{Arg} must be one of: 
\begin{description}
\item[\optarg{batched}{}]
Batched layering\index{batched layers}\index{layering!batched} -- solve layers composed of as many servers as possible from top to bottom.
\item[\optarg{batched-back}{}]
Batched layering with back propagation -- solve layers composed of as many servers as possible from top to bottom, then from bottom to top to improve solution speed.
\item[\optarg{hwsw}{}]
Hardware/software layers\index{hardware-software layers}\index{layers!hardware-software} -- The model is solved using two submodels:
One consisting solely of the tasks in the model, and the other with the tasks calling the processors.
\item[\optarg{mol}{}]
Method Of layers\index{method of layers}\index{layering!method of} -- solve layers using the Method of Layers~\cite{perf:rolia-95-ieeese-mol}\index{Method of Layers}\index{layering!Method of Layers}. Layer spanning is performed by allowing clients to appear in more than one layer.
\item[\optarg{mol-back}{}]
Method Of layers -- solve layers using the Method of Layers.  Software submodels are solved top-down then bottom up to improve solution speed.
\item[\optarg{squashed}{}]
Squashed layers\index{squashed layers}\index{layering!squashed} -- All the tasks and processors are placed into one submodel.
Solution speed may suffer because this method generates the most number of chains in the MVA solution.  See also \flag{P}{}\emph{mva}.
\item[\optarg{srvn}{}]
SRVN layers\index{srvn layers}\index{layering!srvn} -- solve layers composed of only one server.
This method of solution is comparable to the technique used by the \textbf{srvn} solver.  See also \flag{P}{}\emph{mva}.
\end{description}
The default is batched-back.
\item[\optarg{multiserver}{=\emph{arg}}]~\\
This pragma is used to choose the algorithm for solving multiservers\index{multiserver!algorithm}.
\emph{Arg} must be one of: 
\begin{description}
\item[\optarg{bruell}{}]
Use the Bruell multiserver\index{multiserver!Bruell}~\cite{queue:bruell-84-peva-load-dependent} calculation for all multiservers.
\item[\optarg{conway}{}]
Use the Conway multiserver\index{multiserver!Conway}~\cite{queue:deSouzaeSilva-87,queue:conway-88} calculation for all multiservers.
\item[\optarg{reiser}{}]
Use the Reiser multiserver\index{multiserver!Reiser}~\cite{queue:reiser-79} calculation for all multiservers.
\item[\optarg{reiser-ps}{}]
Use the Reiser multiserver calculation for all multiservers. For multiservers with multiple entries, scheduling is processor sharing\index{processor!sharing}, not FIFO. 
\item[\optarg{rolia}{}]
Use the Rolia\index{multiserver!Rolia}~\cite{perf:rolia-92,perf:rolia-95-ieeese-mol} multiserver calculation for all multiservers.
\item[\optarg{rolia-ps}{}]
Use the Rolia multiserver calculation for all multiservers. For multiservers with multiple entries, scheduling is processor sharing\index{processor!sharing}, not FIFO. 
\item[\optarg{schmidt}{}]
Use the Schmidt multiserver\index{multiserver!Schmidt}~\cite{queue:schmidt-97} calculation for all multiservers.
\item[\optarg{suri}{}]
experimental.
\end{description}
The default multiserver\index{multiserver!default} calculation uses the the Conway multiserver for multiservers with less than five servers, and the Rolia multiserver otherwise.

\item[\optarg{mva}{=\emph{arg}}]~\\
This pragma is used to choose the MVA\index{MVA} algorithm used to solve the submodels.
\emph{Arg} must be one of: 
\begin{description}
\item[\optarg{exact}{}]
Exact MVA\index{MVA!exact}.  Not suitable for large systems.
\item[\optarg{fast}{}]
Fast Linearizer
\item[\optarg{linearizer}{}]
Linearizer.\index{MVA!Linearizer}\index{Linearizer}
\item[\optarg{one-step}{}]
Perform one step of Bard Schweitzer approximate MVA for each iteration of a submodel.  The default is to perform Bard Schweitzer approximate MVA until convergence for each submodel.  This option, combined with \flag{P}{}\emph{layering=srvn} most closely approximates the solution technique used by the \textbf{srvn} solver.
\item[\optarg{one-step-linearizer}{}]
Perform one step of Linearizer approximate MVA for each iteration of a submodel.  The default is to perform Linearizer approximate MVA until convergence for each submodel.
\item[\optarg{schweitzer}{}]
Bard-Schweitzer approximate MVA.\index{MVA!Bard-Schweitzer}\index{Bard-Schweitzer}
\end{description}
The default is linearizer.
\item[\optarg{overtaking}{=\emph{arg}}]~\\
This pragma is usesd to choose the overtaking\index{overtaking} approximation.
\emph{Arg} must be one of: 
\begin{description}
\item[\optarg{markov}{}]
Markov phase 2 calculation.\index{overtaking!Markov}
\item[\optarg{none}{}]
Disable all second phase servers.  All stations are modeled as having a single phase by summing the phase information.
\item[\optarg{rolia}{}]
Use the method from the Method of Layers.\index{overtaking!Method of Layers}
\item[\optarg{simple}{}]
Simpler, but faster approximation.
\item[\optarg{special}{}]
?
\end{description}
The default is rolia.
\item[\optarg{processor}{=\emph{arg}}]~\\
Force the scheduling type\index{scheduling!processor}\index{processor!scheduling} of all uni-processors to the type specfied.
\begin{description}
\item[\optarg{fcfs}{}]
All uni-processors are scheduled first-come, first-served.
\item[\optarg{hol}{}]
All uni-processors are scheduled using head-of-line priority.\index{priority!head of line}
\item[\optarg{ppr}{}]
All uni-processors are scheduled using priority, pre-emptive resume.\index{priority!preemptive-resume}
\item[\optarg{ps}{}]
All uni-processors are scheduled using processor sharing.\index{processor sharing}
\end{description}
The default is to use the processor scheduling specified in the model.

\item[\optarg{severity-level}{=\emph{arg}}]~\\
This pragma is used to enable or disable warning messages.
\begin{description}
\item[\optarg{advisory}{}]
\item[\optarg{all}{}]
\item[\optarg{run-time}{}]
\item[\optarg{warning}{}]
\end{description}
The default is all.
\item[\optarg{stop-on-message-loss}{=\emph{arg}}]~\\
This pragma is used to control the operation of the solver when the
arrival rate\index{arrival rate} exceeds the service rate of a server.
\emph{Arg} must be one of: 
\begin{description}
\item[\optarg{false}{}]
Ignore queue overflows\index{overflow} for open arrivals\index{open arrival!overflow} and send-no-reply\index{send-no-reply!overflow} requests.  If a queue overflows, its waiting times is reported as infinite.\index{infinity}\item[\optarg{true}{}]
Stop if messages are lost.
\end{description}
The default is false.
\item[\optarg{tau}{=\emph{arg}}]~\\
Set the tau adjustment factor to \emph{arg}.
\emph{Arg} must be an integer between 0 and 25.
A value of \emph{zero} disables the adjustment.
\item[\optarg{threads}{=\emph{arg}}]~\\
This pragma is used to change the behaviour of the solver when solving
models with fork-join\index{fork}\index{join} interactions.
\begin{description}
\item[\optarg{exponential}{}]
Use exponential values instead of three-point approximations in all approximations\index{three-point approximation}~\cite{perf:jiang-96}.
\item[\optarg{hyper}{}]
Inflate overlap probabilities based on arrival instant estimates.
\item[\optarg{mak}{}]
Use Mak-Lundstrom\index{Mak-Lundstrom}~\cite{perf:mak-90} approximations for join delays.\index{join!delay}
\item[\optarg{none}{}]
Do not perform overlap calculation for forks.\index{overlap calculation}
\end{description}
The default is hyper.
\item[\optarg{variance}{=\emph{arg}}]~\\
This pragma is used to choose the variance\index{variance} calculation used by the solver.
\begin{description}
\item[\optarg{init-only}{}]
Initialize the variances, but don't recompute as the model is solved.\index{variance!initialize only}
\item[\optarg{mol}{}]
Use the MOL variance calculation.\index{variance!Method of Layers}\index{Method of Layers!variance}
\item[\optarg{no-entry}{}]
By default, any task with more than one entry will use the variance calculation.  This pragma will switch off the variance calculation for tasks with only one entry.
\item[\optarg{none}{}]
Disable variance adjustment.  All stations in the MVA submodels are either delay- or FIFO-servers.
\item[\optarg{stochastic}{}]
?
\end{description}

\end{description}
\section{Stopping Criteria}
\label{sec:lqns-stopping-criteria}
\textbf{Lqns} computes the model results by iterating through a set of
submodels until either convergence\index{convergence} is achieved, or the iteration limit\index{iteration limit|textbf}
is hit. Convergence is determined by taking the root of the mean of
the squares of the difference in the utilization of all of the servers
from the last two iterations of the MVA solver over the all of the
submodels then comparing the result to the convergence value specified
in the input file. If the RMS change in utilization is less than
convergence value\index{convergence!value|textbf}, then the results are considered valid.


If the model fails to converge,\index{convergence!failure} three options are available:
\begin{enumerate}
\item reduce the under-relaxation coefficient. Waiting and idle times are
propogated between submodels during each iteration. The
under-relaxation coefficient determines the amount a service time is
changed between each iteration. A typical value is 0.7 - 0.9; reducing
it to 0.1 may help.
\item increase the iteration limit.\index{iteration limit} The iteration limit sets the upper bound
on the number of times all of the submodels are solved. This value may
have to be increased, especially if the under-relaxation coefficient
is small, or if the model is deeply nested. The default value is 50
iterations.
\item increase the convergence test value\index{convergence!value}. Note that the convergence value
is the standard deviation in the change in the utilization of the
servers, so a value greater than 1.0 makes no sense.
\end{enumerate}


The convergence value can be observed using \flag{t}{}\emph{convergence} flag.
\section{Model Limits}
\label{sec:model-limits}
The following table lists the acceptable parameter types for
\textbf{lqns}.  An error will
be reported if an unsupported parameter is supplied except when the
value supplied is the same as the default.


%%--------------------------------------------------------------------
%% Table Begin
%%--------------------------------------------------------------------
\begin{table}[htbp]
  \centering
  \begin{tabular}[c]{ll}
    Parameter&lqns \\
    \hline
    Phases\index{phase!maximum} & 3\\
    Scheduling\index{scheduling} & FIFO, HOL, PRI\index{scheduling!fifo}\index{scheduling!hol}\index{scheduling!pri}\\
    Open arrivals\index{open arrival} & yes\\
    Phase type\index{phase!type} & stochastic, deterministic\\
    Think Time\index{think time} & yes\\
    Coefficient of variation\index{coefficient of variation} & yes\\
    Interprocessor-delay\index{interprocessor delay} & yes\\
    Asynchronous connections\index{asynchronous connections} & yes\\
    Forwarding\index{forwarding} & yes\\
    Multi-servers\index{multi-server} & yes\\
    Infinite-servers\index{infinite server} & yes\\
    Max Entries\index{entry!maximum} & 1000\\
    Max Tasks\index{task!maximum} & 1000\\
    Max Processors\index{processor!maximum} & 1000\\
    Max Entries per Task & 1000\\
    \hline
  \end{tabular}
  \caption{\label{tab:lqns-model-limits}LQNS Model Limits\index{limits!lqns}.}
\end{table}
\section{Diagnostics}
\label{sec:lqns-diagnostics}
Most diagnostic messages result from errors in the input file.
If the solver reports errors, then no solution will be generated for
the model being solved.  Models which generate warnings may not be
correct.  However, the solver will generate output.


Sometimes the model fails to converge\index{convergence!failure}, particularly if there are several
heavily utilized servers in a submodel.  Sometimes, this problem can
be solved by reducing the value of the under-relaxation coefficient.  It
may also be necessary to increase the iteration-limit\index{iteration limit}, particularly if
there are many submodels.  With replicated models, it may be necessary
to use `srvn' layering to get the model to converge.  Convergence can be tracked
using the \flag{t}{}\emph{convergence} option.


The solver will sometimes report some servers with `high' utilization.
This problem is the result of some of the approximations used, in particular, two-phase servers.
Utilizations in excess of 10\% are likely the result of failures in the solver.
Please send us the model file so that we can improve the algorithms.
%%% Local Variables: 
%%% mode: latex
%%% mode: outline-minor 
%%% fill-column: 108
%%% TeX-master: "userman"
%%% End: 

%%  -*- mode: latex; mode: outline-minor; fill-column: 108 -*-
%% Title:  lqsim
%%
%% $HeadURL: http://rads-svn.sce.carleton.ca:8080/svn/lqn/trunk/doc/userman/lqsim.tex $
%% Original Author:     Greg Franks <greg@sce.carleton.ca>
%% Created:             Tue Jul 18 2006
%%
%%
%% ----------------------------------------------------------------------
%% $Id: lqsim.tex 13831 2020-09-18 12:51:41Z greg $
%% ----------------------------------------------------------------------

\chapter{Invoking the Simulator ``lqsim''}
\label{sec:invoking-lqsim}

Lqsim is used to simulate layered queueing networks using the
PARASOL\index{Parasol}~\cite{perf:neilson-91b} simulation system.
Lqsim reads its input from files specified at the command line if
present, or from the standard input otherwise.  By default, output for
an input file \texttt{filename} specified on the command line will be
placed in the file \texttt{filename.out}.  If the \flag{p}{} switch is
used, parseable output\index{output!parseable} will also be written
into \texttt{filename.p}. If XML input\index{input!XML} is used,
results will be written back to the original input file.  This
behaviour can be changed using the \flag{o}{output} switch, described
below.  If several files are named, then each is treated as a separate
model and output will be placed in separate output files.  If input is
from the standard input, output will be directed to the standard
output.  The file name `\texttt{-}' is used to specify standard input.

The \flag{o}{output} option can be used to direct output to the file or directory named \emph{output}
regardless of the source of input.  Output will be XML\index{XML}\index{output!XML} if XML
input\index{input!XML} is used, parseable output if the \flag{p}{} switch is used, and normal output
otherwise; multiple input files cannot be specified.  If \emph{output} is a directory, results will be
written in the directory named \texttt{output}.  Output can be directed to standard output by using
\texttt{-o-} (i.e., the output file name is `\texttt{-}'.)

\section{Command Line Options}
\label{sec:lqsim-options}\index{command line}

\begin{description}
\item[\flag{A}{}, \longopt{automatic}=\emph{run-time[,precision[,skip]]}]~\\
  Use automatic
  blocking\index{automatic blocking}\index{block!automatic} with a
  simulation block
  size\index{block!size}\index{simulation!block}\index{block!simulation}
  of \emph{run-time}\index{run time!simulation}.  The
  \emph{precision}\index{precision!simulation} argument specifies the
  desired mean 95\% confidence level\index{confidence level}.  By
  default, precision is 1.0\%.  The simulator will stop when this
  value is reached, or when 30 blocks have run.
  \emph{Skip}\index{skip period} specifies the time value of the
  initial skip period.
  Statistics\index{statistics}\index{simulation!statistics} gathered
  during the skip period are discarded.  By default, its value is 0.
  When the run completes, the results reported will be the average
  value of the data collected in all of the blocks.  If the \flag{R}{}
  flag is used, the confidence intervals\index{confidence intervals}
  will for the raw statistics will be included in the monitor
  file\index{file!monitor}\index{monitor file}.
\item[\flag{B}{}, \longopt{blocks}=\emph{blocks[,run-time[,skip]]}]~\\
  Use manual blocking\index{manual blocking}\index{block!manual} with
  \emph{blocks} blocks.  The value of \emph{blocks} must be less than
  or equal to 30.  The run time for each block is specified with
  \emph{run-time}\index{run time!simulation}.  \emph{Skip}\index{skip}
  specifies the time value of the initial skip period.
\item[\flag{C}{}, \longopt{confidence}=\emph{precision[,initial-loops[,run-time]]}]~\\
  Use automatic
  blocking\index{automatic blocking}\index{block!automatic}, stopping
  when the specified precision\index{precision!simulation} is met.
  The run time of each block is estimated, based on
  \emph{initial-loops}\index{initial-loops} running on each
  reference\index{task!reference}\index{reference task} task.  The
  default value for \emph{initial-loops} is 500.  The \emph{run-time}
  argument specifies the maximum total run time.
\item[\flag{d}{}, \longopt{debug}]~\\ This option is used to dump task and entry
  information showing internal index numbers.  This option is useful
  for determining the names of the servers and tasks when tracing the
  execution of the simulator since the Parasol output routines do no
  emit this information at present.  Output is directed to stdout
  unless redirected using \flag{m}{file}.
\item[\flag{e}, \longopt{error}=\emph{error}]~\\
  This option is to enable floating point
  exception\index{floating point!exception} handling.
  \begin{description}
  \item[a] Abort immediately on a floating point error (provided the
    floating point unit can do so).
  \item[b] Abort on floating point errors. (default)
  \item[i] Ignore floating point errors.
  \item[w] Warn on floating point errors.
  \end{description}
  The solver checks for floating point overflow\index{overflow},
  division by zero and invalid operations.  Underflow and inexact
  result exceptions are always ignored.
  
  In some instances,
  infinities\index{infinity}\index{floating point!infinity} will be
  propogated within the solver.  Please refer to the
  \pragma{stop-on-message-loss} pragma below.
\item[\flag{h}{output}]~\\
  Generate comma separated values for the service time distribution\index{service time!distribution}
  data\index{distribution!service time}.  If \emph{output} is a directory, the output file name will be the
  name of a the input file with a \texttt{.csv}\index{output!csv} extension.  Otherwise, the output will be
  written to the named file.
\item[\flag{m}{file}]~\\
  Direct all output generated by the various
  debugging\index{file!debug} and tracing\index{file!tracing} options
  to the monitor\index{file!monitor} file \emph{file}, rather than to
  standard output.  A filename of `\texttt{-}' directs output to
  standard output.
\item[\flag{n}{}, \longopt{no-execute}]~\\
  Read input, but do not solve.  The input is checked for validity. No
  output is generated.
\item[\flag{o}{}, \longopt{output}=\emph{output}]~\\
  Direct analysis results to output\index{output}.  A file name of `\texttt{-}' directs output to standard
  output.  If \emph{output} is a directory, all output from the simulator will be placed there with
  filenames based on the name of the input files processed.  Otherwise, only one input file can be
  processed; its output will be placed in \emph{output}.
\item[\flag{p}{}, \longopt{parseable}]~\\
  Generate parseable output\index{output!parseable} suitable as input
  to other programs such as \manpage{MultiSRVN}{1} and
  \manpage{srvndiff}{1}.  If input is from \texttt{filename},
  parseable output is directed to \texttt{filename.p}.  If standard
  input\index{standard input} is used for input, then the parseable
  output is sent to the standard output device.  If the
  \flag{o}{output} option is used, the parseable output is sent to the
  file name output.  (In this case, only parseable output is emitted.)
\item[\flag{P}, \longopt{pragma}=\emph{pragma}]~\\
  Change the default solution strategy.  Refer to the PRAGMAS chapter
  (\S\ref{sec:lqsim-pragmas})\index{pragma} below for more information.
\item[\flag{R}{}, \longopt{raw-statistics}]~\\
  Print the values of the statistical
  counters\index{counters!statistical}\index{statistical counters} to
  the monitor file\index{file!monitor}.  If the \flag{A}{}, \flag{B}{}
  or \flag{C}{} option was used, the mean value, 95th and 99th
  percentile are reported.  At present, statistics are gathered for
  the task\index{cycle-time!task} and entry\index{cycle-time!entry},
  cycle time task\index{utilization!task},
  processor\index{utilization!processor} and
  entry\index{utilization!entry} utilization, and waiting
  time\index{waiting time} for messages.
\item[\flag{S}{}, \longopt{seed}=\emph{seed}]~\\
  Set the initial seed\index{seed} value for the random
  number\index{random number!generation} generator.  By default, the
  system time from time \manpage{time}{3} is used.  The same seed
  value is used to initialize the random number generator for each
  file when multiple input files are specified.
\item[\flag{t}{}, \longopt{trace}=\emph{traceopts}]~\\
  This option is used to set tracing\index{tracing} options which are
  used to print out various steps of the simulation while it is
  executing.  \emph{Traceopts} is any combination of the following:
  \begin{description}
  \item[\optarg{driver}{}] Print out the underlying tracing
    information from the Parasol\index{Parasol} simulation engine.
  \item[\optarg{processor}{=regex}] Trace activity for
    processors\index{processor!trace}\index{trace!processor} whose
    name match \emph{regex}.  If \emph{regex}is not specified,
    activity on all processors is reported.  \emph{Regex} is regular
    expression of the type accepted by \manpage{egrep}{1}.
  \item[\optarg{task}{=regex}] Trace activity for
    tasks\index{task!trace}\index{trace!task} whose name match
    \emph{regex}.  If \emph{regex} is not specified, activity on all
    tasks is reported.  pattern is regular expression of the type
    accepted by \manpage{egrep}{1}.
  \item[\optarg{events}{regex[:regex]}] Display only events matching
    pattern.  The events are: msg-async, msg-send, msg-receive,
    msg-reply, msg-done, msg-abort, msg-forward, worker-dispatch,
    worker- idle, task-created, task-ready, task-running,
    task-computing, task-waiting, thread-start, thread-enqueue,
    thread-dequeue, thread-idle, thread-create, thread-reap,
    thread-stop, activity-start, activity-execute, activity-fork, and
    activity-join.
  \item[\optarg{msgbuf}{}] Show msgbuf allocation and deallocation.
  \item[\optarg{timeline}{}] Generate events for the timeline tool.
  \end{description}
\item[\flag{T}{}, \longopt{run-time}=\emph{run-time}]~\\
  Set the run time\index{run time!simulation} for the simulation.  The
  default is 10,000 units.  Specifying \flag{T}{} after either
  \flag{A}{} or \flag{B}{} changes the simulation block size, but does
  not turn off blocked statistics
  collection\index{simulation!block}\index{statistics!simulation}.
\item[\flag{v}{}, \longopt{verbose}]~\\
  Print out statistics\index{solution!statistics}\index{statistics}
  about the solution on the standard output device.
\item[\flag{V}{}, \longopt{version}]~\\
  Print out version\index{version} and copyright\index{copyright}
  information.
\item[\flag{w}{}, \longopt{no-warnings}]~\\
Ignore warnings.  The default is to print out all warnings.\index{warning!ignore}
\item[\flag{x}{}, \longopt{xml}]~\\
Generate XML output regardless of input format.
\item[\flag{z}{specialopts}]~\\
  This flag is used to select special options.  Arguments of the form
  $n$ are integers while arguments of the form $n.n$ are real
  numbers.  \emph{Specialopts} is any combination of the following:
  \begin{description}
  \item[\optarg{print-interval}{=nn}] Set the printing interval to
    $n$.  Results are printed after $nn$ blocks have run.  The default
    value is 10.
  \item[\optarg{global-delay}{=n.n}] Set the interprocessor
    delay\index{delay!interprocessor} to nn.n for all tasks.  Delays
    specified in the input file will override the global value.
  \end{description}
\item[\longopt{global-delay}]~\\
Set the inter-processor communication delay to n.n.
\item[\longopt{print-interval}]~\\
Ouptut results after n iterations.
\item[\longopt{restart}]~\\
Re-run the LQX\index{LQX} program without re-solving the models unless a valid solution does not exist.  
This option is useful if LQX print statements are changed, or if a subset of simulations has to be re-run.
\item[\longopt{debug-lqx}]~\\
Output debugging informtion as an LQX\index{LQX!debug} program is being parsed.
\item[\longopt{debug-xml}]~\\
Output XML\index{XML!debug} elements and attributes as they are being parsed.   Since the XML parser usually stops when it encounters an error,
this option can be used to localize the error.
\end{description}

\section{Return Status}
\label{sec:lqsim-return-status}

Lqsim exits 0 on success\index{exit!success}, 1 if the simulation
failed to meet the convergence criteria\index{convergence!failure}, 2
if the input was invalid\index{input!invalid}, 4 if a command line
argument was incorrect\index{command line!incorrect}, 8 for file
read/write problems and -1 for fatal errors\index{error!fatal}.  If
multiple input files\index{input!multiple} are being processed, the
exit code is the bit-wise OR of the above conditions.

\section{Pragmas}
\label{sec:lqsim-pragmas}

Pragmas\index{pragma} are used to alter the behaviour of the simulator
in a variety of ways.  They can be specified in the input file with
``\#pragma'', on the command line with the \flag{P}{} option, or
through the environment variable\index{environment variable}
\texttt{LQSIM\_PRAGMAS}\index{LQSIM\_PRAGMAS@\texttt{LQSIM\_PRAGMAS}}.
Command line\index{command line} specification of pragmas overrides
those defined in the environment variable\index{environment
  variable!override} which in turn override those defined in the input
file.

The following pragmas are supported.  An invalid pragma
specification\index{pragma!invalid!command line} at the command line
will stop the solver.  Invalid pragmas defined in the environment
variable or in the input\index{pragma!invalid!input file} file are
ignored as they might be used by other solvers.
\begin{description}
% \item[\optarg{messages}{\emph{=n}}]~\\
%   Set the number of message buffers\index{message!buffers} to $n$.
%   The default is 1000.

\item[\optarg{block-period}{=real}]~\\
  Set the block period to \emph{real}.  This value is used in conjuction with \emph{max-blocks} or
  \emph{precision}.
\item[\optarg{initial-delay}{=real}]~\\
  Set the initial warmup period to \emph{real}.
\item[\optarg{initial-loops}{=real}]~\\
  Run reference tasks \emph{int} times before recording data.
\item[\optarg{max-blocks}{=int}]~\\
  Set the maximum number of blocks to \emph{int}.  \emph{Int} must be no more than 30.
\item[\optarg{precision}{=real}]~\\
  Set the precision of the simulation results, based on the confidence intervals of the utilizations of all
  of the tasks and processors, to \emph{real}.
\item[\optarg{run-time}{=real}]~\\
  Set the run-time of the simulations to \emph{real}.  If used by itself, the simulation will use one block
  and not report confidence intervals.
\item[\optarg{seed-value}{=int}]~\\
  Set the seed for the random number generator to \emph{int}
\item[\optarg{nice}{=int}]~\\
  Set the ``nice'' value (i.e, lower the priority) when runninng the simulation.
\item[\optarg{severity-level}{=enum}]~\\
  Suppress messages with a severity-level lower than \emph{enum}. \emph{Enum} is any one of
  the following:
  \begin{description}
  \item[all] Show all messages.
  \item[warning-only] Suppress warnings.
  \item[advisory] Suppress warnings and advisorys.
  \item[runtime-error] Suppress runtime errors, warnings and advisorys.
  \end{description}
  The default is to report all messages.
\item[\optarg{scheduling-model}{=enum}]~\\
  This pragma is used to select the scheduler used for
  processors\index{processor!scheduling}.  \emph{Enum} is any one of
  the following:
  \begin{description}
  \item[default] Use the scheduler built into parasol for processor
    scheduling.  (faster)
  \item[custom] Use the custom
    scheduler\index{processor!scheduling!custom} for scheduling which
    permits more statistics to be gathered about processor
    utilization\index{utilization!processor}\index{processor!utilization}
    and waiting
    times\index{waiting!processor}\index{processor!waiting}.  However,
    this option invokes more internal tasks, so simulations are slower
    than when using the default scheduler.
  \item[default-natural] Use the parasol scheduler, but don't
    reschedule\index{processor!scheduling!natural} after the end of
    each phase\index{reschedule!phase}\index{phase!reschedule} or
    activity\index{reschedule!activity}\index{activity!reschedule}.
    This action more closely resembles the scheduling of real
    applications.
  \item[custom-natural] Use the custom scheduler; don't reschedule
    after the end of each phase or activity.
  \end{description}
\item[\optarg{reschedule-on-async-send}{\emph{=bool}}]~\\
  In models with send-no-reply\index{send-no-reply} messages, the
  simulator does not reschedule the processor after an asynchronous
  message is sent (unlike the case with synchronous messages).  The
  meanings of \emph{bool} are:
  \begin{description}
  \item[true] reschedule after each asynchronous message.
  \item[false] reschedule after each asynchronous message.
  \end{description}
\item[\optarg{stop-on-message-loss}{\emph{=bool}}] ~\\
  This pragma is used to control the operation of the solver when the
  arrival rate\index{arrival rate} exceeds the service rate of a
  server.  The simulator can either discard the arrival, or it can
  halt.  The meanings of \emph{bool} are:
  \begin{description}
  \item[false] Ignore queue overflows for open
    arrivals\index{open arrival!overflow} and
    send-no-reply\index{send-no-reply!overflow} requests.  If a queue
    overflows\index{overflow}, its waiting times is reported as
    infinite\index{infinity}.
  \item[true] Stop if messages are lost.
  \end{description}
\end{description}

\section{Stopping Criteria}
\label{sec:stopping-criteria}\index{stopping criteria}

It is important that the length of the
simulation\index{length!simulation} be chosen properly.  Results may
be inaccurate if the simulation run is too short.  Simulations that
run too long waste time and resources.

Lqsim uses \emph{batch means}\index{batch means} (or the method of
samples\index{method of samples}) to generate confidence
intervals\index{confidence intervals}.  With automatic
blocking\index{automatic blocking}\index{block!automatic}, the
confidence intervals are computed after the simulations runs for three
blocks plus the initial skip period\index{skip period} If the root or
the mean of the squares\index{root mean square} of the confidence
intervals for the entry service times\index{service time!entry} is
within the specified precision, the simulation stops.  Otherwise, the
simulation runs for another block and repeats the test.  With manual
blocking\index{block!manual}, lqsim runs the number of blocks
specified then stops.  In either case, the simulator will stop after
30 blocks.

Confidence intervals can be tightened by either running additional
blocks or by increasing the block size\index{block!size}.  A rule of
thumb is the block size should be 10,000 times larger than the largest
service time demand\index{service time!demand} in the input model.

\section{Model Limits}
\label{sec:lqsim-model-limits}

The following table lists the acceptable parameter types and limits
for lqsim.  An error will be reported if an unsupported parameter is
supplied except when the value supplied is the same as the default.

\begin{table}[htbp]
  \centering
  \begin{tabular}[c]{ll}
    Parameter&lqsim \\
    \hline
    Phases\index{phase!maximum} & 3 \\
    Scheduling\index{scheduling} & FIFO, HOL, PRI, RAND\index{scheduling!fifo}\index{scheduling!hol}\index{scheduling!pri}\index{scheduling!rand} \\
    Open arrivals\index{open arrival} & yes \\
    Phase type\index{phase!type} & stochastic, deterministic \\
    Think Time\index{think time}  & yes  \\
    Coefficient of variation\index{coefficient of variation} & yes \\
    Interprocessor-delay\index{interprocessor delay} & yes \\
    Asynchronous connections\index{asynchronous connections} & yes \\
    Forwarding\index{forwarding} & yes \\
    Multi-servers\index{multi-server} & yes \\
    Infinite-servers\index{infinite server} & yes \\
    Max Entries\index{entry!maximum} & unlimited \\
    Max Tasks\index{task!maximum} & 1000 \\
    Max Processors\index{processor!maximum} & 1000 \\
    Max Entries per Task & unlimited \\
  \end{tabular}
  \caption{\label{tab:lqsim-model-limits}Lqsim Model Limits\index{limits!lqsim}}
\end{table}

%%% Local Variables: 
%%% mode: latex
%%% mode: outline-minor 
%%% fill-column: 108
%%% TeX-master: "userman"
%%% End: 

%% -*- mode: latex; mode: outline-minor: t; fill-column: 108 -*-
%% Title:  errors
%%
%% $HeadURL: http://rads-svn.sce.carleton.ca:8080/svn/lqn/trunk/doc/userman/errors.tex $
%% Original Author:     Greg Franks <greg@sce.carleton.ca>
%% Created:             Tue Jul 18 2006
%%
%% ----------------------------------------------------------------------
%% $Id: errors.tex 15611 2022-05-31 12:24:50Z greg $
%% ----------------------------------------------------------------------

\chapter{Error Messages}
\label{sec:error-messages}

Error messages are classified into four categories ranging from the
most severe to the least, they are: fatal, error, advisory and
warning.  Fatal errors will cause the program to exit immediately.
All other error messages will stop the solution of the current model
and suppress output generation.  However, subsequent input files will
be processed.  Advisory messages occur when the model has been solved,
but the results may not be correct.  Finally, warnings indicate
possible problems with the model which the solver has ignored.

\section{Fatal Error Messages}
\label{sec:fatal-error-messages}
\begin{itemize}
% /* FTL_INTERNAL_ERROR                  1 */
\item \texttt{Internal error}

  Something bad happened...

% /* FTL_NO_MEMORY                       2 */

\item \texttt{No more memory}

  A request for memory failed.  

% /* ERR_NO_OBJECTS                      3 */

\item \texttt{Model has no }\emph{(activity$|$entry$|$task$|$processor)}

  This should not happen.
  
% /* FTL_ACTIVITY_STACK_FULL           60 */
\item \texttt{Activity stack for "}\emph{identifier}\texttt{" is
    full.}
  
  The stack size\indexerror{stack size} limit for task
  \emph{identifier} has been exceeded.

% /* FTL_MSG_POOL_EMPTY                61 */
\item \texttt{Message pool is empty.  Sending from
    "}\emph{identifier}\texttt{" to "}\emph{identifier}\texttt{".}
  
  Message buffers are used when sending asynchronous send-no-reply
  messages\indexerror{send-no-reply}.  All the buffers have been
  used\indexerror{message!pool}.
\end{itemize}

\section{Error Messages}
\label{sec:normal-error-messages}

\begin{itemize}
% /* ERR_REPLICATION                   62 */
\item \emph{(task$|$processor)}\texttt{ "}\emph{identifier}\texttt{":
    Replication not supported.}\marginpar{lqsim}
  
  The simulator\index{replication!simulator} does not support
  replication\indexerror{replication}.  The model can be
  ``flattened''\index{replication!flatten} using
  \manpage{rep2flat}{1}.

% /* ERR_NON_UNITY_REPLIES               4 */
\item \emph{n.n} \texttt{Replies generated by Entry
    "}\emph{identifier}\texttt{".}
  
  This error occurs when an entry is supposed to generate a reply
  because it accepts rendezvous\index{rendezvous} requests, but the
  activity graph does not generate exactly one
  reply\indexerror{reply}.  Common causes of this error are replies
  being generated by two or more branches of an
  AND-fork\indexerror{AND-fork!reply}, or replies being generated as
  part of a LOOP\indexerror{LOOP!reply}\footnote{Replies cannot be
    generated by branches of loops because the number of iterations of
    the loop is random, not deterministic}.

% /* ERR_IS_START_ACTIVITY               5 */
\item \texttt{Activity "}\emph{identifier}\texttt{" is a start activity.}
  
  The activity named \emph{identifier} is the first activity in an
  activity graph\indexerror{start activity}.  It cannot be used in a
  \emph{post-}precedence\indexerror{post-precedence}
  (\emph{fork-}list\indexerror{fork-list}).

% /* ERR_DUPLICATE_ACTIVITY_RVALUE       6 */
\item \texttt{Activity "}\emph{identifier}\texttt{" previously used in a fork." }
  
  The activity \emph{identifier} has already been used as part of a
  fork expression\indexerror{fork-list}.  Fork lists are on the right
  hand side of the \texttt{->}\index{->@\texttt{->}} operator in the old
  grammar, and are the
  \emph{post}-precedence\indexerror{post-precedence} expressions in
  the XML grammar.  This error will cause a loop in the activity
  graph.

% /* ERR_DUPLICATE_ACTIVITY_LVALUE       7 */
\item \texttt{Activity "}\emph{identifier}\texttt{" previously used in a join." }
  
  The activity \emph{identifier} has already been used as part of a
  join list\indexerror{join-list}.  Join lists are on the left hand
  side of the \texttt{->}\index{\texttt{->}} operator in the old
  grammar, and are the
  \emph{pre}-precedence\indexerror{pre-precedence} expressions in the
  XML grammar.  This error will cause a loop in the activity graph.

% /* ERR_REPLY_NOT_FOUND               63 */
\item \texttt{Activity "}\emph{identifier}\texttt{" requests reply for
    entry "}\emph{identifier}\texttt{" but none pending.}\marginpar{lqsim}
  
  The simulator is trying to generate a reply\indexerror{reply} from
  entry \emph{identifier}, but there are no messages queued at the
  entry.  This error usually means there is a logic error in the
  simulator.

% /* ERR_LQX_COMPILATION                    */
\item \texttt{An error occured while compiling the LQX program found in file: }\emph{filename}\texttt`{.}\marginpar{lqx}
  
  A syntax error was found in the LQX\indexerror{LQX} program found in the file \emph{filename}.  Refer to
  earlier error messages.

% /* ERR_LQX_EXECUTION                      */
\item \texttt{An error occured  executing the LQX program found in file: }\emph{filename}\texttt{.}\marginpar{lqx}
  
  A error occured while executing the the LQX\indexerror{LQX} program found in the file \emph{filename}.
  Refer to earlier error messages.

% /* ERR_ATTRIBUTE_MISSING               8 */
\item \texttt{Attribute "}\emph{attribute}\texttt{" is missing from "}\emph{type}\texttt{" element.}

  The attribute named \emph{attribute} for the \texttt{type}-element
  is missing\indexerror{attribute!missing}.
  
%  ---- SCHEMA
\item \texttt{Attribute '}\emph{attribute}\texttt{' is not declared for element '}\emph{element}\texttt{'}
  
  The attribute named \emph{attribute} for \emph{element} is not defined in the
  schema.\indexerror{attribute!not declared}.  

% /* ERR_LQX_SPEX			    */
\item \texttt{"Both LQX and SPEX found in file } \emph{filename}
  \texttt{.  Use one or the other."}

  XML input allows for the use of LQX or SPEX, but not both at the same time.\indexerror{lqx!spex}\indexerror{spex!lqx}
  
% /* ERR_CANNOT_CREATE_X               64 */
\item \texttt{Cannot create }\emph{(processor$|$processor for task$|$task})
  \texttt{"}\emph{identifier}\texttt{".}\marginpar{lqsim}
  
  Parasol\indexerror{Parasol} could not create an object such as a
  task\indexerror{task creation} or
  processor\indexerror{processor!creation}.

% /* ERR_CYCLE_IN_ACTIVITY_GRAPH         9 */
\item \texttt{Cycle in activity graph for task
    "}\emph{identifier}\texttt{", back trace is
    "}\emph{list}\texttt{".}
  
  There is a cycle in the activity graph for the task named
  \emph{identifier}\indexerror{cycle!activity graph}.  Activity graphs
  must be acyclic.  \emph{List} identifies the activities found in the
  cycle.

% /* ERR_CYCLE_IN_CALL_GRAPH            10 */
\item \texttt{Cycle in call graph,  backtrace is "}\emph{list}\texttt{".}
  
  There is a cycle in the call graph\indexerror{cycle!call graph}
  indicating either a possible deadlock\index{deadlock} or livelock
  condition.  A deadlock can occur if the same task, but via a
  different entry, is called in the cycle of
  rendezvous\index{rendezvous!cycle} indentified by \emph{list}.  A
  livelock\index{livelock} can occur if the same task and entry are
  found in the cycle.

  In general, call graphs must be acyclic.  If a deadlock condition is
  identified, the \pragma{cycles=allow} pragma can be used to suppress
  the error.  Livelock conditions cannot be suppressed as these
  indicate an infinite loop in the call graph\index{infinite loop!call graph}.

% /* ERR_TOO_MANY_PHASES                11 */
\item \texttt{Data for }\emph{n}\texttt{ phases specified.  Maximum permitted is }\emph{m}\texttt{.}
  
  The solver only supports \emph{m} phases (typically
  3)\indexerror{maximum phases}; data for \emph{n} phases was
  specified.  If more than \emph{m} phases need to be specified, use
  activities to define the phases.

% --- Schema --- 
\item \texttt{Datatype error: Type:InvalidDatatypeValueException,
    Message:}\emph{message}

% /* ERR_DELAY_MULTIPLY_DEFINED        65 */
\item \texttt{Delay from processor "}\emph{identifier}\texttt{" to
    processor "}\emph{identifier}\texttt{" previously specified.}\marginpar{lqsim}

  Inter-processor delay...\typeout{Bonne mots: inter-processor delay}

% /* ERR_BOGUS_COPIES                  61 */
\item \texttt{Derived population of }\emph{n.n}\texttt{ for task
    "}\emph{identifier}\texttt{" is not valid."}\marginpar{lqns}
  
  The solver finds populations for the
  clients\indexerror{population!infinite} in a given submodel by
  traversing up the call graphs from all the servers in the submodel.
  If the derived population is infinite\index{submodel!population},
  the submodel cannot be solved.  This error usually arises when open
  arrivals\indexerror{open arrival} are accepted by infinite
  servers\index{server!infinite}\indexerror{infinite server}.

% /* ERR_SRC_EQUALS_DST                 12 */
\item \texttt{Destination entry "}\emph{dst-identifier}\texttt{" must
    be different from source entry "}\emph{src-identifier}\texttt{".}
  
  This error occurs when \emph{src-identifier} and
  \emph{dst-identifier} specify the same
  entry.\indexerror{entry!different}

% /* ERR_NON_INTEGRAL_CALLS             13 */
\item \texttt{Deterministic phase "}\emph{src-identifier}\texttt{"
    makes a non-integral number of calls (}\emph{n.n}\texttt{) to
    entry }\emph{dst-identifier}\texttt{.}
  
  This error occurs when a deterministic
  phase\indexerror{phase!deterministic} or activity makes a
  non-integral number of calls to some other entry.

% --- SCHEMA_DUPLICATE_UNIQUE_VALUE --- 
\item \texttt{Duplicate unique value declared for identity constraint
    of element '}\emph{task}\texttt{'.}
  
  One or more activities are being bound to the same
  entry\indexerror{duplicate!unique value}.  This is not allowed, as
  an entry is only allowed to be bound to one activity.  Check all
  \attribute{bound-to-entry} attributes for all activities to ensure
  this constraint is being met.

% --- SCHEMA --- 
\item \texttt{Duplicate unique value declared for identity constraint
    of element '}\emph{lqn-model}\texttt{'.}
  
  This error indicated that an element has a duplicate
  name\indexerror{element!duplicate name}\indexerror{duplicate!unique value}
  -- the parser gives the line number to the start of the second
  instance of duplicate element.  The following elements must have
  unique name attributes\index{attribute!unique name}, but the
  uniqueness does not span elements.  Therefore a processor and task
  element can have the same name attribute, but two processor elements
  cannot have the same name attribute.
  
  The following elements must have a unique \attribute{name}
  attribute:
  \begin{itemize}
  \item processor
  \item task
  \item entry
  \end{itemize}

% Value '}\emph{value}\texttt{' does not match any member types (of the union).
% Value 'fred' is not in enumeration .

\item \texttt{Empty content not valid for content
    model:'}\emph{element}\texttt{'}

(result-processor,task)

% /* ERR_OPEN_AND_CLOSED_CLASSES        14 */
\item \texttt{Entry "}\emph{identifier}\texttt{" accepts both
    rendezvous and send-no-reply messages.}

  An entry can either accept synchronous messages (to which it
  generates replies), or asynchronous messages (to which no reply is
  needed), but not both.  Send the requests to two separate
  entries.\indexerror{entry!message type}

% /* ERR_INVALID_FORWARDING_PROBABILITY 15 */
\item \texttt{Entry "}\emph{identifier}\texttt{" has invalid
    forwarding probability of }\emph{n.n}\texttt{.}
  
  This error occurs when the sum of all forwarding
  probabilities\indexerror{forwarding!probability} from the entry
  \emph{identifier} is greater than 1\index{probability!forwarding}.

% /* ERR_WRONG_TASK_FOR_ENTRY           16 */
\item \texttt{Entry "}\emph{entry-identifier}\texttt{"  is not part of task
    "}\emph{task-identifier}\texttt{".}
  
  An activity graph part of task \emph{task-identifer}
  replies\indexerror{activity!reply} to \emph{entry-identifier}.
  However, \emph{entry-identifier} belongs to another task.

% /* ERR_ENTRY_NOT_SPECIFIED            17 */
\item \texttt{Entry "}\emph{identifier}\texttt{" is not specified.}
  
  An entry is declared but not defined, either using phases or
  activities\indexerror{entry}.  An entry is
  ``defined''\index{entry!defined} when some parameter such as service
  time is specified.

% /* ERR_REPLY_NOT_GENERATED            18 */
\item \texttt{Entry "}\emph{identifier}\texttt{" must reply; the reply
    is not specified in the activity graph.}
  
  The entry \emph{identifier} accepts rendezvous
  requests\indexerror{activity!reply}.  However, no reply is specified
  in the activity graph.

% /* ERR_MIXED_ENTRY_TYPES              19 */
\item \texttt{Entry "}\emph{identifier}\texttt{" specified using both
    activity and phase methods.}
  
  Entries can be specified either using phases\index{entry!phase}, or
  using activities\index{entry!activity}, but not
  both.\indexerror{entry!type}.

% /* ERR_MIXED_SPECIAL_ENTRY_TYPES              19 */
\item \texttt{Entry "}\emph{identifier}\texttt{" specified as both a signal and wait.}
  
  A semaphore task\index{task!semaphore}\index{semaphore task} must have exactly one
  signal\index{entry!signal}\index{signal} and one wait\index{entry!wait}\index{wait} entry.  Both entries
  have the same type.\indexerror{entry!type}.

% --- SCHEMA ----
% Expected end of tag 'synch-call'

\item \texttt{Expected end of tag '}\emph{element}\texttt{'}

  The closing tag\indexerror{tag!end} for \emph{element} was not found in the input file.

% /* ERR_EXTERNAL_SYNC                 62 */
\item \texttt{External synchronization not supported for task
    "}\emph{identifier}\texttt{" at join
    "}\emph{join-list}\texttt{".}\marginpar{lqns}
  
  The analytic solver does not implement external synchronization\indexerror{synchronization}.

% /* ERR_LQX_VARIABLE_RESOLUTION            */
\item \texttt{External variables are present in file "}\emph{filename}\texttt{, but there is no LQX program
    to resolve them.}\marginpar{lqx}
  
  The input model contains a variable\indexerror{external variable} of the form ``\texttt{\$var}'' as a
  parameter such as a service time, multiplicty, or rate.  The variables are only assigned values when an
  LQX program executes.  Since no LQX program\indexerror{LQX!execution} was present in the model file, the
  model cannot be solved.

% /* ERR_FANIN_MISMATCH                63 */
\item \texttt{Fan-ins from task "}\emph{from-identifier}\texttt{" to
    task "}\emph{to-identifier}\texttt{" are not identical for all
    calls.}\marginpar{lqns}
  
  All requests\indexerror{fan-in} made from task
  \emph{from-identifier} to task \emph{to-identifier} must have the
  same fan-in and fan-out values\indexerror{fan-out}.

% /* ERR_REPLICATION                   68 */
\item \texttt{Fan-out from }\emph{(activity$|$entry$|$task)}\texttt{
    "}\emph{src-identifier}\texttt{" (}\emph{n}\texttt{ *
  }\emph{n}\texttt{ replicas) does not match fan-in to
  }\emph{(entry$|$processor)}\texttt{ "}\emph{dst-identifier}\texttt{"
    (}\emph{n}\texttt{ * }\emph{n}\texttt{).}\marginpar{lqns}
  
  This error occurs when the number of replicas\indexerror{replication}
  at \emph{src-identifier} multiplied by the
  fan-out\indexerror{fan-out} for the request to \emph{dst-identifier}
  does not match the number of replicas at \emph{dst-identifier}
  multiplied by the fan-in\indexerror{fan-in} for the request from
  \emph{src-identifier}. A fan-in or fan-out of zero (a common error
  case) can arise when the ratios of tasks to processors is
  non-integral\index{replication!ratio}.

% /* ERR_LESS_ENTRIES_THAN_TASKS        20 */
\item \texttt{Fewer entries defined (}\emph{n}\texttt{) than tasks
    (}\emph{m}\texttt{).}
  
  A model was specified with more tasks than entries.  Since each task
  must have at least one entry, this model is
  invalid\indexerror{model}.

% /* ERR_NO_TASK_DEFINED_FOR_GROUP        */
\item \texttt{Group "}\emph{identifier}\texttt{" has no tasks.}
  \indexerror{group!tasks}
  
  The group named by \emph{identifier} has no tasks assigned to it.  A group requires a minimum of one task.

% /* ERR_INVALID_SHARE                    */ 
\item \texttt{Group "}\emph{identifier}\texttt{" has invalid share of }\emph{n.n}\texttt{.}
  \indexerror{group!share}

  The share\index{group!share} value of \emph{n.n} for group \emph{identifier} is not between the range of $
  0 < n.n <= 1.0$.

% /* ERR_INFINITE_THROUGHPUT           64 */
\item \texttt{Infinite throughput for task
    "}\emph{identifier}\texttt{".  Model specification
    error.}\marginpar{lqns}
  
  The response time\indexerror{response time} for the task
  \emph{identifier} is zero\indexerror{throughput!infinite}.  The
  likely cause is zero service time for all calls made by the task.

% /* ERR_INIT_DELAY                    66 */
\item \texttt{Initial delay of }\emph{n.n}\texttt{ is too small,
  }\emph{n}\texttt{ client(s) still running.}\marginpar{lqsim}
  
  This error occurs when the \emph{initial-loops}\index{initial-loops}
  parameter for automatic
  blocking\index{automatic blocking}\index{block!automatic} is too
  small.

% /* ERR_INVALID_FANIN                 65 */
\item \texttt{Invalid fan-in of }\emph{n}\texttt{: source task
    "}\emph{identifier}\texttt{" is not replicated.}\marginpar{lqns}
  
  The fan-in\indexerror{fan-in} value for a request specifies the
  number of replicated\indexerror{replication} source tasks making a
  call to the destination.  To correct this error, the source task
  needs to be replicated by a multiple of $n$.

% /* ERR_INVALID_FANOUT                66 */
\item \texttt{Invalid fan-out of }\emph{n}\texttt{: destination task
    "}\emph{identifier}\texttt{" has only }\emph{m}\texttt{
    replicas.}\marginpar{lqns}
  
  The fan-out\indexerror{fan-out} value $n$ is larger than the number
  of destination tasks $m$.  In effect, the source will have more than
  one request arc to the destination.

% /* ERR_JOIN_BAD_PATH                  21 */
\item \texttt{Invalid path to join "}\emph{join-list}\texttt{" for
    task "}\emph{identifier}\texttt{": backtrace is
    "}\emph{list}\texttt{".}
  
  The activity graph\indexerror{activity graph} for task
  \emph{identifer} is invalid because the branches to the
  join\indexerror{join} \emph{join-list} do not all originate from the
  same fork\indexerror{fork}.  \emph{List} is a dump of the activity
  stack when the error occurred.

% /* ERR_INVALID_PROBABILITY            22 */
\item \texttt{Invalid probability of }\emph{n.n}\texttt{.}
  
  The probability\indexerror{probability} of \emph{n.n} is not between
  the range of zero to one inclusive.  The likely cause for this error
  is the sum of the
  probabilities\index{branch!probability}\index{forwarding!probability}\index{probability!branch}\index{probability!forwarding}
  either from an OR-fork\index{OR-fork}, or from
  forwarding\index{forwarding} from an entry, is greater than one.

% /* ERR_MULTIPLY_DEFINED               23 */
\item \texttt{Name "}\emph{identifier}\texttt{" previously defined.}
  
  The symbol
  \emph{identifer}\indexerror{duplicate!identifier}\index{identifier!duplicate}
  was previously defined.  Tasks, processors and entries must all be
  named uniquely.  Activities must be named uniquely within a task.

% /* ERR_NO_REFERENCE_TASKS             24 */
\item \texttt{Model has no reference tasks.}
  
  There are no reference tasks\indexerror{reference task} nor are there
  any tasks with open arrivals\indexerror{open arrival} specified in
  the model.  Reference tasks serve as customers for closed queueing
  models\index{queueing model!customers}\index{queueing model!closed}.
  Open-arrivals serve as sources for open queueing
  models\index{queueing model!open}.

% /* ERR_NO_CALLS_TO_ENTRY             67 */
\item \texttt{No calls from }\emph{(entry$|$activity)}\texttt{
    "}\emph{from-identifier}\texttt{" to entry
    "}\emph{to-identifier}\texttt{".}\marginpar{lqns}
  
  This error occurs when the fan-in\indexerror{fan-in} or
  fan-out\indexerror{fan-out} parameter for a request are specified
  \emph{before} the actual request type.  Switch the order in the
  input file.

% /* ERR_NO_QUANTUM_SCHEDULING            */
\item \texttt{No group specified for task "}\emph{task\_identifier}\texttt{" running on processor
    "}\emph{proc\_identifier}\texttt{" using fair share scheduling.}

  Task \emph{task\_identifier} has no group specified, yet it is running on processor
  \emph{proc\_identifier} which is using completely fair scheduling\indexerror{scheduling!completely fair}.

% /* ERR_NO_SEMAPHORE                 26 */
\item \texttt{No signal or wait specified for semaphore task "}\emph{identifier}\texttt{".}
  
  Task \emph{identifier} has been identified as a semaphore\indexerror{semaphore task}
  task\index{task!semaphore}, but neither of its entries has been designated as a
  signal\index{entry!signal}\index{signal} or a wait\index{entry!wait}\index{wait}.

% /* ERR_NON_REF_THINK_TIME             25 */
\item \texttt{Non-reference task "}\emph{identifier}\texttt{" cannot
    have think time.}
  
  A think time\indexerror{think time} is specified for a non-reference
  task.  Think times for non-reference tasks can only be specified by
  entry\index{think time!entry}.

% /* ERR_NOT_SEMAPHORE_TASK */
\item \texttt{Non-semaphore task "}\emph{identifer}\texttt{" cannot have a }\emph{(signal|wait)}\texttt{ for
    entry "}\emph{entry}\texttt{".}

  The \emph{entry} is designated as either a signal\index{entry!signal}\index{signal} or a
  wait\index{entry!wait}\index{wait}.  However, \emph{identifier} is not a semaphore
  task\index{task!semaphore}\indexerror{semaphore task}. 

% /* ERR_TOO_MANY_X                     26 */
\item \texttt{Number of }\emph{(entries$|$tasks$|$processors)}\texttt{
    is outside of program limits of (1,}\emph{n}\texttt{).}

  An internal program limit has been exceeded.  Reduce the number of
  objects in the model\indexerror{program limit}.

% /* ERR_JOIN_PATH_MISMATCH             27 */
\item \texttt{Number of paths found to AND-Join
    "}\emph{join-list}\texttt{" for task "}\emph{identifier}\texttt{"
    does not match join list." }
  
  During activity graph traversal, one or more of the
  branches\index{branch!AND}\indexerror{join} to the join
  \emph{join-list} either originate from different forks, or do not
  originate from a fork at all\indexerror{fork}.

% /* ERR_ARRIVAL_RATE                  60 */
\item \texttt{Open arrival rate of }\emph{n.n}\texttt{ to task
    "}\emph{identifier}\texttt{" is too high.  Service time is
  }\emph{n.n}\texttt{.}\marginpar{lqns}
  
  The open arrival rate of $n.n$ to entry
  \emph{identifier}\indexerror{open arrival} is too high, so the input
  queue to the task has overflowed.  This error may be the result of a
  transient condition, so the \pragma{stop-on-message-loss} pragma
  (c.f. \S\ref{sec:lqns-pragmas}) may be used to suppress this error.
  If the arrival rate exceeds the service time at the time the model
  converges, then the waiting time results for the entry will show
  infinity\index{infinity}.  Note that if a task accepts both open and
  closed classes\index{class!open}\index{class!closed}, an overflow in
  the open class will result in zero throughput\index{throughput!zero}
  for the closed classes.

% /* ERR_MISSING_OR_BRANCH              28 */
\item \texttt{OR branch probabilities for OR-Fork
    "}\emph{list}\texttt{" for task "}\emph{identifier}\texttt{"
    do not sum to 1.0; sum is }\emph{n.n}\texttt{.}
  
  All branches from an or-fork\indexerror{OR-fork}\index{branch!OR}
  must be specified so that the sum of the
  probabilities\index{branch!probability} equals one.

% /* ERR_INVALID_PROC_RATE              29 */
\item \texttt{Processor "}\emph{identifier}\texttt{" has invalid rate of }\emph{n.n}\texttt{.}
  
  The processor rate\indexerror{processor!rate} parameter is used to scale the speed of the processor.  A
  value greater than zero must be used.

% /* ERR_NO_GROUP_DEFINED_FOR_PR0CESSOR   */

\item \texttt{Processor "}\emph{identifier}\texttt{" using CFS scheduling has no group."}
  
  If the completely fair share\indexerror{scheduling!completely fair} scheduler is being used, there must
  be at least one group\indexerror{group} defined for the processor.

% /* ERR_REF_TASK_FORWARDING            31 */
\item \texttt{Reference task "}\emph{identifier}\texttt{" cannot forward requests.}
  
  Reference tasks\indexerror{reference task} cannot accept messages,
  so they cannot forward\indexerror{forward}.

% /* ERR_REFERENCE_TASK_OPEN_ARRIVALS   32 */
\item \texttt{Reference task "}\emph{task-identifier}\texttt{", entry
    "}\emph{entry-identifier}\texttt{" cannot have open arrival
    stream.}
  
  Reference tasks\indexerror{reference task} cannot accept
  messages.\indexerror{open arrival}

% /* ERR_REFERENCE_TASK_IS_RECEIVER     33 */
\item \texttt{Reference task "}\emph{task-identifier}\texttt{", entry
    "}\emph{entry-identifier}\texttt{" receives requests.}
  
  Reference tasks\indexerror{reference task} cannot accept messages.

% /* ERR_REFERENCE_TASK_REPLIES         34 */
\item \texttt{Reference task "}\emph{task-identifier}\texttt{",
    replies to entry "}\emph{entry-identifier}\texttt{" from activity
    "}\emph{activity-identifier}\texttt{)".}
  
  Reference tasks\indexerror{reference task} cannot accept messages,
  so they cannot generate replies.  The activity
  \emph{activity-identifier} replies to entry
  \emph{entry-identifier}.\indexerror{reply}

% --- SCHEMA
\item \texttt{Required attribute '}\emph{attribute}\texttt{' was not
    provided} 
  
  The attribute named \emph{attribute} is
  missing\indexerror{attribute!missing} for the element.

% /* ERR_ASYNC_REQUEST_TO_WAIT	    */
\item \texttt{Semaphore "wait" entry "}\emph{entry-identifier}\texttt{" cannot accept send-no-reply requests.}

  An entry designated as the semaphore
  ``wait''\index{semaphore!wait}\indexerror{wait} can only accept
  rendezvous-type messages because send-no-reply messages and open
  arrivals cannot block the caller if the semaphore is busy.

% /* ERR_DUPLICATE_START_ACTIVITY       36 */
\item \texttt{Start activity for entry
    "}\emph{entry-identifier}\texttt{" is already defined.  Activity
    "}\emph{activity-identifier}\texttt{" is a duplicate.}
  
  A start activity\index{activity!start} has already been defined.
  This one is a duplicate\indexerror{duplicate!start activity}.

% /* ERR_NOT_DEFINED                    35 */
\item \texttt{Symbol "}\emph{identifier}\texttt{" not previously
    defined.}
  
  All identifiers must be declared before they can be
  used\indexerror{not defined}.

% /* ERR_INFINITE_TASK              30 */
\item \texttt{Task "}\emph{identifier}\texttt{" cannot be an
    infinite server." }
  
  This error occurs whenever a reference task\indexerror{reference task}\index{task!reference} or a
  semaphore task\indexerror{semaphore task}\index{task!semaphore} is designated as an infinite server.
  Reference tasks are the customers in the model so an infinite reference task would imply an infinite
  number of customers\footnote{An infinite source of customers should be represented by open
    arrivals\index{open arrival} instead.}.  An infinite semaphore task implies an infinite number of
  buffers -- no blocking at the wait entry\index{entry!wait} would ever occur.

% /* ERR_NO_START_ACTIVITIES            36 */
\item \texttt{Task "}\emph{identifier}\texttt{" has activities but
    none are reachable.}
  
  None of the activities for \emph{identifier} is reachable.  The most
  likely cause is that the start activity is
  missing\index{activity!start}\indexerror{start activity}\indexerror{not reachable}.

% /* ERR_NO_ENTRIES_DEFINED_FOR_TASK    37 */
\item \texttt{Task "}\emph{identifier}\texttt{" has no entries.}
  
  No entries were defined for \emph{identifier}\indexerror{entry}.

% /* ERR_ENTRY_COUNT_FOR_TASK           42 */
\item \texttt{"Task "}\emph{identifier}\texttt{" has }\emph{n}\texttt{ entries defined, exactly
  }\emph{m}\texttt{ are required.}
  
  The task \emph{identifier} has \emph{n} entries, \emph{m} are required. This error typically occurs with
  semaphore tasks\indexerror{semaphore task}\index{task!semaphore} which must have exactly two entries.

% /* ERR_ACTIVITY_NOT_SPECIFIED         38 */
\item \texttt{Task "}\emph{task-identifier}\texttt{", Activity
    "}\emph{activity-identifer}\texttt{" is not specified.}
  
  An activity is declared but not defined.\indexerror{activity}.  An
  activity is ``defined''\index{activity!defined} when some parameter
  such as service time is specified.

% /* ERR_DUPLICATE_REPLY                39 */
\item \texttt{Task "}\emph{task-identifier}\texttt{", Activity
    "}\emph{activity-identifer}\texttt{" makes a duplicate reply for
    Entry "}\emph{entry-identifier}\texttt{".}
  
  An activity graph is making a reply to entry \emph{entry-identifier}
  even though the entry is already in phase
  two\indexerror{reply!duplicate}.  This error usually occurs when
  more than one reply to \emph{entry-identifier} is specified in a
  sequence of activities.

% /* ERR_INVALID_REPLY                  40 */
\item \texttt{Task "}\emph{task-identifier}\texttt{", Activity
    "}\emph{activity-identifer}\texttt{" makes invalid reply for Entry
    "}\emph{entry-identifier}\texttt{".}
  
  An activity graph is making a reply to entry \emph{entry-identifier}
  even though the activity is not
  reachable.\indexerror{reply!invalid}\indexerror{activity!not reachable}.

% /* ERR_REPLY_SPECIFIED_FOR_SNR_ENTRY  41 */
\item \texttt{Task "}\emph{task-identifier}\texttt{", Activity
    "}\emph{activity-identifer}\texttt{" replies to Entry
    "}\emph{entry-identifier}\texttt{" which does not accept rendezvous requests.}
  
  The activity graph specifies a reply\indexerror{reply!invalid} to
  entry \emph{entry-identifier} even though the entry does not accept
  rendezvous requests.  (The entry either accepts send-no-reply
  requests or open arrivals\index{open arrival}).

% --- Schema ---
\item \texttt{Unknown element '}\emph{element}\texttt{'}
  
  The \emph{element}\indexerror{element!unkown} is not expected at this point in the input file.
  \emph{Element} may not be spelled incorrectly, or if not, in an incorrect location in the input file.

\end{itemize}

\section{Advisory Messages}
\label{sec:advisory-messages}

\begin{itemize}
% /* ADV_CONVERGENCE_VALUE             69 */
\item \texttt{Invalid convergence value of }\emph{n.n}\texttt{, using
  }\emph{m.m}\texttt{.}\marginpar{lqns}
  
  The convergence value\indexerror{convergence!value} specified in the
  input file is not valid.  The analytic solver is using $m.m$
  instead.

% /* WRN_BOGUS_STDDEV                  68 */
\item \texttt{Invalid standard deviation: sum=}\emph{n.n}\texttt{,
    sum\_sqr=}\emph{n.n}\texttt{, n=}\emph{n.n}\texttt{.}
  
  When calculating a standard
  deviation\indexerror{standard deviation}, the difference of the sum
  of the squares and the mean of the square of the sum was negative.
  This usually implies an internal error in the simulator.

% /* ADV_ITERATION_LIMIT               72 */
\item \texttt{Iteration limit of }\emph{n}\texttt{ is too small, using
  }\emph{m}\texttt{.}\marginpar{lqns}
  
  The iteration limit\indexerror{iteration limit} specified in the
  input file is not valid.  The analytic solver is using $m$ instead.

% /* ADV_MESSAGES_DROPPED              42 */
\item \texttt{Messages dropped at task }\emph{identifier}\texttt{ for
    open-class queues.}
  
  Asynchronous send-no-reply\indexerror{send-no-reply} messages were
  \emph{lost} at the task \emph{task}.  This message will occur when
  the \pragma{stop-on-message-loss} pragma
  (c.f.~\S\ref{sec:lqns-pragmas}) is set to ignore open class
  overflows.  Note that if a task accepts both open and closed
  classes\index{class!open}\index{class!closed}, an overflow in the
  open class will result in zero throughput\index{throughput!zero} for
  the closed classes.

% /* ADV_SOLVER_ITERATION_LIMIT        75 */
\item \texttt{Model failed to converge after }\emph{n}\texttt{
    iterations (convergence test is }\emph{n.n}\texttt{, limit is
  }\emph{n.n}\texttt{).}\marginpar{lqns}
  
  Sometimes the model fails to converge\indexerror{convergence},
  particularly if there are several heavily utilized
  servers\index{utilization!high} in a submodel.  Sometimes, this
  problem can be solved by reducing the value of the
  under-relaxation\index{under-relaxation} coefficient.  It may also
  be necessary to increase the iteration-limit\index{iteration limit},
  particularly if there are many submodels.  With replicated
  models\index{replication!convergence}, it may be necessary to use
  `loose' layering\index{layering!loose} to get the model to converge.
  Convergence can be tracked using \flag{t}{convergence}.

% /* ADV_LQX_IMPLICIT_SOLVE               */
\item \texttt{No solve() call found in the lqx program in file: }\emph{filename}\texttt{.  solve() was
    invoked implicitly.}
  
  An LQX\index{lqx} program was found in file \emph{filename}.  However, the function
  \texttt{solve()}\index{solve()!implicit} was not invoked explictity.  The program was executed to
  completion, after which \texttt{solve()} was called using the final value of all the variables found in
  the program.
  
% /* ADV_REPLICATION_ITERATION_LIMIT   73 */
\item \texttt{Replicated Submodel }\emph{n}\texttt{ failed to converge
    after }\emph{n}\texttt{ iterations (convergence test is
  }\emph{n.n}\texttt{, limit}\linebreak[4] \texttt{is }\emph{m.m}\texttt{).}\marginpar{lqns}
  
  The inner ``replication''
  iteration\indexerror{replication!iteration} failed to
  converge....\typeout{Bonne mots: replication convergence}

% /* ADV_SERVICE_TIME_RANGE            74 */
\item \texttt{Service times for }\emph{(processor)}
  \emph{identifier}\texttt{ have a range of }\emph{n.n}\texttt{ -
  }\emph{n.n}\texttt{. Results may not be valid.}\marginpar{lqns}
  
  The range of values of service times\indexerror{service time} for a
  processor using processor sharing\index{processor!sharing}
  scheduling\indexerror{processor!sharing} is over two orders of
  magnitude.  The results may not be valid.

% /* ADV_PRECISION                     67 */
\item \texttt{Specified confidence interval of }\emph{n.n}\texttt{\%
    not met after run time of }\emph{n.n}\texttt{. Actual value is
  }\emph{n.n}\texttt{\%.}\marginpar{lqsim}

% /* ADV_EMPTY_SUBMODEL                70 */
\item \texttt{Submodel }\emph{n}\texttt{ is empty.}\marginpar{lqns}
  
  The call graph\index{call graph} is interesting, to say the least.

% /* ADV_UNDERRELAXATION               76 */
\item \texttt{Underrelaxation ignored.  }\emph{n.n}\texttt{ outside
    range [0-2), using }\emph{m.m}\texttt{.}\marginpar{lqns}
  
  The under-relaxation
  coefficient\indexerror{under-relaxation coefficient} specified in
  the input file is not valid.  The solver is using $m.m$
  instead\footnote{Values of under-relaxation from $1 < n \le 2$ are
    more correctly called over-relaxation\index{over relaxation}.}.

% /* ADV_INVALID_UTILIZATION           71 */
\item \texttt{The utilization of }\emph{n.n}\texttt{ at }\emph{(task$|$processor)}
  \emph{identifier}\texttt{ with multiplicity }\emph{m}{ is too high.}
  
  This problem is the result of some of the approximations
  used\indexerror{utilization!high} by the analytic solver.  The
  common causes are two-phase
  servers\index{phases!approximation!error} and the Rolia
  multiserver\index{multiserver!approximation!error}\index{multiserver!Rolia}.
  If \emph{identifer} is a multiserver, switching to the
  Conway\index{multiserver!Conway} approximation will often help.
  Values of $n.n$ in excess of 10\% are likely the result of failures
  in the solver.  Please send us the model file so that we can improve
  the algorithms.

\end{itemize}

\section{Warning Messages}
\label{sec:warning-messages}

\begin{itemize}
% /* WRN_NOT_USED                       43  */
\item \emph{(activity$|$entry$|$task$|$processor)}\texttt{ "}\emph{identifier}\texttt{" is not used.}
  
  The object is not reachable\indexerror{not reachable}.  This may
  indicate an error in the specification of the model.

% WRN_INFINITE_MULTI_SERVER
\item \emph{(Processor$|$Task)}\texttt{ "}\emph{identifier}\texttt{" is an infinite server with a
  multiplicity of }$n$\texttt{.} 

  Infinite servers must always have a multiplicty of one.\index{multiplicity!infinite server}\indexerror{multiplicity}
  This error is caused by specifying both \emph{delay}\index{scheduling!delay} scheduling and a multiplicity
  for the named task or processor.  The multiplicity attribute is ignored.
  
% /* WRN_SCHEDULING_NOT_SUPPORTED       44 */
\item \emph{sched} \texttt{scheduling specified for}
  \emph{(processor$|$task)}\texttt{ "}\emph{identifier}\texttt{" is not
    supported.}
  
  The solver does not support the specified scheduling
  type\indexerror{scheduling}.  First-in, first-out scheduling will be
  used instead.

% /* WRN_ACT_NO_SERVICE_TIME            45 */
\item \texttt{Activity "}\emph{identifier}\texttt{" has no service
    time specified.}
  
  No service time\indexerror{service time} is specified for
  \emph{identifier}.

% /* WRN_COEFFICIENT_OF_VARIATION      77 */
\item \texttt{Coefficient of variation is incompatible with phase type
    at }\emph{(entry$|$task)} \texttt{"}\emph{identifier}\texttt{"}
  \emph{(phase$|$activity)}
  \texttt{"}\emph{identifier}\texttt{".}\marginpar{lqns}
  
  A coefficient of variation\indexerror{coefficient of variation} is
  specified at a using stochastic phase or activity.

% /* WRN_NO_REQUESTS_TO_ENTRY           46 */
\item \texttt{Entry "}\emph{identifier}\texttt{" does not receive any
    requests.}
  
  Entry \emph{identifier} is part of a non-reference
  task\indexerror{reference task}\indexerror{server!task}\index{task!server}.
  However, no requests are made to this entry.

% /* WRN_NO_SERVICE_TIME                47 */
\item \texttt{Entry "}\emph{identifier}\texttt{" has no service time
    specified for any phase.}
  
  No service time\indexerror{service time} is specified for entry
  \emph{identifier}.

% /* WRN_NO_SERVICE_TIME_FOR_PHASE      48 */
\item \texttt{Entry "}\emph{identifier}\texttt{" has no service time
    specified for phase} \emph{n}\texttt{.}
  
  No service time\indexerror{service time} is specified for entry
  \emph{identifier}, phase $n$.

% /* INFINITE_SERVER_OPEN_ARRIVALS */
\item \texttt{Infinite server "}\emph{identifier}{" accepts either asynchronous messages or open arrivals.}

  The task or processor, \emph{identifier}, is an infinite server.  It processes either asynchronous
  messages or open arrivals.  If the arrival rate exceeds the service rate of the infinite server, the
  number of instances of the infinite server will grow to infinity.
  
% /* WRN_NO_PHASE_FOR_HISTOGRAM         68 */
\item \texttt{Histogram requested for entry "}\emph{identifier}\texttt{", phase~}\emph{n}\texttt{ -- phase is not present.}\marginpar{lqsim}
  
  A histogram\index{histogram!no~phase} for the service time of phase \emph{n} of entry \emph{identifier} was requested.  This entry
  has no corresponding phase.

% /* WRN_INVALID_PRIORITY              70 */
\item \texttt{Priority specified (}\emph{n}\texttt{) is outside of
    range (}\emph{n}\texttt{,}\emph{n}\texttt{). (Value has been
    adjusted to }\emph{n}\texttt{).}\marginpar{lqsim}

  The priority $n$ is outside of the range specified.\indexerror{priority}

% /* WRN_NO_QUANTUM_FOR_PS             69 */
\item \texttt{No quantum specified for PS scheduling discipline.  FIFO
    used." }\marginpar{lqsim}
  
  A processor using processor sharing\indexerror{processor!sharing}
  scheduling\index{scheduling!processor sharing} needs a quantum
  value\index{quantum} when running on the
  simulator\index{lqsim!scheduling}.

% /* WRN_NO_REQUESTS_MADE              79 */
\item \texttt{No requests made from }\emph{from-identifier}\texttt{ to
  }\emph{to-identifier}\texttt{.}\marginpar{lqns}
  
  The input file has a rendezvous\indexerror{rendezvous} or
  send-no-reply\indexerror{send-no-reply} request with a value of
  zero.

% /* WRN_DEFINED_NE_SPECIFIED_X         49 */
\item \texttt{Number of }\emph{(processors$|$tasks$|$entries)}\texttt{
    defined (}\emph{n}\texttt{) does not match number specified
    (}\emph{m}\texttt{).}

  The processor task and entry chapters of the original input
  grammar\index{grammar!original} can specify the number of objects
  that follow.  The number specified does not match the actual number
  of objects.  Specifying \emph{zero} as a record count is valid.

% /* WRN_MULTIPLE_SPECIFICATION         50 */
\item \texttt{Parameter is specified multiple times.}

  A parameter is specified more than one time.  The first occurance is
  used.\indexerror{duplicate!parameter}

% /* WRN_NON_CFS_PR0CESSOR              */
\item \texttt{Processor "}\emph{identifier}\texttt{" is not running fair share scheduling." }
  
  A group\indexerror{scheduling!completely fair}\indexerror{group} was defined in the model and associated
  with a processor using a scheduling discipline other than completely fair scheduling.

% /* WRN_NO_TASKS_DEFINED_FOR_PROCESSOR 51 */
\item \texttt{Processor "}\emph{identifier}\texttt{" has no tasks.}
  
  A processor was defined in the model, but it is not used by any
  tasks\indexerror{processor!not used}.  This can occur if none of the
  entries or phases has any service time\index{service time}.

% /* WRN_QUEUE_LENGTH                  80 */
\item \texttt{Queue Length is incompatible with task type at task
  }\emph{identifier}\texttt{.}\marginpar{lqns}
  
  A queue length\indexerror{queue length} parameter was specified at a
  task which does not support bounded queues.

% /* WRN_NO_SENDS_FROM_REF_TASK         52 */
\item \texttt{Reference task "}\emph{identifier}\texttt{" does not
    send any messages." }
  
  Reference tasks\indexerror{reference task} are customers in the
  model.  This reference task does not visit any servers, so it serves
  no purpose.

% /* WRN_TOO_MANY_ENTRIES_FOR_REF_TASK  53 */
\item \texttt{Reference task "}\emph{identifier}\texttt{" has more
    than one entry defined.}
  
  Reference tasks\index{task!reference}\indexerror{reference task}
  typically only have one entry.  The named reference task has more
  than one.  Requests are generated in proportion to the service times
  of the entries.

% /* WRN_PRIO_TASK_ON_FIFO_PROC         54 */
\item \texttt{Task "}\emph{task-identifier}\texttt{" with priority is
    running on processor
    "}\emph{processor-identifier}\texttt{"}\linebreak[3] \texttt{which
    does not have priority scheduling.}
  
  Processors running with FCFS\index{scheduling!FCFS} scheduling
  ignore priorities\indexerror{priority}.

% /* WRN_INVALID_INT_VALUE             78 */
\item \texttt{Value specified for }\emph{(fanin$|$fanout)}\texttt{,
  }\emph{n}\texttt{, is invalid.}\marginpar{lqns}
  
  The value specified for a fan-in\indexerror{fan-in} or
  fan-out\indexerror{fan-out} is not valid and will be ignored.

% /* WRN_NOT_SUPPORTED                  55 */
\item \texttt{The }\emph{x}\texttt{ feature is not supported in this version.}

  Feature \emph{x} is not supported in this release.

\end{itemize}

\section{Input File Parser Error Messages}

\begin{itemize}
\item \texttt{error: not well-formed (invalid token)}

  This error occurs when an XML input file is expected, but some other
  input file type was given.

\item \texttt{Parse error.}

  An error was detected while processing the XML input file\indexerror{}.  See the list below for more
  explantion:

\begin{itemize}
% : warning: An exception occurred! Type:RuntimeException, Message:Warning: The primary document entity could not be opened. Id=/usr/lqn.xsd
\item \texttt{The primary document entity could not be opened. Id=}\emph{URI}\texttt{ while parsing
  }\emph{filename}\texttt{.}
  
  This error\label{error:primary-document}\indexerror{primary document} is generated by the
  Xerces\indexerror{Xerces} when the Uniform resource indicator \emph{(URI)} specified as the argument to
  the \attribute{xsi:noNamespaceSchemaLocation} attribute of the \schemaelement{lqn-model} element cannot be
  opened.  This argument must refer to a valid location containing the LQN schema\indexerror{schema} files.
% --- SCHEMA_KEY_FOR_IDENTITY_CONSTRAINT ---
\item \texttt{The key for identity constraint of element '}\emph{lqn-model}\texttt{' is not found.}
  
  When this message appears, Xerces\index{Xerces} does \textbf{not} provide many hints on where the actual
  error occurs because it always gives a line number which points to the end of the file (i.e. where the
  terminating tag \verb!</lqn-model>! is).

  In this case, the following three points should be inspected to
  ensure validity of the model:

  \begin{enumerate}
  \item All synchronous calls have a \attribute{dest} attribute which
    refers to a valid entry\indexerror{rendezvous}.
  \item All asynchronous calls have a \attribute{dest} attribute which
    refers to a valid entry\indexerror{send-no-reply}.
  \item All forwarding calls have a \attribute{dest} attribute which
    refers to a valid entry\indexerror{forward}.
  \end{enumerate}

  If it is not practical to manually inspect the model, run the XML
  file through another tool like XMLSpy or XSDvalid which will report
  more descriptive errors.
  
\item \texttt{The key for identity constraint of element '}\emph{task}\texttt{' is not found.}
  
  When this error appears, it means there is something wrong within
  the \schemaelement{task} element indicated by the line number.
  Check that:
  \begin{itemize}
  \item The name \attribute{attribute} of all
    \schemaelement{reply-entry} elements refers to a valid entry name,
    which exists within the same task as the task activity
    graph\indexerror{activity graph}.
  \item All activities which contain the attribute
    \attribute{bound-to-entry} have a valid entry name that exists
    within the same task as the task activity
    graph\indexerror{activity graph}.
  \end{itemize}

% --- SCHEMA_KEY_FOR_IDENTITY_CONSTRAINT ---
\item \texttt{The key for identity constraint of element
    '}\emph{task-activities}\texttt{' is not found.}
  
  When this error appears, it means there is something wrong within
  the \schemaelement{task-activities} element indicated by the line
  number.

  Check that:
  \begin{itemize}
  \item All activities referenced within the
    \schemaelement{precedence} elements refer to activities which are
    defined for that particular task activity
    graph\indexerror{activity graph}.
  \item The \attribute{name} attribute of all
    \schemaelement{reply-activity}\indexerror{reply-activity} elements
    refers to an activity defined within the mentioned
    \schemaelement{task-activities} element.
  \item The head attribute of all \schemaelement{post-loop} elements
    refers to an activity defined within the mentioned
    \schemaelement{task-activities} element.
  \item All post-LOOP elements which contain the optional attribute
    \attribute{end}, refers to an activity defined within the
    mentioned \schemaelement{task-activities} element.
  \end{itemize}

\item \texttt{Not enough elements to match content model :}\emph{elements}

  ((run-control,plot-control,solver-params,processor),slot)
\end{itemize}
\end{itemize}
\section{LQX Error messages}

\begin{itemize}
\item \texttt{Runtime Exception Occured: Unable to Convert From:} \emph{`<<uninitialized>>'} \texttt{To:} \emph{`Array'}

  An unitialized variable is used where an array is expected (like in
  a foreach loop).
\end{itemize}


%%% Local Variables: 
%%% mode: latex
%%% mode: outline-minor 
%%% fill-column: 108
%%% TeX-master: "userman.tex"
%%% End: 

%%  -*- mode: latex; mode: outline-minor; fill-column: 108 -*-
%% Title:  defects
%%
%% $HeadURL: http://rads-svn.sce.carleton.ca:8080/svn/lqn/trunk-V6/doc/userman/defects.tex $
%% Original Author:     Greg Franks <greg@sce.carleton.ca>
%% Created:             Tue Jul 18 2006
%%
%% ----------------------------------------------------------------------
%% $Id: defects.tex 7586 2007-08-02 11:13:34Z greg $
%% ----------------------------------------------------------------------

\chapter{Known Defects}
\label{sec:defects}

\section{MOL Multiserver Approximation Failure}
\label{sec:MOLMultiserver}

The MOL multiserver approximation sometimes fails when the service
time of the clients to the multiserver are significantly smaller than
the service time of the server itself.  The utilization of the
multiserver will be too high.  Sometimes, the problem can be solved by
changing the mol-underrelaxation.  Otherwise, switch to the
more-expensive Conway multiserver approximation.

\section{Chain construction for models with multi- and infinite-servers}
\label{sec:ChainConstruction}

\section{No algorithm for phased multiservers OPEN class.}
\label{sec:PhaseMultiOpen}

\section{Overtaking probabilities are calculated using CV=1}
\label{sec:Overtaking}

\section{Need to implement queue lengths for open classes.}
\label{sec:QueueLengths}



%%% Local Variables: 
%%% mode: latex
%%% mode: outline-minor 
%%% fill-column: 108
%%% TeX-master: "userman"
%%% End: 

\appendix
f%% -*- mode: latex; mode: outline-minor; fill-column: 108 -*-
%% Title:  grammar
%%
%% $HeadURL: http://rads-svn.sce.carleton.ca:8080/svn/lqn/trunk-V6/doc/userman/grammar.tex $
%% Original Author:     Greg Franks <greg@sce.carleton.ca>
%% Created:             Tue Jul 18 2006
%%
%% ----------------------------------------------------------------------
%% $Id: grammar.tex 14587 2021-04-02 18:09:46Z greg $
%% ----------------------------------------------------------------------

\chapter{Traditional Grammar}
\label{sec:old-grammar}

This chapter gives the formal description of Layered Queueing Network
input file and parseable output file grammars in extended BNF form.
For the nonterminals the notation \nt{nonterminal\_id} is used, while
the terminals are written without brackets as they appear in the input
text.  The notation \rep{$\cdots$}{n}{m}, where $n \leq m$ means that
the part inside the curly brackets is repeated at least $n$ times and
at most $m$ times. If $n=0$, then the part may be missing in the input
text. The notation \opt{\nt{$\cdots$}} means that the non-terminal is
optional.

\section{Input File Grammar}
\label{sec:input-file-bnf}

%% -*- mode: latex; mode: outline-minor; fill-column: 108 -*-
%% Title:  grammar
%%
%% $HeadURL: http://rads-svn.sce.carleton.ca:8080/svn/lqn/trunk-V6/doc/userman/input-grammar.tex $
%% Original Author:     Greg Franks <greg@sce.carleton.ca>
%% Created:             Tue Jul 18 2006
%%
%% ----------------------------------------------------------------------
%% $Id: input-grammar.tex 14587 2021-04-02 18:09:46Z greg $
%% ----------------------------------------------------------------------

\newcommand{\altflg}[1]{#1}
\newcommand{\swapflg}{\let\tmpflg\chgflg\let\chgflg\altflg\let\altflg\tmpflg}

\begin{bnf}{multi\_server\_flag}
  \defitem[LQN\_input\_file] \<general\_info> \<processor\_info> \opt{\<group\_info>} 
  \<task\_info> \<entry\_info> \rep{\<activity\_info>}{0}{}
  \oritem \<parameter\_list> \<processor\_info> \opt{\<group\_info>} 
  \<task\_info> \<entry\_info> \rep{\<activity\_info>}{0}{} \opt{\<report\_info>} \opt{\<convergence\_info>}
\end{bnf}

\subsection{SPEX Parameters}
\label{sec:spex-parameters}\index{SPEX!grammar!paramters}

\begin{bnf}{multi\_server\_flag}
  \defitem[parameter\_list]  \rep{\<comma\_expr>}{1}{np}

  \defitem[comma\_expr] \<variable\_def>
  \oritem[ \<expression> , \<variable\_def> ]

  \defitem[variable\_def] \<variable> = \<ternary\_expr>
  \oritem [ \< expression\_list> ]
  \oritem [ \<real> : \<real> , \<real> ]

\end{bnf}

\subsection{General Information}
\label{sec:general-in}

\begin{bnf}{multi\_server\_flag}
  \defitem[general\_info] G \<comment> \<conv\_val> \<it\_limit>
  \?\<print\_int>? \?\<underrelax\_coeff>? \<end\_list>

  \defitem[comment] \<string> \(comment on the model)\index{model!comment}

  \swapflg
  \defitem[conv\_val] \<real> \(convergence value) \chg{\ddag}\index{convergence!value}

  \defitem[it\_limit] \<integer> \(max. nb. of iterations) \chg{\ddag}\index{iteration limit}

  \defitem[print\_int] \<integer> \chg{\ddag}\index{print interval}
  \itemsep\oritemsep
  \item \(intermed. res. print interval)
  \itemsep\defitemsep

  \defitem[underrelax\_coeff] \<real> \(under\_relaxation coefficient)\index{under-relaxation coefficient}
  \chg{\ddag}
  \swapflg

  \defitem[end\_list] -1 \(end\_of\_list mark)

  \defitem[string] " \<text> "
\end{bnf}

\subsection{Processor Information}
\label{sec:processor}

\begin{bnf}{multi\_server\_flag}
  \defitem[processor\_info] P \<np> \<$p$\_decl\_list>

  \defitem[np] \<integer> \(total number of processors)

  \defitem[$p$\_decl\_list] \rep{\<$p$\_decl>}{1}{np} \<end\_list>

  \defitem[$p$\_decl] p \<proc\_id> \<scheduling\_flag> \?\<quantum>?\index{scheduling!processor}
  \?\<multi\_server\_flag>?  \?\<replication\_flag>? \?\<proc\_rate>?\index{replication!processor}

  \defitem[proc\_id] \<integer> \| \<identifier>
  \itemsep\oritemsep
  \item \(processor identifier)
  \itemsep\defitemsep

  \defitem[scheduling\_flag] f \(First come, first served)
  \oritem h \(Head Of Line)
  \oritem p \(Priority, preemeptive)
  \oritem c \<real> \(completely fair scheduling\index{processor!scheduling!completely fair})
  \oritem s \<real> \(processor sharing\index{processor!sharing}\index{processor!scheduling!sharing})
  \oritem i \(Infinite or delay)
  \oritem r \(Random)

  \defitem[quantum] \<real>\index{quantum} \| \<variable>

  \defitem[multi\_server\_flag] m \<copies> \(number of duplicates)
  \oritem i \(Infinite server)

  \defitem[replication\_flag] r \<copies> \(number of replicas)

  \defitem[proc\_rate] R \<ratio> \| \<variable> \(Relative proc. speed)

  \defitem[copies] \<integer> \| \<variable>

  \defitem[ratio] \<real> \| \<variable>

\end{bnf}

\subsection{Group Information}
\label{sec:group}

\begin{bnf}{multi\_server\_flag}
  \defitem[group\_info] U \<ng> \<$g$\_decl\_list> \<end\_list>

  \defitem[ng] \<integer> \(total number of groups)

  \defitem[$g$\_decl\_list] \rep{\<$g$\_decl>}{1}{ng} \<end\_list>

  \defitem[$g$\_decl] g \<group\_id> \<group\_share> \opt{\<cap\_flag>} \<proc\_id>

\defitem[group\_id] \<identifier>

\defitem[group\_share] \<real> \| \<variable>

\defitem[cap\_flag] c
\end{bnf}

\subsection{Task Information}
\label{sec:task}

\begin{bnf}{multi\_server\_flag}
  \defitem[task\_info] T \<nt> \<$t$\_decl\_list>

  \defitem[nt] \<integer> \(total number of tasks)

  \defitem[$t$\_decl\_list] \rep{\<$t$\_decl>}{1}{nt} \<end\_list>

  \defitem[$t$\_decl] t \<task\_id> \<task\_sched\_type>\index{scheduling!task}
  \<entry\_list> \?\<queue\_length>?\index{queue length}
  \<proc\_id> \?\<task\_pri>? \?\<think\_time\_flag>?\index{think time} \?\<tokens>?
  \?\<multi\_server\_flag>? \?\<replication\_flag>?\index{replication!task}
  \?\<group\_flag>?\index{group share}
  \oritem I \<from\_task> \<to\_task> \<fan\_in>\index{fan-in}
  \oritem O \<from\_task> \<to\_task> \<fan\_out>\index{fan-out}

  \defitem[task\_id] \<integer> \| \<identifier>
  \itemsep\oritemsep
  \item \(task identifier)
  \itemsep\defitemsep

  \defitem[task\_sched\_type] r \(reference task)
  \oritem n \(non-reference task)
  \oritem h \(Head of line)
  \oritem f \(FIFO Scheduling)
  \oritem i \(Infinite or delay server)
  \oritem p \(Polled scheduling at entries)
  \oritem b \(Bursty Reference task)
  \oritem S \(Semaphore)\index{scheduling!semaphore}

  \defitem[entry\_list] \rep{\<entry\_id>}{1}{{\it ne\/}_t} \<end\_list>
  \itemsep\oritemsep
  \item \(task $t$ has $ne_t$ entries)
  \itemsep\defitemsep

  \defitem[entry\_id] \<integer> \| \<identifier>
  \itemsep\oritemsep
  \item \(entry identifier)
  \itemsep\defitemsep

  \defitem[task\_pri] \<integer> \(task priority, optional)

  \defitem[queue\_length] q \<integer> \(open class queue length)

  \defitem[group\_flag] g \<identfier> \(Group for scheduling)

  \defitem[tokens] t \<integer> \(Initial tokens)

  \defitem[from\_task] \<task\_id> \(Source task)
  \defitem[to\_task] \<task\_id> \(Destination task)
  \defitem[fan\_in] \<integer> \(fan in to this task)
  \defitem[fan\_out] \<integer> \(fan out from this task)
\end{bnf}


\subsection{Entry Information}
\label{sec:entry}

\begin{bnf}{multi\_server\_flag}
  \defitem[entry\_info] E \<ne> \<entry\_decl\_list>

  \defitem[ne] \<integer> \(total number of entries)

  \defitem[entry\_decl\_list] \rep{\<entry\_decl>}{1}{} \<end\_list>

  \item \(k = maximum number of phases)

  \defitem[entry\_decl] a \<entry\_id> \<arrival\_rate>
  \oritem A \<entry\_id> \<activity\_id> 
  \oritem F \<from\_entry> \<to\_entry> \<p\_forward>\index{forwarding probability}
  \oritem H \<entry\_id> \<phase> \<hist\_min> ':' \<hist\_max> \<hist\_bins> \<hist\_type>\index{histogram}
  \oritem M \<entry\_id> \rep{\<max\_service\_time>}{1}{k} \<end\_list>\index{maximum service time}\index{service time!maximum}
  \oritem P \<entry\_id> \(Signal Semaphore)\index{semaphore!signal}
  \oritem V \<entry\_id> \(Wait Semaphore)\index{semaphore!signal}
  \oritem Z \<entry\_id> \rep{\<think\_time>}{1}{k} \<end\_list>\index{think time}
  \oritem c \<entry\_id> \rep{\<coeff\_of\_variation>}{1}{k} \<end\_list>\index{coefficient of variation}
  \oritem f \<entry\_id> \rep{\<ph\_type\_flag>}{1}{k} \<end\_list>\index{phase!type}
  \oritem p \<entry\_id> \<entry\_priority>\index{priority!entry}
  \oritem s \<entry\_id> \rep{\<service\_time>}{1}{k} \<end\_list>\index{service time}
  \oritem y \<from\_entry> \<to\_entry> \rep{\<rendezvous>}{1}{k} \<end\_list>\index{rendezvous}
  \oritem z \<from\_entry> \<to\_entry> \rep{\<send\_no\_reply>}{1}{k} \<end\_list>\index{send-no-reply}

  \defitem[arrival\_rate] \<real> \| \<variable> \(open arrival rate to entry)
  \defitem[coeff\_of\_variation] \<real> \| \<variable> \(squared service time coefficient of variation)
  \defitem[from\_entry] \<entry\_id> \(Source of a message)
  \defitem[hist\_bins] \<integer> \(Number of bins in histogram.)
  \defitem[hist\_max] \<real> \(Median service time.)
  \defitem[hist\_min] \<real> \(Median service time.)
  \defitem[hist\_type] log \| linear \| sqrt \(bin type.)
  \defitem[max\_service\_time] \<real> \(Median service time.)
  \defitem[p\_forward] \<real> \(probability of forwarding)
  \defitem[phase] 1 \| 2 \| 3 \(phase of entry)
  \defitem[ph\_type\_flag] 0 \(stochastic phase)
  \oritem 1 \(deterministic phase)
  \defitem[rate] \<real> \| \<variable> \(nb. of calls per arrival)
  \defitem[rendezvous] \<real>  \| \<variable> \(mean number of RNVs/ph)
  \defitem[send\_no\_reply] \<real> \| \<variable> \(mean nb.of non-blck.sends/ph)
  \defitem[service\_time] \<real> \| \<variable> \(mean phase service time)
  \defitem[think\_time] \<real> \| \<variable> \(Think time for phase.)
  \defitem[to\_entry] \<entry\_id> \(Destination of a message)
\end{bnf}

\subsection{Activity Information}
\label{sec:activity}

\begin{bnf}{multi\_server\_flag}

  \defitem[activity\_info] \<activity\_defn\_list> \?\<activity\_connections>? \<end\_list>

\item \( Activity definition. )

  \defitem[activity\_defn\_list] \rep{\<activity\_defn>}{1}{\it na}
  \defitem[activity\_defn] c \<activity\_id> \<coeff\_of\_variation> \( Sqr. Coeff. of Var. )\index{coefficient of variation}
  \oritem f \<activity\_id> \<ph\_type\_flag> \( Phase type )\index{phase!type}
  \oritem H \<entry\_id> \<hist\_min> ':' \<hist\_max> \<hist\_bins> \<hist\_type>\index{histogram}
  \oritem M \<activity\_id> \<max\_service\_time>\index{maximum service time}\index{service time!maximum}
  \oritem s \<activity\_id> \<ph\_serv\_time> \( Service time )\index{service time}
  \oritem Z \<activity\_id> \<think\_time> \( Think time )\index{think time}
  \oritem y \<activity\_id> \<to\_entry> \<rendezvous> \( Rendezvous  )\index{rendezvous}
  \oritem z \<activity\_id> \<to\_entry> \<send\_no\_reply> \( Send-no-reply )\index{send-no-reply}

\item \( Activity Connections. )\index{activity!connection|see{precedence}}\index{activity!connection}

  \defitem[activity\_connections] : \<activity\_conn\_list>

  \defitem[activity\_conn\_list] \<activity\_conn> \rep{; \<activity\_conn>}{1}{\it na}

  \defitem[activity\_conn] \<join\_list>
  \oritem \<join\_list> -> \<fork\_list>\index{->@\texttt{->}|see{precedence}}

  \defitem[join\_list] \<reply\_activity>\index{join-list}
  \oritem \<and\_join\_list>\index{AND-join}
  \oritem \<or\_join\_list>\index{OR-join}

  \defitem[fork\_list] \<activity\_id>
  \oritem \<and\_fork\_list>\index{and-fork}
  \oritem \<or\_fork\_list>\index{or-fork}
  \oritem \<loop\_list>\index{LOOP}

  \defitem[and\_join\_list] \<reply\_activity> \rep{\& \<reply\_activity>}{1}{\it na} \opt{\<quorum\_count>}
  \defitem[or\_join\_list] \<reply\_activity> \rep{+ \<reply\_activity>}{1}{\it na}
  \defitem[and\_fork\_list] \<activity\_id> \rep{\& \<activity\_id>}{1}{\it na}
  \defitem[or\_fork\_list] \<prob\_activity> \rep{+ \<prob\_activity>}{1}{\it na}
  \defitem[loop\_list] \<loop\_activity> \rep{, \<loop\_activity>}{0}{na} \opt{\<end\_activity>}

  \defitem[prob\_activity] ( \<real> ) \<activity\_id>
  \defitem[loop\_activity] \<real> * \<activity\_id>
  \defitem[end\_activity] , \<activity\_id>
  \defitem[reply\_activity] \<activity\_id> \opt{\<reply\_list>}

  \defitem[reply\_list] [ \<entry\_id> \rep{, \<entry\_id> }{0}{ne} ]
  \defitem[quorum\_count] ( \<integer> ) \( Quorum )\index{quorum}
\end{bnf}

\subsection{SPEX Report Information}\index{SPEX!grammar!report}
\begin{bnf}{multi\_server\_flag}
  \defitem[report\_info] R \<nr> \<report\_decl\_list>  \<end\_list>
  \oritem R \<nr> \<identifier> ( \<expression\_list> )

  \defitem[report\_decl\_list] \rep{\<$r$\_decl>}{1}{nr}

  \defitem[$r$\_decl] \<variable> = \<ternary\_expr>
  \oritem \<expression>
\end{bnf}

\subsection{SPEX Convergence Information}\index{SPEX!grammar!convergence}
\begin{bnf}{multi\_server\_flag}
  \defitem[convergence\_info] C \<nc> \<convergence\_decl\_list>  \<end\_list>

  \defitem[convergence\_decl\_list] \rep{\<$c$\_decl>}{1}{nr}

  \defitem[$c$\_decl] \<variable> = \<ternary\_expr>
\end{bnf}

\subsection{Expressions}\index{SPEX!grammar!expressions}
\label{sec:spex-expressions}
\begin{bnf}{multi\_server\_flag}

  \defitem[ternary\_expression] \<or\_expression> ? \<or\_expression> : \<or\_expression>
  \oritem \<or\_expression>

  \defitem[or\_expression] \<or\_expression> $|$ \<and\_expression> \( Logical OR )
  \oritem \<and\_expression>

  \defitem[and\_expression] \<and\_epxression> $\&$ \<compare\_expression> \( Logical AND )
  \oritem \<compare\_expression>

  \defitem[compare\_expression] \<compare\_expression> == \<expression>
  \oritem \<compare\_expression> != \<expression>
  \oritem \<compare\_expression> < \<expression>
  \oritem \<compare\_expression> <= \<expression>
  \oritem \<compare\_expression> > \<expression>
  \oritem \<compare\_expression> >= \<expression>
  \oritem \<expression>

  \defitem[expression] \<expression> + \<term>
  \oritem \<expression> $-$ \<term> 
  \oritem \<term>

  \defitem[term] \<term> * \<power>
  \oritem \<term> / \<power>
  \oritem \<term> \% \<power> \( Modulus )
  \oritem \<power>

  \defitem[power] \<prefix> ** \<power> \( Exponentiation, right associative) % \char94 forces raw caret.
  \oritem \<prefix>

  \defitem[prefix] ! \<factor> \( Logical NOT )
  \oritem \<factor>

  \defitem[factor] ( \<expression> )
  \oritem \<identifier> ( \<expression\_list> )
  \oritem \<variable> [ \<expression\_list> ]
  \oritem \<variable>
  \oritem \<double>

  \defitem[expression\_list] \<expression> \rep{, \<expression> }{0}{}

  \defitem[int] \( Non negative integer )
  \defitem[double] \( Non negative double precision number )
  
\end{bnf}

\subsection{Identifiers}
% This paragraph describes the identifiers. Using text was alot easier
% than introducing some new BNF symbols, etc.
% Added July 23, 1991 by alex

Identifiers\index{identifiers} may be zero or more leading underscores
(`\_'), followed by a character, followed by any number of characters,
numbers or underscores. Punctuation characters and other special
characters such as the dollar-sign (`\$') are not permitted.  The
following, {\tt \_p1}, {\tt foo\_bar}, and {\tt \_\_P\_21\_proc} are
valid identifiers, while {\tt \_21} and {\tt \$proc} are not.

\subsection{Variables}\index{SPEX!variables}

SPEX variables all start with a dollar-sign (`\$') followed by any number of characters, numbers or
underscores (`\_').  The following, \texttt{\$s1}, \texttt{\$1}, and \texttt{\$\_x} are all valid variables.
SPEX variables are treated as global symbols by the underlying LQX program.  Variables used to store arrays
will also generate a \emph{local} variable of the same name, except without the leading dollar-sign.  

%%% Local Variables: 
%%% mode: latex
%%% TeX-master: "userman"
%%% End: 


\section{Output File Grammar}
\label{sec:output-file-bnf}

\begin{bnf}{LQN\_output\_file}
  \defitem[LQN\_output\_file] \<general> \?\<bound>?  \?\<waiting>?
  \?\<wait\_var>? \?\<snr\_waiting>?  \?\<snr\_wait\_var>? \?\<drop\_prob>? \?\<join>?
  \?\<service>? \?\<variance>? \?<exceeded>? \rep{<distribution>}{0}{} \<thpt\_ut> \?\<open\_arrivals>?
  \<processor>

  \defitem[from\_entry] \<entry\_name> \(Source entry id.)

  \defitem[to\_entry] \<entry\_name> \(Destination entry id.)

  \defitem[entry\_name] \<identifier>

  \defitem[task\_name] \<identifier>

  \defitem[proc\_name] \<identifier>

  \defitem[float\_phase\_list] \rep{\<real>}{}{} \<end\_list>

  \defitem[real] \<float> \| \<integer>
\end{bnf}

\subsection{General Information}
\label{sec:general-p}

\begin{bnf}{LQN\_output\_file}
  \defitem[general] \<valid> \<convergence> \<iterations> \<n\_processors> \<n\_phases>

  \defitem[valid] V \<yes\_or\_no>

  \defitem[yes\_or\_no] y \| Y \| n \| N

  \defitem[convergence] C \<real>

  \defitem[iterations] I \<integer>

  \defitem[n\_processors] PP \<integer>

  \defitem[n\_phases] NP \<integer>
\end{bnf}


\subsection{Throughput Bounds}
\label{sec:bounds}\index{throughput!bounds}
\begin{bnf}{multi\_server\_flag}
  \defitem[bound] B \<nt> \rep{\<bounds\_entry>}{1}{nt} \<end\_list>

  \defitem[bounds\_entry] \<task\_name> \<real>

  \defitem[nt] \<integer>\(Total number of tasks)
\end{bnf}

\subsection{Waiting Times}
\label{sec:rendezvous-delay-p}\index{rendezvous!delay}

\begin{bnf}{multi\_server\_flag}
  \defitem[waiting] W \<ne> \rep{\<waiting\_t\_tbl>}{1}{nt} \<end\_list>

  \defitem[waiting\_t\_tbl] \<task\_name> : \<waiting\_e\_tbl> \<end\_list> \opt{\<waiting\_a\_tbl>}

  \defitem[waiting\_e\_tbl] \rep{\<waiting\_entry>}{0}{ne}

  \defitem[waiting\_entry] \<from\_entry> \<to\_entry> \<float\_phase\_list>

  \defitem[ne] \<integer> \(Number of Entries)

  \defitem[waiting\_a\_tbl] : \rep{\<waiting\_activity>}{0}{na} \<end\_list>

  \defitem[waiting\_activity] \<from\_activity> \<to\_entry> \<float\_phase\_list>

  \defitem[na] \<integer> \(Number of Activities)
\end{bnf}

\subsection{Waiting Time Variance}
\label{sec:rendezvous-variance-p}\index{rendezvous!variance}

\begin{bnf}{multi\_server\_flag}
  \defitem[wait\_var] VARW \<ne> \rep{\<wait\_var\_t\_tbl>}{1}{nt} \<end\_list>

  \defitem[wait\_var\_t\_tbl] \<task\_name> : \<wait\_var\_e\_tbl> \<end\_list> \opt{\<wait\_var\_a\_tbl>}

  \defitem[wait\_var\_e\_tbl] \rep{\<wait\_var\_entry>}{0}{ne}

  \defitem[wait\_var\_entry] \<from\_entry> \<to\_entry> \<float\_phase\_list>

  \defitem[wait\_var\_a\_tbl] : \rep{\<wait\_var\_activity>}{0}{na} \<end\_list>

  \defitem[wait\_var\_activity] \<from\_activity> \<to\_entry> \<float\_phase\_list>
\end{bnf}

\subsection{Send-No-Reply Waiting Time}
\label{sec:send-no-reply-wait-p}\index{send-no-reply!delay}

\begin{bnf}{multi\_server\_flag}
  \defitem[snr\_waiting] Z \<ne> \rep{\<snr\_waiting\_t\_tbl>}{1}{nt} \<end\_list>

  \defitem[snr\_waiting\_t\_tbl] \<task\_name> : \<snr\_waiting\_e\_tbl> \<end\_list> \opt{\<snr\_waiting\_a\_tbl>}

  \defitem[snr\_waiting\_e\_tbl] \rep{\<snr\_waiting\_entry>}{0}{ne}

  \defitem[snr\_waiting\_entry] \<from\_entry> \<to\_entry> \<float\_phase\_list>

  \defitem[snr\_waiting\_a\_tbl] : \rep{\<snr\_waiting\_activity>}{0}{na} \<end\_list>

  \defitem[snr\_waiting\_activity] \<from\_activity> \<to\_entry> \<float\_phase\_list>

\end{bnf}

\subsection{Send-No-Reply Wait Variance}
\label{sec:send-no-reply-variance-p}\index{send-no-reply!variance}

\begin{bnf}{multi\_server\_flag}
  \defitem[snr\_wait\_var] VARZ \<ne> \rep{\<snr\_wait\_var\_t\_tbl>}{1}{nt} \<end\_list>

  \defitem[snr\_wait\_var\_t\_tbl] \<task\_name> : \<snr\_wait\_var\_e\_tbl> \<end\_list> \opt{\<snr\_wait\_var\_a\_tbl>}

  \defitem[snr\_wait\_var\_e\_tbl] \rep{\<snr\_wait\_var\_entry>}{0}{ne}

  \defitem[snr\_wait\_var\_entry] \<from\_entry> \<to\_entry> \<float\_phase\_list>

  \defitem[snr\_wait\_var\_a\_tbl] : \rep{\<snr\_wait\_var\_activity>}{0}{na} \<end\_list>

  \defitem[snr\_wait\_var\_activity] \<from\_activity> \<to\_entry> \<float\_phase\_list>
\end{bnf}

\subsection{Arrival Loss Probabilities}
\label{sec:arrival-loss-p}\index{open arrival!loss probability}

\begin{bnf}{multi\_server\_flag}
  \defitem[drop\_prob] DP \<ne> \rep{\<drop\_prob\_t\_tbl>}{1}{nt} \<end\_list>

  \defitem[drop\_prob\_t\_tbl] \<task\_name> : \<drop\_prob\_e\_tbl> \<end\_list> \opt{\<drop\_prob\_a\_tbl>}

  \defitem[drop\_prob\_e\_tbl] \rep{\<drop\_prob\_entry>}{0}{ne}

  \defitem[drop\_prob\_entry] \<from\_entry> \<to\_entry> \<float\_phase\_list>

  \defitem[drop\_prob\_a\_tbl] : \rep{\<drop\_prob\_activity>}{0}{na} \<end\_list>

  \defitem[drop\_prob\_activity] \<from\_activity> \<to\_entry> \<float\_phase\_list>

\end{bnf}

\subsection{Join Delays}
\label{sec:join-delay-p}\index{join!delay}

\begin{bnf}{multi\_server\_flag}

  \defitem[join] J \<ne> \rep{\<join\_t\_tbl>}{1}{nt} \<end\_list>

  \defitem[join\_t\_tbl] \<task\_name> : \<join\_a\_tbl> \<end\_list>

  \defitem[join\_a\_tbl] \rep{\<join\_entry>}{0}{na}

  \defitem[join\_entry] \<from\_activity> \<to\_activity> \<real>
  \<real>
\end{bnf}


\subsection{Service Time}
\label{sec:service-time-p}\index{service time}

\begin{bnf}{multi\_server\_flag}
  \defitem[service] X \<ne> \rep{\<service\_t\_tbl>}{1}{nt} \<end\_list>

  \defitem[service\_t\_tbl] \<task\_name> : \<service\_e\_tbl> \<end\_list> \opt{\<service\_a\_tbl>}

  \defitem[service\_e\_tbl] \rep{\<service\_entry>}{0}{ne}

  \defitem[service\_entry] \<entry\_name> \<float\_phase\_list>

  \defitem[service\_a\_tbl] : \rep{\<service\_activity>}{0}{na} \<end\_list>

  \defitem[service\_activity] \<activity\_name> \<float\_phase\_list>
\end{bnf}

\subsection{Service Time Variance}
\label{sec:service-time-variance-p}\index{service time!variance}

\begin{bnf}{multi\_server\_flag}
  \defitem[variance] VAR \<ne> \rep{\<variance\_t\_tbl>}{1}{nt}
  \<end\_list>

  \defitem[variance\_t\_tbl] \<task\_name> : \<variance\_e\_tbl> \<end\_list> \opt{\<variance\_a\_tbl>}

  \defitem[variance\_e\_tbl] \rep{\<variance\_entry>}{0}{ne}

  \defitem[variance\_entry] \<entry\_name> \<float\_phase\_list>

  \defitem[variance\_a\_tbl] : \rep{\<variance\_activity>}{0}{na} \<end\_list>

  \defitem[variance\_activity] \<activity\_name> \<float\_phase\_list>
\end{bnf}

\subsection{Probability Service Time Exceeded}
\label{sec:service-time-exceeded-p}\index{service time!exceeded}

\begin{bnf}{multi\_server\_flag}
  \defitem[variance] VAR \<ne> \rep{\<variance\_t\_tbl>}{1}{nt}
  \<end\_list>

\end{bnf}

\subsection{Service Time Distribution}
\label{sec:service-time-distribution-p}\index{service time!distribution}

\begin{bnf}{multi\_server\_flag}
  \defitem[distribution] D \<entry\_name> \<statistics> \rep{\<hist\_bin>}{0}{}  \<end\_list>
  \oritem D  \<task\_name> \<activity\_name> \<statistics> \rep{\<hist\_bin>}{0}{}  \<end\_list>

  \defitem[statistics] \<phase> \<mean> \<stddev> \<skew> \<kurtosis>

  \defitem[hist\_bin] \<begin> \<end> \<probability> \rep{\<bin\_conf>}{0}{2}

  \defitem[mean] \<real> \(Distribution mean)
  \defitem[stddev] \<real> \(Distribution standard deviation)
  \defitem[skew] \<real> \(Distribution skew)
  \defitem[kurtosis] \<real> \(Distribution kurtosis)
  \defitem[probability] \<real> \( 0.0 - 1.0 )

  \defitem[bin\_conf] \% \<conf\_level> \<real>
\end{bnf}

\subsection{Throughputs and Utilizations}
\label{sec:througput-utilization-p}\index{throughput}\index{utilization!task}

\begin{bnf}{multi\_server\_flag}
  \defitem[thpt\_ut] FQ \<nt> \rep{\<thpt\_ut\_task>}{1}{nt} \<end\_list>

  \defitem[thpt\_ut\_task] \<task\_name> \<net> \rep{<thpt\_ut\_entry>}{1}{net} \<end\_list> \?\<thpt\_ut\_task\_total>?

  \defitem[thpt\_ut\_entry] \<entry\_name> \<entry\_info> \rep{\<thpt\_ut\_confidence>}{0}{}

  \defitem[entry\_info] \<throughput> \<utilization> \<end\_list> \<total\_util>

  \defitem[throughput] \<real> \(Task Throughput)

  \defitem[utilization] \<float\_phase\_list> \(Per phase task util.)

  \defitem[total\_util] \<real>

  \defitem[thpt\_ut\_task\_total] \<entry\_info>

  \rep{\<thpt\_ut\_conf>}{0}{}

  \defitem[thpt\_ut\_conf] \% \<conf\_level> \<entry\_info>

  \defitem[conf\_level] \<integer>
\end{bnf}

\subsection{Arrival Rates and Waiting Times}
\label{sec:open-wait-p}\index{open arrival!waiting time}\index{waiting time!open arrival}

\begin{bnf}{multi\_server\_flag}
  \defitem[open\_arrivals] R \<no> \rep{\<open\_arvl\_entry>}{1}{no} \<end\_list>

  \defitem[no] \<integer> \(Number of Open Arrivals)

  \defitem[open\_arvl\_entry] \<from\_entry> \<to\_entry> \<real> \<real> \oritem \<from\_entry> \<to\_entry> \<real> Inf
\end{bnf}

\subsection{Utilization and Waiting per Phase for Processor}
\label{sec:processor-wait-utilization-p}\index{utilization!processor|textbf}\index{queueing time!processor|textbf}
\begin{bnf}{multi\_server\_flag}
  \defitem[processor] \rep{\<proc\_group>}{1}{n\_processors} \<end\_list>

  \defitem[proc\_group] P \<proc\_name> \<nt> \rep{\<proc\_task>}{1}{nt} \<end\_list> \?\<proc\_total>?

  \defitem[proc\_task] \<task\_name> \<proc\_task\_info> \rep{\<proc\_entry\_info>}{1}{ne} \<end\_list> \?\<task\_total>?

  \defitem[proc\_task\_info] \<ne> \<priority> \?\<multiplicity>?

  \defitem[priority] \<integer> \(Prio. of task)

  \defitem[multiplicity] \<integer> \(Number of task instances)

  \defitem[proc\_info] \<entry\_name> \<proc\_entry\_info> \rep{\<proc\_entry\_conf>}{0}{}

  \defitem[proc\_entry\_info] \<utilization> \<sched\_delay> \<end\_list>

  \defitem[sched\_delay] \<float\_phase\_list> \(Scheduling delay)

  \defitem[proc\_entry\_conf] \% \<integer> \<proc\_entry\_info>

  \defitem[task\_total] \<real> \rep{\<proc\_total\_conf>}{0}{}

  \defitem[proc\_total] \<real> \rep{\<proc\_total\_conf>}{0}{} \<end\_list>

  \defitem[proc\_total\_conf] \% \<integer> \<real>
\end{bnf}


%%% Local Variables: 
%%% mode: latex
%%% mode: outline-minor 
%%% fill-column: 108
%%% TeX-master: "userman"
%%% End: 

\clearpage
\bibliographystyle{plainurl}
\bibliography{userman}
%\bibliography{/home/greg/usr/bib/bibtex-full.bib,/home/greg/usr/bib/queueing,/home/greg/usr/bib/perf,/home/greg/usr/bib/sw,/home/greg/usr/bib/srvn}
\clearpage
\printindex
\end{document} 
