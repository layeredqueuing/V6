%%  -*- mode: latex; mode: outline-minor; fill-column: 108 -*-
%% Title:  model
%% $HeadURL: http://rads-svn.sce.carleton.ca:8080/svn/lqn/trunk/doc/userman/model.tex $
%%
%% Doc: /home/greg/usr/src/doc/userman/model.tex
%% Original Author:     Greg Franks <greg@sce.carleton.ca>
%% Created:             Tue Jul 18 2006
%%
%% ----------------------------------------------------------------------
%% $Id: model.tex 12977 2017-04-04 03:35:54Z greg $
%% ----------------------------------------------------------------------

\chapter{The Layered Queueing Network Model}
\label{sec:lqn}

Figure~\ref{fig:bookstore} illustrates the LQN notation with an
example of an on-line e-commerce system. In an LQN, software
resources\index{resource!software} are all called
``tasks''\index{task}, have queues\index{queue} and provide classes of
service which are called ``entries''\index{entry}.  The
demand\index{demand} for each class of service\index{service!class}
can be specified through ``phases''\index{phase}, or for more complex
interactions, using ``activities''\index{activity}.  In
Figure~\ref{fig:bookstore}, a task is shown as a parallelogram,
containing parallelograms for its entries and rectangles for
activities. Processor\index{processor} resources are shown as circles,
attached to the tasks that use them. Stacked icons\index{icon!stacked}
represent tasks or processors with multiplicity\index{multiplicity},
making it a multiserver\index{multiserver}. A multiserver may
represent a multi-threaded task, a collection of identical users, or a
symmetric multiprocessor with a common scheduler.
Multiplicity\index{multiplicity} is shown on the diagram with a label
in braces. For example there are five copies of the task `Server' in
Figure~\ref{fig:bookstore}.

\begin{figure}[htbp]
  \centering
  \epsfxsize=\textwidth
  \epsffile{bookstore/bookstore.eps}
  \caption{Notation}
  \label{fig:bookstore}
\end{figure}

Entries and activities have directed arcs\index{arcs} to other entries
at lower layers to represent service
requests\index{service!request}\index{request} (or
messages)\footnote{requests may jump over
  layers\index{layer!spanning}, such as the request from the
  Administrator task to the InventoryMgr task.}. A request from an
entry or an activity to an entry may return a
reply\index{request!reply}\index{reply} to the requester (a
synchronous request\index{request!synchronous}, or
\emph{rendezvous}\index{rendezvous}) indicated in
Figure~\ref{fig:bookstore} by solid arrows with closed arrowheads.
For example, task Administrator makes a request to task BackorderMgr
who then makes a request to task InventoryMgr. While task InventoryMgr
is servicing the request, tasks BackorderMgr and Administrator are
blocked\index{request!blocked}.  A request may be
forwarded\index{forwarding}\index{request!forward} to another entry
for later reply, such as from InventoryMgr to CustAccMgr.  Finally a
request may not return any reply at all (an asynchronous
request\index{request!asynchronous} or
\emph{send-no-reply}\index{send-no-reply}, shown as an arrow with an
open arrow head, for example, the request from task ShoppingCart to
CustAccMgr.

The first way that the demand at entries can be specified is through
phases\index{phase}.  The parameters of an
entry\index{entry!parameters} are the mean number of requests for
lower entries (shown as labels in parenthesis on the request arcs),
and the mean total host demand for the entry (in units of time, shown
as a label on the entry in brackets). An entry may continue to be busy
after it sends a reply\index{reply}, in an asynchronous ``second
phase''\index{phase!second}\index{phase!asynchronous} of
service~\cite{perf:franks-99b} so each parameter is an array of values
for the first and second phase.  Second phases are a common
performance optimization, for example for transaction cleanup and
logging, or delayed write operations.

The second way that demand can be specified is through
activities\index{activity}.  Activities are the lowest level of
granularity in a performance model and are linked together in a
directed graph to indicate precedence\index{precedence!activity}.
When a request arrives at an entry, it triggers the first activity of
the activity graph\index{activity graph}.  Subsequent activities may
follow sequentially, or may fork\index{fork} into multiple paths which
later join\index{join}.  The fork may take the form of an `AND' which
means that all the activities on the branch\index{branch!AND} after
the fork can run in parallel, or in the form of an `OR', which chooses
one of the branches\index{branch!OR} with a specified
probability\index{branch!probability}.  In Figure~\ref{fig:bookstore},
a request that is received by entry ``SCE3'' of task ``ShoppingCart''
is processed using an activity called ``SCE3A95'' that represents the
main thread\index{thread} of control, then the main thread is
OR-Forked into two branches, one of which is later
AND-forked\index{AND-fork} into three threads.  The three threads,
starting with activities `AFBA109', `AFBA130' and `AFBA133'
respectively, run in parallel.  The first thread
replies\index{reply!activity}\index{activity!reply} to the entry
through activity `OJA110' then ends.  The remaining two threads join
into one thread at activity `AJA131'.  When both `OJA110' and `AJA131'
terminate, the task can accept a new request.

The holding time\index{holding time} for one class of service is the
entry service time, which is not a constant parameter but is
determined by its lower servers.  Thus the essence of layered queuing
is a form of simultaneous resource
possession\index{resource!possession!simultaneous}. In software
systems delays and congestion are heavily influenced by
synchronous\index{request!synchronous} interactions such as remote
procedure calls\index{remote procedure call} (RPCs) or
rendezvous\index{rendezvous}, and the LQN model captures these delays
by incorporating the lower layer queueing and service into the service
time of the upper layer server.  This ``active
server''\index{server!active}\index{active server}
feature~\cite{srvn:woodside-94-ieeetc-srvn} is the key difference between layered
and ordinary queueing networks.

\section{Model Elements}
\label{sec:elements}

Figure~\ref{fig:meta-model} shows the
\emph{meta-model}\index{meta model} used to describe Layered Queueing
Networks\index{Layered Queueing Network}.  This model is unique in
that it is more closely aligned with the architecture of a software
system that it is with a conventional queueing network model such as
Performance Model Interchange
Format\index{Performance Model Interchange Format}
(PMIF)~\cite{perf:smith-99-jss-pmif,perf:smith-2004-pmif2}.  The latter consists of
stations with queues and visits, whereas a LQN has processors, tasks
and requests.

A Layered Queueing Network is a directed graph.  Nodes in the
graph\index{node} consist of tasks\index{task},
processors\index{processor}, entries\index{entry},
activities\index{activity}, and precedence\index{precedence}.
Arcs\index{arcs} in the graph consist of requests\index{request} from
one node to another.  The model objects are described below.

\begin{figure}[htbp]
  \centering
%  \epsfxsize=\textwidth
  \epsffile{model/meta-model.eps}
  \caption{LQN Meta Model}
  \label{fig:meta-model}
\end{figure}

\subsection{Processors}
\label{sec:processors}

Processors\index{processor|(textbf} are used by the
activities\index{activity} within a performance model to consume
\emph{time}.  They are \emph{pure servers}\index{server!pure} in that
they only accept requests from other servers and clients.  They can be
actual processors in the system, or may simply be place holders for
tasks representing customers and other logical resources.

Each processor has a single queue for requests.  Requests may be
scheduled using the following queueing disciplines:
\begin{description}
\item[FIFO] First-in, first out\index{scheduling!fifo} (first-come,
  first-served).  Tasks are served in the order in which they arrive.
\item[PRI] Priority, preemptive resume\index{priority!preemptive resume}\index{scheduling!pri}.  Tasks with
  priorities\index{processor!priority}\index{priority!processor} higher than the task currently running on
  the processor will preempt the running task.  Priorities\index{priority!highest} range from zero to
  positive infinity, with a priority of zero being the highest.  The default priority for all tasks is zero.
\item[HOL] Head-of-line priority\index{priority!head of line}\index{scheduling!hol}.  Tasks
  with higher priorities will be served by the processor first.  Tasks
  in the queue will not preempt a task running on the processor even
  though the running task may have a lower priority.
\item[PS] Processor sharing\index{processor!sharing|textbf}\index{scheduling!ps}.  The processor
  runs all tasks ``simultaneously''.  The rate of service by the
  processor is inversely proportional to the number of executing
  tasks.  For \emph{lqsim}\index{lqsim!scheduling}, processor
  sharing\index{scheduling!processor sharing} is implemented as
  \emph{round-robin}\index{round robin}\index{scheduling!round robin}
  -- a \emph{quantum}\index{quantum} must be specified.
\item[INF] Infinite (delay)\index{processor!delay}\index{scheduling!delay}\index{scheduling!infinite}. 
\item[RAND] Random scheduling\index{scheduling!random}.  The processor selects a task at random. 
\item[CFS] Completely fair scheduling\index{scheduling!cfs}\index{scheduling!completely fair}~\cite{perf:li-2009-mascots-fairshare}.  Tasks are scheduled within groups\index{group} using
  round-robin scheduling and groups are scheduled according to their share\index{share}.  A
  \emph{quantum}\index{quantum} must be specified.  This scheduling discipline is implemented on the
  simulator only at present.
\end{description}

Each processor can have multiple cores, all of which are served by the common queue (see
\S\protect\ref{sec:multiplicity}).  The processor
multiplicity\index{processor!multiplicity}\index{multiplicity!processor} must be a integer greater than zero
or the special constant
\texttt{@infinity}\index{multiplicity!"@infinity@\texttt{"@infinity}}\index{"@infinity@\texttt{"@infinity}}.  If
the multiplicity is set to \texttt{@infinity}, the processor is coerced to a delay
server\index{processor!delay}.  \index{processor|)}

\subsection{Groups}
\label{sec:groups}

Groups\index{group|(\textbf}\cite{perf:li-2009-mascots-fairshare} are used to divide up a processor's
execution time up into \emph{shares}\index{share}.  The tasks within a group divide the share up among
themselves evenly.  Groups can only be created on processors running the scheduling discipline
\emph{completely fair scheduling},\index{scheduling!completely fair}.  \index{group|)}.

Shares may either be \emph{guaranteed}\index{share!guarantee} or \emph{capped}\index{share!cap}.  Guarantee
shares act as a floor for the share that a group receives.  If surplus CPU time is available (i.e., the
processor is not fully utilized), tasks in a guaranteed group can exceed their share\index{share!exceed}.
Cap shares act as a hard ceiling.  Tasks within these groups will never receive more than their share of CPU
time.  

Note: Completely fair scheduling is a form of priority scheduling\index{scheduling!priority}.  With layered
models, calls made by tasks within groups to lower level servers can cause \emph{priority
  inversion}\index{priority!inversion}.  Cap scheduling tends to behave better than guaranteed scheduling
for these cases.

\subsection{Tasks}
\label{sec:tasks}

Tasks\index{task|(textbf} are used in layered queueing networks to
represent resources.  Resources include, but are not limited to:
actual tasks (or processes) in a computer system,
customers\index{customer}, buffers\index{buffers}, and hardware
devices.  In essence, whenever some entity requires some sort of
service, requests between tasks involved.

A task has a queue\index{task!queue} for requests and runs on a
processor.  Items are served from the queue in a first-come,
first-served manner.  Different classes of
service\index{service!class} are specified using
\emph{entries}\index{entry} (c.f.~\S\ref{sec:entries}).  Tasks may
also have internal concurrency\index{concurrency}, specified using
\emph{activities}\index{activity} (c.f.~\S\ref{sec:activities}).

Requests can be served using the following scheduling
methods\index{scheduling!task}:
\begin{description}
\item[FIFO] First-in, first out\index{scheduling!fifo} (first-come,
  first-served).  Requests are served in the order in which they
  arrive.  This scheduling discipline is the default for tasks.
\item[PRI] Priority, preemptive resume\index{priority!preemptive resume}\index{scheduling!pri}.  Requests
  arriving at entries with priorities\index{entry!priority}\index{priority!entry} higher than entry that
  task is currently processing will preempt the execution of the current entry.
  Priorities\index{priority!highest} range from zero to positive infinity, with a priority of zero being the
  highest.  The default priority for all entries is zero.
\item[INF] Infinite (delay)\index{task!delay}\index{scheduling!delay}\index{scheduling!infinite}. 
\item[HOL] Head-of-line priority\index{priority!head of line}\index{scheduling!hol}.
  Requests arriving at entries with higher priorities will be served
  by the task first.  Requests in the queue will not preempt the
  processing of the current entry by the task.
\end{description}

Each task can have multiple homogenous
threads\index{task!multiplicity}\index{task!threads}\index{threads!homogenous}, all of which are served by
the common queue (see \S\protect\ref{sec:multiplicity}).  The task
multiplicity\index{task!multiplicity}\index{multiplicity!task} must be a integer greater than zero or the
special constant
\texttt{@infinity}\index{multiplicity!"@infinity@\texttt{"@infinity}}\index{"@infinity@\texttt{"@infinity}}.  If
the multiplicity is set to \texttt{@infinity}, the task is coerced to a delay server\index{task!delay}.

The subclasses of \emph{task} are:
\begin{description}
\item[\emph{Reference Task:}] Reference tasks\index{task!reference}\index{reference task|textbf} are used to
  represent customers\index{customer} in the layered queueing network.  They are like normal tasks in that
  they have entries and can make requests.  However, they can never receive requests and are always found at
  the top of a call graph.  They typically generate traffic in the underlying closed queueing
  model\index{queueing model!closed} by making rendezvous\index{rendezvous!reference task} requests to
  lower-level servers.  Reference tasks can also generate traffic in the underlying open queueing
  model\index{queueing model!open} by making send-no-reply requests instead of rendezvous requests.
  However, open class customers are more typically represented using open arrivals which is simply encoded
  as a parameter to an entry.
  
  \emph{Bursty} reference tasks\index{task!reference!bursty|textbf}\index{reference task!bursty|textbf} are
  a special case of reference tasks where the service time for the slices\index{slice} are random variables
  with a \emph{Pareto} distribution\index{Pareto distribution} (c.f.~\S\ref{sec:slices}).
\item[\emph{Semaphore Task:}] Semaphore tasks\index{task!semaphore}\index{semaphore task|textbf} are used to
  model passive resources\index{resource!passive} such as buffers.  They always have two entries which are
  used to \emph{signal}\index{entry!signal}\index{signal} and \emph{wait}\index{entry!wait}\index{wait} the
  semaphore.  The wait entry must be called using a synchronous request whereas the signal entry can be
  called using any type of request.  Once a request is accepted by the wait entry, no further requests will
  be accepted until a request is processed by the signal entry.  The signal and wait entries do not have to
  called from a common task.  However, the two entries must share a common call graph, and the call graph
  must be deterministic.  The entries themselves can be defined using phases or activies and can make
  requests to other tasks.  Counting semaphores\index{semaphore!counting} can be modeled using a
  multiserver.
\item[\emph{Synch Task:}] Synchronization tasks\index{task!synchronization}
  \index{synchronization task|textbf} are used...  Cannot be a multiserver.\typeout{Write me!}
\end{description}\index{task|)}


\subsection{Entries}
\label{sec:entries}

Entries\index{entry|(textbf} service requests and are used to
differentiate the service provided by a task.  An entry can accept
either synchronous\index{message!synchronous}\index{message|see{request}}, or
asynchronous\index{message!asynchronous} requests, but not both.
Synchronous requests are part of the \emph{closed} queueing
model\index{queueing model!closed}\index{closed model} whereas
asynchronous requests are part of the \emph{open} 
model\index{queueing model!open}\index{open model}.  Message types are
described in Section~\ref{sec:requests} below.

Entries also generate the replies for synchronous
requests\index{rendezvous}.  Typically, a reply to a message is
returned to the client who originally sent the message.  However,
entries may also \emph{forward}\index{forwarding} the reply.  The next
entry which accepts the forwarded reply may forward the message in
turn, or may reply back to the originating client.  For example, in
Figure~\ref{fig:bookstore}, entry `IME8' on task `IventoryMgr'
forwards the request from entry `BME2' on task `BackorderMgr' to entry
`CAME5' on task `CustAccMgr'.  The reply from `CAME2' will be sent
directly back to `BME2'.

The parameters for an entry can be specified using either
phases\index{phase} or activities\index{activity}\footnote{The
  meta-model in Figure~\protect\ref{fig:meta-model} only shows
  activities, phases are a notational short-hand.}.  The activity
method is typically used when a task has complex internal behaviour
such as forks\index{fork} and joins\index{join}, or if its behaviour
is specified as an activity graph\index{activity graph} such as those
used by Smith and Williams~\cite{perf:smith-2002}.  The phase method
is simply a short hand notation for specifying a sequence of one to
three activities, with the
reply\index{reply!phase}\index{reply!activity} being generated by the
first activity in the sequence.  Figure~\ref{fig:entry-specification}
shows both methods for specifying a two-phase client calling a
two-phase server.

\begin{figure}[htbp]
  \centering
  \subfloat[Phases]{\epsffile{model/entry-phases.eps}}
  \subfloat[Activities]{\epsffile{model/entry-activities.eps}}
  \caption{Entry Specification}
  \label{fig:entry-specification}
\end{figure}

Regardless of the specification method used for an entry, its
behaviour as a server to its clients is by \emph{phase}\index{phase},
shown in Figure~\ref{fig:phase}.  Phases consume time on processors
and make requests to entries.  Phase one is a \emph{service phase} and
is similar to the service given by a station in a queueing network.
Phase one ends after the server sends a reply\index{reply!phase}.
Subsequent phases are \emph{autonomous}\index{autonomous phase}\index{phase!autonomous} phases
which are launched by phase one.  These phases operate in parallel
with the clients which initiated them.  The simulator and analytic
solver limit the number of phases to three.

\begin{figure}[htbp]
  \centering
  \epsffile{model/phase.eps}
  \caption{Phases for an Entry.\protect\index{phase}}
  \label{fig:phase}
\end{figure}

\subsection{Activities}
\label{sec:activities}

Activities\index{activity|textbf} are the lowest-level of
specification in the performance model.  They are connected together
using ``Precedence'' (c.f.~\S\ref{sec:precedence})\index{precedence}
to form a directed graph\index{directed graph} to represent more than
just sequential execution scenarios.

Activities consume time on processors.  The \emph{service time}\index{service time} is defined by a mean and
variance, the latter through \index{coefficient of variation}\emph{coefficient of variation squared}
\footnote{The squared coefficient of variation is variance divided by the square of the mean.}. The service
time between requests to lower level servers is assumed to be exponentially distributed (with the exception
of \emph{bursty reference tasks}\index{task!reference!bursty}) so the total service time is the sum of a
random number of exponentially distributed random variables.

Activities also make requests to entries on other tasks.  The distribution of requests to lower level
servers is set by the \emph{call order}\index{call order} for the activity which is either
\emph{stochastic}\index{stochastic} or \emph{deterministic}\index{deterministic}.  If the call order is
deterministic, the activity makes the exact number of requests specified to the lower level servers.  The
number of requests is integral; the order of requests to different entries is not defined.  If the call
order is stochastic, the activity makes a random number of requests to the lower level servers.  The mean
number of requests is specified by the value specified.  Requests are assumed to be geometrically
distributed.

For entries which accept rendezvous\index{rendezvous} requests,
replies must be generated.  If the entry is specified using phases,
the reply\index{reply!implicit} is implicit after phase
one\index{phase!reply}.  However, if the entry is specified using
activities\index{activity!reply}, one or more of the activities must
explicitly\index{reply!explicitly} generate the reply.  Exactly one
reply must be generated for each request.\index{entry|)}

\subsubsection{Slices}\index{slice|(textbf}
\label{sec:slices}

Activities consume time by making requests to the processor associated with the task.  The service time
demand specified for an activity is divided into {\em slices} between requests to other entries, shown in
the UML Sequence Diagram in Figure~\ref{fig:slices}. The mean number of slices is always $1 + Y$ where $Y$
is total total number of requests made by the activity.  

\begin{figure}[htbp]
  \centering
  \epsffile{model/slice.eps}
  \caption{Slices\protect\index{slice}.  The \emph{slice time} is shown using the label $\zeta$.}
  \label{fig:slices}
\end{figure}

By default, the demand of a \emph{slice} is assumed to be exponentially distributed~\cite{srvn:woodside-94-ieeetc-srvn}
but a variance may be specified through the \index{coefficient of variation|textbf}\emph{coefficient of
  variation squared} ($\textit{cv}^2 = \sigma^2 / \overline{s}^2$) parameter for the entry or activity.  The
method used to solve the model depends on the solver being used:
\begin{description}
\item[Analytic Solver:] All servers with $\textit{cv}^2 \ne 1$ use the HVFCFS MVA approximation
  from~\cite{queue:reiser-79}.  
\item[Simulator:] The simulator uses the following distributions for generating random variates for slice
  times provided that the task is \emph{not} a bursty reference task.
  \begin{description}
  \item[$\textit{cv}^2 = 0$:] deterministic.
  \item[$0 < \textit{cv}^2 < 1$:] gamma\index{distribution!gamma}.
  \item[$\textit{cv}^2 = 1$:] exponential\index{distribution!exponential}.
  \item[$\textit{cv}^2 > 1$:] bizarro...
  \end{description}
  If the task is a bursty reference task\index{reference task!bursty}, then the simulator generates random
  variates for slice times according to the Pareto distribution\index{distribution!Pareto}. The scale $x_m >
  0$ and shape $k > 0$ parameters for the distribution are derived from the service time $s$ and coefficient
  of variation squared $\textit{cv}^2$ parameters for the corresponding activity (or phase).
  \begin{eqnarray*}
    k & = & \sqrt{\frac{1}{\textit{cv}^2} + 1} + 1 \\
    x_m & = & s \times \frac{(k - 1)}{k}
  \end{eqnarray*}
  On-off\index{on-off behaviour} behaviour can simulated by using two or more phases at the client, where on
  phase corresponds to the on period and makes requests to other servers, while the other phase corresponds
  to the off period.
\end{description}
\index{slice|)}

\subsection{Precedence}
\label{sec:precedence}

\emph{Precedence}\index{precedence|(textbf} is used to connect
activities within a task to from an \emph{activity graph}.  Referring
to Figure~\ref{fig:meta-model}, precedence is subclassed into
`\textbf{Pre}'\index{pre-precedence} (or \emph{`join'}\index{join!precedence})
and `\textbf{Post}'\index{post-precedence} (or \emph{`fork'}\index{fork!precedence}).  To
connect one activity to another, the source activity connects to a
\emph{pre-}precedence (or a \emph{join-}list\index{join-list}).  The
\emph{pre-}precedence then connects to a \emph{post-}precedence (or a
\emph{fork-}list\index{fork-list}) which, in turn, connects to the
destination activity.  Table~\ref{tab:activity-notation} summarizes
the precedence types.

\begin{table}[htbp]
  \begin{center}
    \begingroup\makeatletter\ifx\SetFigFont\undefined%
    \gdef\SetFigFont#1#2#3#4#5{%
      \reset@font\fontsize{#1}{#2pt}%
      \fontfamily{#3}\fontseries{#4}\fontshape{#5}%
      \selectfont}%
    \fi\endgroup%
    \leavevmode
    \begin{tabular}{|m{1in}|m{3cm}|m{3.5in}|}
      \hline
      Name & Icon & Description\\
      \hline
      \hline
%
% Join lists (pre-)
%
      Sequence\index{precedence!sequence}
      & \centering\setlength{\unitlength}{4144sp}%
      \begin{picture}(1104,339)(529,107)
        \thinlines
        \put(541,119){\makebox(1080,315){}}
        \put(1081,389){\line( 0,-1){225}}
      \end{picture}%
      & Transfer of control from an activity to a join-list.\\
      \hline
      And-Join\index{precedence!and-join}\index{and~join} 
      & \centering\setlength{\unitlength}{4144sp}%
      \begin{picture}(1104,429)(529,-28)
        \thinlines
        \put(541,-16){\makebox(1080,585){}}
        \put(1081,119){\circle{180}}
        \multiput(856,344)(5.78571,-5.78571){29}{\makebox(1.5875,11.1125){\SetFigFont{5}{6}{\rmdefault}{\mddefault}{\updefault}.}}
        \put(1018,182){\vector( 1,-1){0}}
        \put(1081, 67){\makebox(0,0)[b]{\smash{{\SetFigFont{10}{12.0}{\rmdefault}{\mddefault}{\updefault}\&}}}}
        \multiput(1306,344)(-5.64286,-5.64286){29}{\makebox(1.5875,11.1125){\SetFigFont{5}{6}{\rmdefault}{\mddefault}{\updefault}.}}
        \put(1148,186){\vector(-1,-1){0}}
      \end{picture}%
      & A Synchronization point for concurrent activities.\\
      \hline
      Quorum-Join\index{precedence!quourm-join}\index{quorum~join} 
      & \centering\setlength{\unitlength}{4144sp}%
      \begin{picture}(1104,429)(529,-28)
        \thinlines
        \put(541,-16){\makebox(1080,585){}}
        \put(1081,119){\circle{180}}
        \multiput(856,344)(5.78571,-5.78571){29}{\makebox(1.5875,11.1125){\SetFigFont{5}{6}{\rmdefault}{\mddefault}{\updefault}.}}
        \put(1018,182){\vector( 1,-1){0}}
        \put(1081, 67){\makebox(0,0)[b]{\smash{{\SetFigFont{10}{12.0}{\rmdefault}{\mddefault}{\itdefault}n}}}}
        \multiput(1306,344)(-5.64286,-5.64286){29}{\makebox(1.5875,11.1125){\SetFigFont{5}{6}{\rmdefault}{\mddefault}{\updefault}.}}
        \put(1148,186){\vector(-1,-1){0}}
      \end{picture}%
      & A Synchronization point for concurrent activities where only $n$ branches must finish.\\
      \hline
      Or-Join\index{precedence!or-join}\index{or~join} 
      & \centering\setlength{\unitlength}{4144sp}%
      \begin{picture}(1104,429)(529,17)
        \thinlines
        \put(541, 29){\makebox(1080,405){}}
        \put(1081,164){\circle{180}}
        \multiput(856,389)(5.64286,-5.64286){29}{\makebox(1.5875,11.1125){\SetFigFont{5}{6}{\rmdefault}{\mddefault}{\updefault}.}}
        \put(1014,231){\vector( 1,-1){0}}
        \multiput(1306,389)(-5.67857,-5.67857){29}{\makebox(1.5875,11.1125){\SetFigFont{5}{6}{\rmdefault}{\mddefault}{\updefault}.}}
        \put(1147,230){\vector(-1,-1){0}}
        \put(1081,112){\makebox(0,0)[b]{\smash{{\SetFigFont{10}{12.0}{\rmdefault}{\mddefault}{\updefault}+}}}}
      \end{picture}%
      & \mbox{} \\
      \hline
      \hline
%
% Fork lists (post-precedence)
%
      Sequence\index{precedence!sequence}
      & \centering\setlength{\unitlength}{4144sp}%
      \begin{picture}(1104,339)(529,107)
        \thinlines
        \put(541,119){\makebox(1080,315){}}
        \put(1081,389){\vector( 0,-1){225}}
      \end{picture}%
      & Transfer of control from fork-list to activity\\
      \hline
      And-Fork\index{precedence!and-fork}\index{and~fork} 
      & \centering\setlength{\unitlength}{4144sp}%
      \begin{picture}(1104,474)(529,-28)
        \thinlines
        \put(541,-16){\makebox(1080,450){}}
        \multiput(1019,192)(-5.62069,-5.62069){30}{\makebox(1.5875,11.1125){\SetFigFont{5}{6}{\rmdefault}{\mddefault}{\updefault}.}}
        \put(856, 29){\vector(-1,-1){0}}
        \multiput(1146,189)(5.71429,-5.71429){29}{\makebox(1.5875,11.1125){\SetFigFont{5}{6}{\rmdefault}{\mddefault}{\updefault}.}}
        \put(1306, 29){\vector( 1,-1){0}}
        \put(1081,254){\circle{180}}
        \put(1081,202){\makebox(0,0)[b]{\smash{{\SetFigFont{10}{12.0}{\rmdefault}{\mddefault}{\updefault}\&}}}}
      \end{picture}%
      & Start of concurrent execution.  There can be
      any number of forked paths. \\
      \hline
      Or-Fork\index{precedence!or-fork}\index{or~fork} 
      & \centering\setlength{\unitlength}{4144sp}%
      \begin{picture}(1104,474)(529,-28)
        \thinlines
        \put(541,-16){\makebox(1080,450){}}
        \put(1081,254){\circle{180}}
        \put(1081,202){\makebox(0,0)[b]{\smash{{\SetFigFont{10}{12.0}{\rmdefault}{\mddefault}{\updefault}+}}}}
        \put(901,119){\makebox(0,0)[rb]{\smash{{\SetFigFont{8}{9.6}{\rmdefault}{\mddefault}{\updefault}$p$}}}}
        \put(1261,119){\makebox(0,0)[lb]{\smash{{\SetFigFont{8}{9.6}{\rmdefault}{\mddefault}{\updefault}$1-p$}}}}
        \multiput(1144,191)(5.78571,-5.78571){29}{\makebox(1.5875,11.1125){\SetFigFont{5}{6}{\rmdefault}{\mddefault}{\updefault}.}}
        \put(1306, 29){\vector( 1,-1){0}}
        \multiput(1017,190)(-5.75000,-5.75000){29}{\makebox(1.5875,11.1125){\SetFigFont{5}{6}{\rmdefault}{\mddefault}{\updefault}.}}
        \put(856, 29){\vector(-1,-1){0}}
      \end{picture}%
      & A branching point where one of the paths
      is selected with probability $p$.  There can be any number of branches.\\
      \hline
      Loop\index{precedence!loop}\index{loop} 
      & \centering\setlength{\unitlength}{4144sp}%
      \begin{picture}(1104,474)(529,-28)
        \thinlines
        \put(541,-16){\makebox(1080,450){}}
        \put(1081,299){\circle{180}}
        \put(1081,209){\vector( 0,-1){180}}
        \put(1017,235){\vector(-1,-1){206}}
        \put(1145,235){\vector( 1,-1){206}}
        \put(856,119){\makebox(0,0)[rb]{\smash{{\SetFigFont{8}{9.6}{\rmdefault}{\mddefault}{\updefault}$n_1$}}}}
        \put(1036, 29){\makebox(0,0)[rb]{\smash{{\SetFigFont{8}{9.6}{\rmdefault}{\mddefault}{\updefault}$n_2$}}}}
        \put(1081,226){\makebox(0,0)[b]{\smash{{\SetFigFont{10}{12.0}{\rmdefault}{\mddefault}{\updefault}*}}}}
      \end{picture}%
      & Repeat the activity an average of $n$ times.\\
      \hline
    \end{tabular}
    \caption{\label{tab:activity-notation}Activity graph notation.}
  \end{center}
\end{table}

The semantics of an activity graph\index{activity graph!semantics} are
as follows.  For AND-forks\index{AND-fork}, AND-joins\index{AND-join} and QUORUM-joins\index{QUORUM-join},
each branch of a join must originate from a common fork, and each
branch of the join must have a matching branch from the fork.
Branches\index{branch!AND} from AND-forks need not necessarily join, either explictily by a ``dangling''
thread not participating in a join, or implicitly through a quorum 
join\index{join!quorum}\index{quorum join}, where only  a subset of the branches must join while ignoring
the rest. However, all threads started by a fork must terminate before the task
will accept a new message (i.e., there is an implied join collecting
all threads at the end of a task's cycle).  Branches to an AND-join do
not necessarily have to originate from a fork -- for this case each
branch must originate from a unique entry.  This case is used to
synchronize\index{server!synchronize}\index{synchronization server}
two or more clients at the server.

For OR-forks\index{OR-fork}, the sum of the
probabilities\index{branch!probability}\index{probability!branch} of
the branches must sum to one -- there is no ``default'' operation.
AND-forks may join at OR-joins\index{OR-join}.  The threads from the
AND-fork implicitly join when the task cycle completes.  OR-joins may
be called directly from entries.  This case is analogous to running
common code for different requests to a task.

LOOPs\index{LOOP} consist of one or more
branches\index{branch!loop count}\index{loop count}, each of which is
run a random number of times with the specified mean, followed by an
optional deterministic branch\index{branch!deterministic}
exit\index{branch!exit} which is followed after all the looping has
completed.

Replies\index{activity!reply} can only occur from activities in
\emph{pre-}precedence (\emph{and-}join) lists.  Activities cannot
reply to entries from a loop branch because the number of times that a
branch is executed is a random number.\index{precedence|)}

\subsection{Requests}
\label{sec:requests}

Service requests\index{request|(textbf} from one task to another can be
one of three types: rendezvous\index{rendezvous},
forwarded\index{forwarding}, and send-no-reply\index{send-no-reply},
shown in Figure~\ref{fig:request-types}.  A rendezvous request is a
blocking synchronous request -- the client is suspended while the
server processes the request.  A send-no-reply request is an
asynchronous request -- the client continues execution after the send
takes place.  A forwarded request results when the reply to a client
is redirected to a subsequent server which, may forward the request
itself, or may reply to the originating client.\index{request|)}

\begin{figure}[htbp]
  \centering
  \subfloat[Rendezvous]{\hspace{1cm}\epsffile{model/request-rnv.eps}\hspace{1cm}}
  \subfloat[Forwarding]{\hspace{1cm}\epsffile{model/request-fwd.eps}\hspace{1cm}}
  \subfloat[Send-no-reply]{\hspace{1cm}\epsffile{model/request-snr.eps}\hspace{1cm}}
  \caption{Request Types.\index{request!types}}
  \label{fig:request-types}
\end{figure}

\section{Multiplicity and Replication}
\label{sec:replication}\label{sec:multiplicity}\index{replication|(textbf}\index{multiplicity|(textbf}

One common technique to improve the performance of a system is to add
copies of servers.  The performance model supports two techniques:
multiplicity\index{multiplicity|textbf} and
replication\index{replication|textbf}.  Multiplicity is the simpler
technique of the two as a single queue is served by multiple servers.
Replication requires a more elaborate specification because the queues
of the servers are also copied, so requests must be routed to the
various queues.  Multi-servers can be replicated.
Figure~\ref{fig:multiplicity-replication} shows the underlying
queueing models for each technique.

\begin{figure}[htbp]
  \centering
  \subfloat[Multi-server]{\hspace{1cm}\epsffile{replication/multiserver.eps}\hspace{1cm}}
  \subfloat[Replicated]{\hspace{1cm}\epsffile{replication/replicaserver.eps}\hspace{1cm}}
  \caption{Multiple copies of servers.}
  \label{fig:multiplicity-replication}
\end{figure}

Replication reduces the number of nodes in the layered queueing model
by combining tasks and processors with identical behaviour into a
single object, shown in Figure~\ref{fig:replicated-flat}.  The left
figure shows three identical clients making requests to two identical
servers.  The right figure is the same model, but specified using
replication.  Labels within angle brackets in
Figure~\ref{fig:replicated-flat}(b) denote the number of replicas.

\begin{figure}[htbp]
  \centering
  \subfloat[Flat]{\epsffile{replication/flat.eps}}
  \subfloat[Replicated]{\epsffile{replication/replicated.eps}}
  \caption{Replicated Model}
  \label{fig:replicated-flat}
\end{figure}

Replication also introduces the notion of
\emph{fan-in}\index{fan-out|textbf} and
\emph{fan-in}\index{fan-in|textbf}, denoted using the
\texttt{O=}\emph{n} and \texttt{I=}\emph{n} labels on the request from
t1 to t2 in Figure~\ref{fig:replicated-flat}(b).  Fan-out represents
the number of replicated servers that a client task calls.  Similarly,
fan-in represents the number of replicated clients that call a server.
The product of the number of clients and the fan-out to a server must
be the same as the product of the number of servers and the fan-in to
the server.  Further, both fan-in and fan-out must be integral and
non-zero.  

The total number of requests that a client makes to a server is the product of the mean number of requests
and the fan-out.  If the performance of a system is being evaluated by varying the replication parameter of
a server, the number of requests to the server must be varied inversely with the number of server replicas
in order to retain a constant number of requests from the client.
\index{replication|)}\index{multiplicity|)}

\section{A Brief History}
\label{sec:history}

LQN~\cite{perf:franks-2009-ieeese-lqn} is a combination of Stochastic Rendezvous
Networks~\cite{srvn:woodside-94-ieeetc-srvn} and the Method of Layers~\cite{perf:rolia-95-ieeese-mol}.


%%% Local Variables: 
%%% mode: latex
%%% mode: outline-minor 
%%% fill-column: 108
%%% TeX-master: "userman"
%%% End: 
