%% -*- mode: latex; mode: outline-minor: t; fill-column: 108 -*-
%% Title:  errors
%%
%% $HeadURL: http://rads-svn.sce.carleton.ca:8080/svn/lqn/branches/merge-V5-V6/doc/userman/errors.tex $
%% Original Author:     Greg Franks <greg@sce.carleton.ca>
%% Created:             Tue Jul 18 2006
%%
%% ----------------------------------------------------------------------
%% $Id: errors.tex 15611 2022-05-31 12:24:50Z greg $
%% ----------------------------------------------------------------------

\chapter{Error Messages}
\label{sec:error-messages}

Error messages are classified into four categories ranging from the
most severe to the least, they are: fatal, error, advisory and
warning.  Fatal errors will cause the program to exit immediately.
All other error messages will stop the solution of the current model
and suppress output generation.  However, subsequent input files will
be processed.  Advisory messages occur when the model has been solved,
but the results may not be correct.  Finally, warnings indicate
possible problems with the model which the solver has ignored.

\section{Fatal Error Messages}
\label{sec:fatal-error-messages}
\begin{itemize}
% /* FTL_INTERNAL_ERROR                  1 */
\item \texttt{Internal error}

  Something bad happened...

% /* FTL_NO_MEMORY                       2 */

\item \texttt{No more memory}

  A request for memory failed.  

% /* ERR_NO_OBJECTS                      3 */

\item \texttt{Model has no }\emph{(activity$|$entry$|$task$|$processor)}

  This should not happen.
  
% /* FTL_ACTIVITY_STACK_FULL           60 */
\item \texttt{Activity stack for "}\emph{identifier}\texttt{" is
    full.}
  
  The stack size\indexerror{stack size} limit for task
  \emph{identifier} has been exceeded.

% /* FTL_MSG_POOL_EMPTY                61 */
\item \texttt{Message pool is empty.  Sending from
    "}\emph{identifier}\texttt{" to "}\emph{identifier}\texttt{".}
  
  Message buffers are used when sending asynchronous send-no-reply
  messages\indexerror{send-no-reply}.  All the buffers have been
  used\indexerror{message!pool}.
\end{itemize}

\section{Error Messages}
\label{sec:normal-error-messages}

\begin{itemize}
% /* ERR_REPLICATION                   62 */
\item \emph{(task$|$processor)}\texttt{ "}\emph{identifier}\texttt{":
    Replication not supported.}\marginpar{lqsim}
  
  The simulator\index{replication!simulator} does not support
  replication\indexerror{replication}.  The model can be
  ``flattened''\index{replication!flatten} using
  \manpage{rep2flat}{1}.

% /* ERR_NON_UNITY_REPLIES               4 */
\item \emph{n.n} \texttt{Replies generated by Entry
    "}\emph{identifier}\texttt{".}
  
  This error occurs when an entry is supposed to generate a reply
  because it accepts rendezvous\index{rendezvous} requests, but the
  activity graph does not generate exactly one
  reply\indexerror{reply}.  Common causes of this error are replies
  being generated by two or more branches of an
  AND-fork\indexerror{AND-fork!reply}, or replies being generated as
  part of a LOOP\indexerror{LOOP!reply}\footnote{Replies cannot be
    generated by branches of loops because the number of iterations of
    the loop is random, not deterministic}.

% /* ERR_IS_START_ACTIVITY               5 */
\item \texttt{Activity "}\emph{identifier}\texttt{" is a start activity.}
  
  The activity named \emph{identifier} is the first activity in an
  activity graph\indexerror{start activity}.  It cannot be used in a
  \emph{post-}precedence\indexerror{post-precedence}
  (\emph{fork-}list\indexerror{fork-list}).

% /* ERR_DUPLICATE_ACTIVITY_RVALUE       6 */
\item \texttt{Activity "}\emph{identifier}\texttt{" previously used in a fork." }
  
  The activity \emph{identifier} has already been used as part of a
  fork expression\indexerror{fork-list}.  Fork lists are on the right
  hand side of the \texttt{->}\index{->@\texttt{->}} operator in the old
  grammar, and are the
  \emph{post}-precedence\indexerror{post-precedence} expressions in
  the XML grammar.  This error will cause a loop in the activity
  graph.

% /* ERR_DUPLICATE_ACTIVITY_LVALUE       7 */
\item \texttt{Activity "}\emph{identifier}\texttt{" previously used in a join." }
  
  The activity \emph{identifier} has already been used as part of a
  join list\indexerror{join-list}.  Join lists are on the left hand
  side of the \texttt{->}\index{\texttt{->}} operator in the old
  grammar, and are the
  \emph{pre}-precedence\indexerror{pre-precedence} expressions in the
  XML grammar.  This error will cause a loop in the activity graph.

% /* ERR_REPLY_NOT_FOUND               63 */
\item \texttt{Activity "}\emph{identifier}\texttt{" requests reply for
    entry "}\emph{identifier}\texttt{" but none pending.}\marginpar{lqsim}
  
  The simulator is trying to generate a reply\indexerror{reply} from
  entry \emph{identifier}, but there are no messages queued at the
  entry.  This error usually means there is a logic error in the
  simulator.

% /* ERR_LQX_COMPILATION                    */
\item \texttt{An error occured while compiling the LQX program found in file: }\emph{filename}\texttt`{.}\marginpar{lqx}
  
  A syntax error was found in the LQX\indexerror{LQX} program found in the file \emph{filename}.  Refer to
  earlier error messages.

% /* ERR_LQX_EXECUTION                      */
\item \texttt{An error occured  executing the LQX program found in file: }\emph{filename}\texttt{.}\marginpar{lqx}
  
  A error occured while executing the the LQX\indexerror{LQX} program found in the file \emph{filename}.
  Refer to earlier error messages.

% /* ERR_ATTRIBUTE_MISSING               8 */
\item \texttt{Attribute "}\emph{attribute}\texttt{" is missing from "}\emph{type}\texttt{" element.}

  The attribute named \emph{attribute} for the \texttt{type}-element
  is missing\indexerror{attribute!missing}.
  
%  ---- SCHEMA
\item \texttt{Attribute '}\emph{attribute}\texttt{' is not declared for element '}\emph{element}\texttt{'}
  
  The attribute named \emph{attribute} for \emph{element} is not defined in the
  schema.\indexerror{attribute!not declared}.  

% /* ERR_LQX_SPEX			    */
\item \texttt{"Both LQX and SPEX found in file } \emph{filename}
  \texttt{.  Use one or the other."}

  XML input allows for the use of LQX or SPEX, but not both at the same time.\indexerror{lqx!spex}\indexerror{spex!lqx}
  
% /* ERR_CANNOT_CREATE_X               64 */
\item \texttt{Cannot create }\emph{(processor$|$processor for task$|$task})
  \texttt{"}\emph{identifier}\texttt{".}\marginpar{lqsim}
  
  Parasol\indexerror{Parasol} could not create an object such as a
  task\indexerror{task creation} or
  processor\indexerror{processor!creation}.

% /* ERR_CYCLE_IN_ACTIVITY_GRAPH         9 */
\item \texttt{Cycle in activity graph for task
    "}\emph{identifier}\texttt{", back trace is
    "}\emph{list}\texttt{".}
  
  There is a cycle in the activity graph for the task named
  \emph{identifier}\indexerror{cycle!activity graph}.  Activity graphs
  must be acyclic.  \emph{List} identifies the activities found in the
  cycle.

% /* ERR_CYCLE_IN_CALL_GRAPH            10 */
\item \texttt{Cycle in call graph,  backtrace is "}\emph{list}\texttt{".}
  
  There is a cycle in the call graph\indexerror{cycle!call graph}
  indicating either a possible deadlock\index{deadlock} or livelock
  condition.  A deadlock can occur if the same task, but via a
  different entry, is called in the cycle of
  rendezvous\index{rendezvous!cycle} indentified by \emph{list}.  A
  livelock\index{livelock} can occur if the same task and entry are
  found in the cycle.

  In general, call graphs must be acyclic.  If a deadlock condition is
  identified, the \pragma{cycles=allow} pragma can be used to suppress
  the error.  Livelock conditions cannot be suppressed as these
  indicate an infinite loop in the call graph\index{infinite loop!call graph}.

% /* ERR_TOO_MANY_PHASES                11 */
\item \texttt{Data for }\emph{n}\texttt{ phases specified.  Maximum permitted is }\emph{m}\texttt{.}
  
  The solver only supports \emph{m} phases (typically
  3)\indexerror{maximum phases}; data for \emph{n} phases was
  specified.  If more than \emph{m} phases need to be specified, use
  activities to define the phases.

% --- Schema --- 
\item \texttt{Datatype error: Type:InvalidDatatypeValueException,
    Message:}\emph{message}

% /* ERR_DELAY_MULTIPLY_DEFINED        65 */
\item \texttt{Delay from processor "}\emph{identifier}\texttt{" to
    processor "}\emph{identifier}\texttt{" previously specified.}\marginpar{lqsim}

  Inter-processor delay...\typeout{Bonne mots: inter-processor delay}

% /* ERR_BOGUS_COPIES                  61 */
\item \texttt{Derived population of }\emph{n.n}\texttt{ for task
    "}\emph{identifier}\texttt{" is not valid."}\marginpar{lqns}
  
  The solver finds populations for the
  clients\indexerror{population!infinite} in a given submodel by
  traversing up the call graphs from all the servers in the submodel.
  If the derived population is infinite\index{submodel!population},
  the submodel cannot be solved.  This error usually arises when open
  arrivals\indexerror{open arrival} are accepted by infinite
  servers\index{server!infinite}\indexerror{infinite server}.

% /* ERR_SRC_EQUALS_DST                 12 */
\item \texttt{Destination entry "}\emph{dst-identifier}\texttt{" must
    be different from source entry "}\emph{src-identifier}\texttt{".}
  
  This error occurs when \emph{src-identifier} and
  \emph{dst-identifier} specify the same
  entry.\indexerror{entry!different}

% /* ERR_NON_INTEGRAL_CALLS             13 */
\item \texttt{Deterministic phase "}\emph{src-identifier}\texttt{"
    makes a non-integral number of calls (}\emph{n.n}\texttt{) to
    entry }\emph{dst-identifier}\texttt{.}
  
  This error occurs when a deterministic
  phase\indexerror{phase!deterministic} or activity makes a
  non-integral number of calls to some other entry.

% --- SCHEMA_DUPLICATE_UNIQUE_VALUE --- 
\item \texttt{Duplicate unique value declared for identity constraint
    of element '}\emph{task}\texttt{'.}
  
  One or more activities are being bound to the same
  entry\indexerror{duplicate!unique value}.  This is not allowed, as
  an entry is only allowed to be bound to one activity.  Check all
  \attribute{bound-to-entry} attributes for all activities to ensure
  this constraint is being met.

% --- SCHEMA --- 
\item \texttt{Duplicate unique value declared for identity constraint
    of element '}\emph{lqn-model}\texttt{'.}
  
  This error indicated that an element has a duplicate
  name\indexerror{element!duplicate name}\indexerror{duplicate!unique value}
  -- the parser gives the line number to the start of the second
  instance of duplicate element.  The following elements must have
  unique name attributes\index{attribute!unique name}, but the
  uniqueness does not span elements.  Therefore a processor and task
  element can have the same name attribute, but two processor elements
  cannot have the same name attribute.
  
  The following elements must have a unique \attribute{name}
  attribute:
  \begin{itemize}
  \item processor
  \item task
  \item entry
  \end{itemize}

% Value '}\emph{value}\texttt{' does not match any member types (of the union).
% Value 'fred' is not in enumeration .

\item \texttt{Empty content not valid for content
    model:'}\emph{element}\texttt{'}

(result-processor,task)

% /* ERR_OPEN_AND_CLOSED_CLASSES        14 */
\item \texttt{Entry "}\emph{identifier}\texttt{" accepts both
    rendezvous and send-no-reply messages.}

  An entry can either accept synchronous messages (to which it
  generates replies), or asynchronous messages (to which no reply is
  needed), but not both.  Send the requests to two separate
  entries.\indexerror{entry!message type}

% /* ERR_INVALID_FORWARDING_PROBABILITY 15 */
\item \texttt{Entry "}\emph{identifier}\texttt{" has invalid
    forwarding probability of }\emph{n.n}\texttt{.}
  
  This error occurs when the sum of all forwarding
  probabilities\indexerror{forwarding!probability} from the entry
  \emph{identifier} is greater than 1\index{probability!forwarding}.

% /* ERR_WRONG_TASK_FOR_ENTRY           16 */
\item \texttt{Entry "}\emph{entry-identifier}\texttt{"  is not part of task
    "}\emph{task-identifier}\texttt{".}
  
  An activity graph part of task \emph{task-identifer}
  replies\indexerror{activity!reply} to \emph{entry-identifier}.
  However, \emph{entry-identifier} belongs to another task.

% /* ERR_ENTRY_NOT_SPECIFIED            17 */
\item \texttt{Entry "}\emph{identifier}\texttt{" is not specified.}
  
  An entry is declared but not defined, either using phases or
  activities\indexerror{entry}.  An entry is
  ``defined''\index{entry!defined} when some parameter such as service
  time is specified.

% /* ERR_REPLY_NOT_GENERATED            18 */
\item \texttt{Entry "}\emph{identifier}\texttt{" must reply; the reply
    is not specified in the activity graph.}
  
  The entry \emph{identifier} accepts rendezvous
  requests\indexerror{activity!reply}.  However, no reply is specified
  in the activity graph.

% /* ERR_MIXED_ENTRY_TYPES              19 */
\item \texttt{Entry "}\emph{identifier}\texttt{" specified using both
    activity and phase methods.}
  
  Entries can be specified either using phases\index{entry!phase}, or
  using activities\index{entry!activity}, but not
  both.\indexerror{entry!type}.

% /* ERR_MIXED_SPECIAL_ENTRY_TYPES              19 */
\item \texttt{Entry "}\emph{identifier}\texttt{" specified as both a signal and wait.}
  
  A semaphore task\index{task!semaphore}\index{semaphore task} must have exactly one
  signal\index{entry!signal}\index{signal} and one wait\index{entry!wait}\index{wait} entry.  Both entries
  have the same type.\indexerror{entry!type}.

% --- SCHEMA ----
% Expected end of tag 'synch-call'

\item \texttt{Expected end of tag '}\emph{element}\texttt{'}

  The closing tag\indexerror{tag!end} for \emph{element} was not found in the input file.

% /* ERR_EXTERNAL_SYNC                 62 */
\item \texttt{External synchronization not supported for task
    "}\emph{identifier}\texttt{" at join
    "}\emph{join-list}\texttt{".}\marginpar{lqns}
  
  The analytic solver does not implement external synchronization\indexerror{synchronization}.

% /* ERR_LQX_VARIABLE_RESOLUTION            */
\item \texttt{External variables are present in file "}\emph{filename}\texttt{, but there is no LQX program
    to resolve them.}\marginpar{lqx}
  
  The input model contains a variable\indexerror{external variable} of the form ``\texttt{\$var}'' as a
  parameter such as a service time, multiplicty, or rate.  The variables are only assigned values when an
  LQX program executes.  Since no LQX program\indexerror{LQX!execution} was present in the model file, the
  model cannot be solved.

% /* ERR_FANIN_MISMATCH                63 */
\item \texttt{Fan-ins from task "}\emph{from-identifier}\texttt{" to
    task "}\emph{to-identifier}\texttt{" are not identical for all
    calls.}\marginpar{lqns}
  
  All requests\indexerror{fan-in} made from task
  \emph{from-identifier} to task \emph{to-identifier} must have the
  same fan-in and fan-out values\indexerror{fan-out}.

% /* ERR_REPLICATION                   68 */
\item \texttt{Fan-out from }\emph{(activity$|$entry$|$task)}\texttt{
    "}\emph{src-identifier}\texttt{" (}\emph{n}\texttt{ *
  }\emph{n}\texttt{ replicas) does not match fan-in to
  }\emph{(entry$|$processor)}\texttt{ "}\emph{dst-identifier}\texttt{"
    (}\emph{n}\texttt{ * }\emph{n}\texttt{).}\marginpar{lqns}
  
  This error occurs when the number of replicas\indexerror{replication}
  at \emph{src-identifier} multiplied by the
  fan-out\indexerror{fan-out} for the request to \emph{dst-identifier}
  does not match the number of replicas at \emph{dst-identifier}
  multiplied by the fan-in\indexerror{fan-in} for the request from
  \emph{src-identifier}. A fan-in or fan-out of zero (a common error
  case) can arise when the ratios of tasks to processors is
  non-integral\index{replication!ratio}.

% /* ERR_LESS_ENTRIES_THAN_TASKS        20 */
\item \texttt{Fewer entries defined (}\emph{n}\texttt{) than tasks
    (}\emph{m}\texttt{).}
  
  A model was specified with more tasks than entries.  Since each task
  must have at least one entry, this model is
  invalid\indexerror{model}.

% /* ERR_NO_TASK_DEFINED_FOR_GROUP        */
\item \texttt{Group "}\emph{identifier}\texttt{" has no tasks.}
  \indexerror{group!tasks}
  
  The group named by \emph{identifier} has no tasks assigned to it.  A group requires a minimum of one task.

% /* ERR_INVALID_SHARE                    */ 
\item \texttt{Group "}\emph{identifier}\texttt{" has invalid share of }\emph{n.n}\texttt{.}
  \indexerror{group!share}

  The share\index{group!share} value of \emph{n.n} for group \emph{identifier} is not between the range of $
  0 < n.n <= 1.0$.

% /* ERR_INFINITE_THROUGHPUT           64 */
\item \texttt{Infinite throughput for task
    "}\emph{identifier}\texttt{".  Model specification
    error.}\marginpar{lqns}
  
  The response time\indexerror{response time} for the task
  \emph{identifier} is zero\indexerror{throughput!infinite}.  The
  likely cause is zero service time for all calls made by the task.

% /* ERR_INIT_DELAY                    66 */
\item \texttt{Initial delay of }\emph{n.n}\texttt{ is too small,
  }\emph{n}\texttt{ client(s) still running.}\marginpar{lqsim}
  
  This error occurs when the \emph{initial-loops}\index{initial-loops}
  parameter for automatic
  blocking\index{automatic blocking}\index{block!automatic} is too
  small.

% /* ERR_INVALID_FANIN                 65 */
\item \texttt{Invalid fan-in of }\emph{n}\texttt{: source task
    "}\emph{identifier}\texttt{" is not replicated.}\marginpar{lqns}
  
  The fan-in\indexerror{fan-in} value for a request specifies the
  number of replicated\indexerror{replication} source tasks making a
  call to the destination.  To correct this error, the source task
  needs to be replicated by a multiple of $n$.

% /* ERR_INVALID_FANOUT                66 */
\item \texttt{Invalid fan-out of }\emph{n}\texttt{: destination task
    "}\emph{identifier}\texttt{" has only }\emph{m}\texttt{
    replicas.}\marginpar{lqns}
  
  The fan-out\indexerror{fan-out} value $n$ is larger than the number
  of destination tasks $m$.  In effect, the source will have more than
  one request arc to the destination.

% /* ERR_JOIN_BAD_PATH                  21 */
\item \texttt{Invalid path to join "}\emph{join-list}\texttt{" for
    task "}\emph{identifier}\texttt{": backtrace is
    "}\emph{list}\texttt{".}
  
  The activity graph\indexerror{activity graph} for task
  \emph{identifer} is invalid because the branches to the
  join\indexerror{join} \emph{join-list} do not all originate from the
  same fork\indexerror{fork}.  \emph{List} is a dump of the activity
  stack when the error occurred.

% /* ERR_INVALID_PROBABILITY            22 */
\item \texttt{Invalid probability of }\emph{n.n}\texttt{.}
  
  The probability\indexerror{probability} of \emph{n.n} is not between
  the range of zero to one inclusive.  The likely cause for this error
  is the sum of the
  probabilities\index{branch!probability}\index{forwarding!probability}\index{probability!branch}\index{probability!forwarding}
  either from an OR-fork\index{OR-fork}, or from
  forwarding\index{forwarding} from an entry, is greater than one.

% /* ERR_MULTIPLY_DEFINED               23 */
\item \texttt{Name "}\emph{identifier}\texttt{" previously defined.}
  
  The symbol
  \emph{identifer}\indexerror{duplicate!identifier}\index{identifier!duplicate}
  was previously defined.  Tasks, processors and entries must all be
  named uniquely.  Activities must be named uniquely within a task.

% /* ERR_NO_REFERENCE_TASKS             24 */
\item \texttt{Model has no reference tasks.}
  
  There are no reference tasks\indexerror{reference task} nor are there
  any tasks with open arrivals\indexerror{open arrival} specified in
  the model.  Reference tasks serve as customers for closed queueing
  models\index{queueing model!customers}\index{queueing model!closed}.
  Open-arrivals serve as sources for open queueing
  models\index{queueing model!open}.

% /* ERR_NO_CALLS_TO_ENTRY             67 */
\item \texttt{No calls from }\emph{(entry$|$activity)}\texttt{
    "}\emph{from-identifier}\texttt{" to entry
    "}\emph{to-identifier}\texttt{".}\marginpar{lqns}
  
  This error occurs when the fan-in\indexerror{fan-in} or
  fan-out\indexerror{fan-out} parameter for a request are specified
  \emph{before} the actual request type.  Switch the order in the
  input file.

% /* ERR_NO_QUANTUM_SCHEDULING            */
\item \texttt{No group specified for task "}\emph{task\_identifier}\texttt{" running on processor
    "}\emph{proc\_identifier}\texttt{" using fair share scheduling.}

  Task \emph{task\_identifier} has no group specified, yet it is running on processor
  \emph{proc\_identifier} which is using completely fair scheduling\indexerror{scheduling!completely fair}.

% /* ERR_NO_SEMAPHORE                 26 */
\item \texttt{No signal or wait specified for semaphore task "}\emph{identifier}\texttt{".}
  
  Task \emph{identifier} has been identified as a semaphore\indexerror{semaphore task}
  task\index{task!semaphore}, but neither of its entries has been designated as a
  signal\index{entry!signal}\index{signal} or a wait\index{entry!wait}\index{wait}.

% /* ERR_NON_REF_THINK_TIME             25 */
\item \texttt{Non-reference task "}\emph{identifier}\texttt{" cannot
    have think time.}
  
  A think time\indexerror{think time} is specified for a non-reference
  task.  Think times for non-reference tasks can only be specified by
  entry\index{think time!entry}.

% /* ERR_NOT_SEMAPHORE_TASK */
\item \texttt{Non-semaphore task "}\emph{identifer}\texttt{" cannot have a }\emph{(signal|wait)}\texttt{ for
    entry "}\emph{entry}\texttt{".}

  The \emph{entry} is designated as either a signal\index{entry!signal}\index{signal} or a
  wait\index{entry!wait}\index{wait}.  However, \emph{identifier} is not a semaphore
  task\index{task!semaphore}\indexerror{semaphore task}. 

% /* ERR_TOO_MANY_X                     26 */
\item \texttt{Number of }\emph{(entries$|$tasks$|$processors)}\texttt{
    is outside of program limits of (1,}\emph{n}\texttt{).}

  An internal program limit has been exceeded.  Reduce the number of
  objects in the model\indexerror{program limit}.

% /* ERR_JOIN_PATH_MISMATCH             27 */
\item \texttt{Number of paths found to AND-Join
    "}\emph{join-list}\texttt{" for task "}\emph{identifier}\texttt{"
    does not match join list." }
  
  During activity graph traversal, one or more of the
  branches\index{branch!AND}\indexerror{join} to the join
  \emph{join-list} either originate from different forks, or do not
  originate from a fork at all\indexerror{fork}.

% /* ERR_ARRIVAL_RATE                  60 */
\item \texttt{Open arrival rate of }\emph{n.n}\texttt{ to task
    "}\emph{identifier}\texttt{" is too high.  Service time is
  }\emph{n.n}\texttt{.}\marginpar{lqns}
  
  The open arrival rate of $n.n$ to entry
  \emph{identifier}\indexerror{open arrival} is too high, so the input
  queue to the task has overflowed.  This error may be the result of a
  transient condition, so the \pragma{stop-on-message-loss} pragma
  (c.f. \S\ref{sec:lqns-pragmas}) may be used to suppress this error.
  If the arrival rate exceeds the service time at the time the model
  converges, then the waiting time results for the entry will show
  infinity\index{infinity}.  Note that if a task accepts both open and
  closed classes\index{class!open}\index{class!closed}, an overflow in
  the open class will result in zero throughput\index{throughput!zero}
  for the closed classes.

% /* ERR_MISSING_OR_BRANCH              28 */
\item \texttt{OR branch probabilities for OR-Fork
    "}\emph{list}\texttt{" for task "}\emph{identifier}\texttt{"
    do not sum to 1.0; sum is }\emph{n.n}\texttt{.}
  
  All branches from an or-fork\indexerror{OR-fork}\index{branch!OR}
  must be specified so that the sum of the
  probabilities\index{branch!probability} equals one.

% /* ERR_INVALID_PROC_RATE              29 */
\item \texttt{Processor "}\emph{identifier}\texttt{" has invalid rate of }\emph{n.n}\texttt{.}
  
  The processor rate\indexerror{processor!rate} parameter is used to scale the speed of the processor.  A
  value greater than zero must be used.

% /* ERR_NO_GROUP_DEFINED_FOR_PR0CESSOR   */

\item \texttt{Processor "}\emph{identifier}\texttt{" using CFS scheduling has no group."}
  
  If the completely fair share\indexerror{scheduling!completely fair} scheduler is being used, there must
  be at least one group\indexerror{group} defined for the processor.

% /* ERR_REF_TASK_FORWARDING            31 */
\item \texttt{Reference task "}\emph{identifier}\texttt{" cannot forward requests.}
  
  Reference tasks\indexerror{reference task} cannot accept messages,
  so they cannot forward\indexerror{forward}.

% /* ERR_REFERENCE_TASK_OPEN_ARRIVALS   32 */
\item \texttt{Reference task "}\emph{task-identifier}\texttt{", entry
    "}\emph{entry-identifier}\texttt{" cannot have open arrival
    stream.}
  
  Reference tasks\indexerror{reference task} cannot accept
  messages.\indexerror{open arrival}

% /* ERR_REFERENCE_TASK_IS_RECEIVER     33 */
\item \texttt{Reference task "}\emph{task-identifier}\texttt{", entry
    "}\emph{entry-identifier}\texttt{" receives requests.}
  
  Reference tasks\indexerror{reference task} cannot accept messages.

% /* ERR_REFERENCE_TASK_REPLIES         34 */
\item \texttt{Reference task "}\emph{task-identifier}\texttt{",
    replies to entry "}\emph{entry-identifier}\texttt{" from activity
    "}\emph{activity-identifier}\texttt{)".}
  
  Reference tasks\indexerror{reference task} cannot accept messages,
  so they cannot generate replies.  The activity
  \emph{activity-identifier} replies to entry
  \emph{entry-identifier}.\indexerror{reply}

% --- SCHEMA
\item \texttt{Required attribute '}\emph{attribute}\texttt{' was not
    provided} 
  
  The attribute named \emph{attribute} is
  missing\indexerror{attribute!missing} for the element.

% /* ERR_ASYNC_REQUEST_TO_WAIT	    */
\item \texttt{Semaphore "wait" entry "}\emph{entry-identifier}\texttt{" cannot accept send-no-reply requests.}

  An entry designated as the semaphore
  ``wait''\index{semaphore!wait}\indexerror{wait} can only accept
  rendezvous-type messages because send-no-reply messages and open
  arrivals cannot block the caller if the semaphore is busy.

% /* ERR_DUPLICATE_START_ACTIVITY       36 */
\item \texttt{Start activity for entry
    "}\emph{entry-identifier}\texttt{" is already defined.  Activity
    "}\emph{activity-identifier}\texttt{" is a duplicate.}
  
  A start activity\index{activity!start} has already been defined.
  This one is a duplicate\indexerror{duplicate!start activity}.

% /* ERR_NOT_DEFINED                    35 */
\item \texttt{Symbol "}\emph{identifier}\texttt{" not previously
    defined.}
  
  All identifiers must be declared before they can be
  used\indexerror{not defined}.

% /* ERR_INFINITE_TASK              30 */
\item \texttt{Task "}\emph{identifier}\texttt{" cannot be an
    infinite server." }
  
  This error occurs whenever a reference task\indexerror{reference task}\index{task!reference} or a
  semaphore task\indexerror{semaphore task}\index{task!semaphore} is designated as an infinite server.
  Reference tasks are the customers in the model so an infinite reference task would imply an infinite
  number of customers\footnote{An infinite source of customers should be represented by open
    arrivals\index{open arrival} instead.}.  An infinite semaphore task implies an infinite number of
  buffers -- no blocking at the wait entry\index{entry!wait} would ever occur.

% /* ERR_NO_START_ACTIVITIES            36 */
\item \texttt{Task "}\emph{identifier}\texttt{" has activities but
    none are reachable.}
  
  None of the activities for \emph{identifier} is reachable.  The most
  likely cause is that the start activity is
  missing\index{activity!start}\indexerror{start activity}\indexerror{not reachable}.

% /* ERR_NO_ENTRIES_DEFINED_FOR_TASK    37 */
\item \texttt{Task "}\emph{identifier}\texttt{" has no entries.}
  
  No entries were defined for \emph{identifier}\indexerror{entry}.

% /* ERR_ENTRY_COUNT_FOR_TASK           42 */
\item \texttt{"Task "}\emph{identifier}\texttt{" has }\emph{n}\texttt{ entries defined, exactly
  }\emph{m}\texttt{ are required.}
  
  The task \emph{identifier} has \emph{n} entries, \emph{m} are required. This error typically occurs with
  semaphore tasks\indexerror{semaphore task}\index{task!semaphore} which must have exactly two entries.

% /* ERR_ACTIVITY_NOT_SPECIFIED         38 */
\item \texttt{Task "}\emph{task-identifier}\texttt{", Activity
    "}\emph{activity-identifer}\texttt{" is not specified.}
  
  An activity is declared but not defined.\indexerror{activity}.  An
  activity is ``defined''\index{activity!defined} when some parameter
  such as service time is specified.

% /* ERR_DUPLICATE_REPLY                39 */
\item \texttt{Task "}\emph{task-identifier}\texttt{", Activity
    "}\emph{activity-identifer}\texttt{" makes a duplicate reply for
    Entry "}\emph{entry-identifier}\texttt{".}
  
  An activity graph is making a reply to entry \emph{entry-identifier}
  even though the entry is already in phase
  two\indexerror{reply!duplicate}.  This error usually occurs when
  more than one reply to \emph{entry-identifier} is specified in a
  sequence of activities.

% /* ERR_INVALID_REPLY                  40 */
\item \texttt{Task "}\emph{task-identifier}\texttt{", Activity
    "}\emph{activity-identifer}\texttt{" makes invalid reply for Entry
    "}\emph{entry-identifier}\texttt{".}
  
  An activity graph is making a reply to entry \emph{entry-identifier}
  even though the activity is not
  reachable.\indexerror{reply!invalid}\indexerror{activity!not reachable}.

% /* ERR_REPLY_SPECIFIED_FOR_SNR_ENTRY  41 */
\item \texttt{Task "}\emph{task-identifier}\texttt{", Activity
    "}\emph{activity-identifer}\texttt{" replies to Entry
    "}\emph{entry-identifier}\texttt{" which does not accept rendezvous requests.}
  
  The activity graph specifies a reply\indexerror{reply!invalid} to
  entry \emph{entry-identifier} even though the entry does not accept
  rendezvous requests.  (The entry either accepts send-no-reply
  requests or open arrivals\index{open arrival}).

% --- Schema ---
\item \texttt{Unknown element '}\emph{element}\texttt{'}
  
  The \emph{element}\indexerror{element!unkown} is not expected at this point in the input file.
  \emph{Element} may not be spelled incorrectly, or if not, in an incorrect location in the input file.

\end{itemize}

\section{Advisory Messages}
\label{sec:advisory-messages}

\begin{itemize}
% /* ADV_CONVERGENCE_VALUE             69 */
\item \texttt{Invalid convergence value of }\emph{n.n}\texttt{, using
  }\emph{m.m}\texttt{.}\marginpar{lqns}
  
  The convergence value\indexerror{convergence!value} specified in the
  input file is not valid.  The analytic solver is using $m.m$
  instead.

% /* WRN_BOGUS_STDDEV                  68 */
\item \texttt{Invalid standard deviation: sum=}\emph{n.n}\texttt{,
    sum\_sqr=}\emph{n.n}\texttt{, n=}\emph{n.n}\texttt{.}
  
  When calculating a standard
  deviation\indexerror{standard deviation}, the difference of the sum
  of the squares and the mean of the square of the sum was negative.
  This usually implies an internal error in the simulator.

% /* ADV_ITERATION_LIMIT               72 */
\item \texttt{Iteration limit of }\emph{n}\texttt{ is too small, using
  }\emph{m}\texttt{.}\marginpar{lqns}
  
  The iteration limit\indexerror{iteration limit} specified in the
  input file is not valid.  The analytic solver is using $m$ instead.

% /* ADV_MESSAGES_DROPPED              42 */
\item \texttt{Messages dropped at task }\emph{identifier}\texttt{ for
    open-class queues.}
  
  Asynchronous send-no-reply\indexerror{send-no-reply} messages were
  \emph{lost} at the task \emph{task}.  This message will occur when
  the \pragma{stop-on-message-loss} pragma
  (c.f.~\S\ref{sec:lqns-pragmas}) is set to ignore open class
  overflows.  Note that if a task accepts both open and closed
  classes\index{class!open}\index{class!closed}, an overflow in the
  open class will result in zero throughput\index{throughput!zero} for
  the closed classes.

% /* ADV_SOLVER_ITERATION_LIMIT        75 */
\item \texttt{Model failed to converge after }\emph{n}\texttt{
    iterations (convergence test is }\emph{n.n}\texttt{, limit is
  }\emph{n.n}\texttt{).}\marginpar{lqns}
  
  Sometimes the model fails to converge\indexerror{convergence},
  particularly if there are several heavily utilized
  servers\index{utilization!high} in a submodel.  Sometimes, this
  problem can be solved by reducing the value of the
  under-relaxation\index{under-relaxation} coefficient.  It may also
  be necessary to increase the iteration-limit\index{iteration limit},
  particularly if there are many submodels.  With replicated
  models\index{replication!convergence}, it may be necessary to use
  `loose' layering\index{layering!loose} to get the model to converge.
  Convergence can be tracked using \flag{t}{convergence}.

% /* ADV_LQX_IMPLICIT_SOLVE               */
\item \texttt{No solve() call found in the lqx program in file: }\emph{filename}\texttt{.  solve() was
    invoked implicitly.}
  
  An LQX\index{lqx} program was found in file \emph{filename}.  However, the function
  \texttt{solve()}\index{solve()!implicit} was not invoked explictity.  The program was executed to
  completion, after which \texttt{solve()} was called using the final value of all the variables found in
  the program.
  
% /* ADV_REPLICATION_ITERATION_LIMIT   73 */
\item \texttt{Replicated Submodel }\emph{n}\texttt{ failed to converge
    after }\emph{n}\texttt{ iterations (convergence test is
  }\emph{n.n}\texttt{, limit}\linebreak[4] \texttt{is }\emph{m.m}\texttt{).}\marginpar{lqns}
  
  The inner ``replication''
  iteration\indexerror{replication!iteration} failed to
  converge....\typeout{Bonne mots: replication convergence}

% /* ADV_SERVICE_TIME_RANGE            74 */
\item \texttt{Service times for }\emph{(processor)}
  \emph{identifier}\texttt{ have a range of }\emph{n.n}\texttt{ -
  }\emph{n.n}\texttt{. Results may not be valid.}\marginpar{lqns}
  
  The range of values of service times\indexerror{service time} for a
  processor using processor sharing\index{processor!sharing}
  scheduling\indexerror{processor!sharing} is over two orders of
  magnitude.  The results may not be valid.

% /* ADV_PRECISION                     67 */
\item \texttt{Specified confidence interval of }\emph{n.n}\texttt{\%
    not met after run time of }\emph{n.n}\texttt{. Actual value is
  }\emph{n.n}\texttt{\%.}\marginpar{lqsim}

% /* ADV_EMPTY_SUBMODEL                70 */
\item \texttt{Submodel }\emph{n}\texttt{ is empty.}\marginpar{lqns}
  
  The call graph\index{call graph} is interesting, to say the least.

% /* ADV_UNDERRELAXATION               76 */
\item \texttt{Underrelaxation ignored.  }\emph{n.n}\texttt{ outside
    range [0-2), using }\emph{m.m}\texttt{.}\marginpar{lqns}
  
  The under-relaxation
  coefficient\indexerror{under-relaxation coefficient} specified in
  the input file is not valid.  The solver is using $m.m$
  instead\footnote{Values of under-relaxation from $1 < n \le 2$ are
    more correctly called over-relaxation\index{over relaxation}.}.

% /* ADV_INVALID_UTILIZATION           71 */
\item \texttt{The utilization of }\emph{n.n}\texttt{ at }\emph{(task$|$processor)}
  \emph{identifier}\texttt{ with multiplicity }\emph{m}{ is too high.}
  
  This problem is the result of some of the approximations
  used\indexerror{utilization!high} by the analytic solver.  The
  common causes are two-phase
  servers\index{phases!approximation!error} and the Rolia
  multiserver\index{multiserver!approximation!error}\index{multiserver!Rolia}.
  If \emph{identifer} is a multiserver, switching to the
  Conway\index{multiserver!Conway} approximation will often help.
  Values of $n.n$ in excess of 10\% are likely the result of failures
  in the solver.  Please send us the model file so that we can improve
  the algorithms.

\end{itemize}

\section{Warning Messages}
\label{sec:warning-messages}

\begin{itemize}
% /* WRN_NOT_USED                       43  */
\item \emph{(activity$|$entry$|$task$|$processor)}\texttt{ "}\emph{identifier}\texttt{" is not used.}
  
  The object is not reachable\indexerror{not reachable}.  This may
  indicate an error in the specification of the model.

% WRN_INFINITE_MULTI_SERVER
\item \emph{(Processor$|$Task)}\texttt{ "}\emph{identifier}\texttt{" is an infinite server with a
  multiplicity of }$n$\texttt{.} 

  Infinite servers must always have a multiplicty of one.\index{multiplicity!infinite server}\indexerror{multiplicity}
  This error is caused by specifying both \emph{delay}\index{scheduling!delay} scheduling and a multiplicity
  for the named task or processor.  The multiplicity attribute is ignored.
  
% /* WRN_SCHEDULING_NOT_SUPPORTED       44 */
\item \emph{sched} \texttt{scheduling specified for}
  \emph{(processor$|$task)}\texttt{ "}\emph{identifier}\texttt{" is not
    supported.}
  
  The solver does not support the specified scheduling
  type\indexerror{scheduling}.  First-in, first-out scheduling will be
  used instead.

% /* WRN_ACT_NO_SERVICE_TIME            45 */
\item \texttt{Activity "}\emph{identifier}\texttt{" has no service
    time specified.}
  
  No service time\indexerror{service time} is specified for
  \emph{identifier}.

% /* WRN_COEFFICIENT_OF_VARIATION      77 */
\item \texttt{Coefficient of variation is incompatible with phase type
    at }\emph{(entry$|$task)} \texttt{"}\emph{identifier}\texttt{"}
  \emph{(phase$|$activity)}
  \texttt{"}\emph{identifier}\texttt{".}\marginpar{lqns}
  
  A coefficient of variation\indexerror{coefficient of variation} is
  specified at a using stochastic phase or activity.

% /* WRN_NO_REQUESTS_TO_ENTRY           46 */
\item \texttt{Entry "}\emph{identifier}\texttt{" does not receive any
    requests.}
  
  Entry \emph{identifier} is part of a non-reference
  task\indexerror{reference task}\indexerror{server!task}\index{task!server}.
  However, no requests are made to this entry.

% /* WRN_NO_SERVICE_TIME                47 */
\item \texttt{Entry "}\emph{identifier}\texttt{" has no service time
    specified for any phase.}
  
  No service time\indexerror{service time} is specified for entry
  \emph{identifier}.

% /* WRN_NO_SERVICE_TIME_FOR_PHASE      48 */
\item \texttt{Entry "}\emph{identifier}\texttt{" has no service time
    specified for phase} \emph{n}\texttt{.}
  
  No service time\indexerror{service time} is specified for entry
  \emph{identifier}, phase $n$.

% /* INFINITE_SERVER_OPEN_ARRIVALS */
\item \texttt{Infinite server "}\emph{identifier}{" accepts either asynchronous messages or open arrivals.}

  The task or processor, \emph{identifier}, is an infinite server.  It processes either asynchronous
  messages or open arrivals.  If the arrival rate exceeds the service rate of the infinite server, the
  number of instances of the infinite server will grow to infinity.
  
% /* WRN_NO_PHASE_FOR_HISTOGRAM         68 */
\item \texttt{Histogram requested for entry "}\emph{identifier}\texttt{", phase~}\emph{n}\texttt{ -- phase is not present.}\marginpar{lqsim}
  
  A histogram\index{histogram!no~phase} for the service time of phase \emph{n} of entry \emph{identifier} was requested.  This entry
  has no corresponding phase.

% /* WRN_INVALID_PRIORITY              70 */
\item \texttt{Priority specified (}\emph{n}\texttt{) is outside of
    range (}\emph{n}\texttt{,}\emph{n}\texttt{). (Value has been
    adjusted to }\emph{n}\texttt{).}\marginpar{lqsim}

  The priority $n$ is outside of the range specified.\indexerror{priority}

% /* WRN_NO_QUANTUM_FOR_PS             69 */
\item \texttt{No quantum specified for PS scheduling discipline.  FIFO
    used." }\marginpar{lqsim}
  
  A processor using processor sharing\indexerror{processor!sharing}
  scheduling\index{scheduling!processor sharing} needs a quantum
  value\index{quantum} when running on the
  simulator\index{lqsim!scheduling}.

% /* WRN_NO_REQUESTS_MADE              79 */
\item \texttt{No requests made from }\emph{from-identifier}\texttt{ to
  }\emph{to-identifier}\texttt{.}\marginpar{lqns}
  
  The input file has a rendezvous\indexerror{rendezvous} or
  send-no-reply\indexerror{send-no-reply} request with a value of
  zero.

% /* WRN_DEFINED_NE_SPECIFIED_X         49 */
\item \texttt{Number of }\emph{(processors$|$tasks$|$entries)}\texttt{
    defined (}\emph{n}\texttt{) does not match number specified
    (}\emph{m}\texttt{).}

  The processor task and entry chapters of the original input
  grammar\index{grammar!original} can specify the number of objects
  that follow.  The number specified does not match the actual number
  of objects.  Specifying \emph{zero} as a record count is valid.

% /* WRN_MULTIPLE_SPECIFICATION         50 */
\item \texttt{Parameter is specified multiple times.}

  A parameter is specified more than one time.  The first occurance is
  used.\indexerror{duplicate!parameter}

% /* WRN_NON_CFS_PR0CESSOR              */
\item \texttt{Processor "}\emph{identifier}\texttt{" is not running fair share scheduling." }
  
  A group\indexerror{scheduling!completely fair}\indexerror{group} was defined in the model and associated
  with a processor using a scheduling discipline other than completely fair scheduling.

% /* WRN_NO_TASKS_DEFINED_FOR_PROCESSOR 51 */
\item \texttt{Processor "}\emph{identifier}\texttt{" has no tasks.}
  
  A processor was defined in the model, but it is not used by any
  tasks\indexerror{processor!not used}.  This can occur if none of the
  entries or phases has any service time\index{service time}.

% /* WRN_QUEUE_LENGTH                  80 */
\item \texttt{Queue Length is incompatible with task type at task
  }\emph{identifier}\texttt{.}\marginpar{lqns}
  
  A queue length\indexerror{queue length} parameter was specified at a
  task which does not support bounded queues.

% /* WRN_NO_SENDS_FROM_REF_TASK         52 */
\item \texttt{Reference task "}\emph{identifier}\texttt{" does not
    send any messages." }
  
  Reference tasks\indexerror{reference task} are customers in the
  model.  This reference task does not visit any servers, so it serves
  no purpose.

% /* WRN_TOO_MANY_ENTRIES_FOR_REF_TASK  53 */
\item \texttt{Reference task "}\emph{identifier}\texttt{" has more
    than one entry defined.}
  
  Reference tasks\index{task!reference}\indexerror{reference task}
  typically only have one entry.  The named reference task has more
  than one.  Requests are generated in proportion to the service times
  of the entries.

% /* WRN_PRIO_TASK_ON_FIFO_PROC         54 */
\item \texttt{Task "}\emph{task-identifier}\texttt{" with priority is
    running on processor
    "}\emph{processor-identifier}\texttt{"}\linebreak[3] \texttt{which
    does not have priority scheduling.}
  
  Processors running with FCFS\index{scheduling!FCFS} scheduling
  ignore priorities\indexerror{priority}.

% /* WRN_INVALID_INT_VALUE             78 */
\item \texttt{Value specified for }\emph{(fanin$|$fanout)}\texttt{,
  }\emph{n}\texttt{, is invalid.}\marginpar{lqns}
  
  The value specified for a fan-in\indexerror{fan-in} or
  fan-out\indexerror{fan-out} is not valid and will be ignored.

% /* WRN_NOT_SUPPORTED                  55 */
\item \texttt{The }\emph{x}\texttt{ feature is not supported in this version.}

  Feature \emph{x} is not supported in this release.

\end{itemize}

\section{Input File Parser Error Messages}

\begin{itemize}
\item \texttt{error: not well-formed (invalid token)}

  This error occurs when an XML input file is expected, but some other
  input file type was given.

\item \texttt{Parse error.}

  An error was detected while processing the XML input file\indexerror{}.  See the list below for more
  explantion:

\begin{itemize}
% : warning: An exception occurred! Type:RuntimeException, Message:Warning: The primary document entity could not be opened. Id=/usr/lqn.xsd
\item \texttt{The primary document entity could not be opened. Id=}\emph{URI}\texttt{ while parsing
  }\emph{filename}\texttt{.}
  
  This error\label{error:primary-document}\indexerror{primary document} is generated by the
  Xerces\indexerror{Xerces} when the Uniform resource indicator \emph{(URI)} specified as the argument to
  the \attribute{xsi:noNamespaceSchemaLocation} attribute of the \schemaelement{lqn-model} element cannot be
  opened.  This argument must refer to a valid location containing the LQN schema\indexerror{schema} files.
% --- SCHEMA_KEY_FOR_IDENTITY_CONSTRAINT ---
\item \texttt{The key for identity constraint of element '}\emph{lqn-model}\texttt{' is not found.}
  
  When this message appears, Xerces\index{Xerces} does \textbf{not} provide many hints on where the actual
  error occurs because it always gives a line number which points to the end of the file (i.e. where the
  terminating tag \verb!</lqn-model>! is).

  In this case, the following three points should be inspected to
  ensure validity of the model:

  \begin{enumerate}
  \item All synchronous calls have a \attribute{dest} attribute which
    refers to a valid entry\indexerror{rendezvous}.
  \item All asynchronous calls have a \attribute{dest} attribute which
    refers to a valid entry\indexerror{send-no-reply}.
  \item All forwarding calls have a \attribute{dest} attribute which
    refers to a valid entry\indexerror{forward}.
  \end{enumerate}

  If it is not practical to manually inspect the model, run the XML
  file through another tool like XMLSpy or XSDvalid which will report
  more descriptive errors.
  
\item \texttt{The key for identity constraint of element '}\emph{task}\texttt{' is not found.}
  
  When this error appears, it means there is something wrong within
  the \schemaelement{task} element indicated by the line number.
  Check that:
  \begin{itemize}
  \item The name \attribute{attribute} of all
    \schemaelement{reply-entry} elements refers to a valid entry name,
    which exists within the same task as the task activity
    graph\indexerror{activity graph}.
  \item All activities which contain the attribute
    \attribute{bound-to-entry} have a valid entry name that exists
    within the same task as the task activity
    graph\indexerror{activity graph}.
  \end{itemize}

% --- SCHEMA_KEY_FOR_IDENTITY_CONSTRAINT ---
\item \texttt{The key for identity constraint of element
    '}\emph{task-activities}\texttt{' is not found.}
  
  When this error appears, it means there is something wrong within
  the \schemaelement{task-activities} element indicated by the line
  number.

  Check that:
  \begin{itemize}
  \item All activities referenced within the
    \schemaelement{precedence} elements refer to activities which are
    defined for that particular task activity
    graph\indexerror{activity graph}.
  \item The \attribute{name} attribute of all
    \schemaelement{reply-activity}\indexerror{reply-activity} elements
    refers to an activity defined within the mentioned
    \schemaelement{task-activities} element.
  \item The head attribute of all \schemaelement{post-loop} elements
    refers to an activity defined within the mentioned
    \schemaelement{task-activities} element.
  \item All post-LOOP elements which contain the optional attribute
    \attribute{end}, refers to an activity defined within the
    mentioned \schemaelement{task-activities} element.
  \end{itemize}

\item \texttt{Not enough elements to match content model :}\emph{elements}

  ((run-control,plot-control,solver-params,processor),slot)
\end{itemize}
\end{itemize}
\section{LQX Error messages}

\begin{itemize}
\item \texttt{Runtime Exception Occured: Unable to Convert From:} \emph{`<<uninitialized>>'} \texttt{To:} \emph{`Array'}

  An unitialized variable is used where an array is expected (like in
  a foreach loop).
\end{itemize}


%%% Local Variables: 
%%% mode: latex
%%% mode: outline-minor 
%%% fill-column: 108
%%% TeX-master: "userman.tex"
%%% End: 
