%% -*- mode: latex; mode: outline-minor; fill-column: 108 -*-
%% Title:  schema
%%
%% $HeadURL: http://rads-svn.sce.carleton.ca:8080/svn/lqn/trunk/doc/userman/schema.tex $
%% Original Author:     Greg Franks <greg@sce.carleton.ca>
%% Created:             Tue Jul 18 2006
%%
%% ----------------------------------------------------------------------
%% $Id: schema.tex 11982 2014-04-15 21:32:15Z greg $
%% ----------------------------------------------------------------------

\chapter{XML Grammar}
\label{sec:xml-grammar}

The definition of LQN models using XML\index{XML Grammar|(textbf}\index{Grammar!XML|(textbf} is an evolution of
the original SRVN file format (c.f.~\S\ref{ch:srvn} and~Appendix~\ref{sec:input-file-bnf}).  The
new XML format is based on the work done in~\cite{perf:wu-2003}, with
further refinement for general usage.  There are new features in the
XML format to support new concepts for building and assembling models
using components\index{components}.  The normal LQN tool suite (like
\manpage{lqns}{1} and \manpage{lqsim}{1}) do not support these new
features, however other tools outside the suite are being written to
utilize the new parts of the XML format.

\section{Basic XML File Structure}
\label{sec:xml-file-structure}

In XML, layered models are specified in a bottom-up order, which is
the reverse of how layered models are typically presented.  First, a
processor is defined, then within the processor block, all the tasks
than run on it are defined.  Similarly, within each task block all the
entries that are associated with it are defined, etc.  A simplified
layout of an incomplete LQN model written in XML is shown in
Figure~\ref{fig:xml-file-layout}.

\lstset{language=XML,basicstyle=\ttfamily,numbersep=10pt}
\begin{lstlisting}[float,caption={XML file layout.},label=fig:xml-file-layout]
<lqn-model>
   <solver-params>
      <pragma/>
   </solver-params>
   <processor>
      <task>
         <entry>
            <entry-phase-activities>
               <activity>
                  <synch-call/>
                  <asynch-call/>
               </activity>
               <activity> ... </activity>
            </entry-phase-activities>
         </entry>
         <entry> ... </entry>
         <task-activities>
            <activity/>
            <precedence/>
         </task-activities>
      </task>
      <task> ... </task>
   </processor>
   <processor> ... </processor>
</lqn-model>
\end{lstlisting}

Activity graphs (specified by task-activities) belong to a task, and
hence are siblings to entry elements.  The element
entry-activity-graph specifies an activity graph contained within one
entry, but is not supported by any of the LQN tools.  The concept of
phases still exists, but now each phase is an activity, and is defined
in the entry-phase-activities element.

\section{Schema Elements}
\label{sec:schema-elements}

The XML definition for layered models consists of three files:
\begin{description}
\item[\texttt{lqn.xsd}\index{lqn.xsd}:] lqn.xsd is the root of the schema.
\item[\texttt{lqn-sub.xsd}\index{lqn-sub.xsd}] ...
\item[\texttt{lqn-core.xsd}\index{lqn-core.xsd}] lqn-core is the
  actual model specfication and is included by lqn.xsd.
\end{description}
All three files should exist in the same location.  If the solver cannot located the \texttt{lqn.xsd} file,
it will emit an error\footnote{See the error message ``The primary document entity could not be opened'' on
  \pageref{error:primary-document}.} and stop\indexerror{primary document}.

Figure~\ref{fig:Schema} shows the schema for Layered Queueing Networks
using Unified Modeling Language notation.  The model is defined starting from
\texttt{lqn-model}.  Unless otherwise specified in the figure, the order of
elements in the model is from left to right, i.e., \texttt{<solver-params>}
always preceeds \texttt{<processor>} in the input file.  Optional elements are
shown using a multiplicity of zero for an association.  Note that results
(optional, shown in blue) are part of the schema.

\begin{figure}[htbp]
%  \epsfxsize=\textwidth 
  \epsffile{xml-schema/schema.eps}
  \caption[LQN Schema]{LQN Schema.  Elements shown in \color{blue}blue\color{black}\mbox{} are results found in the output.
    Elements shown in \color{red}red\color{black}\mbox{} are not implemented.  Unless otherwise indicated, all elements are
    ordered from left to right.}
  \label{fig:Schema}
\end{figure}

\subsection{LqnModelType}
\label{sec:LqnModelType}

The first element in a layered queueing network XML input\index{input!XML} file is
\schemaelement{lqn-model}, which is of type \schematype{LqnModelType} and is shown in
Figure~\ref{fig:LqnModelType}.  \textbf{LqnModelType} has five elements, namely:
\schemaelement{run-control}, \schemaelement{plot-control}, \schemaelement{solver-params},
\schemaelement{processor} and \schemaelement{slot}.  \texttt{Run-control} and \texttt{plot-control} are not
not implemented.  \texttt{Processor} is described under Section~\ref{sec:ProcessorType}.  \texttt{Slot} is
described in~\cite{perf:wu-2003}.  The attributes for \textbf{LqnModelType} are shown in
Table~\ref{tab:LqnModelType}.

\begin{figure}[htbp]
%  \epsfxsize=\textwidth 
  \epsffile{xml-schema/lqn-model-schema.eps}
  \caption{Top-level LQN Schema.}
  \label{fig:LqnModelType}
\end{figure}

\begin{table}[htbp]
  \centering
  \begin{tabular}[l]{|l|l|l|l|p{2.5in}|}
    \hline
    \textbf{Name} & \textbf{Type} & \textbf{Use} & \textbf{Default} &
    \textbf{Comments} \\
    \hline
    \attribute{name} & string & optional & & The name of the model. \\
    \hline
    \attribute{description} & string & optional & & A description
    of the model. \\
    \hline
    \attribute{lqn-schema-version} & integer & fixed & 1.0 & The version of the schema
    (used by the solver in case of substantial schema changes for model
    conversion.) \\
    \hline
    \attribute{lqncore-schema-version} & integer & fixed & 1.0 & \\
    \hline
    \attribute{xml-debug} & boolean & optional & false & \\
    \hline
  \end{tabular}
  \caption{\label{tab:LqnModelType}Attributes for elements of type \schematype{LqnModelType}
    from Figure~\protect\ref{fig:LqnModelType}.}
\end{table}

The element \schemaelement{solver-params} is used to set various
operating parameters for the analytic solver, and to record various
output statistics after a run completes.  It contains the elements
\schemaelement{result-general} and \schemaelement{pragma}.  The
attributes for \schemaelement{solver-params} are shown in
Table~\ref{tab:solver-params}.  These attributes are mainly used to
control the analytic solver.  Refer to
Section~\ref{sec:lqns-stopping-criteria} for more information.  The
attributes for \schemaelement{result-general} are shown in
Table~\ref{tab:result-general}.  Refer to
Sections~\ref{sec:analytic-header-out} and
\ref{sec:simulator-header-out} for the interpretation of header
information.  The attributes for \schemaelement{pragma} are show in
Table~\ref{tab:pragma}.  Refer to Section~\ref{sec:lqns-pragmas} for
the pragmas supported by lqns and to Section~\ref{sec:lqsim-pragmas}
for the pragmas supported by lqsim.

\begin{table}[htbp]
  \centering
  \begin{tabular}[l]{|l|l|l|l|p{3in}|}
    \hline
    \textbf{Name} & \textbf{Type} & \textbf{Use} & \textbf{Default} &
    \textbf{Comments} \\
    \hline
    \attribute{conv\_val}   & float   & optional & 1 & Convergence
    value\index{lqns!convergence value} for lqns
    (c.f~\S\protect\ref{sec:lqns-stopping-criteria}).  Ignored by lqsim.\\
    \hline
    \attribute{it\_limit}   & integer & optional & 50 & Iteration limit\index{iteration limit} for lqns
    (c.f~\S\protect\ref{sec:lqns-stopping-criteria}).  Ignored by lqsim.\\
    \hline
    \attribute{print\_int}  & integer & optional & 0 & Print interval for intermediate
    results.  The \flag{t}{print} must be specified to lqns\index{print interval!lqns} to generate
    output after \emph{it\_limit} iterations.  Blocked
    statistics\index{statistics!blocked} must be specified to
    lqsim\index{print interval} using
    the \flag{A}{n}, \flag{B}{n}, or \flag{C}{n} flags. \\
    \hline
    \attribute{underrelax\_coeff} & float & optional & 0.5 & Under-relaxation
    coefficient for lqns (c.f~\S\protect\ref{sec:lqns-stopping-criteria}).
    Ignored by lqsim. \\
    \hline
  \end{tabular}
  \caption{\label{tab:solver-params}Attributes of element \schemaelement{solver-params} from
    Figure~\protect\ref{fig:LqnModelType}.} 
\end{table}

\begin{table}[htbp]
  \centering
  \begin{tabular}[l]{|l|l|l|l|p{2.8in}|}
    \hline
    \textbf{Name} & \textbf{Type} & \textbf{Use} & \textbf{Default} &
    \textbf{Comments} \\
    \hline
    \hline
    \attribute{conv-val} & float & required & & Convergence value
    (c.f.~\protect\ref{sec:analytic-header-out}) \\
    \hline
    \attribute{valid} & enumeration & required & & Either \texttt{YES} or
    \texttt{NO}. \\
    \hline
    \attribute{iterations} & float & optional & & The number of iterations of the
    analytic solver or the number of blocks for the simulator.\\
    \hline
    \attribute{elapsed-time} & string & optional & & The wall-clock time used by
    the solver. \\
    \hline
    \attribute{system-cpu-time} & string & optional & & The CPU time spent in
    kernel-mode. \\
    \hline
    \attribute{user-cpu-time} & string & optional & & The CPU time spent in user
    mode. \\
    \hline
    \attribute{platform-info} & string & optional & & The operating system and CPU
    type. \\
    \hline
    \attribute{solver-info} & string & optional & & The version of the solver. \\
    \hline
  \end{tabular}
  \caption{\label{tab:result-general}Attributes of element \schemaelement{result-general} from
    Figure~\protect\ref{fig:LqnModelType}.} 
\end{table}

\begin{table}[htbp]
  \centering
  \begin{tabular}[l]{|l|l|l|l|p{3in}|}
    \hline
    \textbf{Name} & \textbf{Type} & \textbf{Use} & \textbf{Default} &
    \textbf{Comments} \\
    \hline
    \hline
    \attribute{param} & string & required & & The name of the parameter.
    (c.f.~\protect\ref{sec:lqns-pragmas}, \S\ref{sec:lqsim-pragmas}) \\
    \hline
    \attribute{value} & string & required & & the value assigned to the pragma. \\
    \hline
  \end{tabular}
  \caption{\label{tab:pragma}Attributes of element \schemaelement{pragma} from
    Figure~\protect\ref{fig:LqnModelType}.} 
\end{table}

%%
%% Processor
%%

\subsection{ProcessorType}
\label{sec:ProcessorType}

Elements of type \schematype{ProcessorType}, shown in Figure~\ref{fig:ProcessorType} are used to define the
processors in the model.  They contain an optional \texttt{result-processor} element and elements of either
\schematype{GroupType} or \schematype{TaskType}.  The \attribute{scheduling} attribute must by set to
\texttt{cfs}, for completely fair scheduling\index{scheduling!completely fair}, if \schematype{GroupType}
elements are present and to any other type if \schematype{GroupType} are not found.  \schematype{GroupType}
and \schematype{TaskType} elements may not be both be defined in a processor.

Element \schemaelement{result-processor} is of type \schematype{OutputResultType} and is described in
Section~\ref{sec:OutputResultType}.  Element \schemaelement{task} is described in
Section~\ref{sec:TaskType}.  The attributes of \schematype{ProcessorType}, described in \ref{sec:processor},
are shown in Table~\ref{tab:ProcessorType}.

\begin{figure}[htbp]
%  \epsfxsize=\textwidth 
  \centering
  \epsffile{xml-schema/processor-schema.eps}
  \caption{Processor Schema.}
  \label{fig:ProcessorType}
\end{figure}
\begin{table}[htbp]
  \centering
  \begin{tabular}[l]{|l|l|l|l|p{3in}|}
    \hline
    \textbf{Name} & \textbf{Type} & \textbf{Use} & \textbf{Default} &
    \textbf{Comments} \\
    \hline
    \attribute{name}         & string         & required & & \\
    \hline
    \attribute{multiplicity} & integer        & optional & 1 & See \S\protect\ref{sec:multiplicity} \\
    \hline
    \attribute{speed-factor} & float          & optional & 1.0 & Scaling factor for
    the processor. \\
    \hline
    \attribute{scheduling}   & enumeration & optional & fcfs & The allowed
    scheduling types are \texttt{fcfs}, \texttt{hol}, \texttt{pp},
    \texttt{rand}, \texttt{inf}, \texttt{ps-hol}, \texttt{ps-pp} and \texttt{cfs}. See
    \S\protect\ref{sec:processors}\index{scheduling!processor}\index{scheduling!head of line}. \\
    \hline
    \attribute{replication}  & integer        & optional & 1 & See \S\protect\ref{sec:replication} \\
    \hline
    \attribute{quantum}      & float          & optional & 0.0 & Mandatory for
    processor sharing\index{processor!sharing} scheduling when using lqsim. \\
    \hline
  \end{tabular}
  \caption{\label{tab:ProcessorType}Attributes for elements of type \schematype{ProcessorType}.}
\end{table}
%%
%%
%% Group
%%
\subsection{GroupType}
\label{sec:GroupType}

Optional elements of type \schematype{GroupType}, shown in Figure~\ref{fig:ProcessorType}, are used to
define groups of tasks for processors running completely fair scheduling\index{scheduling!completely fair}.
Each group must contain a minimum of one task.  The attributes of \schematype{GroupType} are shown in
Table~\ref{tab:GroupType}. 

\begin{table}[htbp]
  \centering
  \begin{tabular}[l]{|l|l|l|l|p{2.8in}|}
    \hline
    \textbf{Name} & \textbf{Type} & \textbf{Use} & \textbf{Default} &
    \textbf{Comments} \\
    \hline
    \attribute{name}         & string         & required & & \\
    \hline
    \attribute{share}        & float          & required & & The fraction of the processor allocated to this
    group. \\
    \hline
    \attribute{cap}          & boolean        & optional & false & If true, shares are
    \emph{caps}\index{share!cap} (ceilings).  Otherwise, shares are guarantees\index{share!guarantee}
    (floors) \\
    \hline
  \end{tabular}
  \caption{\label{tab:GroupType}Attributes for elements of type \schematype{GroupType}}
\end{table}
%%
%%
%% Task
%%
\subsection{TaskType}
\label{sec:TaskType}

Elements of type \schematype{TaskType}, shown in Figure~\ref{fig:TaskType}, are used to define the tasks in
the model.  These elements contain an optional \schemaelement{result-task} element, one or more elements of
\textbf{EntryType}, and optionally, elements of \schemaelement{service} and \schemaelement{task-activities}.
Element \schemaelement{result-task} is of type \schematype{OutputResultType}, and is described in
Section~\ref{sec:OutputResultType}.  Element \schemaelement{entry} is described in
Section~\ref{sec:EntryType}.  The attributes of \schematype{TaskType}, described in Section~\ref{sec:task},
are shown in Table~\ref{tab:TaskType}.

\begin{figure}[htbp]
  \centering
  \epsffile{xml-schema/task-schema.eps}
  \caption{TaskType}
  \label{fig:TaskType}
\end{figure}

\begin{table}[htbp]
  \centering
  \begin{tabular}[l]{|l|l|l|l|p{2.8in}|}
    \hline
    \textbf{Name} & \textbf{Type} & \textbf{Use} & \textbf{Default} &
    \textbf{Comments} \\
    \hline
    \attribute{name}         & string         & required & & \\
    \hline
    \attribute{multiplicity} & integer & optional & 1 & See \S\protect\ref{sec:multiplicity}.\\
    \hline
    \attribute{priority}     & integer & optional & 0 & The priority used by the
    processor for scheduling.  See \S\protect\ref{sec:processors}. \\
    \hline
    \attribute{queue-length} & integer & optional & 0 & Maximum queue size (for
    open-class requests only).  See \S\protect\ref{sec:tasks}. \\
    \hline
    \attribute{replication}  & integer & optional & 1 & See \S\protect\ref{sec:replication}\\
    \hline
    \attribute{scheduling}   & enumeration & optional & FCFS & The scheduling
    of requests at the task.  The allowed
    scheduling types are \texttt{ref}, \texttt{fcfs}, \texttt{hol}, \texttt{pri},
    \texttt{inf}, \texttt{burst}, and \texttt{poll} and \texttt{semaphore}. See \S\protect\ref{sec:tasks}.\index{scheduling!task} \\
    \hline
    \attribute{activity-graph} & enumeration & required & &
    \texttt{yes} or \texttt{no}\\
    \hline
    \hline
    \attribute{think-time}   & float         & optional & 0 & Reference tasks only.  Customer think time. \\
    \hline
    \hline
    \attribute{initially}    & integer       & optional & \emph{multiplicity} & Semaphore tasks only.  Set the initial
    number of semaphore tokens to zero.  By default, the number of tokens is set to the multiplicity of the task.  \\
    \hline
  \end{tabular}
  \caption{\label{tab:TaskType}Attributes for elements of type \schematype{TaskType}}
\end{table}

\subsection{FanInType and FanOutType}
\label{sec:FanInType}
\label{sec:FanOutType}

\begin{table}[htbp]
  \centering
  \begin{tabular}[l]{|l|l|l|l|p{2.5in}|}
    \hline
    \textbf{Name} & \textbf{Type} & \textbf{Use} & \textbf{Default} &
    \textbf{Comments} \\
    \hline
    \attribute{source}     & integer & required &  & (See \S\ref{sec:replication}) \\
    \hline
    \attribute{value}      & integer & required &  & (See \S\ref{sec:replication}) \\
    \hline
  \end{tabular}
  \caption{\label{tab:FanInType}Attributes for elements of type \schematype{FanInType}.}
\end{table}

\begin{table}[htbp]
  \centering
  \begin{tabular}[l]{|l|l|l|l|p{2.5in}|}
    \hline
    \textbf{Name} & \textbf{Type} & \textbf{Use} & \textbf{Default} &
    \textbf{Comments} \\
    \hline
    \attribute{dest}       & integer & required &  & (See \S\ref{sec:replication}) \\
    \hline
    \attribute{value}      & integer & required &  & (See \S\ref{sec:replication}) \\
    \hline
  \end{tabular}
  \caption{\label{tab:FanOutType}Attributes for elements of type \schematype{FanOutType}.}
\end{table}

\subsection{EntryType}
\label{sec:EntryType}

Elements of type \schematype{EntryType}, shown in Figure~\ref{fig:EntryType}, are used to define the entries
of tasks.  Entries can be specified one of three ways, based on the attribute \attribute{type} of an
\schemaelement{entry} element, namely:
\begin{figure}[htbp]
  \centering
%  \epsfxsize=\textwidth 
  \epsffile{xml-schema/entry-schema.eps}
  \caption{Schema for type \schematype{EntryType}.}
  \label{fig:EntryType}
\end{figure}
\begin{description}
\item[\texttt{ph1ph2}] The entry is specified using phases.  The
  phases are specified using an \schemaelement{entry-phase-activities}
  element which is of the \schematype{ActivityPhasesType} type.
  Activities defined within this element must have a unique
  \attribute{phase} attribute.  
\item[\texttt{graph}] The entry is specified as an activity
  graph\index{activity graph} defined within the entry.  The demand is
  specified using elements of type \schematype{ActivityEntryDefType}.
  This method of defining an entry is not supported currently.
\item[\texttt{none}] The entry is specified using an activity graph
  defined within the task.  A \schemaelement{task-activities} element
  of type \schematype{ActivtyDefType} must be present and one of the
  activities defined within this element must have a
  \attribute{bound-to-entry} attribute.  The
  \schematype{TaskActivityGraph} type is defined in
  Section~\ref{sec:TaskActivityGraph}.
\end{description}
\schematype{ActivityPhasesType},   \schematype{ActivityEntryDefType} and
\schematype{ActivtyDefType} are all based on
\schematype{ActivityDefBase}, described in
Section~\ref{sec:ActivityDefBase}.  They only differ in the way the
start of the graph is identified, and in the case of
\schematype{ActivityPhasesType}, the way the activities are connected.  

The attributes for \schematype{EntryType}, described in
Section~\ref{sec:entry}, are shown in Table~\ref{tab:EntryType}.  The optional
element \schemaelement{result-entry} is of type
\textbf{OutputResultType}, and is described in
Section~\ref{sec:OutputResultType}.  The optional element
\schemaelement{forwarding} is used to describe the probability of
forwarding a request to another entry; it is described in
Section~\ref{sec:MakingCallType}.

\begin{table}[htbp]
  \centering
  \begin{tabular}[l]{|l|l|l|l|p{2.5in}|}
    \hline
    \textbf{Name} & \textbf{Type} & \textbf{Use} & \textbf{Default} &
    \textbf{Comments} \\
    \hline
    \attribute{name} & string & required & & The entry name\\
    \hline
    \attribute{type} & enumeration & required & & \texttt{PH1PH2}, \texttt{GRAPH}, or \texttt{NONE} \\
    \hline
    \attribute{open-arrival-rate} & float & optional & \index{open arrival} & \\
    \hline
    \attribute{priority} & integer & optional & & (c.f.~\ref{sec:tasks}) \\
    \hline
    \attribute{sempahore} & enumeration & optional & & \texttt{signal} or \texttt{wait} (c.f.~\ref{sec:tasks}) \\
    \hline
  \end{tabular}
  \caption{\label{tab:EntryType}Attributes for elements of type \schematype{EntryType}.}
\end{table}

\subsection{ActivityGraphBase}
\label{sec:ActivityGraphBase}

Elements of type \schematype{ActivityGraphBase}, shown in
Figure~\ref{fig:ActivityGraphBase}, are used to define activities
(c.f.~\ref{sec:activities})\index{activity} and their relationships to
each other.  They are used by elements of both \schematype{EntryType} and
\schematype{TaskActivityGraph} types.

\begin{figure}[htbp]
  \centering
%  \epsfxsize=\textwidth 
  \epsffile{xml-schema/activity-schema.eps}
  \caption{Schema diagram for the type \schematype{ActivityGraphBase}}
  \label{fig:ActivityGraphBase}
\end{figure}

Elements of the \schematype{ActivityGraphBase} consist of a sequence
of one or more \schemaelement{activity} elements followed by a
sequence of \schemaelement{precedence} elements.  \texttt{Activity}
elements are used to store the demand for an
activity\index{activity!demand}\index{demand} and requests to other
servers (through the \schematype{ActivityDefType}) and, optionally,
results through elements of \schematype{ActivityDefType}.
\texttt{Precedence} elements are defined by the
\schematype{PrecedenceType} in Section~\ref{sec:PrecedenceType}.

\subsection{TaskActivityGraph}
\label{sec:TaskActivityGraph}

Task Activity Graphs\index{activity graph!task}, defined using
elements of type \schematype{TaskActivityGraph} and shown in
Figure~\ref{fig:ActivityGraphBase}, are used to specify the behaviour
of a task using activities.  This type is almost the same as
\schematype{EntryActivityGraph}, except that the activity that replies
to an entry\index{reply!activity} must explicitly specify the entry
for which the reply is being generated.  The actual activity graph is
defined using elements of type \schematype{ActivityGraphBase},
described in Section~\ref{sec:ActivityGraphBase}.  The attributes for
elements \schemaelement{reply-entry} and
\schemaelement{reply-activity} are shown in
Tables~\ref{tab:reply-entry} and \ref{tab:reply-activity}
respectively.

\begin{table}[htbp]
  \centering
  \begin{tabular}[l]{|l|l|l|l|p{3in}|}
    \hline
    \textbf{Name} & \textbf{Type} & \textbf{Use} & \textbf{Default} &
    \textbf{Comments} \\
    \hline
    \attribute{name} & string & required & & The name of the
    entry for which the list of \schemaelement{reply-activity} elements
    generate replies.\\
    \hline
  \end{tabular}
  \caption{\label{tab:reply-entry}Attributes of element \schemaelement{reply-entry} from
    Figure~\protect\ref{fig:ActivityGraphBase}.} 
\end{table}

\begin{table}[htbp]
  \centering
  \begin{tabular}[l]{|l|l|l|l|p{3in}|}
    \hline
    \textbf{Name} & \textbf{Type} & \textbf{Use} & \textbf{Default} &
    \textbf{Comments} \\
    \hline
    \attribute{name} & string & required & & The name of the
    activity which generates a reply.  The entry is either implicitly
    defined if this element is defined within an
    \schematype{EntryType}, or part of list defined within a
    \schemaelement{reply-element}. \\
    \hline
  \end{tabular}
  \caption{\label{tab:reply-activity}Attributes of element \schemaelement{reply-activity} from
    Figure~\protect\ref{fig:ActivityGraphBase}.} 
\end{table}

\subsection{ActivityDefBase}
\label{sec:ActivityDefBase}

The type \schematype{ActivityDefBase}, shown in Figure~\ref{fig:ActivityGraphBase}, is used to define the
parameters for an activity, such as demand\index{demand} and call-order\index{call order}.  This type is
extended by \schematype{ActivityPhasesType}, \schematype{EntryActivityDefType}, and
\schematype{ActivityDefType} to define the requests\index{request} from an activity to an entry, and to
connect the activity graph\index{activity graph!connection} to the requesting entry.
Table~\ref{tab:ActivityDefBase} lists the parameters used as attributes and the attributes used by the three
sub-types.  Refer to Section~\ref{sec:activity} for more information on these parameters.  Refer to
\schematype{MakingCallType} (\S\ref{sec:MakingCallType}) for the \schemaelement{Activity-CallGroup} used to
make requests to other entries\footnote{\schemaelement{Call-List-Group} is not defined at present.}. Refer
to \schematype{OutputResultForwardingANDJoinDelay} (\S\ref{sec:OutputResultJoinDelayType}) for
\schemaelement{result-join-delay} and \schemaelement{result-forwarding} for join-delay\index{join!delay} and
forwarding\index{forwarding} results respectively.  Refer to \schematype{OutputDistributionType}
(\S\ref{sec:OutputDistributionType}) for \schemaelement{service-time-distribtion}.  Finally, refer to
\schematype{OutputResultType} (\S\ref{sec:OutputResultType}) for \schemaelement{result-activity}.  This
element contains most of the results for an activity\index{activity!results}\index{results!activity} or
phase\index{results!phase}\index{phase!results}.

\begin{table}[htbp]
  \centering
  \begin{tabular}[l]{|l|l|l|l|p{2in}|}
    \hline
    \textbf{Name} & \textbf{Type} & \textbf{Use} & \textbf{Default} &
    \textbf{Comments} \\
    \hline
    \attribute{name}             & string & required & & \\
    \hline
    \attribute{host-demand-mean} & float  & required & & The mean
    service time\index{service time} demand for the activity.\\
    \hline
    \attribute{host-demand-cvsq} & float  & optional & 1.0 & The
    squared coefficient of variation~\index{coefficient of variation} for the activity.\\
    \hline
    \attribute{think-time}       & float  & optional & 0.0 & \\
    \hline
    \attribute{max-service-time} & float  & optional & 0.0 & \\
    \hline
    \attribute{call-order}       & enumeration & optional &
    STOCHASTIC & \texttt{STOCHASTIC} or \texttt{DETERMINISTIC} \\
    \hline
    \hline
    \multicolumn{5}{|c|}{\schematype{ActivtyPhasesType}}\\
    \hline
    \attribute{phase} & integer & required & & 1, 2, or 3 \\
    \hline
    \hline
    \multicolumn{5}{|c|}{\schematype{ActivtyEntryDefType}}\\
    \hline
    \attribute{first-activity} & string & required & &  \\
    \hline
    \hline
    \multicolumn{5}{|c|}{\schematype{ActivtyDefType}} \\
    \hline
    \attribute{bound-to-entry}   & string & optional & & If set, this
    activity is the start of an activity
    graph\index{activity graph!start}. \\
    \hline
  \end{tabular}
  \caption{\label{tab:ActivityDefBase}Attributes for elements of type \schematype{ActivityDefBase}.}
\end{table}


\subsection{MakingCallType}
\label{sec:MakingCallType}

The type \schematype{MakingCallType}, shown in
Figure~\ref{fig:MakingCallType}, is used to define the parameters for
requests\index{request} to entries.  This type is extended by
\schematype{ActivityMakingCallType} and
\schematype{EntryMakingCallType} to defined requests from activities
to entries and for forwarding requests from entry to entry
respectively.  Requests from activities to entries can be either
synchronous, (i.e., a \emph{rendezvous}\index{rendezvous}), through a
\schemaelement{sync-call} element, or asynchronous (i.e., a
\emph{send-no-reply}\index{send-no-reply}), through a
\schemaelement{async-call} element.  Section~\ref{sec:requests}
defines the parameters for a request.
Table~\ref{tab:MakingCallType} lists the attributes for the
types.

\begin{figure}[htbp]
  \centering
%  \epsfxsize=\textwidth 
  \epsffile{xml-schema/call-schema.eps}
  \caption{Schema diagram for the group \schematype{MakingCallType}.}
  \label{fig:MakingCallType}
\end{figure}

\begin{table}[htbp]
  \centering
  \begin{tabular}[l]{|l|l|l|l|p{3in}|}
    \hline
    \textbf{Name} & \textbf{Type} & \textbf{Use} & \textbf{Default} &
    \textbf{Comments} \\
    \hline
    \attribute{dest}       & string  & required &   & The name of the entry to which the requests are made. \\
    \hline
    \hline
    \multicolumn{5}{|c|}{\schematype{ActivityMakingCallType}}\\
    \hline
    \hline
    \attribute{calls-mean} & float   & required &   & The mean number of requests. \\
    \hline
    \hline
    \multicolumn{5}{|c|}{\schematype{EntryMakingCallType}}\\
    \hline
    \hline
    \attribute{prob} & float   & required &   & The probability of forwarding requests. \\
    \hline
  \end{tabular}
  \caption{\label{tab:MakingCallType}Attributes for elements of type \schematype{MakingCallType}.}
\end{table}

%
% Precedence
%

\subsection{PrecedenceType}
\label{sec:PrecedenceType}

The type \schematype{PrecedenceType}, shown in
Figure~\ref{fig:PrecedenceType}, is used to connect one activity to
another within an activity graph\index{activity graph}.  Each element
of this type contains exactly one \schemaelement{pre} element and,
optionally, one \schemaelement{post} element.  The pre elements are
referred to as \emph{join}-lists\index{join-list} as all of the
branches associated with the activities in the join-list must finish
(i.e.~``join'') before the activities in the subsequent post element
can start.  The post element itself is referred to as a
\emph{fork}-list\index{fork-list}.

\begin{figure}[htbp]
  \centering
  \epsffile{xml-schema/precedence-schema.eps}
  \caption{Schema diagram for the type \schematype{PrecedenceType}.}
  \label{fig:PrecedenceType}
\end{figure}

Elements of \schematype{PrecedenceType} can be of one of five types:
\begin{description}
\item[\schematype{SingleActivityListType}:] Elements of this type have
  no attributes and a sequence of exactly one \schemaelement{activity}
  element of \schematype{ActivityType}.
\item[\schematype{ActivityListType}:] Elements of this type have no
  attributes and a sequence one or more \schemaelement{activity}
  elements of \schematype{ActivityType}.
\item[\schematype{AndJoinListType}:] Elements of this type have an optional \schemaelement{quorum} element
  and a sequence of one or more or more \schemaelement{activity} elements of \schematype{ActivityType}.
  Table~\ref{tab:AndJoinListType} show the attributes of \schematype{AndJoinListType}.
\item[\schematype{OrListType}:] Elements of this type have no
  attributes and a sequence one or more \schemaelement{activity}
  elements of \schematype{ActivityOrType}.  These elements specify an
  activity name and a branch
  probability\index{branch!probability}\index{probability!branch}.
  Table~\ref{tab:ActivityOrType} show the attributes of
  \schematype{ActivityOrType}.
\item[\schematype{ActivityLoopListType}:] Elements of this type have
  one optional attribute and a sequence one or more
  \schemaelement{activity} elements of \schematype{ActivityLoopType}.
  These elements specify an activity name and a loop
  count\index{branch!loop count}\index{loop count}.  The optional
  attribute is used to specify the activity that is executed after all
  the loop\index{loop} branches complete.
  Tables~\ref{tab:ActivityLoopListType} and \ref{tab:ActivityLoopType}
  show the attributes of \schematype{ActivityLoopListType} and
  \schematype{ActivityLoopType} respectively.
\end{description}

\begin{table}[htbp]
  \centering
  \begin{tabular}[l]{|l|l|l|l|p{3in}|}
    \hline
    \textbf{Name} & \textbf{Type} & \textbf{Use} & \textbf{Default} &
    \textbf{Comments} \\
    \hline
    \attribute{name} & string & required & & \\
    \hline
    \attribute{quorum} & integer & optional & 0 & The number of branches which must complete for the join to
    finish.  If this attribute is not specified, then all of the branches must finish, which makes this
    object an AND-Join\index{join!and}\index{join!quorum}\index{quorum~join}\\
    \hline
  \end{tabular}
  \caption{\label{tab:AndJoinListType}Attributes for elements of type \schematype{AndJoinListType}.}
\end{table}

\begin{table}[htbp]
  \centering
  \begin{tabular}[l]{|l|l|l|l|p{3in}|}
    \hline
    \textbf{Name} & \textbf{Type} & \textbf{Use} & \textbf{Default} &
    \textbf{Comments} \\
    \hline
    \attribute{name} & string & required & & \\
    \hline
    \attribute{prob} & float & optional & 1.0 & The probability that the branch is
    taken, on average (c.f.~\S\ref{sec:precedence}) \\
    \hline
  \end{tabular}
  \caption{\label{tab:ActivityOrType}Attributes for elements of type \schematype{ActivityOrType}.}
\end{table}

\begin{table}[htbp]
  \centering
  \begin{tabular}[l]{|l|l|l|l|p{3in}|}
  \hline
  \textbf{Name} & \textbf{Type} & \textbf{Use} & \textbf{Default} &
  \textbf{Comments} \\
  \hline
  \attribute{end} & string & required & & \\
  \hline
  \end{tabular}
  \caption{\label{tab:ActivityLoopListType}Attributes for elements of type \schematype{ActivityLoopListType}.}
\end{table}

\begin{table}[htbp]
  \centering
  \begin{tabular}[l]{|l|l|l|l|p{3in}|}
    \hline
    \textbf{Name} & \textbf{Type} & \textbf{Use} & \textbf{Default} &
    \textbf{Comments} \\
    \hline
    \attribute{count} & float & optional & 1.0 & The number of times the loop is
    executed, on average (c.f.~\S\ref{sec:precedence})\\
    \hline
  \end{tabular}
  \caption{\label{tab:ActivityLoopType}Attributes for elements of type \schematype{ActivityLoopType}.}
\end{table}

\subsection{OutputResultType}
\label{sec:OutputResultType}

The type \schematype{OutputResultType}, shown in Figure~\ref{fig:OutputResultType}, is used to create
elements that store results described earlier in Section~\ref{sec:results}.  \schematype{OutputResultType}
is a subtype of \schematype{ResultContentType}.  This latter type defines the result element's attributes.
Elements of this \schematype{OutputResultType} can contain two elements of type
\schematype{ResultContentType}, which contain the $\pm 95\%$ and $\pm 99\%$ confidence intervals, provided
that these results are available.  The attributes for elements of \schematype{ResultContentType} are listed
in Table~\ref{tab:ResultContentType} and are used to store the actual results produced by the solver.  Note
that all the attributes are optional: elements of this type will only have those attributes which are
relevant.

\begin{figure}[htbp]
  \centering
  \epsffile{xml-schema/result-schema.eps}
  \caption{Schema diagram for type \schematype{OutputResultType}}
  \label{fig:OutputResultType}
\end{figure}

\begin{table}[htbp]
  \centering
  \begin{tabular}[l]{|l|l|p{2.5in}|c|}
    \hline
    \textbf{Name} & \textbf{Type} & \textbf{Comments} & \textbf{(xref)}\\
    \hline
    \hline
    \attribute{proc-utilization} & float & Processor utilization for a task, entry, or activity. & \S\ref{sec:processor-wait-utilization-out}\index{utilization!processor}\\
    \hline
    \attribute{proc-waiting} & float & Waiting time at a processor for an activity. & \S\ref{sec:processor-wait-utilization-out}\index{queueing time!processor} \\
    \hline
    \attribute{phaseX-proc-waiting} & float & Waiting time at a processor for phase \emph{X} of an entry. & \S\ref{sec:processor-wait-utilization-out})\index{queueing time!processor} \\
    \hline
    \hline
    \attribute{open-wait-time} & float & Waiting time for open arrivals. \index{waiting time!open arrival} & \S\ref{sec:open-wait-out} \\
    \hline
    \attribute{service-time} & float & Activity service time. & \S\ref{sec:service-time-out}\index{service time}\\
    \hline
    \attribute{loss-probability} & float & Probability of dropping an asynchronous message. & \S\ref{sec:rendezvous-delay-out}\index{loss probability}\\
    \hline
    \attribute{phaseX-service-time} & float & Service time for phase X of an entry. & \S\ref{sec:service-time-out}\index{service time}\\
    \hline
    \attribute{service-time-variance} & float & Variance for an activity. & \S\ref{sec:service-time-variance-out}\index{service time!variance} \\
    \hline
    \attribute{phaseX-service-time-variance} & float & Variance for phase \emph{X} of an entry. & \S\ref{sec:service-time-variance-out}\index{service time!variance} \\
    \hline
    \attribute{phaseX-utilization} & float & Utilization for phase \emph{X} of an entry. & \S\ref{sec:througput-utilization-out} \\
    \hline
    \attribute{prob-exceed-max-service-time} & float & & \S\ref{sec:service-time-distribution-out}\index{service time!probability exceeded} \\
    \hline
    \attribute{squared-coeff-variation} & float & Squared coefficient of variation\index{coefficient of variation} over all phases of an entry & \S\ref{sec:service-time-variance-out}\index{service time!variance} \\
    \hline
    \attribute{throughput-bound} & float & Throughput bound for an entry. & \S\ref{sec:bounds-out}\index{throughput!bounds}\\
    \hline
    \hline
    \attribute{throughput} & float & Throughput for a task, entry or activity. & \S\ref{sec:througput-utilization-out}\index{throughput}\\
    \hline
    \attribute{utilization} & float & Utilization for a task, entry, activity. & \S\ref{sec:througput-utilization-out}\index{utilization!task}\index{utilization!entry}\\
    \hline
    \hline
    \attribute{waiting} & float & Rendezvous delay & \S\ref{sec:rendezvous-delay-out}\index{rendezvous!delay}\\
    \hline
    \attribute{waiting-variance} & float & Variance of delay for a rendezvous & \S\ref{sec:rendezvous-variance-out}\index{rendezvous!variance}\\
    \hline
  \end{tabular}
  \caption{\label{tab:ResultContentType}Attributes for elements of type \schematype{ResultContentType}.}
\end{table}

\subsection{OutputResultJoinDelayType}
\label{sec:OutputResultJoinDelayType}

The type \schematype{OutputResultJoinDelayType} is similar to \schematype{OutputResultType}.  The attributes
of this type are shown in Table~\ref{tab:OutputResultJoinDelay}.

\begin{figure}[htbp]
  \centering
  \epsffile{xml-schema/result-join-delay-schema.eps}
  \caption{Schema diagram for type \schematype{OutputResultJoinDelayType}}
  \label{fig:OutputResultJoinDelayType}
\end{figure}

\begin{table}[htbp]
  \centering
  \begin{tabular}[l]{|l|l|p{2.5in}|c|}
    \hline
    \textbf{Name} & \textbf{Type} & \textbf{Comments} & \textbf{(xref)}\\
    \hline
    \hline
    \attribute{join-waiting} & float & Join delay\index{join!delay} & \S\ref{sec:join-delay-out} \\
    \hline
    \attribute{join-variance} & float & Join delay variance\index{join!variance} & \S\ref{sec:join-delay-out} \\
    \hline
  \end{tabular}
  \caption{\label{tab:OutputResultJoinDelay}Attributes for elements of type
    \schematype{OutputResultJoinDelayType}.}
\end{table}

\subsection{OutputDistributionType}
\label{sec:OutputDistributionType}

Elements of type \schematype{OutputDistributionType}, shown in Figure~\ref{fig:OutputDistributionType}, are
used to define and store histograms\index{histogram} of phase and activity service times\index{service time!histogram}.
The optional \schemaelement{underflow-bin}, \schemaelement{overflow-bin} and \schemaelement{histogram-bin}
elements, all the elements are of type \schematype{HistogramBinType}, are used to store results.  

The attributes of \schematype{OutputDistributionType} elements are used to both store the parameters for the
histogram, and output statistics.  Refer to Table~\ref{tab:OutputDistributionType}

\begin{figure}[htbp]
  \centering
%  \epsfxsize=\textwidth 
  \epsffile{xml-schema/distribution-schema.eps}
  \caption{Schema for type \schematype{OutputDistributionType}.}
  \label{fig:OutputDistributionType}
\end{figure}

\begin{table}[htbp]
  \centering
  \begin{tabular}[l]{|l|l|l|l|p{2.8in}|}
    \hline
    \textbf{Name} & \textbf{Type} & \textbf{Use} & \textbf{Default} & \textbf{Comments} \\
    \hline
    \attribute{min} & float & required & & The lower bound of the collected histogram data. \\
    \hline
    \attribute{max} & float & required & & The upper bound of the collected histogram data. \\
    \hline
    \attribute{number-bins} & integer & optional & 20 & The number of bins in the distribution. \\
    \hline
    \hline
    \attribute{mid-point} & float & optional & & \\
    \hline
    \attribute{bin-size} & float & optional & &  \\
%     \hline
%     \attribute{mean} & float & optional & & The mean of the distribution. \\
%     \hline
%     \attribute{std-dev} & float & optional & & The standard deviation of the distribution. \\
%     \hline
%     \attribute{skew} & float & optional & & The skew of the distribution. \\
%     \hline
%     \attribute{kurtosis} & float & optional & & The kurtosis of the distribution. \\
    \hline
  \end{tabular}
  \caption{\label{tab:OutputDistributionType}Attributes for elements of type \schematype{OutputDistributionType}.}
\end{table}

\subsection{HistogramBinType}
\label{sec:HistogramBinType}

\begin{table}[htbp]
  \centering
  \begin{tabular}[l]{|l|l|p{2.5in}|c|}
    \hline
    \textbf{Name} & \textbf{Type} & \textbf{Comments} & \textbf{(xref)}\\
    \hline
    \hline
    \attribute{begin} & float & Lower limit of the bin. & \\
    \hline
    \attribute{end} & float & Upper limit of the bin. & \\
    \hline
    \attribute{prob} & float & The probability that the measured value lies within
    \attribute{begin} and \attribute{end}. & \\
    \hline
    \attribute{conf-95} & float & & \\
    \hline
    \attribute{conf-99} & float & & \\
    \hline
  \end{tabular}
  \caption{\label{tab:HistogramBinType}Attributes for elements of type
    \schematype{HistogramBinType}.}
\end{table}

\section{Schema Constraints}
\label{sec:schema-constraints}

The schema contains a set of constraints\index{schema!constraints}
that are checked by the Xerces\index{Xerces} XML
parser~\cite{sw:xerces} to ensure that the model file is
valid.  XML editors can also enforce these constraints so that the
model is somewhat correct before being passed to the simulator or
analytic solver.  The constraints are as follow:
\begin{itemize}
\item All processor must have a unique name.
\item All tasks must have a unique name.
\item All entries must have a unique name.
\item All activities must have a unique name within a given task.
\item All synchronous requests must have a valid destination.
\item All asynchronous requests must have a valid destination.
\item All forwarding requests must have a valid destination.
\item All activity connections (in precedence blocks) must refer to
  valid activities.\index{activity!connection}\index{precedence}
\item All activity replies must refer to a valid entry.
\item All activity loops must refer to a valid activities.
\item Each entry has only one activity bound to it.
\item Phases are restricted to values one through three.
\item All phase attributes\index{attribute!unique phase} within an
  entry must be unique.
\end{itemize}
Further validation is performed by the solver itself.  Refer to
Section~\ref{sec:error-messages} for the error messages generated.

One downside of using the Xerces\index{Xerces!error messages} XML
parser library is that the Xerces tends to give rather cryptic error
messages when compared to other tools.  If an XML file fails to pass
the validation phase, and the error looks cryptic, chances are very
good that there is a genuine problem with the XML input file.  Xerces\index{Xerces}
has a bad habit of coming back with cryptic errors when constraint
checking\index{constraint checking} fails, and only gives you the
general area in the file where the actual problem is.

One easy and convenient solution around this problem is to validate
the XML\index{XML!validation} file using another XML tool.  Tools that
have been found to give more user friendly feedback are
XMLSpy\index{XMLSpy} (any edition), and XSDvalid\index{XSDvalid} (Java
based, freely available).  Another solution is to check if a
particular tool can de-activate schema validation and rely on the
actual tool to do its own internal error checking.  Currently this is
not supported in any of the LQN tools which are XML enabled, but it
maybe implemented later on.

If the XML file validates using other tools, but fails validation with
Xerces\index{Xerces!validation}, or if the XML file fails validation on other tools, but passes
with Xerces then please report the problem.  The likelihood of
validation passing with Xerces and not other tools will be much higher
then the reverse scenario, because Xerces does not rigorously apply
the XML Schema standard as other tools.  Other sources of problems
could be errors in the XML schema itself, or some unknown bug in the
Xerces library.
\index{XML Grammar)|}\index{Grammar!XML)|}

%%% Local Variables: 
%%% mode: latex
%%% mode: outline-minor 
%%% fill-column: 108
%%% TeX-master: "userman"
%%% End: 
