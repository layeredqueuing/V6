%% -*- mode: latex; mode: outline-minor; fill-column: 108 -*-
%% Title:  grammar
%%
%% $HeadURL: http://rads-svn.sce.carleton.ca:8080/svn/lqn/trunk/doc/userman/infile.tex $
%% Original Author:     Greg Franks <greg@sce.carleton.ca>
%% Created:             Tue Jul 18 2006
%%
%% ----------------------------------------------------------------------
%% $Id: infile.tex 12964 2017-03-27 19:17:20Z greg $
%% ----------------------------------------------------------------------

\documentclass{article}
\usepackage{times}
\usepackage{bnf}

\oddsidemargin  18pt
\evensidemargin  18pt
\textwidth   6.5in
%\topmargin   0pt
%\textheight  8.5in
%\marginparwidth 0pt
%\marginparsep 0pt
\setcounter{secnumdepth}{0}

\begin{document}

\title{SRVN Input File Format\\ {\small\rm\$Id: infile.tex 12964 2017-03-27 19:17:20Z greg $\$$}}
\author {Dorina C. Petriu \and R. Gregory Franks \and Alex Hubbard}
\date{\today} 
\maketitle

\section{SRVN Input File Format}
The Stochastic Rendezvous Network (SRVN) is a modelling method under
development, and is currently being upgraded and extendend. Due to
this fact, the input file format was chosen to be as flexible as
possible, and easy to be expand when the need for new input data
occurs.

The SRVN input file has four sections, each consisting of a variable
length list, possibly with other lists nested therein.  The {\em end
of list} marker is $-1$.

The input file may be annotated with comments.\marginpar{\chgflg{3}}
Comments start with the octothorpe symbol, `{\tt \#}', and finish at the
end of the line.  Comments may appear anywhere except within strings.

\begin{description}
\item [General information section] gives input parameters related to
  the numerical computation.  It consists of five fields:
  \begin{enumerate}
  \item the convergence value,
  \item the underrelaxation coefficient,
  \item the maximum number of iterations,
  \item the number of iterations after which intermediate results
    are printed,
  \item and a field for {\em comments on model} which is not
    interpreted by the analysis program.
  \end{enumerate}
  
\item [Processor information section] declares the total number of
  processors in the system, and the corresponding scheduling policy for
  each processor.  The scheduling policy may be First Come First Served
  (FCFS) (the default value), Priority Preemptive Resume (PRI) or Head
  Of the Line (HOL)\footnote{Also known as Priority, Non-preemptive.}.
  The processors are named using symbolic identifiers.
  
\item [Task information section] declares the total number of tasks in
  the system, and the following parameters for each task: task type
  (reference or non-reference), the list of entries belonging to this
  task, the processor on which the task is allocated, and task priority
  (if it applies). The tasks are named using symbolic identifiers. Each
  task must have at least one entry because the workload parameters are
  associated with entries.  As a particular case, a reference task has
  one and only one entry, which is never called by other tasks
  (entries).
  
\item [Entry information section] declares the total number of entries
  and the workload parameters associated with each. The entries are
  named using symbolic identifiers. Each entry is associated to a task
  by way of an {\em entry list} for the task. (The model description is
  invalid if an entry is associated to more than one task, or no task at
  all).  In the BNF description below, the non-terminals {\em
  from\_entry\/} and {\em to\_entry\/} must belong to separate tasks.

The entry workload is divided in phases.  The input file format does
not limit the number of phases: it uses a variable length list with
parameters ordered by phase.  Default values are assumed for all
undeclared parameters.  The parameters declared for entries are:

\begin{itemize}
\item open arrival rate to entry (default = 0);
\item list of mean service times per phase (default = 0);
\item list of phase type flags (default = 0):
  \begin{enumerate}
  \item exponentially distributed service time and geometrically
    distributed number of calls;
  \item deterministic service time and deterministic number of
    calls;
  \item exponential service time and deterministic number of calls;
  \end{enumerate}
\item coefficient of variation of the service time slice between two
  calls, given by phase (default = 1.0);
\item mean number of RNVs per phase from this entry to other entries
  (default = 0.0);
\item mean number of nonblocking sends per phase from this entry to
  other entries (default = 0.0).
\item probability of forwarding a rendezvous to another entry (default
  = 0.0).
\item internal messaging between entries.
\end{itemize}
\end{description}

\section{BNF Description of SRVN Input File Format}

%% -*- mode: latex; mode: outline-minor; fill-column: 108 -*-
%% Title:  grammar
%%
%% $HeadURL: http://rads-svn.sce.carleton.ca:8080/svn/lqn/trunk-V6/doc/userman/input-grammar.tex $
%% Original Author:     Greg Franks <greg@sce.carleton.ca>
%% Created:             Tue Jul 18 2006
%%
%% ----------------------------------------------------------------------
%% $Id: input-grammar.tex 14587 2021-04-02 18:09:46Z greg $
%% ----------------------------------------------------------------------

\newcommand{\altflg}[1]{#1}
\newcommand{\swapflg}{\let\tmpflg\chgflg\let\chgflg\altflg\let\altflg\tmpflg}

\begin{bnf}{multi\_server\_flag}
  \defitem[LQN\_input\_file] \<general\_info> \<processor\_info> \opt{\<group\_info>} 
  \<task\_info> \<entry\_info> \rep{\<activity\_info>}{0}{}
  \oritem \<parameter\_list> \<processor\_info> \opt{\<group\_info>} 
  \<task\_info> \<entry\_info> \rep{\<activity\_info>}{0}{} \opt{\<report\_info>} \opt{\<convergence\_info>}
\end{bnf}

\subsection{SPEX Parameters}
\label{sec:spex-parameters}\index{SPEX!grammar!paramters}

\begin{bnf}{multi\_server\_flag}
  \defitem[parameter\_list]  \rep{\<comma\_expr>}{1}{np}

  \defitem[comma\_expr] \<variable\_def>
  \oritem[ \<expression> , \<variable\_def> ]

  \defitem[variable\_def] \<variable> = \<ternary\_expr>
  \oritem [ \< expression\_list> ]
  \oritem [ \<real> : \<real> , \<real> ]

\end{bnf}

\subsection{General Information}
\label{sec:general-in}

\begin{bnf}{multi\_server\_flag}
  \defitem[general\_info] G \<comment> \<conv\_val> \<it\_limit>
  \?\<print\_int>? \?\<underrelax\_coeff>? \<end\_list>

  \defitem[comment] \<string> \(comment on the model)\index{model!comment}

  \swapflg
  \defitem[conv\_val] \<real> \(convergence value) \chg{\ddag}\index{convergence!value}

  \defitem[it\_limit] \<integer> \(max. nb. of iterations) \chg{\ddag}\index{iteration limit}

  \defitem[print\_int] \<integer> \chg{\ddag}\index{print interval}
  \itemsep\oritemsep
  \item \(intermed. res. print interval)
  \itemsep\defitemsep

  \defitem[underrelax\_coeff] \<real> \(under\_relaxation coefficient)\index{under-relaxation coefficient}
  \chg{\ddag}
  \swapflg

  \defitem[end\_list] -1 \(end\_of\_list mark)

  \defitem[string] " \<text> "
\end{bnf}

\subsection{Processor Information}
\label{sec:processor}

\begin{bnf}{multi\_server\_flag}
  \defitem[processor\_info] P \<np> \<$p$\_decl\_list>

  \defitem[np] \<integer> \(total number of processors)

  \defitem[$p$\_decl\_list] \rep{\<$p$\_decl>}{1}{np} \<end\_list>

  \defitem[$p$\_decl] p \<proc\_id> \<scheduling\_flag> \?\<quantum>?\index{scheduling!processor}
  \?\<multi\_server\_flag>?  \?\<replication\_flag>? \?\<proc\_rate>?\index{replication!processor}

  \defitem[proc\_id] \<integer> \| \<identifier>
  \itemsep\oritemsep
  \item \(processor identifier)
  \itemsep\defitemsep

  \defitem[scheduling\_flag] f \(First come, first served)
  \oritem h \(Head Of Line)
  \oritem p \(Priority, preemeptive)
  \oritem c \<real> \(completely fair scheduling\index{processor!scheduling!completely fair})
  \oritem s \<real> \(processor sharing\index{processor!sharing}\index{processor!scheduling!sharing})
  \oritem i \(Infinite or delay)
  \oritem r \(Random)

  \defitem[quantum] \<real>\index{quantum} \| \<variable>

  \defitem[multi\_server\_flag] m \<copies> \(number of duplicates)
  \oritem i \(Infinite server)

  \defitem[replication\_flag] r \<copies> \(number of replicas)

  \defitem[proc\_rate] R \<ratio> \| \<variable> \(Relative proc. speed)

  \defitem[copies] \<integer> \| \<variable>

  \defitem[ratio] \<real> \| \<variable>

\end{bnf}

\subsection{Group Information}
\label{sec:group}

\begin{bnf}{multi\_server\_flag}
  \defitem[group\_info] U \<ng> \<$g$\_decl\_list> \<end\_list>

  \defitem[ng] \<integer> \(total number of groups)

  \defitem[$g$\_decl\_list] \rep{\<$g$\_decl>}{1}{ng} \<end\_list>

  \defitem[$g$\_decl] g \<group\_id> \<group\_share> \opt{\<cap\_flag>} \<proc\_id>

\defitem[group\_id] \<identifier>

\defitem[group\_share] \<real> \| \<variable>

\defitem[cap\_flag] c
\end{bnf}

\subsection{Task Information}
\label{sec:task}

\begin{bnf}{multi\_server\_flag}
  \defitem[task\_info] T \<nt> \<$t$\_decl\_list>

  \defitem[nt] \<integer> \(total number of tasks)

  \defitem[$t$\_decl\_list] \rep{\<$t$\_decl>}{1}{nt} \<end\_list>

  \defitem[$t$\_decl] t \<task\_id> \<task\_sched\_type>\index{scheduling!task}
  \<entry\_list> \?\<queue\_length>?\index{queue length}
  \<proc\_id> \?\<task\_pri>? \?\<think\_time\_flag>?\index{think time} \?\<tokens>?
  \?\<multi\_server\_flag>? \?\<replication\_flag>?\index{replication!task}
  \?\<group\_flag>?\index{group share}
  \oritem I \<from\_task> \<to\_task> \<fan\_in>\index{fan-in}
  \oritem O \<from\_task> \<to\_task> \<fan\_out>\index{fan-out}

  \defitem[task\_id] \<integer> \| \<identifier>
  \itemsep\oritemsep
  \item \(task identifier)
  \itemsep\defitemsep

  \defitem[task\_sched\_type] r \(reference task)
  \oritem n \(non-reference task)
  \oritem h \(Head of line)
  \oritem f \(FIFO Scheduling)
  \oritem i \(Infinite or delay server)
  \oritem p \(Polled scheduling at entries)
  \oritem b \(Bursty Reference task)
  \oritem S \(Semaphore)\index{scheduling!semaphore}

  \defitem[entry\_list] \rep{\<entry\_id>}{1}{{\it ne\/}_t} \<end\_list>
  \itemsep\oritemsep
  \item \(task $t$ has $ne_t$ entries)
  \itemsep\defitemsep

  \defitem[entry\_id] \<integer> \| \<identifier>
  \itemsep\oritemsep
  \item \(entry identifier)
  \itemsep\defitemsep

  \defitem[task\_pri] \<integer> \(task priority, optional)

  \defitem[queue\_length] q \<integer> \(open class queue length)

  \defitem[group\_flag] g \<identfier> \(Group for scheduling)

  \defitem[tokens] t \<integer> \(Initial tokens)

  \defitem[from\_task] \<task\_id> \(Source task)
  \defitem[to\_task] \<task\_id> \(Destination task)
  \defitem[fan\_in] \<integer> \(fan in to this task)
  \defitem[fan\_out] \<integer> \(fan out from this task)
\end{bnf}


\subsection{Entry Information}
\label{sec:entry}

\begin{bnf}{multi\_server\_flag}
  \defitem[entry\_info] E \<ne> \<entry\_decl\_list>

  \defitem[ne] \<integer> \(total number of entries)

  \defitem[entry\_decl\_list] \rep{\<entry\_decl>}{1}{} \<end\_list>

  \item \(k = maximum number of phases)

  \defitem[entry\_decl] a \<entry\_id> \<arrival\_rate>
  \oritem A \<entry\_id> \<activity\_id> 
  \oritem F \<from\_entry> \<to\_entry> \<p\_forward>\index{forwarding probability}
  \oritem H \<entry\_id> \<phase> \<hist\_min> ':' \<hist\_max> \<hist\_bins> \<hist\_type>\index{histogram}
  \oritem M \<entry\_id> \rep{\<max\_service\_time>}{1}{k} \<end\_list>\index{maximum service time}\index{service time!maximum}
  \oritem P \<entry\_id> \(Signal Semaphore)\index{semaphore!signal}
  \oritem V \<entry\_id> \(Wait Semaphore)\index{semaphore!signal}
  \oritem Z \<entry\_id> \rep{\<think\_time>}{1}{k} \<end\_list>\index{think time}
  \oritem c \<entry\_id> \rep{\<coeff\_of\_variation>}{1}{k} \<end\_list>\index{coefficient of variation}
  \oritem f \<entry\_id> \rep{\<ph\_type\_flag>}{1}{k} \<end\_list>\index{phase!type}
  \oritem p \<entry\_id> \<entry\_priority>\index{priority!entry}
  \oritem s \<entry\_id> \rep{\<service\_time>}{1}{k} \<end\_list>\index{service time}
  \oritem y \<from\_entry> \<to\_entry> \rep{\<rendezvous>}{1}{k} \<end\_list>\index{rendezvous}
  \oritem z \<from\_entry> \<to\_entry> \rep{\<send\_no\_reply>}{1}{k} \<end\_list>\index{send-no-reply}

  \defitem[arrival\_rate] \<real> \| \<variable> \(open arrival rate to entry)
  \defitem[coeff\_of\_variation] \<real> \| \<variable> \(squared service time coefficient of variation)
  \defitem[from\_entry] \<entry\_id> \(Source of a message)
  \defitem[hist\_bins] \<integer> \(Number of bins in histogram.)
  \defitem[hist\_max] \<real> \(Median service time.)
  \defitem[hist\_min] \<real> \(Median service time.)
  \defitem[hist\_type] log \| linear \| sqrt \(bin type.)
  \defitem[max\_service\_time] \<real> \(Median service time.)
  \defitem[p\_forward] \<real> \(probability of forwarding)
  \defitem[phase] 1 \| 2 \| 3 \(phase of entry)
  \defitem[ph\_type\_flag] 0 \(stochastic phase)
  \oritem 1 \(deterministic phase)
  \defitem[rate] \<real> \| \<variable> \(nb. of calls per arrival)
  \defitem[rendezvous] \<real>  \| \<variable> \(mean number of RNVs/ph)
  \defitem[send\_no\_reply] \<real> \| \<variable> \(mean nb.of non-blck.sends/ph)
  \defitem[service\_time] \<real> \| \<variable> \(mean phase service time)
  \defitem[think\_time] \<real> \| \<variable> \(Think time for phase.)
  \defitem[to\_entry] \<entry\_id> \(Destination of a message)
\end{bnf}

\subsection{Activity Information}
\label{sec:activity}

\begin{bnf}{multi\_server\_flag}

  \defitem[activity\_info] \<activity\_defn\_list> \?\<activity\_connections>? \<end\_list>

\item \( Activity definition. )

  \defitem[activity\_defn\_list] \rep{\<activity\_defn>}{1}{\it na}
  \defitem[activity\_defn] c \<activity\_id> \<coeff\_of\_variation> \( Sqr. Coeff. of Var. )\index{coefficient of variation}
  \oritem f \<activity\_id> \<ph\_type\_flag> \( Phase type )\index{phase!type}
  \oritem H \<entry\_id> \<hist\_min> ':' \<hist\_max> \<hist\_bins> \<hist\_type>\index{histogram}
  \oritem M \<activity\_id> \<max\_service\_time>\index{maximum service time}\index{service time!maximum}
  \oritem s \<activity\_id> \<ph\_serv\_time> \( Service time )\index{service time}
  \oritem Z \<activity\_id> \<think\_time> \( Think time )\index{think time}
  \oritem y \<activity\_id> \<to\_entry> \<rendezvous> \( Rendezvous  )\index{rendezvous}
  \oritem z \<activity\_id> \<to\_entry> \<send\_no\_reply> \( Send-no-reply )\index{send-no-reply}

\item \( Activity Connections. )\index{activity!connection|see{precedence}}\index{activity!connection}

  \defitem[activity\_connections] : \<activity\_conn\_list>

  \defitem[activity\_conn\_list] \<activity\_conn> \rep{; \<activity\_conn>}{1}{\it na}

  \defitem[activity\_conn] \<join\_list>
  \oritem \<join\_list> -> \<fork\_list>\index{->@\texttt{->}|see{precedence}}

  \defitem[join\_list] \<reply\_activity>\index{join-list}
  \oritem \<and\_join\_list>\index{AND-join}
  \oritem \<or\_join\_list>\index{OR-join}

  \defitem[fork\_list] \<activity\_id>
  \oritem \<and\_fork\_list>\index{and-fork}
  \oritem \<or\_fork\_list>\index{or-fork}
  \oritem \<loop\_list>\index{LOOP}

  \defitem[and\_join\_list] \<reply\_activity> \rep{\& \<reply\_activity>}{1}{\it na} \opt{\<quorum\_count>}
  \defitem[or\_join\_list] \<reply\_activity> \rep{+ \<reply\_activity>}{1}{\it na}
  \defitem[and\_fork\_list] \<activity\_id> \rep{\& \<activity\_id>}{1}{\it na}
  \defitem[or\_fork\_list] \<prob\_activity> \rep{+ \<prob\_activity>}{1}{\it na}
  \defitem[loop\_list] \<loop\_activity> \rep{, \<loop\_activity>}{0}{na} \opt{\<end\_activity>}

  \defitem[prob\_activity] ( \<real> ) \<activity\_id>
  \defitem[loop\_activity] \<real> * \<activity\_id>
  \defitem[end\_activity] , \<activity\_id>
  \defitem[reply\_activity] \<activity\_id> \opt{\<reply\_list>}

  \defitem[reply\_list] [ \<entry\_id> \rep{, \<entry\_id> }{0}{ne} ]
  \defitem[quorum\_count] ( \<integer> ) \( Quorum )\index{quorum}
\end{bnf}

\subsection{SPEX Report Information}\index{SPEX!grammar!report}
\begin{bnf}{multi\_server\_flag}
  \defitem[report\_info] R \<nr> \<report\_decl\_list>  \<end\_list>
  \oritem R \<nr> \<identifier> ( \<expression\_list> )

  \defitem[report\_decl\_list] \rep{\<$r$\_decl>}{1}{nr}

  \defitem[$r$\_decl] \<variable> = \<ternary\_expr>
  \oritem \<expression>
\end{bnf}

\subsection{SPEX Convergence Information}\index{SPEX!grammar!convergence}
\begin{bnf}{multi\_server\_flag}
  \defitem[convergence\_info] C \<nc> \<convergence\_decl\_list>  \<end\_list>

  \defitem[convergence\_decl\_list] \rep{\<$c$\_decl>}{1}{nr}

  \defitem[$c$\_decl] \<variable> = \<ternary\_expr>
\end{bnf}

\subsection{Expressions}\index{SPEX!grammar!expressions}
\label{sec:spex-expressions}
\begin{bnf}{multi\_server\_flag}

  \defitem[ternary\_expression] \<or\_expression> ? \<or\_expression> : \<or\_expression>
  \oritem \<or\_expression>

  \defitem[or\_expression] \<or\_expression> $|$ \<and\_expression> \( Logical OR )
  \oritem \<and\_expression>

  \defitem[and\_expression] \<and\_epxression> $\&$ \<compare\_expression> \( Logical AND )
  \oritem \<compare\_expression>

  \defitem[compare\_expression] \<compare\_expression> == \<expression>
  \oritem \<compare\_expression> != \<expression>
  \oritem \<compare\_expression> < \<expression>
  \oritem \<compare\_expression> <= \<expression>
  \oritem \<compare\_expression> > \<expression>
  \oritem \<compare\_expression> >= \<expression>
  \oritem \<expression>

  \defitem[expression] \<expression> + \<term>
  \oritem \<expression> $-$ \<term> 
  \oritem \<term>

  \defitem[term] \<term> * \<power>
  \oritem \<term> / \<power>
  \oritem \<term> \% \<power> \( Modulus )
  \oritem \<power>

  \defitem[power] \<prefix> ** \<power> \( Exponentiation, right associative) % \char94 forces raw caret.
  \oritem \<prefix>

  \defitem[prefix] ! \<factor> \( Logical NOT )
  \oritem \<factor>

  \defitem[factor] ( \<expression> )
  \oritem \<identifier> ( \<expression\_list> )
  \oritem \<variable> [ \<expression\_list> ]
  \oritem \<variable>
  \oritem \<double>

  \defitem[expression\_list] \<expression> \rep{, \<expression> }{0}{}

  \defitem[int] \( Non negative integer )
  \defitem[double] \( Non negative double precision number )
  
\end{bnf}

\subsection{Identifiers}
% This paragraph describes the identifiers. Using text was alot easier
% than introducing some new BNF symbols, etc.
% Added July 23, 1991 by alex

Identifiers\index{identifiers} may be zero or more leading underscores
(`\_'), followed by a character, followed by any number of characters,
numbers or underscores. Punctuation characters and other special
characters such as the dollar-sign (`\$') are not permitted.  The
following, {\tt \_p1}, {\tt foo\_bar}, and {\tt \_\_P\_21\_proc} are
valid identifiers, while {\tt \_21} and {\tt \$proc} are not.

\subsection{Variables}\index{SPEX!variables}

SPEX variables all start with a dollar-sign (`\$') followed by any number of characters, numbers or
underscores (`\_').  The following, \texttt{\$s1}, \texttt{\$1}, and \texttt{\$\_x} are all valid variables.
SPEX variables are treated as global symbols by the underlying LQX program.  Variables used to store arrays
will also generate a \emph{local} variable of the same name, except without the leading dollar-sign.  

%%% Local Variables: 
%%% mode: latex
%%% TeX-master: "userman"
%%% End: 


\subsection{Other Solvers}

\subsubsection{Lqsim}
The SRVN input file is also used by the layered queueing network simulator.
\marginpar{\chgflg{3}} Certain optional nonterminals only apply to the
simulator (for example, {\em cpu\_quantum}).  These nonterminals are
identified by a `\dag'.  If they are present as input to the SRVN
solver, they are ignored.  Mandatory fields marked by a `\ddag' are
ignored by the simulator.

\subsubsection{PetriSRVN}

Mandatory fields marked by a `\ddag' are ignored by the Petri net
solver.

\subsubsection{Layered Queueing Network Solver}


\subsubsection{Bounds Solver}

\end{document}
