\C 
\C $Id: population.tex 5439 2005-01-02 02:30:31Z greg $
\C 
\C $Log$
\C Revision 31.0  2005/01/02 02:30:29  greg
\C Bring all files up to revision 31.
\C
\C Revision 30.0  2003/11/12 18:56:52  greg
\C Bring all files up to revision 30.0
\C
\C Revision 29.0  2003/06/13 00:48:38  greg
\C bounce version number.
\C
\C Revision 1.4  1995/01/11 21:05:12  greg
\C Revised for R12 of MVA software.
\C
\C Revision 1.3  1994/08/04  20:52:11  greg
\C Intermediate save
\C
\C Revision 1.2  1994/06/24  19:11:32  greg
\C Format changes.
\C
\C Revision 1.1  1994/06/23  19:11:01  greg
\C Initial revision
\C
\C ----------------------------------------------------------------------
\section{Population}
\label{sec:population}

Population objects are like arrays except that the index to elements
stored therein is a \emph{vector} rather than a scalar; the vector
used as the index is called a \emph{Population Vector} and has the
type \dfn{PopVector}.  The class \code{Population} itself is an
abstract superclass.  Populations should be either of class
\link{\code{GeneralPop}}{sec:general-population} or of class
\link{\code{SpecialPop}}{sec:special-population}.

General populations allow the number of customers to vary from zero to
the maximum number set by the
\link{\code{capacity()}}{sec:population-capacity} function.  This
class is used with \link{Exact MVA}{sec:mva-exact}.  The number of
elements allocated for the \link{\var{array}}{sec:population-ivars} is
given by the expression \math[n1*n2*...*nk]{n_1 \times n_2 \times ...
  \times n_k} where \math[nk]{n_k} is the maximum number of customers
in class \math[k]{k}.

Special Populations are used by all of the \link{approximate
  MVA}{sec:mva} solvers.  It uses far less array space.

Population Vectors can be generated automatically for General
Populations using the \link{\file{PopulationIterator}}[
(\Sec\Ref)]{sec:popiter} class.

\subsection{Class Population --- Abstract Superclass}
\subsubsection{Instance Variables}
\label{sec:population-ivars}

\begin{verbatim}
protected:
unsigned nClasses;
Vector<unsigned> maxCust;
Vector<unsigned> stride;
double *array;
unsigned arraysize;
\end{verbatim}

\begin{description}
\item[nClasses] \texonly{---} The number of classes defined for this
  \file{Population} object.
\item[maxCust] \texonly{---} The maximum number of customers in each class.
\item[stride] \texonly{---} Array strides for each class.
\item[array] \texonly{---} Data storage.
\item[arraysize] \texonly{---} Allocated array size.
\end{description}

\subsubsection{Constructors for Population}

Population objects are created with the following constructor.  Due to
quirks in C++ regarding the allocation arrays of objects, the object
is intialized with the \code{capacity()} call after it has been
created.

\code{Population();}

\subsubsection{Initialization}

\begin{description}
\item[capacity] \texonly{---} Public member function\\
  \label{sec:population-capacity}
  \code{void capacity( const PopVector &N );}

  The \code{capacity()} call is used to allocated storage for the
  Population objects.  The argument \var{N} specifies the maximum number
  of customers in each customer class (and is used to set the instance
  variable \link{\var{maxCust}}{sec:population-ivars}.  Classes range
  from 1 to \link{\var{nClasses}}{sec:population-ivars}.  Offset zero of
  the vector \var{N} is not used.
\end{description}


\subsubsection{Copying}
\label{sec:population-copy}

The \code{save()} and \code{restore()} functions are used to copy
subsets of the population space from one population object to another.
They are used by the \link{linearizer}{sec:mva-linearizer} MVA solvers.
\begin{description}

\item[save] \texonly{---} Public member function\\
  \code{void save( Population & src, const PopVector & N );}

  \var{save()} copies the element at {\bf N} and \math{{\bf N} - e_j}
  for all \math{j} in \link{\var{nClasses}}{sec:population-ivars}.

\item[Restore] \texonly{---} Public member function\\
  \code{void restore( Population & src, const PopVector & N );}

  \var{restore()} performs the inverse operation \emph{except} that it
  does not copy the value at {\bf N}.

\end{description}

\subsubsection{Instance Variable Access}

\begin{description}
\item[Index] \texonly{---} Operator\\
  \code{double& operator[]( const PopVector & N );}\\
  \code{double operator[]( const PopVector & N );}\\
  \code{double& operator[]( const unsigned n );}

  Elements in \code{Population} objects are accessed throught the C++
  operator `\code{[]}', so that population objects look like arrays.
  However, unlike \code{Vector} objects and conventional arrays which
  take scalars, the \emph{index} to \code{Population} objects is of type
  \code{Vector<unsigned>}.  The vector argument must have at least as
  many elements as \link{\var{nClasses}}{sec:population-ivars} and begin
  at offset 1.  Each element must be no larger than the value specified
  by its corresponding element in
  \link{\var{maxCust}}{sec:population-ivars}.  Range checking is
  performed on both the size of the population vector, and the elements
  therin.

  A special index operator taking a simple scalar \var{n} is also
  supplied.  However, {\bf no} range checking is performed.  This
  operator is used in conjunction with the \link{\code{offset()}
    function}{sec:population-offset} described elsewhere and is provided
  to improve the performance of the MVA solvers.

\item[e\_j] \texonly{---} Special Access Functions\\
  \code{double& e_j( const PopVector & N, const unsigned j );}\\
  \code{double& e_c_e_j( const unsigned c, const unsigned j );}

  \code{e_j()} and \code{e_c_e_j()} are also provided to store and
  retrieve data in \code{Population} objects.  The \code{e_j()} function
  performs the operation \math{{\bf N} - e_j}, that is, it accesses the
  element in the population space at the population vector {\bf N} with
  one customer removed from class \var{j}.  There must be at least one
  customer in class \var{j}.  \code{e_c_e_j()} performs a similar
  function except that customers are removed from both class \var{c} and
  \var{j} (See \link{offset functions}[~Section
  \Ref]{sec:population-offset}).  Class \var{c} and \var{j} may, in
  fact, be the same class.

  Both the \code{[]} operator and the \code{e_j()} and \code{e_c_e_j()}
  functions can be used on either side of the assignment operator (i.e.,
  they can be used to both store and retrive values).
\end{description}

\subsubsection{Queries}

\begin{description}

\item[offset] \texonly{---} Array offset functions\\
  \label{sec:population-offset}
  \code{unsigned offset( const PopVector& N)}\\
  \code{unsigned offset_e_j( const PopVector& N, const unsigned j )}\\
  \code{unsigned offset_e_c_e_j( const unsigned c, const unsigned j )}

  The \code{offset()} function is used to translate the population vector
  \var{N} to an offset into \var{array}.  It range checks both the
  number of elements in \var{N}, and the size of each of the elements.

  \code{Offset_e_j()} performs the same function as \code{offset()},
  except one customer is removed from class \var{j} first.  There must
  be at least one customer in class \var{j}.

  \code{Offset_e_c_e_j()} also computes an offset into \var{array}.
  However, one customer is removed from both class \var{c} and \var{j}
  at the maximum population value
  \link{\var{maxCust}}{sec:population-ivars}.  Both \var{c} and \var{j}
  may be the same class, or may be zero.  A value of zero for either
  \var{c} or \var{j} means that no customers are removed for the
  corresponding variable.  It is assumed that there are sufficient
  customers in each class.

\item[Not] \texonly{---} Operator\\
  \code{int operator!() const;}

  The \code{!} operator will return true if the value of the population
  space at the maximum customer level equals zero.  Otherwise, it
  returns false.

\item[maxPopValue] \texonly{---} Public member function\\
  \code{double maxPopValue() const;}

  \code{maxPopValue()} returns the value of the population space at the
  maximum number of customers.


\item[n] \texonly{---} Public member function\\
  \code{double n( const unsigned k ) const;}

  \code{n()} returns the maximum number of customers for class \var{k}.

\end{description}

\subsubsection{Input and Output}
\label{sec:population-io}
\begin{description}
\item[Insertion] \texonly{---} Operator\\
  \code{friend ostream& operator<<( ostream&, const Population& );}

  Print out the the maximum customer population
  \link{\var{maxCust}}{sec:population-ivars} and its value.
\end{description}

\subsection{Class GeneralPop}
\label{sec:general-population}

General populations store values for all possible population vectors
(which is necessary for \link{exact MVA}{sec:mva-exact}).
Consequently, the amount of array space necessary is the product of
the number of customers in each customer class.  

\subsubsection{Initialization}
\begin{description}
\item[capacity] \texonly{---} Public member function\\
  \code{void capacity( const PopVector &N );}
\end{description}

\subsubsection{Queries}

\begin{description}

\item[offset] \texonly{---} Array offset functions\\
  \code{unsigned offset( const PopVector& N)}\\
  \code{unsigned offset_e_c_e_j( const unsigned c, const unsigned j )}

  Return the array offset for the specified population vector.
  \code{offset_e_c_e_j()} removes a customer from classes \var{c} and
  \var{j}.  A value of zero for either \var{c} or \var{j} denotes that
  no customer for the argument is to be removed.

\end{description}

\subsection{Class SpecialPop}
\label{sec:special-population}

Special populations only store values for a restricted range of
populations:
\begin{itemize}
\item the maximum population vector.
\item with one customer removed from class \var{c}.
\item with one customer removed from class \var{j}.
\end{itemize}
If class \var{c} and class \var{j} are the same, then two customers
are removed from class \var{c}.  

\htmlonly{The figure below}\texonly{Figure~\ref{fig:specialpop}} shows
the method which data is stored in this class.  The row and column
labels \var{c} and \var{j} correspond to the removal of \bold{one}
customer from class \emph{c} and \emph{j} respectively.  The
relationship \math[\var{c} >= \var{j}]{c >= j} must always hold.  The removal
of two customers from one class is accomplished by setting
\math[\var{c} == \var{j}]{c = j}.  The full population is represented
by setting \math[\var{c} = \var{j} = 0]{c = j = 0}.

\begin{figure}[htbp]
  \label{fig:specialpop}
  \begin{center}
    \T \tex \leavevmode \input{specialpop.epic}
    \caption{Special population array structure.}
    \H \htmlimage{mva.gif}
  \end{center}
\end{figure}

The index of the emphasized elements in the array (\link{cells 0, 4
  and 7 }[in Figure~\Ref]{fig:specialpop}) correspond to the elements
of the \link{\var{stride}}{sec:population-ivars} array, except in
reverse order.  The total array space allocated,
\link{\var{arraysize}}{sec:population-ivars}, is the sum of the maximum
number of customers in each class.

\subsubsection{Initialization}
\begin{description}
\item[capacity] \texonly{---} Public member function\\
  \code{void capacity( const PopVector &N );}
\end{description}

\subsubsection{Queries}

\begin{description}
\item[offset] \texonly{---} Array offset functions\\
  \code{unsigned offset( const PopVector& N )}\\
  \code{unsigned offset_e_c_e_j( const unsigned c, const unsigned j )}

  Compute and return array offset.  The population vector passed to
  \code{offset()} can have no more than two customers removed from the
  population space.  A run time exception will be generated otherwise.

\end{description}

\subsection{Class PopulationIterator}
\label{sec:popiter}

The \file{PopulationIterator} class is used to generate all feasible
population vectors from 1 customer to the maximum number of customers
in all classes.  Instances of this class define a function which is
used to update a user-supplied population vector with the next
population value.  If the user changes the value of the population
vector between calls to the iterator, the ``next'' population will be
derived from the value set by the user.  The iterator function returns
a non-zero value for updates which result in a valid population
vector.  Otherwise, the population vector is reset to all zeros, and
the function returns zero (0).

This class is subclassed as necessary to generate population vectors
using different constraints.

\subsubsection{Instance Variables}
\label{sec:popiter-ivars}
\begin{verbatim}
protected:
    PopVector limit;
\end{verbatim}

\begin{description}
\item[limit] \texonly{---} The maximum number of customers permitted
  in each class.
\end{description}

\subsubsection{Constructors for PopulationIterator}

Population Iterators are created with the \code{PopulationIterator(N)}
constructor where the argument \var{N} is a population vector defining
the maximum number of customers in each class.  The size of the
population vector \var{N} determines the number of classes. 

\begin{verbatim}
    PopulationIterator( const PopVector& N);
\end{verbatim}

\subsubsection{Computation}

\begin{description}
\item[Next Population] \texonly{---} Operator\\
  \code{int operator()(PopVector& N);}

  Compute the next population value.  Return 1 if the new population is
  valid (i.e., the number of customers in each class is less that or
  equal to the number of customers in the corresponding position in the
  \var{limit} vector.).  Return 0 and reset the argument \var{N} to all
  zeros otherwise.

\item[step] \texonly{---} Private member function\\
  \code{int step( PopVector& N, const unsigned k );}

  Increase the number of customers for class \var{k} by one.  If the
  number of customers exceeds the limit for this class, then reset the
  count to zero, and \code{step()} the next class.

\end{description}

\C
\C ----------------------------------------------------------------------
\C
\subsection{Class B_Iterator}
\label{sec:pop-BIter}

The \file{B_Iterator} class is used to generate all feasible
populations for the \var{xr} term of \link{first-come-first-served
multi-servers}[~(\Sec\Ref)]{sec:multiserver} (\link{Eqn
(16)}[~\Cite]{queue:deSouzaeSilva-87}).  The following constraints are
enforced:

\begin{iftex}
  \tex
  \begin{eqnarray*}
    |{\bf n}| & = & J \\
    {\bf n} & \leq & ({\bf N} - e_k)
  \end{eqnarray*}
\end{iftex}
\begin{ifhtml}
  \begin{tabular}
    n1 + ... + nk & = & J \\
    n & <= & (N - ek)
  \end{tabular}
\end{ifhtml}

\subsubsection{Instance Variables}
\label{sec:pop-BIter-ivars}
\begin{verbatim}
    const unsigned J;           /* Max number.                  */
    const unsigned K;           /* Number of classes            */
    unsigned index;             /* for iterator.                */
\end{verbatim}

\begin{description}
\item[J] \texonly{---} The total number of customers over all classes
  which the \code{()} operator will enforce.
\item[K] \texonly{---} The total number of classes.  This variable is
  set by the argument \var{N} to the constructor.
\item[index] \texonly{---} A local variable used for generating
  populations. 
\end{description}

\subsubsection{Constructors for B_Iterator}

\begin{description}
\item[N] \texonly{---} Maximum number of customers per class.
\item[j] \texonly{---} The number of customers.
\item[k] \texonly{---} Class from which one customer of \var{N} is removed.
\end{description}

\begin{verbatim}
    B_Iterator( const unsigned j, const PopVector& N, const unsigned k );
\end{verbatim}

\subsubsection{Computation}

\begin{description}
\item[next] \texonly{---} Next Population operator\\
  \code{int operator()(PopVector& N);}

  Compute the next population value.  Return 1 if the new population is
  valid (i.e., the number of customers in each class is less that or
  equal to the number of customers in the corresponding position in the
  \var{limit} vector.).  Return 0 and reset the argument \var{N} to all
  zeros otherwise.

\item[step] \texonly{---} Generate next population.\\
  \code{int step( PopVector& N, const unsigned i, const unsigned n );}

  Ultra-funky recursive step function.  It returns 1 if the population
  vector was updated properly, and zero otherwise.  The inherited
  value of limit sets an upper bound to the population in a given
  class.  Note: an index value of zero will initialize the vector to
  the first feasible population.
\end{description}

\C
\C ----------------------------------------------------------------------
\C
\subsection{Class A_Iterator}
\label{sec:pop-AIter}

The \file{A_Iterator} class is used to generate all feasible
populations for the \var{xe} term of \link{first-come-first-served
  multi-servers}[~(\Sec\Ref)]{sec:multiserver} (\link{Eqn
  (17)}[~\Cite]{queue:deSouzaeSilva-87}).  This class differs from
\file{B_Iterator} in that one chain, \var{i}, must always have one
customer.

The following constraints are
enforced:

\begin{iftex}
  \tex
  \begin{eqnarray*}
    |{\bf n}| & = & J \\
    {\bf n} & \leq & ({\bf N} - e_k) \\
    n_i & \geq & 1
  \end{eqnarray*}
\end{iftex}
\begin{ifhtml}
  \begin{tabular}
    n1 + ... + nk = J \\
    n    <=  (N - ek) \\
    ni   >=  1
  \end{tabular}
\end{ifhtml}

\subsubsection{Instance Variables}
\label{sec:pop-AIter-ivars}

\begin{verbatim}
private:
    const unsigned class_i;             /* index of class with at least 1 cust. */
\end{verbatim}

\begin{description}
\item[class_i] \texonly{---} Class that must always have at least 1 customer.
\end{description}

\subsubsection{Constructors for A_Iterator}

\begin{description}
\item[N] \texonly{---} Maximum number of customers per class.
\item[j] \texonly{---} The number of customers.
\item[i] \texonly{---} Class for which at least one customer must
  always be present.
\item[k] \texonly{---} Class from which one customer of \var{N} is removed.

\end{description}

\begin{verbatim}
    A_Iterator( const unsigned j, const unsigned i, const PopVector& N, const unsigned k );
\end{verbatim}

\subsubsection{Computation}

\begin{description}
\item[next] \texonly{---} Next Population operator\\
  \code{int operator()(PopVector& n);}

  Compute the next population value.  Return 1 if the new population is
  valid (i.e., the number of customers in each class is less that or
  equal to the number of customers in the corresponding position in the
  \var{limit} vector.).  Return 0 and reset the argument \var{n} to all
  zeros otherwise.
\end{description}


\C Local Variables: 
\C mode: latex
\C TeX-master: "lqns"
\C End: 
