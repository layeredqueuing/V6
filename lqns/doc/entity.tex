\C 
\C $Id: entity.tex 5439 2005-01-02 02:30:31Z greg $
\C 
\C $Log$
\C Revision 31.0  2005/01/02 02:30:29  greg
\C Bring all files up to revision 31.
\C
\C Revision 30.0  2003/11/12 18:56:52  greg
\C Bring all files up to revision 30.0
\C
\C Revision 29.0  2003/06/13 00:48:38  greg
\C bounce version number.
\C
\C Revision 1.4  1995/01/11 21:05:12  greg
\C Revised for R12 of MVA software.
\C
\C Revision 1.3  1994/12/09  02:02:36  greg
\C update.
\C
\C Revision 1.2  1994/08/04  20:52:11  greg
\C Intermediate save
\C
\C Revision 1.1  1994/06/24  19:11:32  greg
\C Initial revision
\C
\C ----------------------------------------------------------------------
\section{Entity}
\label{sec:entity}

Abstract superclass for \link{tasks}{sec:task} and
\link{processors}{sec:processor}. 

\subsection{Class Entity}
\subsubsection{Instance Variables}
\label{sec:entity-ivars}

\begin{verbatim}

public:
    Server * station;               /* Used by MVA solver.          */
private:
    const char * myName;            /* Name of this entity.         */
    short traceFlag;                /* trace ref to this queue.     */
    scheduling_type myScheduling;   /* Pseudo task scheduling.      */
protected:
    Vector<Entry *> * entryList;    /* Entries for this entity.     */
    short depth;                    /* My depth in tree.  1 == top  */
\end{verbatim}

\begin{description}
\item[myName] \texonly{---} The name of the entity.
\item[traceFlag] \texonly{---} Boolean for debugging purposes.  Not
  used at present.
\item[myScheduling] \texonly{---} Scheduling type (FIFO, HOL, etc) for
  processors and tasks.
\item[entryList] \texonly{---} The entries for the entity.
\item[depth] \texonly{---} Depth in the tree.
\item[station] \texonly{---} Instance variable set to the
  \link{station}{sec:server} to reference results from the
  \link{MVA}{sec:mva} solvers.
\end{description}

\subsubsection{Constructors for Layerize}

\begin{verbatim}
Entity( const char * aStr, const scheduling_type aType );
Entity( const char * aStr, Vector<Entry *> * entries );
\end{verbatim}

\subsubsection{Initialization}
\subsubsection{Copying}
\subsubsection{Instance Variable Access}

\begin{description}
\item[trace] \texonly{---} \\
\code{Entity& trace( const int flag );}\\
\code{int trace() const;}

Set or return \link{\var{traceFlag}}{sec:entity-ivars} for this
\bold{entity}.  If \var{traceFlag} is set, debugging information will be
printed out for this entity.

\item[copies] \texonly{---} \\
\code{Entity& copies( const unsigned );}\\
\code{unsigned copies() const;}

Set or return the number of copies of the \bold{entity}.  Implemented
by subclasses.

\item[scheduling] \texonly{---} \\
\code{Entity& scheduling( const scheduling_type aType );}\\
\code{scheduling_type scheduling() const;}

Set or return \link{\var{myScheduling}}{sec:entity-ivars}.

\item[entries] \texonly{---} \\
\code{const Vector<Entry *>& entries() const;}

Return the \link{\var{entryList}}{sec:entity-ivars} of the entity.  

\end{description}

\subsubsection{Queries}

\begin{description}
\item[name] \texonly{---} Return the name of the task.\\
\code{const char * name() const;}

\item[typeStr] \texonly{---} Return the type of the entity.\\
\code{const char * typeStr() const;}

\item[isTask] \texonly{---} Am I a task?\\
\code{int isTask() const;}

\item[isProcessor] \texonly{---} Am I a processor?\\
\code{int isProcessor() const}

\item[nEntries] \texonly{---} Return the number of \link{\bold{entries}}{sec:entries}\\
\code{unsigned nEntries() const;}

\end{description}

\subsubsection{MVA Interface}

The following member functions form the interface between the layered
queueing network model and the \link{MVA}{sec:mva} solver.  Note that
some functions will cause segementation violations if called outside
the realm of the \link{layerizer}{sec:layerize}.

\begin{description}

\label{sec:entity-makeServer}
\item[makeServer] \texonly{---} Make a Server.\\
\code{Server * makeServer( const unsigned );}

Generate a station for the MVA model for an entity which acts as a
\bold{server}.  Implemented in:
\begin{itemize}
\item \link{Task}{sec:task-mva}
\item \link{Processor}{sec:processor-mva}
\end{itemize}

\item[freeStation] \texonly{---} Release MVA Model Station.\\
\code{freeStation();}

\label{sec:entity-saveServerResults}
\item[saveServerResults] \texonly{---} Save Results for a Server.\\
\code{double saveServerResults();}

Called by \link{layerizer}{sec:layerize} after a layer has been
solved. 

\item[R] \texonly{---} Return Residence Time\\
\code{double R( const unsigned e, const unsigned k ) const;}

\end{description}
\subsubsection{Input and Output}

\begin{description}
\item[print] \texonly{---} Print entity information.\\
\code{ostream& print( ostream& ) const;}

\item[printStation] \texonly{---} Print corresponding MVA station information.\\
\code{ostream& printStation( ostream& ) const;}

\end{description}

\C Local Variables: 
\C mode: latex
\C TeX-master: "lqns"
\C End: 
